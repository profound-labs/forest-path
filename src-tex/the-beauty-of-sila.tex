
\setChapterAuthor{Ajahn Jayasāro}
\setChapterNote{An edited version of a Dhamma talk given to the Sangha during the 2541 (1998) Rains Retreat.}

\chapter{The Beauty of Sīla}
\markright{\chapterAuthor}

Ajahn Chah taught us to bear constantly in mind that we are
\emph{samanas}; we have left behind the household life for an existence
single-mindedly devoted to peace and awakening. He would say that now we
must die to our old worldly habits, behaviour and values, and surrender
to a new higher standard.

But how exactly do we follow the way of the \emph{samana}? In the
\emph{Ovāda Pātimokkha}\footnote{Not to do evil, to do good, to purify
  the mind -- this is the teaching of the Buddhas.} the Buddha laid
down the most basic and important guidelines for the \emph{samana}'s
path, and there we find that harmlessness is the principle he most
emphasized. Through our way of life as \emph{samanas} we offer the gift
of harmlessness to the world. People may be inspired by how we live our
lives, they may be indifferent, or they may even be contemptuous of us,
but whatever the various reactions people might have to a Buddhist monk,
fear is highly unlikely to count among them. People see a Buddhist monk
and they know that he is not dangerous to them. Animals see a Buddhist
monk and they sense that he is no threat to them. This is a singular
thing.

It's very unusual to be so scrupulous and so caring for even the
smallest kind of creature -- not just human beings, not just the cuddly
lovable kinds of creatures like Shetland ponies and fluffy cats, but
even poisonous centipedes, geckos and biting ants. You find that after
you've been keeping the Vinaya precepts sincerely for a while, the idea
of depriving even a venomous snake or a small poisonous insect of its
life becomes almost inconceivable. With the cultivation of \emph{sīla}
and \emph{mettā}-\emph{bhāvanā} it's just no longer an option. Through
our practice as \emph{samanas} we are able to observe how closely the
devotion to moral precepts is connected to being truly benevolent and
altruistic. The best austerity is patient endurance. The Buddhas say
\emph{Nibbāna} is supreme. One who has gone forth does not harm others;
one who harms others is not a \emph{samana}.

Not to revile, not to do any harm, to practise the Pātimokkha restraint,
to be moderate in taking food, to dwell in a secluded place, to devote
oneself to the higher mental training -- this is the teaching of the
Buddhas, benevolent and altruistic. If we continue to harm other beings
by body and speech, our expressions of \emph{mettā} remain hollow and
cannot lead us to peace. At the same time, if we attempt to uphold a
strict level of \emph{sīla} without a spirit of goodwill and compassion,
without a kind and forgiving heart, we can easily fall into the traps of
self-righteousness, a false sense of superiority and contempt for the
unvirtuous. This is what is called losing the plot.

Our practice of \emph{sīla} and \emph{mettā} starts to mature when we
don't consider that our life and our comfort have any more ultimate
significance than those of even a housefly or a mosquito. Why should our
life be any more valuable than that of a malarial mosquito? I can't
think of any logical reason myself. The Buddha said that as all living
beings desire to be happy and fear pain just as we do, we should abstain
from all actions that deprive beings of happiness or increase their
pain. \emph{Sīla} is an offering of \emph{dāna}, a gift of fearlessness
and protection to all sentient beings. To live our lives within the
boundaries defined by the Precepts, mindfulness of our commitment has to
be constantly maintained; sensitivity and skill are continually called
for. With wisdom and understanding of the law of \emph{kamma}, we
abstain from killing, harming, and hurting any sentient being through
our actions and speech. Gradually, our good intentions unbetrayed by our
actions, we are able to tame our unruly minds.

`Not-doing' or refraining is a kind of creativity. I very much admire
Chinese brush paintings. In these works of art only a very small portion
of the canvas is painted on; the effect and power of the picture are
conveyed by the relationship between the painted form or the painted
area, and that which is not painted. In fact, the large blank area of
the white canvas is what gives the black brush strokes their power and
beauty. So if you were to say to a Chinese landscape painter, `What a
waste of good paper, there's a big white area there that you haven't
painted on at all', he would probably snort with derision. But where
human behaviour is concerned, sometimes we don't see that. I think it's
rarely appreciated that certain things we do have weight, beauty,
integrity, nobility, precisely because of other things that we don't do.
And that skilful abstention from actions, from certain kinds of speech,
or from certain kinds of proliferation or imagination -- that is the
creativity.

Artists and writers mention this often. They tend to agree that the art
lies in the editing -- in what is left out. Many writers say that it is
much more difficult to write in a simple style than in an ornate,

complicated one. Simplicity is a skill to be learned; it does not come
easily. And this is another aspect of our life; making simplicity a
standard to return to. We must seek to refrain not only from the immoral
but also from the needlessly disturbing. We can measure our practice by
how simple our life is. We can ask ourselves, `Is my life getting more
complicated?' If it is, maybe we need to re-establish our attention on
the basics. Pictures need frames. We need wise limits for our actions.
Otherwise our lives become cluttered and our energies dissipated.

Appreciating the austere beauty of the simple, taking joy in simplicity,
leads the mind to peace. What could be more simple than the
\emph{samatha} object in meditation? Whether it is the breathing process
or the word `Buddho`, the experience of unifying the mind in meditation
goes against the whole tendency for mental and emotional proliferation
(\emph{papañca}). Through meditation we acquire the taste for simplicity
in every aspect of our lives. In the external sphere, in our
relationships with others and the physical world, we rely on certain
abiding principles that support the simplicity we seek. The most
important of these is non-oppression of oneself and others. As
\emph{samanas} we seek to imbue our actions with a reverence for life, a
spirit of kindheartedness, benevolence and altruism. And we learn to
make that reverence for life unqualified. The sanctity of life, and the
potential of all beings for awakening form the basis for the 227
precepts of the Pātimokkha. When Ajahn Chah asked Ajahn Mun about the
discipline and voiced his fears that there were just too many rules to
make it a practical guide for conduct, Ajahn Mun pointed to
\emph{hiri}\footnote{\emph{A sense of shame regarding unwholesome
  actions.}} and \emph{ottappa}\footnote{\emph{An intelligent fear of
  the consequences of unwholesome actions.}} as the heart of Vinaya.
Develop these two things, he said, and your practice of the Vinaya will
be impeccable. The commentaries state that these two dhammas are based
respectively on self-respect and respect for others. Respect for life,
our own and others, is the foundation of noble conduct. So we train to
strengthen our devotion to harmlessness -- harmlessness to others,
harmlessness to oneself -- always to bear the welfare of self and
others in mind. The more you open up to the pervasive nature of
suffering, the more compassion arises and the more care you take about
the quality of your actions. You realize that whenever you are not part
of the solution, you're bound to be part of the problem.

In fact, the welfare of self and the welfare of others are
complementary. If we truly understand what our own welfare is, we don't
neglect the welfare of others, because in helping others we grow in
virtue. If we really understand what the welfare of others is, we don't
neglect our own welfare, because the more peaceful and wise we are, the
more we are able truly to benefit others. When there seems to be a
conflict between our welfare and that of others, it is usually a sign of
confusion about the nature of welfare.

A second fundamental principle underlying our lives as \emph{samanas} is
that of contentment. We are taught to cultivate gratitude and
appreciation for the robes, alms-food, lodgings and medicines that we
receive, whatever their quality. We go against the worldly desire for
the biggest, the finest and best. We're willing to make do with second
best or third best. We find we can be happy with the worst, the things
that nobody else wants. That is a wonderful discovery. Whatever we are
given, we remind ourselves, is good enough - beggars should not be
choosers. Even the coarsest requisites that we use have been offered
freely with faith, and have been purified by the benevolence of the
donor. It is our responsibility to make use of the requisites that are
given to us with mindfulness and wisdom. The Buddha said that the merit
gained by the donor is directly affected by the purity of mind by which
we receive and make use of the gift. Thus, even in solitude, our life is
always being affected by and affecting others.

To be content means that we don't waste our time scheming about getting
things that we don't have or don't have a right to. It frees the body
for more wholesome activities and frees the mind for more wholesome
thoughts. As \emph{samanas} we do not covet the \emph{borikan} of other
monks with narrow beady eyes. We don't even touch the possessions of
others unless we have been invited to do so.

The Vinaya lays down many detailed rules concerning our behaviour
towards the material world. In the forest tradition we're taught that
the second expulsion offence can be incurred by theft of even the
smallest object, something the value of one baht (less than one US
cent). In the formal announcementswhich arepart of the ordination
ceremony, the preceptor instructs the new monk to take nothing
whatsoever that does not belong to him, not even as much as a blade of
grass.

To take on that standard -- a single blade of grass -- is the essence
of `leaving home'. It entails a radical shift of perspective from lay
attitudes. Such a standard differs not only from that of criminals and
thieves, but also from that of most `law-abiding' citizens. There are
few people who would not take advantage of some kind of little loophole
in the law if they were absolutely sure they could get away with it --
`Everyone does it, I'd be a fool not to.' Moral rectitude is not unknown
outside the walls of monasteries, of course -- I'm not by any means
suggesting we have the monopoly on honesty -- but for a whole community
to abide scrupulously by such principles is extremely rare.

The essence of our daily life as \emph{samanas} consists of putting
forth effort to abandon defilements and develop wholesome qualities
through meditation practice. We spend hours a day sitting cross-legged
and walking on our \emph{jongrom} paths. Even if we may not always be
satisfied with the results of our efforts, we can at least take heart
from the fact that we've done something practical to purify our minds.
By comparison, the training in \emph{sīla} seems nondescript and its
effects intangible.

To maintain our devotion to precepts and \emph{kor wat,} we need to
remember that spiritual life is not just about doing, it's also about
not doing. Abstaining from things is neither immediately inspiring nor
dramatic. We don't see sudden progress in non-harming or in
non-acquisitiveness, or in not coveting things which are not ours, in
the same way as we might from a good meditation sitting or a good
retreat. But there is movement, even if it is like that of the hour hand
of a clock. And \emph{sīla} is a treasure. It is merit, it is
\emph{pāramī}. How wonderful it is that through living this life
sincerely, \emph{sīla} is steadily accumulating and maturing in our
heart. The Buddha said that \emph{sīla} is the most beautiful adornment
for a human being; it's the only fragrance that is all-pervasive. But
the skill is to remember it, to recollect the beauty of virtue, bringing
it up to refresh and give joy to your heart and mind.

The third principle underlying the life of a \emph{samana} deals with
integrity, restraint and chastity in matters related to the sexual
instincts of the body. That a group of young men -- most monks here are
young -- are able to lead a completely celibate life is almost
unbelievable to many people in the world. They assume that we have some
kind of sexual release, that we must have homosexual relationships or
else that we masturbate. They don't think it's possible to live this
way. People these days can hardly credit the idea that a community of
men can live in a completely chaste way and not be utterly screwed-up,
repressed or misogynist. Maybe we are! -- if we were repressed we

wouldn't realize it, would we? But I don't think so. I think our
community is living the `Holy Life' in a resolute and intelligent way.
And though it's not difficult all the time, for almost everyone there
are periods when it is definitely challenging. It's a struggle, and it
is fitting to feel a sense of wholesome pride in the fact that you can
do it.

It's only through adopting this impeccable standard that we can begin to
understand the whole nature of sexuality. We begin to see its
conditioned nature, how it arises and passes away. We begin to see the
suffering inherent in any attachment to it, how impersonal it is, what
feeds it, what gives power to it -- whether physical conditions, food,
lack of sense restraint or indulgence in imagination. We begin to see it
as a conditioned phenomenon. But we can only have a distance from it, be
able to reflect on it and see it for what it is by refraining from its
physical and verbal expressions.

There is an important point about defilements here, that we have to pin
them down on the mental level before we can let go of them. And the way
we pin something down on the mental plane is by consciously refraining
from, or enduring through, the intention to express it physically or
verbally. This is where the relationship between \emph{sīla},
\emph{samādhi} and \emph{paññā} becomes very clear. As long as we're
still expressing sexual feelings physically or indulging in lascivious
or careless speech about sexual matters, we can never isolate them.
They're moving, They're still receiving energy. We're still keeping them
in motion, we're still feeding the flames. So we seek to counter the
stream of craving, and to do that successfully we must aspire to
transcend sexuality altogether. It is that aspiration as much as the
actual restraint which distinguishes the \emph{samana} from the
layperson.

So as celibate monks we take a whole new stance toward our sexual
feelings, towards women -- half of the human race. We practise looking
on women who are older than us as mothers, and as older sisters if
they're just a few years older than us, or as younger sisters if they're
a few years younger than us. We substitute wholesome perceptions of
women for sensual ones. This is a beautiful gift that we can give women.
An attractive woman comes into the monastery and we refrain from
indulging in sexual perceptions, sexual thoughts about that woman,
replacing them with wholesome reflections, whether by consciously trying
to perceive the woman as a sister, or wishing her freedom from
suffering. Practising \emph{mettā,} we reflect, `May she be well'. We
offer women the gift of a wholesome response to them as human beings,
rather than following the instinctive attraction or obsession with their
bodies or some aspect of their physical appearance. Through that
intention we experience an immediate elevation from the blind,
instinctive level of our being to the uniquely human. It is a movement
from the coarse to the refined. Indeed, the Pali word
\emph{brahmacariya}, which we translate into English as the Holy Life
literally means `the way of the gods'. In other words, within the human
realm, a chaste life led voluntarily and with contentment is the most
refined, sublime and happy form of existence.

As a fourth principle in the Dhamma, we have a love of truth. We
endeavour to uphold integrity and honesty in every aspect of our life.
Honesty includes non-deceit, non-trickiness, non-hypocrisy, not trying
to appear in a way that is not a true reflection of how we are. This
includes not trying to hide our faults or exaggerate our good points.
The goal is to develop clarity and straightness. This may also be seen
as a two-way process. The more honest we are with ourselves about what
we're feeling and thinking, the easier we find it to be honest with
other people. Similarly, the more we train ourselves to be honest in our
external dealings, the easier it becomes to be honest about what's going
on within us.

An important obstacle to honesty is the sense of self. We often attach
to an idea of how we are or how we would like to be. We find it
difficult to own up to those things that don't fit the picture of
ourselves with which we identify, we feel resistance. We feel
embarrassed, ashamed, we find good reasons to dissemble. Everybody likes
to be liked and respected. Nobody likes to lose face. Integrity demands
courage. An unflinching gaze and a devotion to truth -- these are
powers, strengths to be developed.

Truth is power. In many lifetimes the \emph{Bodhisatta} harnesses the
power of his speech through an \emph{adhitthāna}, a resolute
determination: `I have never in my life done this or done that. By the
power of these words may such and such happen.' And it happens. In some
profound inexplicable way, the truth exerts a tangible impact on the
physical world. It can affect events in the most marvellous kind of way.
When one has built up that power of truth, one can draw on that power of
integrity with a sincere, solemn declaration.

So in the path of awakening we take a joy and wholesome pride in caring
for the truth. We contemplate a word the Buddha would use,
\emph{saccanurak}, having fidelity to the truth, loving the truth, being
devoted to truth, and being careful to be honest about what we really
know. It means having the clarity, when we speak, only to utter words
that we know to be accurate; being open to receiving others viewpoints
and not thinking that what we know now is timeless absolute truth;
learning to distinguish between what we know, what we believe, what we
think and what we perceive, and not confusing them. Often when people
say they know something, they mean they believe it. Religious people may
often consider their strong faith to be direct knowledge. The Buddha
said that we care for the truth by being very scrupulous in
distinguishing what we know as a direct experience from what we believe
to be the truth.

Lastly, the fifth principle and precept is devotion to sobriety. The
word `sobriety' doesn't have such a pleasant ring to it. In my mind it
used to bring up an image of thin pinched people in tight clothes,
sitting on the edge of chairs in rooms with flowered wallpaper, sipping
tea and talking about the weather. Carlos Castaneda's use of the word
rescued it for me. Now I've come to like its hard edges. Here I am using
sobriety to signify that clarity and sharpness of mind that is so
infinitely superior to confused, dull or altered states of
consciousness.

After my travels and adventures in the East as a layman, on my way back
to England I stopped off to visit a friend in the mountains in Austria.
She was away for a while and I was in her house by myself. Flicking
through some of her books, I came across a pamphlet called
\emph{Questions and Answers with Lama Govinda}, a transcript of a
session he'd had with some Westerners in Darjeeling. I was particularly
taken by one of his replies. Somebody asked him, `What do you think of
mind-expanding drugs?' and he said, `Well, if you've got an ignorant
mind, all you get is expanded ignorance'. That was it for me: game, set
and match to sobriety.

Mind-expanding, mind-altering drugs and altered states of consciousness
are all still within the sphere of darkness. This is still playing
around with different modes of ignorance. Even if you experience
different dimensions of reality, without the wisdom and discernment of
\emph{paññā} you can't benefit from them. You may transcend one
particular room of ignorance, but you're still in the same building of
unknowing. When you're not out of the building, you're still in the
prison. So this sobriety means turning away from the whole razzmatazz of
abnormal experiences, visions, and physiological and mental states that
are available through liquids, fumes, powders and pills. It means
grounding yourself in the simple down-to-earth clarity of awareness --
of the eyes seeing forms, ears hearing sounds, nose contacting odours,
tongue tasting flavours, body experiencing sensations and mind cognizing
ideas. Seeing the true nature of these things. Being with these things
as they are. And taking joy in that. Coming more and more to focus on
`the one who knows', the

knower and the knowing. This is the great mystery of our life. We don't
want to fuzz and confuse that. We want to clarify it. As we start to
clear away all the garbage of the mind, the sense of knowing becomes
clearer and sharper.

Ajahn Tate, one of Ajahn Mun's senior disciples, stresses the sense of
knowing. He talks about the \emph{jit} and the \emph{jy}. By `\emph{jy'}
he means the sense of equanimity, the clarity of knowing; `\emph{jit'}
refers to thinking, feeling, perceiving. This is his way of talking. And
he gives a very simple means of understanding what he's talking about.
He advises holding your breath for a few moments. Your thinking stops.
That's \emph{jy}. You start breathing again, and as the thinking
reappears, that's \emph{jit}. And he talks about getting more and more
in contact with \emph{jy}, as the mind becomes calm in meditation. He
doesn't talk about a \emph{samādhi} \emph{nimitta} or a mental
counterpart to the breath, he talks about turning towards the one who
knows the breath. So as the breath becomes more and more refined, the
sense of knowing the breath becomes more and more prominent. He says you
should then turn away from the breath and go into the one who knows the
breath. That will take you to \emph{appanā-samādhi}.

So this is sharpening this sense of knowing, knowing the one who knows,
and that's what will take you to peace. But this ability to go from
obsession with the content of experience back to the state of
experiencing and that which is experiencing is simplifying, bringing the
mind more and more together. The mind becomes more and more composed,
and more and more one-pointed.

So I invite you to contemplate these principles which give grace, beauty
and meaning to life. Recognize the extent to which they are already a
part of your life, and continue to cultivate them consciously:
principles of harmlessness, honesty, integrity, chastity, love of and
devotion to truth, sobriety and, above all, the constant clarification
and sharpening of the sense of knowing.

