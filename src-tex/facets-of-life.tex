
\setChapterAuthor{Ajahn Vipassi}

\chapter{Facets of Life at Wat Pah Nanachat}
\markright{\chapterAuthor}

The north-east of Thailand is flat -- the once thick forests are long
gone -- and when one drives along the long straight roads one passes
through mile after mile of flat, scrubby land given over to cultivation,
mainly of rice. There are trees, but just here and there in the open
spaces, occasionally providing a bit of shade; there's no hint of the
majestic and almost impenetrable forest that once dominated the
north-eastern region. Then villages would have been linked by rough
jungle tracks, human beings cutting back the undergrowth here and there
to grow their crops; and they would have had to keep cutting lest nature
return to reclaim these clearings for herself. These days nature here is
firmly under man's thumb.

Driving along the main Si Saket Road out of Warin, the first distant
sight of Wat Pah Nanachat is a long, high white wall, behind which is a
forest. The trees are tall and the growth is thick, a noticeable
contrast to the surrounding terrain. Arriving at the wat on a hot
afternoon, one's first impression on being put down at the gate is of
being about to enter a different world. The view up the drive is like
looking up a tunnel, a tunnel of trees. On venturing up the drive one
immediately feels the cool of the shade -- the forest canopy is thick
and the sun can only glint through the trees, finding an opening here
and there down which to pour a pool of fierce light. The wide, swept
concrete drive opens out after a hundred metres or so into a circle, as
one comes to a long low building on the right, the kitchen, and further
on the large unadorned \emph{sala}. In the centre of the circle, around
which cars can turn but beyond which they cannot proceed, there is a
strange rectangular brick structure, loosely covered with a few scraps
of corrugated iron. In a few days there might be a large crowd of people
gathered here, for this is the place where the villagers cremate their
dead, something which was going on here long before Wat Pah Nanachat was
ever thought of.

The wat came into being twenty-five years ago in a rather unlikely way.
Ajahn Sumedho, who had already been training with Luang Por Chah at Wat
Pah Pong for years, and a group of other Western monks were wanting to
fire some alms-bowls. This is a process whereby a rustproof coating is
baked onto an iron bowl, and it requires heating the bowls in an intense
fire for several hours. In the forest at Wat Pah Pong it was difficult
to come across sufficient quantities of firewood -- there were so many
monks and firewood was needed all the time for dyeing and washing robes.
So Ajahn Sumedho was recommended to go with the other monks to the
forest at the nearby village of Bung Wai, where there were plentiful
supplies of fallen branches and dry bamboo. The group of monks came to
the forest, put up their \emph{glots} (large umbrellas) and mosquito
nets and began their work. This soon attracted the attention of the
local villagers, who were impressed that these \emph{farang} monks had
the courage to pitch their umbrellas and camp out there, for this was
their cremation forest -- a place haunted by ghosts and spirits, and so
feared by the locals that it was left unused.

As often happens on such occasions, when the monks were ready to move on
the local villagers begged them to stay. And it just so happened that
Luang Por Chah had already decided it would be good to start a branch of
Wat Pah Pong specifically for the \emph{farangs}. Ajahn Sumedho, who had
been with Luang Por Chah for eight years, would be the teacher, and the
\emph{farangs} could train in their own language. So, as has happened so
many times in Thailand, the simple act of a monk hanging his umbrella
from a tree was the seed that sprouted and grew into a flourishing
monastery.

During the twelve years of my monastic life in England I heard many
things about Wat Pah Nanachat, and met and lived with many of the monks
who began their monastic careers there. Several of the Wat Pah Nanachat
monks who came to live with us in England ended up disrobing. Several of
our monks who went out to stay there did the same. It appeared that each
of our situations represented a last chance for the other: `If it won't
work in Thailand, at least try it in England before calling it quits',
and vice versa. It was impossible not to form impressions based upon
what I'd heard, but I knew from experience that things are never quite
like you imagine -- no matter how good somebody's description, the
actuality is always far richer and more multifaceted than can be
conveyed by words. So having spent more than ten years in a non-Buddhist
country where one is part of a small group of monastics trying somehow
to model Theravāda monasticism for the surrounding culture, and where
there is often a feeling of learning to be a monk somewhat `at second
hand', I decided to go East and experience Theravāda Buddhism in some of
its native settings. Wat Pah Nanachat seemed the natural terminus for my
journey, but I was in no hurry.

I arrived one hot evening in July 1997, the journey having taken a year
and a half, much of that spent staying in several different locations in
Sri Lanka, followed by a stint in a Wat Pah Pong branch monastery in
central Thailand, attempting to get a foothold in the Thai language. The
first impression was of the size and scale of things, the large rather
gloomy \emph{sala} and the numbers of people. On my first morning I
watched as two large coaches pulled up and disgorged their contents, a
posse of faithful Thai people come to make offerings before the start of
the Rains Retreat. The \emph{sala} was full -- maybe 150 people
gathered to hear Ajahn Jayasāro convey some words of wisdom. `This is
impossible!' I thought, `How can the monks here survive if these kinds
of numbers of people are descending upon the place?' However, I later
realized that this impression lacked a context. I was just seeing things
in terms of the situation in England, where people may often be coming
to the monastery for the first time and have a full bag of questions to
ask. They may also be carrying a large number of inaccurate
preconceptions about Buddhist teachings -- 'Is the Buddha a God?'; `You
Buddhists believe that life is suffering, don't you, and you're trying
to take the easy way out' -- which have to be slowly and patiently
dismantled so that sufficient openness appears for the teaching to begin
to penetrate. Not so here, where the people already have faith, where we
are only one among thousands of monasteries and this group of monks is
not solely responsible for presenting and modelling Buddhism for the
culture.

In fact there are not so many tour bus parties, but things do build up
just before the Rains Retreat period as people go off on pilgrimage for
a few days, each day perhaps visiting six monasteries (and making two
shopping trips). After they've been to Wat Pah Pong, they simply must
come and have a look at where the \emph{farangs} live. Receiving these
visitors can be quite straightforward. Usually they have just come for a
quick look and often don't expect much teaching to speak of; it can be
enough to see foreigners with shaven heads and robes for them to be
profoundly affected. However, this kind of superficial interest is
changing somewhat as Thai laypeople seem to be getting more involved in
practising the Dhamma. Although the conversations may begin with a few
apparently innocent and superficial questions -- `How many monks are
there here?' or `Do you eat once a day?' -- it is more and more
noticeable that the conversation will move on to questions about
meditation and how to practise Dhamma in daily life.

On that first morning, as so often happens, the wave of visitors receded
as quickly as it had flooded into the monastery, and there prevailed a
humid stillness soaked with the high-pitched sounds of cicadas. So to
pick up the question, how many monks do we have? These days it is
getting on for twenty monks and novices of about twelve nationalities.
The number of monks who began here and still live in Thailand is
considerably greater. At any one time we will have four or five junior
monks placed at some of the Wat Pah Pong branch monasteries, having been
sent there to learn the ropes of living with a Thai community and speak
the language. After spending his first five years training under
guidance at Wat Pah Nanachat, the monk is usually `freed from
dependence' and from then on it is up to him. Some monks go off walking
on \emph{tudong}, visiting other teachers and regions. Some settle in
other places and some go abroad, but people still keep in touch and
usually regard Wat Pah Nanachat as some kind of home base, coming back
to check in once in a while. This means that at certain times of the
year there is a lot of coming and going -- in fact, the population of
the monastery can sometimes fluctuate from week to week. Thai monks also
happen by, usually on \emph{tudong}, and more often than not when we
really get down to it they are interested in learning English. This is
not enough of a reason to stay beyond three nights, says the Abbot, and
off they go. We usually do have two or three Thai monks here, but they
already speak English and have some prior Dhamma connection with Ajahn
Jayasāro or the community. For instance, one Thai monk here at the
moment was working as a doctor in America when he met Ajahn Jayasāro,
and his faith arose there upon hearing the Ajahn teach.\footnote{He
  describes this event himself in `No Thai No Farang' on page 85

  .}

The monastery serves several different and quite distinct groups of
people, and for the Abbot this is quite a balancing act. There are the
many guests from all over the world who, for many different reasons,
spend time here developing their understanding and practice of Buddhism
through experience of monastic life. Long-term and loyal support, of
course, come from the local Bung Wai villagers, about a dozen of whom
come every day to cook and help out, and many regular supporters come to
the monastery from the local towns of Warin and Ubon. There is a
sizeable following of Bangkok people who come and stay when they can --
one group of air hostesses even arrange their schedules so that they can
fly up to Ubon on the evening flight, spend all Wan Phra night
meditating and then fly back down to Bangkok on the morning flight. In
fact, on the weekly Wan Phra observance days it is common for some
hundred people to be observing the Eight Precepts and staying to
practise and hear the Dhamma in the monastery until the following dawn.
On these observance days the Abbot and the second monk divide their
attention between the various groups, talks being given simultaneously
in Thai and English in different locations.

A steady stream of non-Thai visitors come and stay for varying lengths
of time. Usually the initial period is limited to three days, but in
most cases this can be extended, depending upon the availability of
accommodation. We require people to write beforehand and will only take
those who turn up unannounced if there is space. Demands on
accommodation are getting tighter these days, so quite often we have to
ask people to come back at a later date. Interest in the monastic life
can be sparked off through coming to stay at the monastery. Men are
asked to wear white and shave their heads after three days, while women
wear a white blouse and black skirt but keep their hair. These gestures
give them a chance to feel they are part of the monastic community for
the time being, and indeed they are perceived as such by the local
people. For many the level of renunciation required is quite demanding
-- living according to the Eight Precepts, eating just one meal a day,
following a routine which requires getting up at 3.00 a.m., and having
many hours of the day with no form or structure. All this can be quite a
challenge.

For men who wish to go further, the next step is to request to become an
\emph{anāgārika}, someone who formally joins the community in a ceremony
where he is given the Eight Precepts in front of the Sangha.
\emph{Anāgārikas} wear a white sarong and white sash, and begin their
training in the rudiments of monastic life under the guidance of the
senior monks. There are no equivalent facilities for women to train
here, but on occasions committed women who can manage to fit into what
is undoubtedly a male-oriented atmosphere have been granted permission
to stay for periods of time.

When people have been with us for some time as \emph{anāgārikas} and
wish to make a deeper commitment, we consider arranging for them to
become \emph{sāmanera s} (novices) -- taking the brown robe and looking
and behaving to all intents and purposes like the \emph{bhikkhus},
except that their code of discipline is less demanding. The have
alms-bowls and go on alms-round with the monks, are given a Pali name
and are expected to commit themselves to training for one year. Those
wishing to take higher ordination can request to do so, and on taking
full ordination are expected to stay for five years as \emph{bhikkhus}
under the guidance of the Abbot.

Community members and, as far as is possible, lay guests are each
assigned a \emph{kuti}, a simple wooden hut on stilts, about thirty of
which are scattered around in the forest (about 150 acres, 48 hectares
or 300 \emph{rai}). The first letters of Pali names are derived from the
day of the week on which the individual is born, and their meaning
usually provides an ideal to which to aspire. Accommodation is basic --
there is no electricity in any but a few \emph{kutis}, and a trip to the
toilet can mean a walk through the forest. At night it is not uncommon
to encounter snakes and other creepy-crawlies. Life at Wat Pah Nanachat
was once described to me as being `total insect attack'; this is an
exaggeration, but it does convey something of the flavour of the
experience. From time to time people are forced to evacuate their
\emph{kuti} as ants or termites invade their living space, which they
have usually already had to reckon on sharing with geckos (lizards about
20 cm long which punctuate the stillness of the night with a loud
`gekk-kko' call), bats, spiders and sometimes the odd snake which
decides to coil itself round the rafters. Rats also compete for the
space and help themselves to anything which can be eaten.

The daily routine varies according to the season. Usually there is a
period of morning chanting and meditation at 3.30 a.m. in a large open
\emph{sala} on the edge of the forest, followed by a leaf-sweeping
period for the lay guests while the monks go out at dawn on their
alms-round. The meal is taken at 8.00 a.m. and is followed by a period
of cleaning chores. From then until mid-afternoon there is free time,
and besides spending that time in meditation people will make use of the
well-equipped library to read and study. At 4.30 p.m. the community
gathers for tea, which is an informal affair where questions can be
raised and things discussed in a good-humoured spirit. A couple of days
a week are kept as silent days, one when all formal meetings are
cancelled and another on which the community follows a structured
practice routine together. On these evenings a formal talk is given.

The atmosphere of the monastery also varies according to the season.
During the three months of the Rains Retreat the community is quite
stable, as sangha members are not allowed to travel away for more than
six days during this period. It is a time of focused practice and study;
in particular, study of the monastic discipline (the \emph{Vinaya}) is
undertaken during these three months. At the end of the Rains Retreat
comes the \emph{Kathina}, the ceremonial presentation by the laity of
cloth which is collectively sewn into a robe by the members of the
community who spent the Rains Retreat together. This is one of the
biggest festivals of the year, and draws the community together before
monks move on to other monasteries or return from other places to live
here. There is also a tradition amongst the branch monasteries of Wat
Pah Pong to attend one another's \emph{Kathina} ceremonies, and so it is
a month of travelling here and there, listening all through the night to
Dhamma talks, and trying to stay awake and centred amid the swirling
changes going on around. For new monks who are just starting to find
their feet in their first Rains Retreat, this time can be quite
disorienting.

When the wind swings round to blow from the north, the local people say
that this marks the beginning of the cold season. As the rain stops and
the weather turns cooler, people fly kites in the almost continuous
breeze, flying them high over the rice fields. They attach a device to
them which plays a low, melancholy kind of tune over and over, and this
characterizes the atmosphere of the cold season. This is really the most
pleasant time of the year here, and it is common for senior monks from
England to come visiting during this period from late October until
February. During the last cold season we had visits from Bhante
Gunaratana from Virginia in October, Ajahn Munindo in November, Ajahns
Pasanno and Viradhammo in December and Ajahns Sumedho and Attapemo in
January. Luang Por Sumedho comes to Thailand annually for the
commemorative celebrations for Luang Por Chah, which are held at Wat Pah
Pong in the week leading up to the anniversary of his death and
cremation (which was one year later) on 16 January.

The cold season is also a time when frequent trips are made to the
nearest of our small hermitages. Poo Jom Gom, which means `little
pointed hill', is situated on the Laotian border, about 150 km from
here, and is set in a large area of national parkland. Four or five
monks stay there most of the time, living spread out over an area of
about two square miles, some in caves, others in simple thatched
\emph{kutis}. Some of these dwellings look out over the great Mekong
river that forms the border between Thailand and Laos and flows south
from China, touching Burma, Thailand, Laos and Cambodia, before reaching
the sea in Southern Vietnam. It's one of the world's great rivers,
comparable to the Amazon, the Nile and the Mississippi in length. At the
moment just a few little motorized canoes ply back and forth linking the
villages on either bank, which share a language and set of customs. In a
few years this area will probably develop and be much like the rest of
Thailand, but at present it is still quite remote and is touched only
lightly by the hand of modern culture.

At the end of February almost the entire community travels across the
country to our other hermitage, Dtao Dam, on the Burmese border in
Saiyok National Park beyond Kanchanaburi. This leaves just a skeleton
crew minding the monastery, and so things quieten down as the hot season
begins. Wat Pah Nanachat remains quiet for two months, until the Sangha
returns at the beginning of May. During the following months leading up
to the Rains Retreat there are more comings and goings, people returning
to Wat Pah Nanachat to spend the Rains Retreat here, and young monks
being sent off to Thai branch monasteries to spend a year away. By the
time when Luang Por Chah's birthday is celebrated at Wat Pah Pongon 17
June, it is usually clear who is going to be where for the next four
months or so, and the monastery starts to take on a much more settled
and stable atmosphere.

It was this situation I encountered when I first arrived here, and for
the first few months the impressions I formed were based on this
background feeling of stability in the community. It was some surprise
to see what happened here in the months after the Rains Retreat, when
all of a sudden there was a lot of coming and going. This is quite
difficult and challenging for people who are still fairly young in
training. As a young monk spending your first Rains Retreat here, you're
just starting to get your bearings and settle into the training with the
group of companions you've been living with over the last three months.
Then suddenly the Rains Retreat is over and two people have disrobed,
three people have shown up from other places, three more have left for
other monasteries and the character of the Sangha has completely
changed. This is quite a contrast to monastic life in England, where
there simply aren't the opportunities to leave and go elsewhere. You can
go to Aruna Ratanagiri or Chithurst if you've been at Amaravati for a
long time and are feeling in need of a change, but that's about it apart
from going abroad. Hence the atmosphere in the communities there is
often more stable, and I think it is easier there in some ways to stay
put and endure the difficulties you might have. And from that you can
learn and gain strength.

The wealth of different monastic opportunities here in Thailand is both
a blessing and a curse. One is surrounded by a culture that still
carries the monastic form with considerable confidence, and this can be
tremendously uplifting for a monk who has grown up in the UK. For me,
simply going on almsround every day in the traditional way has felt like
a shot in the arm after thirteen years of not having had the
opportunity. The faith of the laypeople in Thailand is an unending
source of support, and there are opportunities here to meet and live
with monks of wide experience and great wisdom. On the negative side, it
can be difficult to settle. There are always people coming and going
with tales of this place or that, this Ajahn or that, and for someone
who has a lot of restlessness or discontent, the temptation to go off
and explore new pastures is indeed great.

Reflecting on this, I feel quite grateful to have spent my first
thirteen years or so as a monk living in just three places. When you
stick with one thing, you see how moods and atmospheres in a place
change on their own. When things are difficult, uprooting and going
somewhere else is not always the answer. As Luang Por Chah once said of
a monk who was forever going off in search of a better place to
practise: `He's got dog-shit in his bag. He gets to a place and thinks,
``Oh, this looks promising, nice and peaceful, good teacher, good
community, I should be OK here'', and he puts his bag down and settles
in. Then after a while, ``Hmmm, what's that bad smell? I can't stand for
that, the place seems to be full of it. Oh well, better try and find
somewhere else'', and he picks up his bag and off he goes.'

So when just staying put in one place one has to be willing to roll with
the changes, which can help to develop some internal stability. One has
to investigate things, reflecting again and again that this and that is
not going to last, and just letting go, letting go and not creating
problems over how things change. It's a relief to realize that one does
not have to fix things or try to hold them steady in order to feel at
ease. The problem comes from being convinced that they should be
otherwise, when, perennially, it is `just this way'.

One factor that has brought an increased sense of stability to Wat Pah
Nanachat is the decision by Ajahn Jayasāro to stay put here for five
years as Abbot. In the past Ajahn Jayasāro and Ajahn Pasanno would take
it in turns to administer the monastery for a year at a time, which
allowed each of them to have a period of retreat every other year.
Looking back, however, I think that Ajahn Jayasāro wonders how good this
was for the community. An additional thing that has made being Abbot
more workable is the new Abbot's \emph{kuti}. The previous one was
virtually open on all sides, like living on a platform, and only a
stone's throw from the \emph{sala}, which meant that visitors could seek
the Abbot out at any time of the day or night. No wonder it was
stressful -- the Abbot had very little privacy there. I thought it a
healthy sign, then, when I saw that Ajahn Jayasāro was having a new
Abbot's \emph{kuti} built at quite a distance from the \emph{sala}, in a
less conspicuous location and with a much greater feeling of privacy to
it. `That's significant', I remember thinking. `If the Abbot knows how
to look after himself, can take space and find some recuperative
solitude here, he won't feel the need to escape to get some time on his
own. That seems like a healthy direction.'

Ajahn Jayasāro has commented that there is a more harmonious atmosphere
here these days than he can ever remember. Whereas in the old days monks
used to look forward to getting past the five Rains Retreat mark so that
they could go off on their own, there is less of this kind of talk now,
and monks who have grown up here in the last five years seem to regard
Wat Pah Nanachat as home. When the community is harmonious the Abbot is
better supported, and he is more effective at what he does. So it
becomes a more attractive prospect to stay here.

Here, then, are just a few fleeting impressions of this mysterious,
multifaceted place. One of the things I've heard Ajahn Jayasāro comment
upon more than once is how he feels when people talk about what Wat Pah
Nanachat is -- `Oh, you don't want to go to Wat Pah Nanachat. It's like
this or that', or `Wat Pah Nanachat is a really great place.' He says
that he's been around Wat Pah Nanachat for over twenty years and the
place is constantly changing. You can't say for certain what it is, even
though people try. They take away a snapshot of how it might have been
at a particular time when they visited or lived there, and then they
tell people, `Wat Pah Nanachat is like this', grinding out the same old
stale impressions year after year when in fact it has long since
changed. Well, if the Abbot himself declares that he doesn't really know
what Wat Pah Nanachat is like, who are the rest of us to presume to say?

The Author

\emph{Ajahn Vipassi left Thailand a short time after this piece was
written. He stayed in various monasteries in Europe, then in 2000
decided to return to lay life. He has since returned to live in the UK
and built up a computer support business.}

