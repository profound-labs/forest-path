% Title: Dtao Dam Forest Monastery
% Author: Ajahn Jayasāro

Dtao Dam Forest Monastery is a branch monastery of Wat Pah Nanachat, 
located in the mountainous, thickly-forested jungle of Mae Nam Noy, Sai
Yok National Park in the Kanchanaburi Province of Thailand. Mostly
through the tireless efforts of the monastery's main lay supporter, a
brave and determined woman called Tivaporn Srivorakul, the pristine, 
lush quality of the Dtao Dam forest has been well preserved. Tivaporn
operates a tin mine in the Dtao Dam area, employing Burmese, Karen and
Mon workers who live in this remote border region in order to escape the
social and political strife of present-day Burma. Despite great pressure
on her for many years, she has stood up to all those interested in
destroying the forest. 

At the onset of each hot season, the monks and novices of Wat Pah
Nanachat make a three-day \emph{tudong} through the National Park into
the monastery, where they spend two months in retreat. This year (April
 1999), the \emph{tudong} was cancelled due to cattle and
drug-smuggling activity in the outlying area, forcing the Sangha to be
brought in by four-wheel drive vehicles. The following interview
covering these and other issues related to Dtao Dam was conducted by Tan
Paññāvuddho at Ajahn Jayasāro's \emph{dtieng} in the Dtao Dam forest. 

\emph{Tan Paññāvuddho}: Tan Ajahn, to begin, could you give a brief
history of Dtao Dam Forest Monastery? How was this place founded, and
how has it developed and evolved to reach its present state today? 

\emph{Ajahn Jayasāro}: In 1979 Ajahn Pasanno was on \emph{tudong} in
Kanchanaburi. He got to know Khun Sunan, the owner of another mine in
this part of Kanchanaburi, and she built a \emph{kuṭī} for him in the
forest near her mine where he spent the Rains Retreat. That mine is now
abandoned -- we pass it as we come into Dtao Dam. Years later the owner
of the Dtao Dam mine, Yom Tivaporn, was struggling to preserve the
forest here in the National Park. She thought this would be a wonderful
place for forest monks to live and train, and hoped that their
presence might also deter hunters and loggers. So she invited Ajahn
Pasanno to bring some monks in and do a retreat here. She offered to
make sure food and any other requisites would be provided, as there's no
village for alms-round. The idea was that if it was appropriate some
basic \emph{kuṭīs} would be built and two or three monks could stay on
for the Rains Retreat. Everything worked out as she hoped. 

\emph{Tan Paññāvuddho}: So in what year did Ajahn Pasanno first bring
the monks? 

\emph{Ajahn Jayasāro}: About eight years ago. That year two monks spent
the Rains Retreat here in \emph{kuṭīs} built at the foot of this
mountain. Since then monks have spent the rainy season here on a regular
basis. Gradually, over the years, there have been developments. A few
years ago we had a tractor come in and cut a road up to the upper \emph{sāla}.
Three \emph{kuṭīs} have been built up on the ridge there. 

As you know, where we are now (to the west of the upper \emph{sāla} on a
different mountain, where the inner \emph{sāla} is located) is a very
densely-forested plateau through which the stream meanders. We have
twenty \emph{dtiengs} scattered throughout the forest for the monks to
use during the hot season. On this plateau the climate is very pleasant
in the hot season, but extremely damp and unpleasant in the rainy
season; there are many leeches, and if you put a robe out to dry after
washing it, it may still be wet after twenty-four hours. Hence the need
to build the upper \emph{sāla} and three small \emph{kuṭīs} on the more
exposed ridge over to the east of here, so monks can stay at Dtao Dam
during the rainy season. 

We also built a lower \emph{sāla} down at the base of these two
mountains. The monks walk down in the early morning, take their daily
meal and then come straight back up the mountain afterwards. There's no
village for monks to go on alms-round, which makes this quite an unusual
place for monks to live and puts serious restrictions on how many can
stay here on a long-term basis. But we have a lot of lay supporters, 
particularly in Kanchanaburi, who give Yom Tivaporn food and provisions
to bring to the monastery when the whole community comes in the hot
season. There are a couple of Burmese families who cook for the monks,
particularly one lady.

\emph{Tan Paññāvuddho}: Is it common for monks to inhabit forest in
National Parks?

\emph{Ajahn Jayasāro}: About six years ago there was a period in which
the presence of forest monks was looked upon by the government as being
detrimental to the forests. In a number of highly publicized cases monks
were accused of felling trees to build lodgings and developing
monasteries in a way that harmed the forest. A government plan proposed
having all monks living in national parks or reserved forests ejected
from them. In fact, although there have been isolated instances of monks
being insensitive to forests they were living in, this has never been
one of the major problems facing forests in this country. But anyone who
lives out in country areas knows that the presence of monks is the major
factor inhibiting deforestation. It's hard to say to what extent the
plan was devised out of ignorance, and to what extent it was influenced
by the forces who want to get the monks out of the way. Fortunately, 
however, there was an about-face -- the plan aroused a lot of
opposition. Tan Chao Khun Payutto wrote an influential pamphlet
explaining the importance of the forest to Buddhism and Buddhist
monasticism, and subsequently a new plan emerged which involved giving
opportunities for monastic communities to help to care for the forest. 
The new plan allowed monasteries to apply for permission from the Royal
Forestry Department to look after a certain area of land, from a
thousand \emph{rai} upwards to five thousand \emph{rai}. Looking after
the forest in this case means little more than living in it in very
basic dwellings. The respect that monks command, at least amongst the
local people, is acknowledged to inhibit the destruction of the forest. 
The Forestry Department is seriously understaffed and underfunded. The
U-turn regarding the role of monks in forest conservation was more or
less an admission of defeat. They know that they can't prevent the
forest from disappearing and the government prefers to spend its money
on other things. 

So we applied to participate in this programme. In fact, we asked for a
lot more land than we were eventually given. The head of the National
Park (suspected by many to be deeply implicated in illegal logging in
the park and now transferred) was not supportive and cut our application
down to a thousand \emph{rai}. The only condition laid down for us to be
here is that we don't do a lot of building and don't cut down any trees. 
We have fulfilled our plan to build the three small \emph{sālas} and the
three \emph{kuṭīs}, so as for the material development of Dtao Dam, I
can't see that there is really much more that needs to be done. Anyway, 
Tivaporn feels that if there was to be any more building, the officials
hostile to her would use it as a pretext to accuse her of something. 

\emph{Tan Paññāvuddho}: Who are these officials? Why are they apparently
so hostile to Tivaporn? 

\emph{Ajahn Jayasāro}: Well, she's a thorn in the side of the
`baddies'. She stands in the way of certain people making a lot of
money. Every branch of the Thai civil service is plagued by corruption. 
It seems, at least from the outside, that the corrupt officials
outnumber the honest ones to a frightening degree. In many places
corruption is institutionalized, difficult to avoid being sucked into
because social pressures to conform in the workplace are so powerful. Of
course there are some who manage it, but if you stay honest it means
you're unlikely to advance very far in your career. Your boss may well
not put you forward for promotions. Some people who are ambitious and
feel they have something to offer society justify their corruption by
arguing that it is the only way they can get into a position where they
can change things for the better. 

In the eyes of many people in this country, forests mean money. Dtao Dam
is, to put it bluntly, big bucks. Some of these people who hate Tivaporn
don't see nature as we do when they see a forest, they don't see
wildlife -- they see money in a previous existence. Trees are money. 
Land and animals are money. And because there is so much corruption, 
there's so much influence-peddling up to the very highest levels. That
means the laws don't have the kind of irrevocable fixed quality that
they have in the West, at least in most people's minds. In Thailand you
always feel that there's a way to get around things. You can often get
things changed, get exceptions made, if you know the right people, or if
you are the right people yourself. There actually was a law prohibiting
gas pipelines through national forests, so a special law was passed to
make it all right. At the same time, a sub-clause allowing mining
operations in national parks somehow slipped in. 

There are various kinds of scams for forestry and national park
officials, the most obvious one being turning a blind eye to the illegal
logging interests. As time goes on and good timber becomes increasingly
rare, the whole business becomes more and more lucrative. National parks
can be eligible for grants for reforestation, but the area in question
has first to be classified as degraded forest. So a common ploy is to
light a fire or cut down some trees and then get the designation of the
land changed. Once it has been declared `degraded forest' you can apply
for funds for reforestation. Then you can use one part of the grant for
reforestation but keep the rest for yourself and your henchmen. You can
also let the loggers have more of the original big trees. And so on. 

On the national level, forests all over the country are under pressure
from the growing human population. The whole question of land settlement
has become a political hot yam and it brought down the government before
last. The question that has arisen with some urgency over the past few
years is: what do you do when poor and desperate people encroach on
national park or reserved forest in order to clear the land and grow
themselves some food? If you evict them, where are you going to put
them? What will be the political repercussions for political parties
which seek to present themselves as protectors of the poor? The answer
given by the previous government was to pass a law giving these people
squatters' rights, which in effect encouraged people all over the
country to clear land in national parks for farming. 

So there are many pressures on Dtao Dam. Corrupt local politicians, 
government officials, businessmen, soldiers, border police: a lot of
ethically challenged people have their eyes on the place. Theoretically
it should not be too hard to protect Dtao Dam. To get wood out there's
only one road, and there's a border police barrier across it and a
barracks overlooking it. But of course, all the people at the police
post are on a very low wage, far from home. People at the National Park
are involved. Everyone takes their share. 

\emph{Tan Paññāvuddho}: Tan Ajahn, a couple of weeks ago several of us
climbed to the top of the tallest local mountain peak, where one can get
a view in all four directions. From there it is forest as far as the eye
can see. You have mentioned before that there are plans to build a road
from Burma into Thailand though this mountain pass. Why are various
parties so interested in building a road through such remote mountainous
forest? 

\emph{Ajahn Jayasāro}: If you look on a map, you'll find this pass is
the shortest and most direct route between the city of Kanchanaburi and
the Burmese port of Tavoy, which is maybe forty kilometres away. That's
not a long way to cut a road. It would provide Thai industry with direct
access from Bangkok, through Kanchanaburi, to the Andaman Sea -- port
facilities, holiday resorts and so on. The army officers, politicians
etc. who've been buying up land on either side of the prospective road
would make huge profits when the land prices go up. And of course, once
the road is cut you open up the entire forest for exploitation. In every
case that I can think of, a road built through a forested area has
signalled the beginning of the end for that forest. I don't think
there's any question that the road will be built -- it's just a
question of what route it will take. The pass here happens to be the
easiest traverse of this mountain range. Further to the south and to the
north the mountains are much more difficult to cross. This is why in the
past, during conflicts between the Thai or Burmese, this was the pass
that the invading armies would march through. It's a kind of gap in the
mountain range. From where we are sitting right now at an elevation of
about seven hundred metres, we're only about a fifteen-minute walk from
Burma. From an engineering point of view it would be a relatively easy
job. Driving along this route, you'd hardly notice the gradient. 

\emph{Tan Paññāvuddho}: So out there, what interested parties besides
the monks are working to preserve this Dtao Dam forest? 

\emph{Ajahn Jayasāro}: Well, the Thai environmental movement is in many
ways still in its infancy. Probably only in the last five to ten years
has it had any kind of muscle. It is only very recently that the
authorities have even felt a need to pay lip-service to environment
protection. Of course there have always been people warning against the
destruction of the environment, but during the boom economy that
preceded the economic collapse, many people just assumed that
environmental degradation was the price you had to pay for prosperity. 
Preserving forest was considered the concern of romantics or people not
living in the real world. After all, what does a tree contribute to the
gross national product? Now, of course, with the economy on the rocks, 
the reasoning is that there are more pressing priorities. 

Another telling factor is the perception of forest. Until recently
forests were associated in Thailand, and indeed throughout Asia
generally, with backwardness. Cities mean civilization. For example, you
have the Thai word \emph{Pah-thuean}. `\emph{Pah}' means `forest' and
`\emph{thuean}' means `uncivilized', implying that people who live in
the forest are backward and uncivilized, and that everything associated
with nature is the past. That's been a strong underlying idea in the
view of the urban population, and particularly perhaps among businessmen
and politicians. There has never been any sense of the forest as a
national heritage that should be looked after. And this prejudice is
still very strong. It's only recently that there have been attempts to
introduce a more progressive understanding of nature in the school
curriculum, with the aim of creating a new respect for nature. Perhaps a
period of alienation from nature is necessary. As far as I know, love of
nature did not develop in the West until the Industrial Revolution
either. If you look at the history of Western art, for example, it
wasn't until the eighteenth century that there were pure landscapes, 
paintings in which nature was considered an interesting topic in itself, 
rather than as merely a backdrop for human beings. 

So there are various non-governmental environmental groups at work. But
as far as the government itself is concerned, the Ministry of the
Environment is officially called the Ministry of Industry, Technology
and the Environment, which gives you some idea of the priorities and
conflict of interests there. 

\emph{Tan Paññāvuddho}: All three interests lumped together? 

\emph{Ajahn Jayasāro}: All lumped together. And it's the same with the
Forestry Department. It has a dual role: firstly, caring for the forest, 
secondly, the promotion of agro-forestry. So it's inevitably tied up
with big business and not free to preserve the forest. 

There are good people, educated people particularly, who are starting to
appreciate nature and coming together to protect what is left, but they
often feel somewhat helpless and intimidated by all the forces against
them. These people live in the cities, a long way away. As you can see, 
I am not particularly optimistic, but there is a bright side. Here at
Dtao Dam we have representatives from the palace helping us. 

Right from the first days of the Thai kingdom in Sukhothai, there was a
custom whereby on Wan Phra anyone could go to the palace and ring a
certain bell to request an audience with the King in which to let him
know their grievances. You could discuss a problem where you hadn't been
given a fair deal by the government authorities, for instance, or where
you'd been the victim of corruption. It was a very special appeal court
that bypassed the judicial system. This custom has come down to the
present day. Now it takes the form of an office in the palace to which
anyone can write and appeal. The people who work in this office liaise
with the Queen's private secretaries to investigate these claims, so you
go right to the top. And since the power of the monarchy is strong, this
office exercises a lot of influence outside the conventional power
structures of parliament. People from this office have been instrumental
in getting some of the most corrupt officials transferred, for example. 
We also have friends in various conservation bodies, as well as a senior
minister in the present government. 

\emph{Tan Paññāvuddho}: What about Nature Care, the environmental
preservation organization founded by Ajahn Pasanno? How is it involved? 

\emph{Ajahn Jayasāro}: Nature Care is an NGO, a non-governmental
organization, set up originally in Ubon to help preserve the forest
bordering the Mekong River. Subsequently, with our association with
Tivaporn, Nature Care established a branch in Kanchanaburi. This
facilitates applications for funds from businesses and government bodies
for conservation purposes. 

Returning to Dtao Dam again, one of the things that complicates the
issue here is Tivaporn's mine. She has been running this mine since well
before the area was declared a National Park, and she's deeply in debt. 
The mine also functions as a kind of welfare programme for over 100
workers and their families, most of whom are undocumented immigrants
from Burma. Her mining concession will last for another four or five
years. But people who wish she wasn't here -- she is the main obstacle
to people destroying the forest -- have gone as far as to threaten her
life on a number of occasions. Others have instigated various rumours to
blacken her reputation. One of the things that people say is that she
invited the monks here as a front. Or they say that she's trying to
create the image of being an environmentalist, when all she is
interested in is keeping her mine going. She has even been accused of
building a private luxury resort here. 

\emph{Tan Paññāvuddho}: That's pretty unbelievable! 

\emph{Ajahn Jayasāro}: Well, you'd think so, but as the people
spreading it were powerful and influential, this accusation apparently
reached the highest circles of the government and the royal family. As
you know, last year Tivaporn was given a prize as Thai Citizen of the
Year by a leading charity of which the Crown Princess is the patron. The
charity was worried they'd been deceived and decided to investigate. Of
course, they found the charge was baseless. But these kinds of slurs are
leaked to the newspapers. Also, senior army officers are given folders
containing facts about Dtao Dam which distort the picture. For instance, 
Tivaporn was shown a folder two weeks ago in which her signature for an
application for funds for reforestation of Dtao Dam had been forged. 
It's not clear whether someone is attempting to embezzle these funds or
is trying to accuse her of receiving the funds and not using the money
for reforestation. But this kind of thing is going on all the time. All
the various parties involved, even the environmental groups, have their
own contacts, and they hear these things. Naturally, they're not always
sure what to believe. So the policy for us has been to invite these
people in to see for themselves. 

Another problem down the road is when the mining concession ends. One of
the plans is to have various kinds of scientific projects going on, 
especially ones related to biodiversity and botanical research. This is
an area rich in biodiversity. A botanist was here a few weeks ago and he
was extremely excited by what he saw, all kinds of things that he'd
never come across before. So there is a hope that there will be some
kind of botanical or biological station at the base of the mountain, 
where graduate students can come and do research. 

\emph{Tan Paññāvuddho}: What about the animal life? We've seen all
kinds of exotic animals, from elephants to black panthers to white
tigers, and bizarre-looking creatures unlike anything I've ever seen
before. 

\emph{Ajahn Jayasāro}: That's why we have those animal-sighting forms, 
to document that these animals are really here. It's a matter of getting
this information into the hands of the people with the right intention
towards forests. 

\emph{Tan Paññāvuddho}: Tan Ajahn, you've alluded to it somewhat, but
could you articulate your role as Abbot and as a forest monk in
addressing these problems? 

\emph{Ajahn Jayasāro}: Well, as you know conventionally speaking, I am
the Abbot of Dtao Dam, but most of the year I live at Wat Pah Nanachat. 
I'm able to come out for a month or two in the hot season, but I keep in
contact by telephone with Tivaporn when I'm back in Ubon, to stay
abreast of what's going on and give her support and encouragement. She
gets bullied and slandered a lot. It's a lot to put up with, apart from
the ordinary pressures of running a business in adverse circumstances
and struggling to make enough money to keep going. You know what an
incredible drive it is in and out from here, and you know how often she
does that. She goes out for a day, then comes back in again, then drives
a truck all the way to Phuket to sell the ore from the mine, then drives
all the way back up again to meet with all these academics and
scientists here for a visit tomorrow. I've asked her to phone me right
away if anything really difficult or heavy comes up. For example, she
gave us the advance warning about the cattle and drug-smuggling going on
across the border this year. 

Also, a role that one plays as a forest monk who is also an Abbot is to
be a liaison or central figure. Being the Abbot of a large monastery, 
and having been in Thailand for many years, I've come to know a lot of
people and I can help interested parties get in contact with each other. 
So to summarize my role, I'd say it involves first, giving moral, 
spiritual support, Dhamma teaching, encouragement and reflections. 
Second, in the social role as the Abbot of a well-known monastery who
knows lots of people, I can help the right people get in touch with each
other. 

\emph{Tan Paññāvuddho}: Seems potentially like a pretty adversarial
situation. How do you manage to maintain a non-partisan position in a
scenario that is very partisan? 

\emph{Ajahn Jayasāro}: The abbot of a monastery usually plays the role
of being the referee or the impartial resort for both sides in a
dispute, whereas here it's a little different in that we're part of it. 
I'm not absolutely equanimous about this. I'm not totally impartial. I'm
definitely on the side of the people who want to save this forest. But I
find it important to avoid the `us' and `them' way of thinking. Also, I
don't personally have to confront these people trying to destroy the
forest the way Tivaporn does. I've met very few of the leading figures. 
And Thai society being what it is, one always keeps up social
proprieties. If someone were to wish bad things for the forest monastery
and curse us to our backs, if he met me he'd probably bow and speak very
politely. 

For a monk the principle is always, `What's correct according to
Dhamma-Vinaya?; what's wholesome, what's right?', and standing up for
that in certain circumstances, while being sensitive to time and place
and the way to go about things. And not to come across as being
adversarial. For example, if something illegal is going on in the
forest, I wouldn't confront the person directly, but I might try to see
their superior, or superior's superior. Rather than being a problem
between me and someone, it's a problem in the wider community that calls
for the right people to be alerted. 

\emph{Tan Paññāvuddho}: In personal terms, training here in this
remote, wild forest has been a very enjoyable and profound experience. 
Could you put into words why it is so important for monks to train in
forests? What are the advantages? How do we reflect on and learn from
nature in the context of Dhamma practise? 

\emph{Ajahn Jayasāro}: Well, the practise of Dhamma is one in which
it's very important to develop the ability to calm the mind, to make the
mind peaceful and concentrated. That being so, it's essential to have a
conducive environment in which there is nothing too jarring or too
exciting. So we lead a very simple life, one bared down to the
essentials, not surrounded by anything man-made or anything that's going
to pull you out of yourself. Living in a forest, there's nothing really, 
nowhere for your eyes to go -- just greenness and trees all around. It
automatically encourages you to incline inwards. The natural rhythms of
the forest, of the trees and the streams, give a sense of uplift and
well-being to our minds. They ground us. This provides a very important
foundation for the meditation practise. It comes to feel natural to be
by yourself and you come to delight in solitude. Sitting meditation and
walking meditation become obvious ways to spend time, not something you
have to push yourself into. I think that for most people support from
the environment is still vitally important for success in practise. 
Ajahn Chah would often talk of the relationship between physical
seclusion and seclusion from the Hindrances. 

The forest is not quiet, but it is tranquil and it is teaching you the
laws of nature all the time. The things you see around you are just
natural phenomena. You're surrounded by birth, ageing, sickness and
death, arising and passing away in the most raw and obvious forms. As
you reflect on those principles internally, your contemplations find a
resonance outside of you. The phases of the moon, dawn and dusk, the
play of heat and cold, the whole natural environment attain an increased
profundity, because they express the nature of things you're
investigating internally. You feel a sense of harmony and a seamless
unity between the inner and the outer. 

Living at the foot of a tree, keeping the \emph{dhutaṅga} practices, we
also have this wonderful feeling of being the inheritors of a tradition
that stretches back for over 2,500 years. We are not living so
differently at all from the way the great monks of the Buddha's time
lived. That sense of being a part of something larger, something noble
that stretches in an unbroken line right back to the Buddha: I think
that's a very wholesome feeling, one that a monk may cherish. 

This particular forest, being home to so many wild animals, gives us the
opportunity to look at fear, anxiety and attachment to the body in a
very direct way, seeing the effect they have on our mind, the sense of
urgency they give us. While living in the forest, a lot of these
phrases: `a sense of urgency', `making every moment count' -- teachings
we've read about and studied -- really come alive. In a way it's
difficult to articulate, I feel a sense of rightness, a feeling of `This
is exactly how I should be living', and `This is how a monk lives'. 

\emph{Tan Paññāvuddho}: In the Suttas, so many passages from the
Buddha's enlightenment to the \emph{Parinibbāna} take place under trees. The
Buddha always lived and practised in the forest when possible. With the
forest disappearing, and the subsequent likelihood that this will be the
last generation of forest monks, how do you see a Buddhist monk
responding to a predicament like this? 

\emph{Ajahn Jayasāro}: Well, you don't have any choice really. There's
not so much that can be done. As you become a more senior monk and you
have more responsibilities and opportunity to teach the Dhamma to lay
Buddhists, you can at least point out the value of the forest, how
little remains and the need to look after it. It's also important to
point out the relationship between the forest and the Buddhist religion. 
Then as a younger monk, just make the best of the forested areas
available while you can. But I think it's really important to have young
monastics come out and experience this way of practise right from the
beginning of their monastic career, because it can create such a strong
impression. You know you'll remember this for the rest of your life. 
Hopefully you will continue to have the possibility to keep coming here
or to places like this to train. But even if that is not the case, for
monks to have the experience of living simply in a forest like this, 
even once, has a ripple effect. As monks become more senior and have
their own monasteries and their own disciples, they will pass on their
love of nature and appreciation of the role of solitude in monastic
life. 

But you know, I must admit I sometimes feel that this destruction is not
going to stop until every last tree outside of private hands is gone. To
effect a real change, it has to start in the schools. Last week, when I
went out to see what was going on with the various arson fires in the
area, Tivaporn was running a retreat for school kids in the village at
the edge of the forest. And the kids loved it -- they really responded
to the teaching about nature very well. At one point the children were
asked what their parents do for a living, and three of them replied that
it was illegal logging. It's the same everywhere. In Huay Ka Kaeng, just
north-east of here, there is a lot of wild forest. It's labelled a World
Heritage Site. It's also the place where a forest park official called
Seup Nakasatheean killed himself as a gesture to call attention what is
happening to the forests across Thailand. A foundation was set up in his
name and the army was sent in to look after and patrol the land. But
still there is cutting going on to this day. I myself don't see any
fundamental changes being made until there comes a point where laws are
laws, and whoever breaks the law is wrong and is dealt with
appropriately. 

But I feel reasonably confident about the prospect of saving Dtao Dam. 
It is really hard to say when you don't know everything going on, really
hearing everything second or third-hand. Still, overall, I think there
are enough people with influence who know about Dtao Dam to keep this
tract of forest intact. 

\emph{Tan Paññāvuddho}: You mentioned that you went to a conference
about Dtao Dam last year. What was that like? 

\emph{Ajahn Jayasāro}: Well, yes, it was strange for me. One has the
idea that Dtao Dam is a forest far away from anywhere else, where we
come on retreat every year. But at this conference I walked into a large
room with academics who all seemed to be experts on Dtao Dam. I was
wondering where these people came from. They were talking about the
biology and topography and all kinds of esoteric subjects. 

\emph{Tan Paññāvuddho}: Ajahn, when I'm practising \emph{sīla, samādhi, 
paññā} in long periods of solitude in the forest, I sometimes get the
sense that this in itself feels like the most direct, authentic response
to the threat of this forest being destroyed, not to mention all the
other environmental, economic and socio-political calamities on earth. 
But I find it challenging to articulate this. If I were to try to
explain it to somebody who hasn't had much experience with meditation, I
wouldn't know how to put it into words. But the sense of authenticity
seems true and real. 

\emph{Ajahn Jayasāro}: Yes, I agree, but I also have the same kind of
difficulty in explaining it, why it is best to practise in this way. One
thing I said the other day: if we look at the root of all mankind's
self-imposed difficulties, there is a common underlying cause. We find
that because mankind doesn't know himself, he constantly acts in
conflict with his own best interests, living merely as the puppet of
desires, fears and delusions. But practising \emph{sīla, samādhi, paññā}
deals with these things at their very roots. That way one really works
with the whole structure of what is going on, rather than just
responding to a particular expression of it. We study the mind to
understand what greed is, what hatred is, what delusion is, even in
their most subtle forms, while developing the skilful means to abandon
them, to let them go. That seems to me to be as an intelligent response
as any other. 

But with trying to save the forest or whatever, I myself am always wary
of falling into the trap of `I've got to do it', or `We've got to do
this'. Once you fall into this `We've got to!' mentality, you've lost it
already. Yes, I will do what I can. But who knows what will happen? 
These things occur due to causes and conditions, many of them way out of
my control. 

\clearpage

\section{Recent developments in Dtao Dam}

\emph{In the following little sequel, Ajahn Siripañño, who has been the
abbot of Dtao Dam Monastery for the last five years, gives an overview
of some of the developments which have taken place since the original
article was written.}

Since the publication of this article, a whole host of factors regarding
the situation at Dtao Dam have changed, mainly for the better. What
remain almost entirely unchanged are the pristine nature of the jungle
there and the ongoing presence of a small number of forest monks
dedicated to living in the open, under the beautiful tree canopy, with
the call of gibbons and crickets echoing around. 

Regarding Khun Tivaporn and the mine, in 2003 the mine lease expired, 
and with it permission for the hermitage. Certain forces within the
National Park Department tried their best to force the monks to leave, 
presumably hoping to exploit the area commercially in some way. However, 
pressure from monastery supporters, the national press and, finally, 
from within the National Parks Department itself, resulted in the
monastery being given permission to stay on, with the agreement to be
renewed every five years. More stability was achieved when in 2009 the
Thai government created a nationwide `Buddhist National Parks' project, 
with the specific aim of enabling legitimate monasteries to stay in
forest areas and make use of national park land for Dhamma practise, 
while at the same time relying on the Sangha for help in preserving the
forests by keeping an eye open for illegal hunting, logging, forest
fires and other dangers. In fact, with the closure of the mine and the
departure from the area of Tivaporn's family and all the workers, the
hermitage has become a completely isolated spot right on the Burmese
border, with very few people coming in and out other than a seasoned
bunch of off-road vehicle drivers. 

As for Tivaporn herself, this hardy woman, now approaching 70, runs a
community centre on the property surrounding her house on the outskirts
of Kanchanaburi Town, teaching seminars on environmental matters and
sustainable livelihood. She also maintains a centre in the village
nearest to Dtao Dam, Tung Ma Sa Yo, which provides work opportunities
for her former mine-workers, many of whom remain undocumented due to
Thai bureaucratic complexities. 

The road linking Bangkok and Davoy (now usually marked on maps as Tawei) 
is currently under construction. Luckily, the possibility that it might
actually pass right through the Dtao Dam area was averted when
sufficient pressure from environmental groups made it clear that this
would not be acceptable. The route now passes some twenty kilometres
south of the monastery. 

Food is mainly brought in by monastery supporters, cooked by one or two
resident workers and supplemented by a vegetable garden and forest
fruits and vegetables, roots, shoots and herbs which can be collected. 

Every year a group of monks and novices from Wat Pah Nanachat go there
to spend the hot season in the same way as they have done for some
twenty years. The three original \emph{kuṭīs} provide enough shelter for
a small group of monastics to spend each Rains Retreat there. During the
last two Decembers a group of students, parents and teachers from
Panyaprateep school have visited for a few days with Ajahn Jayasāro. 

Animals, large and small, are still in evidence. Two years ago a herd of
elephants strolled through the hermitage, completely demolishing our
inner \emph{sāla}. Tigers and other forest cats are still being sighted, and
recently a protective mother bear rushed at one of our monks high up on
a mountain after her cub had wandered towards him. (Admittedly the monk
was using a mobile phone at the time, which might have alarmed her even
more. Both the monk and the monastery Nokia survived unscathed). A
rustling in the bushes may turn out to be the reclusive Dtao Dam itself
 (a six-legged black turtle), a porcupine with eight-inch quills or the
bizarre Malayan tapir, a cow-sized beast resembling a combination of
elephant and rhino. These incidents, though, are very rare. What are
experienced daily are the beautiful singing of gibbons and the chatter
of monkeys. A strange rhythmic whooshing sound high in the sky will be
the flight of a giant hornbill, or even a pair mated for life, as they
fly through the valleys looking for their regular spots to feast on figs
and insects in the tree canopy. All this and more, set to music: night
and day the jungle noises beat out a samba to rival any carnival. The
streams are still flowing, and the waterfall cascades down several
levels before levelling off at the foot of the mountain on the top of
which the monks dwell. The sun rises in the east over Thailand and sets
in the west over Burma; the timeless rhythm of nature undisturbed by
man's whims and fancies. Long may it be so. 

