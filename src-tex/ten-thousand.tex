% Title: Ten Thoughts Joys and Ten Thousand Sorrows
% Author: Christine Lem

When I was asked to write a piece for this Wat Pah Nanachat publication, 
I found it was an opportunity to reflect on and digest the challenges, 
experiences and insights that arose during my extended period of
practising the Dhamma at the monastery. When I arrived at the monastery
I was similar to many women who came with the hope of devoting focused
energy to the practice. Admittedly, many times during my stay I
questioned the benefits of living in a primarily male monastic
environment, but I stayed on because I was open to experiencing the way
of training in the community, and to learning more about the reality of
life as taught by the Buddha. Over time I eventually gained insights
into myself and the nature of life which astonished me. I continually
found it fascinating to learn and understand more about myself through
practising the Dhamma. 

One of the first teachings which amazed me was related to anxiety. I
remember that on four different occasions I asked completely different
questions with different story lines. The same monk began every response
with, `Well, anxiety is like this\ldots{}' I could not understand why he
was focusing on anxiety and not addressing my questions. It took me a
month to figure out that I had anxiety and the story lines were
meaningless. I thought it was not possible that anxiety could exist in
my life. My lifestyle was on the nomadic side, so I had high coping
mechanisms, and this should have meant that I was in tune with
impermanence. But to my surprise, as I began to investigate more deeply, 
I uncovered areas of my life that needed to be reflected on. Spiritual
urgency was ignited. 

The practice situation for women at Wat Pah Nanachat presented the
opportunity to embrace whatever arose in the present moment, whether it
was a pleasant, unpleasant or neutral experience. Living in a monastic
environment gave me the time and space to investigate deeply and examine
what caused my mind to move. Was it due to external conditions or was it
coming from a place inside? What seemed so real and solid in my mind one
minute was not the same the next. During my stay at the monastery I
considered all situations and incidents which moved my mind as valuable
and as opportunities for potential insights. Having to be with things I
didn't always find pleasant required a lot of work. It took energy to go
against the grain. Mindfulness was needed to hold back views and
opinions, and it took a lot of time to be able to see through the
superficial story lines, in order to search internally for a path
leading out of \emph{dukkha}. Idealism, anxiety, views, opinions and
perceptions, as well as joy, insights, understanding, compassion, 
respect, humility and inspiration were all part and parcel of the
teaching of my life. 

While living at Wat Pah Nanachat it was sometimes difficult to assess
the progress in my practice, but after being away from the monastery for
several months I realized that the positive fruits of the practice could
be seen clearly. I have already experienced glimpses of insight that
have had a profound effect on my life. With the encouragement and
support of the Wat Pah Nanachat community, my practice continues to gain
momentum. So where did the experiences and challenges of Wat Pah
Nanachat begin? 

\section{Relating to monks}

The first evident challenge for lay visitors to Wat Pah Nanachat, 
especially women, is the question of how to fit into a monastic training
centre for males. In particular, due to the monastic discipline and
conventions, women might find that they don't fit into Wat Pah Nanachat
as well as they might wish. For me this was an ongoing challenge, though
I felt more at ease as the practice gained strength. The hierarchical
system was a part of monastic training, which helped define how
monastics interacted with each other as well as how they interacted with
laypeople. In addition, the Thai culture has placed monks on a pedestal. 
In some respects it seemed as if they were no longer ordinary beings. An
explanation given by a Thai lay supporter gave me insight into why monks
received special treatment with which Westerners might not be familiar. 
She said, `I'm a householder with many responsibilities and duties, 
that's my \emph{kamma}. It gives me great joy that I can support the
monks, because they can live and dedicate their lives fully to
practising the Dhamma, which I can't do right now.' The generosity of
the Thais affects not only the monastics, but also those laypeople who
visit and live in the monastery. Often Westerners do not fully
understand the inter-dependent relationship between lay Thai Buddhists
and monastics. The laypeople support the monks with simple material
requisites, and the monks support the laypeople by passing on and
exemplifying the Buddha's teachings. 

What strikes Western women at the monastery as surprising are the old
traditional ways of behaviour. In Thai monasteries laypeople sit and eat
after the monks, kneel down or stand lower when speaking with a monk, 
and fulfil the role of cooking and serving the monks. Western women
wonder, `Where does equality fit in at Wat Pah Nanachat? This is the
twentieth century, is it not? These are Western monks who surely must be
aware of modern times.' Western women may suddenly find themselves in a
position where they don't know how to act or relate to Western men in
robes. The question of appropriate protocol, coupled with the need to be
consistent with one's own familiar comfort zone, can produce heightened
levels of anxiety. What is considered appropriate conduct and what is
not? Anxiety increases, especially for those women who are unsure of
themselves or are unfamiliar with the formalities of the monastery. 
Regardless of the fact that they might be new to Thai culture and
monasticism, some women found they needed to know exactly what to do and
how to act in every situation. I also went through this phase. 

It was important to remember that the monks follow a code of discipline
composed of 227 training rules; these were their rules, not mine. This
was a good reminder to relax at Wat Pah Nanachat. I noticed that
Westerners tended to worry about the monks' code of discipline more than
necessary. It was not as though we had adopted the 227 precepts the
moment we entered the monastery. There was a period when I thought Thai
people knew best about the formalities of the monastery. The reality was
that the monks knew best. They were the ones who lived and abided by the
training rules. Sometimes, even when I was familiar with the protocol of
the monastery, anxiety would still arise. I found the mind proliferated
as follows: `There are Thai monks and Western monks. Should I be formal
with Thai monks and informal with Western monks? But then how do I act
with the Westernized Thai monks, or the Malaysian or Japanese monks? 
Maybe I should be formal with all the monks at all times? But if I were
a Western monk, perhaps I would think it was not necessary to be so
formal? Should I continue to talk with the monks I was friends with when
they were laymen?' The thoughts proliferated on and on. After much
anxiety, I learnt to relax a little and search for balance. I found that
each situation and each person was different, and one acted
appropriately according to the circumstances. There were no fixed rules. 
I found that if I was respectful and polite, and was sensitive to the
cultural differences, I couldn't go wrong. 

I went through other phases as well. In Thai society women have to
maintain a certain physical distance from monks, as if women were
dangerous. I would think, `If I were in the West the distance between
males and females would not be a big issue. This monastery is a bit odd. 
What am I doing here? I'm living in a male monastic community!' I
questioned my stay in the monastery. Monastic life and Thai culture are
interesting combinations for Western women. After making a commitment to
the monastery for a period of time, I felt a rebellious rumble in my
mind. I would express exactly what was on my mind. Once I admonished a
bossy novice who rudely interrupted my conversation so that I could
complete his duties. I used strong words and told him to tidy up his own
mess and take responsibility for his actions. I thought I was speaking
the truth, but harmful speech is not considered `right speech' either. 
It was a good teaching for me, as I was beginning to understand that
true and honest thoughts could be dangerous mind states. This was one of
many insights which meant my stay at the monastery was challenging and
fruitful. Although the monks tried to explain the situation at Wat Pah
Nanachat, and how and why the rules and conventions exist as they do, 
Western women said it was difficult to listen to or even hear the words
of these privileged monastics in robes, explaining the way things were
in the monastery. Monks were explaining the reasons for their own
privileges. Women found that it was easier to hear things when a Western
female with Asian roots, living in the monastery for an extended period
of time, explained matters there. 

I once thought that monks at Wat Pah Nanachat were constantly mindful
and that they probably practised twenty-four hours each day. What else
would monks do? I sometimes forgot that I only saw the monks four times
a day, at morning meditation and chanting, the daily meal, teatime, and
the evening meditation and chanting. Well, of course they were mindful
all the time then! Monks were supposed to behave this way during formal
meetings, and especially when sitting on the \emph{āsana} before the
meal with so many people looking in their direction. The brief moments
when I saw the monks gave an impression of heightened mindfulness and
awareness. 

Wat Pah Nanachat can be an intense place for Western women when small
and trivial matters are magnified. There were times when there was not
enough tea, sweets, cushions or chanting books for the people sitting at
the lower end of the hierarchy. Did this happen every time? Was it due
to an inexperienced person setting up, or perhaps due to someone working
with greed that day? Depending on my mind state I was sensitive to these
things and sometimes took them personally, as if someone was directly
trying to create \emph{dukkha} for me. There were many different ways
the mind could proliferate in response to a given situation. 

I also experienced monks who were sensitive to the position of women in
the monastery. I noticed gestures of kindness and concern. There was a
novice who made a separate kettle of cocoa for women so that they
wouldn't have to wait so long, and there were other novices who would
personally pass on the spare sitting cushions to women. At other times
monks would ask if there was enough food or tea for the women, or a monk
would make a sincere apology for his lack of mindfulness towards women. 
All this took effort, energy, humility and awareness of the sensitivity
of women living in the monastery, which was much appreciated. Honest
mistakes and miscalculations do happen. 

It takes at least a few visits and sometimes more to understand Wat Pah
Nanachat. To understand the monastic training set-up more deeply, I
needed to take the time and energy to keep asking questions. At the same
time, I had to be open to receive answers. I found it was important to
dispel doubts by asking questions, in order to benefit from the new
experiences gained from living in a Thai monastery. My doubts or
misunderstandings arose because there were many ways of perceiving a
given situation. I encountered one monk who did not use eye-contact
during a Dhamma discussion with the lay residents. My judgemental mind
said, `He is a young monk and he must lack confidence, be insecure and
have women issues to deal with in his practice.' It was part of my
conditioning that not to make eye contact is odd or impolite, but in a
monastery little or no eye-contact was appropriate for a `good' monk. It
was obvious he was a `good' monk because his Dhamma responses were sharp
and clear, with compassion and wisdom. 

Timing and not taking everything too seriously are also important. I
remember one occasion when someone asked me, `How come all the monks are
so young?' I said, `It's all that meditation they do.' I'm not quite
sure what role humour plays in monasticism, but it certainly helps to
lower unnecessary barriers between people and allow space for new
growth. 

\section{Language and cultural challenges}

The position of women in a male monastic community can blind Western
women to seeing the benefits of the opportunity to practice at Wat Pah
Nanachat. Inability to communicate in the native language adds to the
challenge of practice there. Female Western guests at Wat Pah Nanachat
can find their anxiety level increase due to the lack of
English-speaking Thai women. Thoughts about what to do, where to go and
how to do things might be overwhelming when you cannot understand the
answers, but the Thais will come to your assistance if you look lost. It
is also important to smile. If I ever felt at a loss for words and did
not know what to do, I just gave a great big smile. It is the Thai way. 
Smiling is considered a higher realm of communication. 

Not knowing the Thai language can also be a blessing. I found it was
interesting to observe the movement around the community and the people
in it. Not knowing the right words, and speaking little, I had an
opportunity to experience the monastery on a different level. Many times
the challenges and aversions that came with community living were
limited to thoughts. At those times I saved myself from creating any
unwholesome \emph{kamma} which I would have regretted later -- what a
relief! It was also an advantage not to understand or become acquainted
with kitchen gossip. Reduced mental proliferation was always conducive
to my meditation practice. 

In addition to the language barrier, unfamiliarity with Thai customs
added to my anxiety. In general I found it was a good idea to follow in
the footsteps of the Thais, literally and figuratively, especially when
there were ceremonies at the monastery. The best policy was to go where
they went, sit where they sat and do as they did. For me Thais were good
means of gauging what was deemed appropriate in a monastic community, 
until I developed the confidence to know what was best based on my own
experience. I remember one incident when I placed bananas randomly in a
bowl that was going to be offered to the monks. They underwent a
transformation when I briefly left the kitchen. They were wiped and
cleaned with a wet cloth, both ends were snipped for a tidy appearance
and then they were neatly arranged in a bowl. It was hard for me to
believe, but the Thais staying at the monastery had even stronger
perfectionist tendencies than I did. They were meticulous in everything
they presented to the monks. I made a point of observing this in order
to smooth out any future incidents. Most of the time, if I made a
mistake, it was not actually a major error on my part. Interestingly, I
learnt a lot when I made mistakes, and the memory was recorded in my
mind to ensure the same mistake did not happen again. This put me at
ease on some levels. I felt as if someone was showing me insights into
Thai culture, and protecting me from making any major blunders that
might cause embarrassment to myself and others in the community. 

I found the challenge in the kitchen was that I needed to keep paying
attention to what was useful, and to take the initiative of performing
tasks that were appreciated. The Thais expressed delight and encouraged
me when I cooked Western food. The villagers inquired about what I was
cooking and what kinds of ingredients I used. I took all this interest
as a sign of Thai-style praise, which may have been a bit of projection
on my part. It didn't seem to matter if the Western food looked, tasted
or smelled pleasing to their eye, tongue or nose contact. The important
thing was that it was Western food, and the pot came back fairly empty
so the monks must have liked it. 

One area that the Thais did not encourage me to emulate occurred when
preparing fruits. They were skilled at carving and arranging fruit into
beautiful shapes and designs. Thais cut with the hands in mid-air, 
rarely using a chopping board, so I did the same. Also, the Thai-style
of using a knife was to cut away from the body, rather than towards the
body as in Western style. I did try a few times to cut and design fruit
in a manner pleasing to the eye, using the Thai method with a knife. The
results were disastrous. I could not manage to carve a piece of fruit
with a machete-like knife without the fruit turning into juice in my
hands. I soon noticed that Thai people offered very subtle hints with
regard to where my talents were most useful and beneficial in the
community. I learnt to be alert to how I could be useful, especially in
the kitchen where communication was in the form of body language. When
preparing the meal, the Thais would rarely suggest what to do to Western
women. 

Overall, insights and challenges from my daily experiences and
interactions with Thai culture brought an enormous amount of joy. The
Thais' kindness, generosity, respect and humble nature were qualities I
continuously tried to rediscover in myself and aspire towards. These
qualities, which the Thai community exemplified in their day-to-day
lives, had a tremendous impact on my attitude towards life. The relaxed
and easygoing style of Thai people influenced nearby Westerners. I
examined my resistance to these wholesome qualities which the Thai
community continuously expressed to others. I learnt a lot by simply
noticing, listening, observing reactions, going against the grain of my
cravings, using my favourite word, `whatever', talking little and
smiling a lot. I believe living for any period of time in a Thai
monastic community might enable people to make great progress in their
practice. I was able to let go of fears and anxieties related to my own
cultural conditioning, and open up to a new set of heartfelt wholesome
qualities. What a joy and relief it was to let down my barriers of
resistance! 

Another cultural difference that deeply affected me occurred during Thai
Buddhist funerals, where the dead bodies were brought to the monastery
to be cremated in the open air. I appreciated the increased contact with
death, which reduced any fears I had about it. Watching a body being
cremated was a powerful reminder that my body was of the same nature. It
was interesting to witness that these funerals were ordinary, peaceful
and simple, quite a contrast to my experience of Chinese funerals. The
bodies were decaying naturally and not decorated to look beautiful or
full of life. I felt for the first time that funerals were authentic. I
noticed that less fuss and distraction regarding funeral arrangements
made the death more real for those concerned. Anyone who wished was able
to watch the body burn and melt from skin to bones to ashes. The heat
element from the fire was raging and powerful. It had a life force of
its own which felt natural and freeing. It felt as if life and death
were coming together in harmony and balance. Burning a body in the open
air in the centre of a forest monastery seemed quite appropriately in
tune with nature. I was very moved by these funerals. When I die I might
consider Wat Pah Nanachat as an option for my cremation. 

Through interacting with the Thais, I realized that being born in the
West with Chinese roots, I perceived the world in a way that appeared to
be a contrast to the reality of experience. It was an incredible
experience to discover that I knew so little about the true meaning of
life, which I felt was being revealed to me in the monastery. 

\section{Practice in communal life}

I've spent a great deal of time in numerous communities, usually passing
through them. At Wat Pah Nanachat I lived for the first time in a
community for an extended period of time. This made a considerable
difference to my practice. Making a commitment to a community meant
there was no escape from confronting my habits of body, speech and mind. 
I learnt to live with the \emph{kamma} created in the present and to
resolve the \emph{kamma} created from the past. Living in a community
was intense and sensitive buttons were pushed. When practising with
others I found that I was no longer dealing with just my own five
\emph{khandhas}: I had taken on board the entire community's
\emph{khandhas}. If there were twenty people in the community, there
were one hundred \emph{khandhas} to interact and deal with in a
compassionate and skilful way. 

I was also aware that in most communities there always seem to be one or
two odd characters who test all those living in the monastery. They came
in a variety of forms: monks and nuns, Western and Thai, men and women, 
young and old. I noted how these types of characters affected the entire
community, even though I first thought I was the only person in the
monastery who was experiencing major \emph{dukkha} with that particular
person. One Western layman who visited the monastery did not want to
speak to or have any contact with any of the women. At the same time, he
continued to frequent areas where women gathered, like the kitchen. 
These characters were challenging and could lead me to react
unskilfully. I noticed that if I was in a good mood they did not move my
mind; instead, compassion arose. My thoughts about this situation were: 
`I wonder what it would feel like to be living in this person's shoes. 
Sounds like a lot of anger. I wonder where it comes from. His
\emph{dukkha} can't come from me. We don't know each other. We have
never met before. I can't take this personally.' Confusion and
\emph{dukkha} arose as I reacted to his attitude towards women on the
external level. I noticed my thoughts, and what immediately came to my
mind was that he was not supposed to be acting that way or saying those
absurd things: `What is he doing here? This is a monastery!' But in
reality it was not his comments that were causing the \emph{dukkha}. It
was my own internal struggle with idealism that created true
\emph{dukkha}. The scenario was all happening so fast that I could
barely keep up with being mindful.

I reacted to external scenarios and
blamed the external world for my \emph{dukkha}, but the real
\emph{dukkha} was my own internal issues. Whenever mindfulness returned, 
the amount of confusion I could create for myself amazed me. Idealism
was an ongoing struggle because it seemed so real and solid. I was
absolutely positive beyond a shadow of a doubt that the \emph{dukkha}
was coming from the people or places who caused me to react with
negative mind states. As I continued to experience scenario after
scenario, I hoped to learn from each and every incident, rather than
react to the endless incidents that were all integral parts of life. 

As well as the odd men who came through Wat Pah Nanachat, there were
also women who were challenging. One woman constantly complained about
living at Wat Pah Nanachat, but she continued to stay in the monastery
for quite some time. We had a discussion, and slowly her personal
history, perceptions and feelings began to unfold. I was startled to
hear the amount of suffering she experienced. It was a strong and solid
reality for her. It was difficult to relate or connect with this
\emph{dukkha} because it didn't affect me in the same way. After our
meeting there was a shift in my attitude: I found space and compassion
to allow this person to be who she was. To begin with I reacted
negatively to her continuous stream of complaints. I felt they were
unrealistic or exaggerated. Idealism affected my mind state. I would
think, `If she doesn't like the monastery, why continue living here?' I
lacked the patience to accept what was happening in the present moment. 

I noted how idealism took up a vast amount of my time and energy, so as
to cover up the present reality and create the reality I wanted. I
realized it would be wiser to bring up the effort to work with the
issues occurring in the present moment. My ideal was that I loved
everybody because that was the practice, but in reality I avoided those
I didn't love, which made it easier to love everybody -- what a delusion! 
It would be better to learn to be with people with whom I did not wish
to be. Community living was like that. I was continuously learning and
gaining insight about others and myself in the community. It was an
opportunity to take on the challenge and work with what moved the mind. 
I didn't get overwhelmed with my \emph{dukkha} or try to leave the
monastery by the back door. If I was not going to practice with the
difficulties that came from living in the monastery, what was I waiting
for?

I had a tendency to wait for the right moment to really put in the
effort to practice. I'd say things like, `When I find the right teacher
I'll practice; when I find the right community I'll practice; when I'm
in a happy mood I'll practice; when I'm not so tired I'll practice; when
I'm in solitude I'll practice.' The list goes on. The right moment is
here and now, because I might die before the next moment happens. Death
does not wait for the right moment. It's known for bad timing. Imagine
dying waiting for `the right moment to practise'. To talk about the
practice more than practising the practice sounds like hungry
ghost-realm material. It was always good to remind myself that it was
useful to take every opportunity to practice, and not waste one precious
moment. The practice continues. 

Wat Pah Nanachat was an inspiration because people were practising, 
living and working with reality, life as it was happening. I tried to
spend my time being with whatever was in the present moment, and going
against those old habits that continuously pushed me around. I examined
exactly what was pushing my buttons. Did it ever really come from
external conditions? I noticed a variety of incidents that caused my
mind to move in different directions. Sometimes I let it go; other times
I reacted because `I was right and they were wrong'; and there were
times when fire came flying out of my mouth with zero mindfulness. It
was all about watching the mind, fully reaping the consequences and
learning from these incidents. If I was not making the effort to
practice and learn in a conducive and supportive environment like the
monastery, where else was I going to do it? Would I ever find the
perfect place so that I could have perfect meditations? 

\section{Investigating idealism}

After I left Wat Pah Nanachat, a major insight revealed itself to me. 
These reflective periods outside the monastery offered an opportunity to
step back and create some space for the challenging situations I had
experienced when living in a community. The key issues that entered my
mind were idealism, perfectionism and anxiety, which extended into
honesty, trust and refuge. I noticed that I had difficulty in being
honest with myself due to idealism, particularly in relation to
feelings, opinions, thoughts and decision-making. I had practically
taken refuge in idealism; I had put my trust and faith in it because it
wouldn't fail me. That perfectionist idealistic tendency invaded my
mind. I continually denied the reality of where I was at, because I was
not the ideal. I was not as honest as I could be in my practice because
of fear that I was not living up to my idealism. This was a lot of
\emph{dukkha}. 

I remember I was surprised and impressed by talks given by a visiting
Ajahn. I noticed he was exceptionally honest about his past and present
experiences. He openly admitted and discussed his challenges, 
difficulties and complaints. I had thought Ajahns would be above and
beyond these mundane issues. He used personal experiences because he was
not afraid to express his situation freely. He was not embarrassed
because he didn't have anything to hide, while I was caught up in my own
idealism and did not allow enough space for the practice to unfold
naturally.

The influence of a highly respected senior monk who spoke in
an alternative style helped me let go of my ideals about the practice. I
suddenly felt I had a lighter load to carry. It was easy to forget that
the Ajahns were just ordinary sentient beings. It was
difficult to think they still had \emph{dukkha} like anyone else. It was
their approach, and how they responded to \emph{dukkha} in a
compassionate and wise way, which made the difference. The skilfulness, 
clarity and honesty of their body, speech and mind were beautiful to
experience and inspiring to watch. The presence of senior monks was a
powerful teaching. 

I found Wat Pah Nanachat was conducive to deepening my practice. I felt
the foundation of my practice shake and move, because it immediately
reflected back to me the question of how honest I was with where I was
in it. It was not as though I was going around lying to people about my
practice -- it was that my living, breathing reality consisted of
thoughts related to incidents, such as irritation which arose due to
loud and disrespectful lay visitors to Wat Pah Nanachat. I thought, `I
should not be sinking into these low-level mind states! This is not how
one trains the mind.' I would reflect on how wonderful it was that
people had discovered the Dhamma, and how great it was that they had a
chance to visit the monastery. That was the ideal -- it was a pleasant
thought, but the \emph{dukkha} still wouldn't go away that easily.

Sometimes I couldn't clearly see the difference between reality and
idealism. In community life the ideal was that people should be aware of
what tasks needed to be done and then do them, but the reality was that
some people didn't make the effort or didn't care. Then I usually went
through a mind-spin and wondered what kind of practices they were doing. 
How could people use and abuse such a wonderful place? Didn't they hear
the evening talk? It was so powerful. When the reality doesn't match
idealism, \emph{dukkha} escalates. How many times did this happen? I
lost count. A monk once said something to me that had a powerful effect. 
He said, `It's good to have you here, Christine, it's good to have
someone we can trust.' He used the word `trust' often with me, and I
thought to myself, `Well, of course you can trust me. I've got too much
idealism, I would never ever think of doing anything intentionally
harmful to the monastery.' With hindsight, however, I realized that he
was trying to say, `Christine, we trust you, therefore you should trust
yourself.' For me this meant I needed to give myself some credit, and
probably believe and trust that I was okay. I was fine most of the time, 
but I got thrown off course because sometimes my feelings, memories, 
opinions, views and criticisms all seemed accurate, justifiable and
absolute truth. At other times I was very clear they were \emph{dukkha}, 
\emph{aniccā}, \emph{anattā} and not worth holding on to. I needed to
work on letting go of those absurd fixed ideas. Self-criticism arose, 
and I wondered how I could possibly trust myself when I made so many
deluded mistakes. It was interesting to watch and notice the anxiety, 
idealism and perfectionist tendencies when I could catch them. It was
better than spinning in them -- such a waste of my time and energy. 

In the monastery I was a long-term female lay resident who lived and
worked near the kitchen, spoke English and a little Thai, and had a
shaved head, which led people to believe I was the local information
source. Sometimes I felt I was engaged in idle conversations more than
necessary, which meant less time to do the 101 things on my wish list. 
This increased my craving for solitude. I thought that if I had enough
solitude my whole practice would come together. I was forever
complaining that there was not enough time in the day to do all that I
wished to do. If only I had more time, I would be happy. I would
renounce tea or fast in order to increase the time leading to happiness. 

One of the subjects for frequent recollection is: `Do I delight in
solitude or not? This should be reflected upon again and again by one
who has gone forth.' For me the answer is, `YES. Which mountain, cave or
jungle do I get to meditate in so as to delight in solitude?'
Unfortunately, it was `not recommended' for women to stay in some branch
monasteries unless there were two women together. It was also `not
recommended' for women to go to certain areas of a monastery. The `not
recommended' advice was frustrating for me because I believed that deep
and heartfelt insights came from intensive solitary practice. Women
practitioners who wished to meditate in more isolated areas were bound
to come into contact with the fears, worries and anxieties expressed by
Thai people. Thais asked, `What if ghosts should appear?' What was
considered appropriate, safe and respectable for women were factors that
made it challenging for women who wished to delight in solitude.

In
hindsight there was enough solitude, but I let \emph{dukkha} get the
better of me. Living in my \emph{kuṭī} in the forest, sweeping and
meditating were periods of solitude which were overlooked because they
didn't match my ideal type of solitude. It was pure \emph{dukkha} to
have high standards and not be able to maintain or keep up with them. I
experienced the `ideal' type of solitude a few times, but using that
memory to compare the daily life experience of solitude only created
unease in my mind. I didn't appreciate or make good use of ordinary, 
everyday solitude, which probably would have measured up to the dream
type of solitude I frequently envisioned in my mind. 

I noticed that some days I would love Wat Pah Nanachat. It was the best
monastery, with the best monks. It was the most authentic and pure
practice place in all Thailand. Everyone was wholeheartedly into the
practice. We had the most faithful and dedicated villagers who came
every day to support the monastery. Then there were other days when I
said, `I'm leaving tomorrow and I'm never coming back. It's all wrong
here, they should be updating and adjusting their rules of discipline to
keep up with modern times, I'm not going along with these outdated, 
old-fashioned ways of interacting with one another.' Which was real and
which was not? I couldn't believe in these passing thoughts or moods. 
They were too much \emph{dukkha}. I found it both time-consuming and
tiring to resist present conditions at the monastery. I spent too much
energy not accepting the way things were. It was less exhausting to let
go and conform to the way things were, because Wat Pah Nanachat is
actually a fine place to practice. In the monastery I meditated, ate a
little, slept a little and lived simply in the forest. After a while I
noticed, `What else do I really need in life?' I heard myself saying
things like, `I love Wat Pah Nanachat 95\% of the time in order to keep
room to express my frustration and complaints.' Was the dissatisfaction
reality or not? On different days different mind states appeared, 
directed at different situations and different people. The mind's moods
kept changing. What could be so important as to take away my peace of
mind? 

\section{Closing}

I continued to make frequent visits to Wat Pah Nanachat, because I loved
the simplicity, authenticity and purity of the living Dhamma that
existed in the monastery. I felt my life opening powerfully when I came
into contact with the straightforward teachings of the three
characteristics of existence, \emph{aniccā, dukkha} and \emph{anattā}. 
Wat Pah Nanachat provided a good setting in which to continue to
investigate these truths at ever-deepening levels. I kept going back
because realistically there was no going back to my old ways of
thinking. In retrospect I can see how Wat Pah Nanachat shook the
foundations under my feet. The experience of living for an extended
period of time in a community of full-time Dhamma practitioners, people
with whom I had not individually chosen to live, became the teaching of
my life. Although Wat Pah Nanachat always managed to provide an array of
challenges, I learnt that the monastery had no inherent \emph{dukkha}, 
just as it had no inherent \emph{sukha} (happiness). I needed to take
responsibility for my own experience of life. 

Women who visit Wat Pah Nanachat will perhaps feel it is similar to a
boys' club. Some women might wish to be part of the boys' club. Who
likes the feeling of being excluded or left out? This is the challenge
for all women who visit Wat Pah Nanachat. But whether one is male or
female, the practice is to keep watching the movement of the mind and
protect it from falling into states of greed, hatred and delusion. Some
women find Wat Pah Nanachat a useful place to practice for a few days, 
others for weeks and months, and some are not yet ready for it. This is
all just fine. It is good to acknowledge where we are at in the present
moment. 

My experience of living in a monastery was intense because it meant I
had to be with things I did not wish to be with, I was separated from
the things I wished to be with and I didn't get what I wished for. There
was no escape except to face the \emph{dukkha}. \emph{Dukkha} was
crystal-clear, and so was the path leading out of \emph{dukkha}. These
were the sorrows and joys of living at Wat Pah Nanachat. It was an
opportunity to deepen my understanding and reflect on my defilements and
habits, in order to learn and know more about myself. I saw myself
investing energy in confronting situations and striving for ideals which
it was beyond my ability to do anything about. Meanwhile, what it was in
my power to change did not seem as appealing to work with. While living
in the monastery I experienced what `real life' was all about. I heard
lay visitors express comments such as, `The monastery isn't real life.'
I felt that in the monastery I began to learn about my multi-dimensional
mind states. My attitude began to change and insights came through
experiencing community life, understanding Thai culture, relating to
monks and observing my idealism. The simplicity of living in a forest
and the ascetic practices of monastic life resonated well. I felt in
tune with this lifestyle because I was drawn to simplicity and to what
was `real', elements which were lacking in my life. For me `real life'
meant reflecting and working with the \emph{dukkha} that appeared in the
present moment. This meant that there was a lot of `real life' in the
monastery! 

I would like to express my sincere appreciation and deep gratitude to
Ajahn Jayasāro, Ajahn Vipassi and the entire Wat Pah Nanachat community, 
and all those wonderful beings who have supported, lived at, and visited
Wat Pah Nanachat. Thank you for your kindness, generosity and the
opportunity to practice at Wat Pah Nanachat. Thank you for the teaching
in my life. 

I hope that what I have written is helpful for all beings who find
themselves at Wat Pah Nanachat, especially women, who might find the
experiences of monasticism, Thai culture and Buddhism challenging. 

\clearpage

\section{The Author}

Born in Canada of Chinese origin, Christine initially spent six
years in Thailand and India practising the Dhamma. Her introduction to
the Dhamma began with Tibetan Buddhism in India. Eventually her travels
led her to Southern Thailand and a retreat at Wat Suan Mokkh in 1993, 
where she first encountered the teachings of Theravāda Buddhism. Soon
after that retreat she visited Wat Pah Nanachat for the first time in
August 1993. The monastery was part of the Dhamma trail and on her
`list'. During the period when this article was written she was staying
at Abhayagiri Forest Monastery in America for a few months, and
eventually travelled to England to spend a year as an \emph{anāgārika} at
Chithurst Monastery. After leaving the monastery, having decided not to
pursue the monastic life there, she returned to Vancouver and alternated
work with travelling to Asia and spending time practising in forest
monasteries and meditation centres there. Between 2006 and 2008 she
lived in Burma, mainly at Pa-Auk Forest Monastery and Shwe Oo Min. Over
the years she has often called in at Wat Pah Nanachat for a few days or
weeks at a time.

