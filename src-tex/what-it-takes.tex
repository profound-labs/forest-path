
During the first few years of his monastic career, a young monk's
training is divided between Wat Pah Nanachat and other branch
monasteries of Wat Pah Pong. One of the disciples of Ajahn Chah who has
helped to train Wat Pah Nanachat monks is Tan Ajahn Piak, abbot of Wat
Pah Cittabhavana, a branch monastery situated to the north of Bangkok.
The following conversation with Tan Chandako took place in 1998.

\emph{Tan Ajahn Piak}: The \emph{Kruba Ajahns‎‎} rarely say anything
directly about Nibbāna because it is beyond a normal person's realm of
possible experience. Even if the people listening believe the
explanation, it still doesn't actually help them much, and if they don't
believe it they may make a lot of bad \emph{kamma} for themselves. So
the \emph{Kruba Ajahns‎‎} usually refer to it using metaphors or refuse
to speak of it at all, only teaching the path to get there.

The important thing is to keep going straight without stopping. For
example, say you want to go to Fa Kram Village over there; if you follow
the path and keep walking you'll get there in a short time. If you stop
to take a look at something and then chat with people, then go off with
them to see something else, it will take a long time before you reach Fa
Kram, if ever. But the reality is that almost everybody gets sidetracked
or at least stuck in \emph{samādhi}, thinking that they've arrived
already. Even Luang Por Chah was stuck for a while; Tan Ajahn Mahā Boowa
for six years; Ajahn Tate for ten years; Ajahn Sot (Wat Pak Nam) for
twenty years.

\emph{Tan Chandako}: Because to all intents and purposes it appears to
be full enlightenment?

\emph{Tan Ajahn Piak}: Yes. There seem to be no \emph{kilesas}
whatsoever. Everything is clear. Many people don't make it past this
stage. Other people practise for five Rains Retreats, ten Rains
Retreats, and still feel they haven't made much progress and get
discouraged. But one has to keep in mind that it is always only a very
few people who have the \emph{pāramī} to reach the goal. Compare it with
the US President or the Thai King. Out of an entire nation of millions
of people, only one person at a time has the \emph{pāramī} to be in the
top position. You have to think in terms of what you are going to do to
set yourself above the crowd, creating the causes and conditions for
future liberation. Effort in the practice is what makes the difference.
There are thousands of monks in Thailand who ordain with the sincere
intention of realizing Nibbāna. What sets people apart, why some succeed
while others don't, is mainly due to their level of effort, as well as
the effort they've put forth in the past. A person has to train himself
to the point where it becomes an ingrained character trait to be
continuously putting forth effort, whether he's around other people or
alone. Some people are very diligent as long as there is a teacher or
other monks watching, but as soon as they're alone their effort
slackens.

When I was a young monk and my body was strong, I'd stay up later than
everyone else walking \emph{jongrom} and see the candles in the other
\emph{kutis} go out one by one. Then I'd get up before the others and
watch the candles gradually being lit. It wasn't that I had it easy. The
\emph{kilesas} in my heart were always trying to convince me to take a
rest: `Everyone else has crashed out. Why shouldn't you do the same?'
The two voices in my head would argue: `You're tired. You need a rest.
You're too sleepy to practise.' `What are you going to do to overcome
sleepiness? Keep going.' Sometimes the \emph{kilesas} would win, but
then I'd start again and eventually they weakened.

\emph{Tan Chandako}: It's often when \emph{samādhi} or \emph{vipassanā}
has been going well that \emph{kilesas} seem to arise the most. At such
times it seems I've got more \emph{kilesas} than ever. Is that normal?

\emph{Tan Ajahn Piak}: Very normal. The average person has a huge amount
of \emph{kilesas}. Just to recognize that one has a lot of
\emph{kilesas} is already a big step. Even the \emph{sotāpanna} has many
\emph{kilesas} to become free from, much work to be done. Even at that
stage it's not as if everything is \emph{sabai}. It's as if there is a
vast reservoir of \emph{kilesas} below us which gradually come to the
surface, and it's not easy to know how much is remaining. Just when you
think you've fully gone beyond a particular \emph{kilesa}, it will arise
again. This happens over and over. The only thing to do is to keep using
\emph{paññā} to keep pace with the \emph{kilesas}, meet and let go of
them as they arise in the present.

\emph{Tan Chandako}: Have you ever met or heard of anyone who has
attained \emph{magga/phala} by only contemplating and not practising
\emph{samādhi}?

\emph{Tan Ajahn Piak}: No, if you want a straight answer. \emph{Samādhi}
is essential for the mind to have enough power to cut thoroughly through
the \emph{kilesas}. However, if one is practising \emph{vipassanā} with
the understanding and intention that it will lead to the development of
\emph{samādhi} at a later stage, this is a valid way to go about it.

The character of almost all meditation monks, both Thais and those born
in Western countries, is such that they need to use a lot of
\emph{paññā} right from the very beginning in order to gradually make
their minds peaceful enough to be able to develop \emph{samādhi}. Only a
very small percentage of Thais, and possibly no Westerners, are the type
to develop \emph{samādhi} fully before beginning \emph{vipassanā}.

\emph{Tan Chandako}: Can it be said how deep and strong \emph{samādhi}
must be in order to attain \emph{magga/phala}?

\emph{Tan Ajahn Piak}: It must be strong enough to be still and unified
as one, without any thinking whatsoever. There will still be awareness
--- knowing what one is experiencing.

\emph{Tan Chandako}: According to whether one is in a remote location or
in a busy monastery, should one's Dhamma practice change or remain the
same?

\emph{Tan Ajahn Piak}: Dhamma practice takes on a different character if
you are in the city or are busy with duties in a monastery. In the
forest there are few external distractions and it is easy to make the
mind peaceful. If you have many sense contacts and dealings with other
people, it is essential to figure out how not to pick up other people's
emotional vibes (\emph{arom}). Otherwise what happens is that the people
around us feel lighter, while we feel heavier and heavier. It's
necessary to be able to completely drop mental engagement as soon as
interactions with other people have finished. Otherwise all the
conversations and emotions of the day are floating around in the
\emph{citta} when one goes to sit in meditation.

It's easy to say, `Just be mindful' and `Don't pick up other people's
baggage', but it is very difficult to do. Luang Por Chah could take on
the problems and sufferings of others without picking up any of them
himself, because his \emph{citta} was very strong. The people around him
didn't know what was happening. They just knew that they felt cool and
happy around Luang Por. But this is not a practice for beginners. Most
people just get burned out. Practising in the forest is easier, and I
recommend that you should try as much as possible not to get involved
with too many responsibilities, especially being an abbot. If someone
tries to tell you that you are selfish and should be helping others,
reflect that this is due in large part to the conditioning from Western
society. If the Buddha had thought that way, we never would have had a
Buddha. In order to put your mind at rest, reflect on the goodness
you've done and rejoice in the \emph{pāramī} that you're creating. Those
who try to help others too much before they've helped themselves will
never be able to teach or help beyond the superficial. If their
teachings mislead others due to their own ignorance, they can make a lot
of negative \emph{kamma}. Many of the Wat Pah Pong monks try to emulate
Luang Por in his later years, when he would talk with people all day,
rather than his early years of difficult practice. But it was precisely
those years in the forest that made Luang Por into the great teacher
that he was.

\emph{Tan Chandako}: Have you ever heard of anyone attaining
\emph{magga/phala} by any means other than analyzing the body into its
component parts and elements?

\emph{Tan Ajahn Piak}: No. At the very least, when the \emph{citta} is
clearly known as \emph{anattā}, the knowing mind will return to knowing
the body thoroughly as \emph{anattā} as well.

\emph{Tan Chandako}: In one of Luang Por Chah's Dhamma talks he says
that even for \emph{arahants} there are still \emph{kilesas}, but like a
bead of water rolling off a lotus petal: nothing sticks. How do you
understand this?

\emph{Tan Ajahn Piak}: Luang Por liked to use language in unconventional
ways in order to get people's attention and make them think. What he was
referring to was the body --- the result of previous \emph{kamma} ---
but the \emph{citta} was completely devoid of \emph{kilesas}. Normally
people use other terms to refer to the body and the physical
\emph{dukkha} of an \emph{arahant}, but Luang Por was quite creative in
his use of the convention of language.

\emph{Tan Chandako}: I've heard that while still a student, before you'd
met Luang Por Chah, you had a vision of him.

\emph{Tan Ajahn Piak}: That's right. I'd intended to return, {[}to New
York, to finish a master's degree in business management{]} but soon
after I'd begun to meditate I had a clear vision of a monk whom I didn't
recognize, chewing betel nut. I went to see many of the famous
\emph{Kruba Ajahns‎‎} at that time --- Luang Pu Fun, Luang Pu Waen ---
but when I met Luang Por Chah I recognized him from the vision and
figured that he would be my teacher.

When I began to consider ordaining instead of completing my studies, my
family tried hard to dissuade me, but I found meditation so peaceful
that everything else felt like \emph{dukkha}.



The Authors

Tan Ajahn Piak and Tan Chandako

\emph{Tan Ajahn Piak still lives in his monastery to the north of
Bangkok. Any fields surrounding it are long gone and now the Bangkok
suburban sprawl has engulfed Wat Pah Cittabhavana. The 2011 flooding saw
the monastery submerged under a couple of metres of water. However,
Ajahn Piak still provides a refuge for those seeking the Buddha's path.
His reputation as a meditation teacher has grown, and his emphasis on
combining the cultivation of samādhi with staying up all night brings
many people to practise under him. Despite poor health he has begun
travelling and teaching abroad in recent years, most notably in
Malaysia, Singapore, Australia and New Zealand.}

\emph{Tan Chandako carried on training in Thailand under various
teachers, and also spent periods of time in Perth, living in Bodhinyana
Monastery. He spent a year in Wat Pah Nanachat as Vice-Abbot in 2002,
before seeking a place to settle down. A Rains Retreat in the Czech
Republic led to his return to Australia and finally to Auckland, New
Zealand, where in 2004 he was invited by the ABTA (Auckland Theravāda‎
Buddhist Association) to establish a monastic residence on their
recently-acquired property not too far from the city. Thus Vimutti
Monastery was born, and an extensive programme of tree-planting and
construction has been under way since then. Additional land has been
purchased to provide something of a buffer zone.}

As well as his responsibility for running the monastery, Ajahn Chandako
provides regular teaching and retreats both at the monastery and in
various other parts of New Zealand. Every year he comes to Thailand and
visits his home in the US, where he also conducts retreats.

