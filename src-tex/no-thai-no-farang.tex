% Title: No Thai, No Farang
% Author: n/a

\emph{In addition to the monks from abroad, a number of Thai monks also
come to Wat Pah Nanachat to live and practice. Tan Jayasiri, Tan
Jotimanto and Tan Dhirapañño are three such monks. Each has lived at Wat
Pah Nanachat for a number of years and has served the community greatly
by acting as secretary to the Abbot. This interview was conducted by Tan
Paññāvuddho.}

\emph{Tan Paññāvuddho}: All three of you grew up in Thailand. I'd be
curious to know what were the first impressions you can remember from
your childhood when you saw Buddhist monks.

\emph{Tan Jayasiri}: There was a branch of Wat Pah Pong (Ajahn Chah's
main monastery) near my home in the countryside outside Si Saket. When I
was a little boy I went to the monastery every day. It was clear to me
from an early age that I wanted to become a monk in the future. I liked
the way the monks shaved their heads and wore their robes, not dressing
like laypeople. When I first went to the monastery I was four years old.
I went to the `Sunday school' there.

\emph{Tan Paññāvuddho}: What do you remember about the first time that
you saw the monks in meditation? What feelings arose for you?

\emph{Tan Jayasiri}: I felt special. It was so different from what
worldly people did. I thought I would like to be like them in the
future. I also noticed that the monks did not live together, but in the
forest in their own \emph{kutis}. That interested me. At the forest
monastery I observed that each monk would put all his food in one alms
bowl (curries and sweets together), and eat it at one sitting in the
morning. When I was young I had the opportunity to eat the food left
over from the monks' bowls, as is common for laypeople at the monastery.
It smelled so different from normal food. I thought it smelled like the
scent of heaven, very strange for me. While I went to school during the
week, my grandmother went to the wat. At midday she would bring the
monks' food back from the wat, so I would go home from school during the
lunch break to eat it. If she didn't have any food that day, I felt
upset. Years later, on the first day I became a monk, I tried to smell
the food in my bowl during the meal. It smelled normal, nothing special
at all. Maybe the smell was just bait to get me here!

\emph{Tan Paññāvuddho} : When you were little, did you ever have the
opportunity to meet any of the great \emph{Kruba Ajahns  } like Ajahn
Chah?

\emph{Tan Jayasiri}: Sure. When I was a young boy about seven years old,
I went to see Luang Por Chah at Wat Pah Pong. At that time the monastery
was still very simple. Although many monks lived there, after the meal
the place looked deserted because all the monks went back to their
\emph{kutis} to practise. Luang Por Chah, however, always received many
visitors. He smiled all the time. When I looked at him I felt something
very special.

\emph{Tan Paññāvuddho} : Why do you think you had so much faith at an
early age?

\emph{Tan Jayasiri}: I don't know. It was just normal for me. When I was
little and saw suffering in the world, I would think about the monks,
their robes and their bowls. Ajahn Chah's smile is always in my memory,
it has never gone. Even today it is very clear. It is a very special
feeling for me to be a monk in his tradition. I have never had any
doubts about Luang Por Chah and the \emph{Kruba Ajahns  }.

\emph{Tan Paññāvuddho} : Tan Jotimanto, can you recall your experience
when you first became aware of Buddhist monks?

\emph{Tan Jotimanto}: Actually, my grandmother was my main influence.
When I was young I went to the monastery quite often with her. There was
a forest monastery nearby where Luang Por Poot Thaniyo was the abbot. My
family was very close to Luang Por Poot. From an early age I was taught
to pay great respect to the monks. During the Rains Retreat my
grandmother always went to the monastery on the Wan Phra days, spending
all day and night observing the Eight Precepts. Sometimes I would go and
stay with her.

\emph{Tan Paññāvuddho} : When you observed the monks in the forest
monastery environment, what did you experience?

\emph{Tan Jotimanto}: At that time I thought the monks were very special
people, and that they must have psychic powers because they taught us
about heavens and hells. Actually, as is the case with many Thai
children, I was afraid of ghosts and I thought the monks could help. I
felt that since the monks had good \emph{sīla}, the ghosts would be
afraid of them.

\emph{Tan Paññāvuddho} : Tan Dhirapañño, how about yourself? What were
your first impressions of the Buddha-\emph{Sāsana}?

\emph{Tan Dhirapañño}: In my case they were similar to Tan Joti. When I
was young I grew up with my grandmother in Chonburi, out in the
countryside. She used to go and \emph{sai baht} \footnote{\emph{sai baht
  (Thai): to give almsfood (literally: to put food in a monks-bowl)}}
every morning. On Wan Phra she would wake up extra early in the morning
to make special food for the monks. You see, as a little boy I always
slept in the same bed with my grandmother, so on the days when I woke up
and did not see her beside me, I knew that day had to be Wan Phra.

\emph{Tan Paññāvuddho} : Can you remember when you were young and your
grandmother taught you to \emph{sai baht}?

\emph{Tan Dhirapañño} : I remember her telling me to kneel down, put my
hands in \emph{añjalī}, be quiet and carefully put food into the monks'
bowls without touching the brim. That was in the morning and it was not
that difficult. On certain special days she would take my cousins and me
to the monastery. That was the hardest part because I didn't understand
most of what the monks were teaching, and I had to sit with my legs
politely folded behind me in the \emph{papiap} position. I would sit
with my grandmother in the \emph{sala} while my younger cousins played
loudly outside. Anyhow, my grandmother always seemed to be very happy on
that day, and being around her, somehow I felt very happy too.

\emph{Tan Paññāvuddho} : Tan Jayasiri, what gave you the inspiration to
be ordained as a monk?

\emph{Tan Jayasiri}: Actually, I always had a strong ambition to become
a monk. I was thinking about it when I was very young,

\emph{Tan Paññāvuddho} : Did you ever have any doubt, like when you had
a girlfriend, maybe?

\emph{Tan Jayasiri}: It wasn't my nature to think like that. I was
fortunate to come from a happy home. Still, I saw that family life
involved a lot of suffering. I valued the ideals of simplicity and
renunciation, and I was always drawn to the monk's life, dedicated to
contemplation and peace. From the time I turned fifteen, of course I
would be around a lot of girls at school. But for some reason my mind
would always turn to the monks. When it came to having a girlfriend, I
wasn't averse to it, but I always thought that I should practise the
path of the Buddha first. When I turned twenty I asked my mother's
permission to become a monk immediately. I asked her several times, and
finally she said `Okay.' She said she would be happy for me to be a
monk, but not for too long, maybe a few years. She felt that a short
while would be enough for me and then I should return to her and the
family.

\emph{Tan Paññāvuddho} : What does she say now that you have been a monk
for almost ten years?

\emph{Tan Jayasiri}: She is quite happy now. Not like in the first few
years, though, when she would always wish for me to go back to lay life.

\emph{Tan Paññāvuddho} : Tan Joti, what gave you the inspiration to go
forth as a monk?

\emph{Tan Jotimanto}: It also happened when I was young. I had the
chance to see Luang Por Poot frequently with my grandmother. Once at his
monastery a senior monk from Luang Por Mun's monastery pointed to me and
said, `You should become a monk.' This statement stuck in my mind all
through the years. When I was a teenager I went to Wat Pah Pong and had
the chance to see Luang Por Chah. He was still healthy at that time.
During those days I had heard Ajahn Chah was very strict and fierce, and
you know, when you are young, you are not that interested in the strict
monks.

Later my cousin introduced me to Wat Pah Nanachat. At that time I was
working as a lawyer in the nearby city, Ubon. That was quite a stressful
period in my life. When I visited Wat Pah Nanachat I was very impressed
with the peaceful quietude of the forest, and the mindfulness and
kindness of the monks. My mind started to calm down. Ajahn Pasanno and
Ajahn Jayasāro were the senior monks at that time.

\emph{Tan Paññāvuddho} : Was that before you went to live in New York
City to help run a large restaurant?

\emph{Tan Jotimanto}: Yes, it was. After my first visit, I came and went
to and from Wat Pah Nanachat for two or three days at a time as a layman
during one and a half years,. Then I moved to the States. I worked hard
at the restaurant in Manhattan and earned enough money to support the
family back home. I had the financial independence to have almost
anything I wanted, but I was not happy. I did not know what the point
was. One day I went shopping with my friends. They asked me what I
wanted to buy and I stopped. I felt a profound sense of boredom with
worldly things and experiences. I had had enough. It was a very free and
liberating kind of feeling. Reflecting back on my life, I realized that
what I really aspired to was to practise the Dhamma and be a monk. But I
had to prepare for that and it took time.

\emph{Tan Paññāvuddho} : How did your parents feel when you said you
wanted to become a monk?

\emph{Tan Jotimanto}: They were happy that I would become a monk for
just one Rains Retreat. It is the tradition for Thai men to be ordained
for one rainy season, and my two other brothers had already done so. But
I did not want to ordain for only one Rains Retreat. I wanted to stay a
monk as long as I felt it was meaningful, and disrobe if I did not like
it. It was quite difficult for my parents at that time. They had to
think about it, and finally they gave me permission. At that time both
of my parents were elderly and not in good health. Even though it was
very difficult for them to allow me to ordain, they wanted what was best
for me and were generous enough to make the sacrifice. This willingness
on their part showed me how much they really loved me. Once they saw me
as a monk, especially as a forest monk, they liked it and were inspired
themselves to practise. Through Dhamma practice they really changed the
way they lived their lives. After I had been a monk for just one year,
they said, `Don't ever disrobe.' {[}Laughter{]}

\emph{Tan Paññāvuddho} : That was quick.

\emph{Tan Jotimanto}: They were very happy. I felt very lucky to have
the chance to repay my debt of gratitude to my parents, who passed away
soon after, by helping to teach them more about the Dhamma. In the last
few years of their life we discussed the Dhamma many times, and I shared
with them some books and tapes of Luang Por Chah such as `Our Real
Home'. I noticed my parents became much more contented and at ease with
life. They deepened their understanding of impermanence and reflected on
the inevitable separation at death. They started to practise every day.
My father died first, but my mother continued to be strong in spirit due
to her Dhamma practice, despite her poor health. She died four months
later, and at the time that she died she was about to make an offering
to the Sangha. The day she died it was a blessing to know that she was
happy and at peace.

\emph{Tan Paññāvuddho} : Tan Dhira, what about you? You also were
working in America before coming to be ordained, weren't you? What made
you change direction to become a monk?

\emph{Tan Dhirapañño} : After I finished medical school in Thailand, I
had to decide what to do. First I wanted to be a pediatrician like my
mother, and I wanted to do my internship in the West. In those days,
however, it was difficult to train in America. Finally I was accepted
for the programme in Michigan. During one of the first teaching rounds
in the hospital, the attending doctor asked the residents where they
came from originally. When the doctor found out that I came from
Thailand, he smiled broadly and asked me, `So you're from Thailand, are
you? What is Nirvana?` I was shocked. I did not expect this kind of
question in the West. I don't remember what I answered, but it made me
think hard. I had gone all the way to America in search of knowledge,
but I wondered if what I was looking for was in my own country.

\emph{Tan Paññāvuddho} : So how did you end up at Wat Pah Nanachat?

\emph{Tan Dhirapañño} : In Detroit the local Thai community would have a
pot-luck meal at my aunt's house every weekend. They would talk and then
sit in meditation together. I joined the group once in a while. The
benefit of attending was the plentiful and delicious food, and my aunt
always insisted that I take the leftover food, because she knew that I
lived by myself and was not good at cooking. I have to admit that this
was the original inspiration for my meditation practice. One day several
years later I found a post on the Thai student internet that Ajahn
Jayasāro was coming to teach a meditation retreat in Maine. This
`backpack retreat' took place in Acadia National Park, and it turned out
to be a life-changing experience for me. In the morning and the evening
we chanted a \emph{puja} together in Pali and Thai. My grandmother had
taught me to chant every night before I went to bed, and I had been
doing that since I was a child, but I knew very little about what it
meant. During the retreat I was very moved to learn the deep meaning of
what I had been chanting all along. During the day we hiked on different
trails. When we got tired we sat in meditation. During the questions and
answers session each day, I was very impressed by Ajahn Jayasāro's
wisdom. On the last day of the retreat we had the ceremony of asking for
forgiveness from the Ajahn. It was a beautiful tradition. While we were
walking to the park for this ceremony, Ajahn Jayasāro handed me his
bowl. That was the first time that I had carried a monk's bowl and I
felt very happy. I don't know how to describe it, but it was a special
feeling that I will never forget. Tears filled my eyes. Time seemed to
stand still. At that moment I felt that I too could become a monk and
strive to be like the great \emph{Arahants}. That was the moment I
decided to become a monk.

\emph{Tan Paññāvuddho} : Your family had put you through medical school
and supported your training as a doctor in America -- was this a shock
to them?

\emph{Tan Dhirapañño} : I would say yes. The year that I met Ajahn
Jayasāro and decided to become a monk, I told them of my decision, but
they did not believe it. They thought I was broken-hearted or something
{[}laughter{]}. But I persisted. For the next two years, while I was
waiting to finish my training, I tried to keep the Five Precepts on a
regular basis and the Eight Precepts once a week. My friends started to
think that I was weird, but I found I was more and more peaceful. I
would spend more time on meditation retreats, where I felt very much at
home. When I finished my fellowship I came back to Thailand and found my
way to Wat Pah Nanachat. I asked Ajahn Jayasāro to ordain me so I could
continue on the path. Of course, my parents had high expectations for
me. But I think I did not let them down, although at the time they might
have thought differently. For me, becoming a monk is more challenging.
There are already many doctors in my family, but there is not a single
monk. There are many meaningful and worthwhile things to develop and
cultivate in this holy life.

\emph{Tan Paññāvuddho} : Have they grown to accept your decision by now,
or they still hoping that you will come back?

\emph{Tan Dhirapañño} : As time passes, they have begun to appreciate
what I am doing as a monk. To be honest, my parents' generous, loving
support throughout my life gave me the emotional strength to feel ready
to go forth as a monk. For me the monastic life is not so much a
rejection of the family life as a way to evolve one step further. If I
weren't a monk in this life, I would feel in my heart that I had not
fulfilled my responsibility to the family in a higher sense.

\emph{Tan Paññāvuddho} : Tan Jayasiri, what was it that originally
brought you to Wat Pah Nanachat? How did you decide to come here to
train when there also are many other good Thai forest monasteries?

\emph{Tan Jayasiri}: Just before I ordained I went to Wat Pah Pong and
met Luang Por Sumedho and Ajahn Jayasāro. From that moment I wanted to
be with them. I was ordained in Ayudhya, and after spending the first
five years there I came to Wat Pah Nanachat.

\emph{Tan Paññāvuddho} : What was it that attracted you to Luang Por
Sumedho and Ajahn Jayasāro?

\emph{Tan Jayasiri}: It's hard to put into words -- I just wanted to be
with them. Their presence was peaceful and inspiring. I felt that to
practise under their guidance would be beneficial.

\emph{Tan Paññāvuddho} : Did you feel that they were different from the
Thai Ajahns?

\emph{Tan Jayasiri}: No, similar.

\emph{Tan Paññāvuddho} : Having come to practise at Wat Pah Nanachat,
how do you find practising with Western monks, compared to the Thai
Sangha?

\emph{Tan Jayasiri}: Actually, in my practice, I never feel that I'm
practising with special people. Everywhere I stay I always just keep
Ajahn Chah's practice.

\emph{Tan Paññāvuddho} : Which part of Luang Por Chah's teaching do you
find most important in your practice?

\emph{Tan Jayasiri}: To watch and investigate feelings and the mind.
Seeing impermanence.

\emph{Tan Paññāvuddho} : Why do you think that in Thailand, where there
are many great \emph{Kruba Ajahns  }, so many of the \emph{farang} monks
have been ordained with Luang Por Chah or in his specific lineage?

\emph{Tan Jayasiri}: Yes, there are many great \emph{kruba}
\emph{ajahns} in Thailand, but Luang Por Chah was a special monk. He had
his own style of explaining the Dhamma in a simple yet profound way. It
was easy to understand his teachings. His presence made an impact.
People were happy to see him. Everyone liked to be around him. He set up
a style of training that is helpful and suitable for many different
kinds of people.

\emph{Tan Paññāvuddho} : Tan Joti, what made you decide to be ordained
at Wat Pah Nanachat as opposed to another Thai forest monastery?

\emph{Tan Jotimanto}: As I had known about Wat Pah Nanachat for some
time, I knew that I liked the way of life and I felt the monks were very
sincere and diligent in their practice. Still, I went to visit Tan Ajahn
Dtun before ordination. I liked it there too, but the monastery was very
close to my family's house in Chonburi. I wanted to ordain away from my
home, so I came here to Ubon. Also, I felt inspired to ordain with Ajahn
Jayasāro.

\emph{Tan Paññāvuddho} : How would you compare living with Western monks
to living with Thai monks?

\emph{Tan Jotimanto}: In many ways things are the same, but there can be
some cultural differences. People at Wat Pah Nanachat come from many
different places. Although the community is usually very harmonious,
from time to time difficulties or misunderstandings can arise. As Thais
we are taught to keep things inside ourselves. We don't know how to
express ourselves. But when I came here it seemed everybody knew how to
express himself. I have been trying to learn to do this skilfully, but
it is still difficult. I think that if people in a monastic community
express their feelings too much, it can create some problems. For
example, people will not learn to cultivate patience. On the other hand,
I think that open communication is usually valuable because it increases
understanding between people. For Thai people, if there are problems we
just don't talk about them. The ideal is to find the middle way.

\emph{Tan Paññāvuddho} : Tan Dhira, how about you? Was it your
connection with Ajahn Jayasāro that decided you to be ordained here as
opposed to a Thai monastery?

\emph{Tan Dhirapañño} : At that time I didn't know much about other
monasteries in Thailand. I did retreats at Ajahn Buddhadāsa's monastery,
Suan Mokkh, and a couple of other places. But I thought my weak point
was the Vinaya, the discipline. So I tried to find a place where there
was a strong sense of community and strict monastic training. I first
came to Wat Pah Nanachat mainly because of my connection with Ajahn
Jayasāro, and I liked it right away. I particularly loved the feeling of
the forest. The paths were well swept and there was a good environment
for meditation. And perhaps most importantly, I immediately connected
with the Sangha. I felt at home here.

\emph{Tan Paññāvuddho} : Now I have a question for any of you. Many
people think that in Thailand today there is a crisis with the
\emph{sāsana}. There have been scandals with monks not keeping the
Vinaya, and some Western-educated lay Buddhists are critical of a Sangha
that they feel is stuck in old-fashioned ways. What do you see as the
future of the \emph{sāsana} in Thailand?

\emph{Tan Jotimanto}: I think people are more interested in the
\emph{sāsana} now. They understand what is going on more. Before people
held the \emph{sāsana} in very high esteem, maybe too high and it became
out of touch. Now lay Buddhists are freer to speak their views and to be
heard.

\emph{Tan Paññāvuddho} : Do you think this is due to Western influence?

\emph{Tan Jotimanto}: It could be.

\emph{Tan Paññāvuddho} : How do you think this new attitude will affect
the monks' practice? Some contend that the more worldly orientation and
values of lay life can be counter-productive when brought into the
sphere of monasticism. Any opinions?

\emph{Tan Jayasiri}, \emph{Tan Jotimanto}, \emph{Tan Dhirapañño} :
{[}Long pause, then laughter.{]} It is a difficult balance.

\emph{Tan Paññāvuddho} : In the West there is a stress on the value of
equality, but the Buddha noted that the view `I am equal' is still a
form of attachment to self-identity view. Some monks contend that is
important not to sacrifice \emph{samana saññā} (the perception of being
a peaceful renunciant) at the expense of conforming to current worldly
norms.

\emph{Tan Jotimanto}: One example of this can be the way we use
hierarchy. Thais maintain very high respect for the Ajahn. When Thais
live in a Western community we still put the Ajahn in a high position.
In effect there is a sense of formality and respectful distance.
However, I've noticed that Westerners are more informal and relaxed
around the teacher. For them, their relationship with the Ajahn can be
like with a friend. For Thais this would be uncomfortable. I wouldn't
allow myself to play the role of a friend to the Ajahn, even if I wanted
to. Thai monks can be very close to the teacher, but there is always a
formality in the relationship.

\emph{Tan Paññāvuddho} : That's very interesting. It seems common in
various Buddhist traditions to keep the teacher in a formal position, so
as to maintain proper respect. If one doesn't have that respect it is
very difficult to learn, to have pure communication from the teacher to
the student. Do you think Westerners may miss out on learning something
because of the casualness with which they can relate to the teacher?

\emph{Tan Jotimanto}: Yes. In Thailand, even if the Ajahn is young or
has about the same number of Rains Retreats, he still is accorded formal
respect as a teacher. It may be natural to feel a level of friendship in
this situation, or not to want to listen to the Ajahn as someone senior
in the hierarchy. But in a Thai wat this would rarely happen, because
the Ajahn is always placed so high.

\emph{Tan Dhirapañño} : Take Tan Jayasiri, for example. He will turn
into an Ajahn very soon, and it will be difficult for us to relate to
him. We used to be close to him and tell jokes around him. When he
becomes an Ajahn we will have to change that perception. He will be put
in the special position of a teacher soon. {[}Laughter{]}

\emph{Tan Jayasiri}: I think there is a time and place for that. It's
not always the case that the Ajahn plays the role of the formal teacher.
When a situation in which respect is appropriate arises, we act
accordingly, like a young son with his father. The feeling of a close
relationship makes it easier to talk with each other.

\emph{Tan Paññāvuddho} : For Western monks it usually requires some
effort to adapt to hierarchical social structures in a skilful way. For
example, in Western culture one tends to relate to people directly, not
in terms of a formal hierarchy. One shows respect through mutual
friendship, openness and trust. However, in the Buddhist tradition,
respect and trust of one's teacher and fellow monks can be conveyed by
selflessly and harmoniously acting in accordance with one's formal place
in the relationship. This can be a subtle and refined thing, something
many of us Westerners can learn to do better. It is a challenge to adopt
a new cultural form naturally, without fabricating a new identity that
isn't authentic. When I live in Thai monasteries, however, it seems
comfortable to fit into the hierarchy and relate to the Ajahn formally.
Formal separation feels natural. It can also be useful. The Dhamma
teachings flow much better this way. But when I am with many senior
Western monks, it often feels appropriate not to overdo a formality that
is unnatural in our culture. But the form of these relationships changes
depending on the context, whether in Asia or the West, as individuals or
in a group, inside the monastery or out in public. One has to be alert
and sensitive in order to adapt.

\emph{Tan Dhirapañño} : I think there are both benefits and drawbacks to
the `equality' form. In Thailand the hierarchy system is so strong that
when I was young I had very little idea about what monks do. The closest
I came to the monks was offering alms-food or listening to a Dhamma
talk. That was it. It never occurred to me that the monk's life was
another lifestyle that I might choose. Today is different from the old
days, when the monastery was also the local school and a community
centre, and good monks were role models for the whole community.

When I met a Western monk, however, somehow that rigid hierarchy was
weakened. When I spent a week hiking in America, listening and talking
to Ajahn Jayasāro, I felt very close to him. I felt that I could ask any
question I liked and his answers touched me personally. Perhaps my
Western education enabled me to understand the Dhamma better in Western
terms. But I have to say that Ajahn Jayasāro is more Thai than many Thai
people I know. Even when I first met him, during many talks we had, he'd
always correct my use of Thai words. I also think the personality of the
teacher is important. When we like the teacher, we like the subject too.
I have much gratitude for that chance to have a close relationship to
the Ajahn while in America, something I had not yet experienced in
Thailand.

But there are also drawbacks to the `equality' form that I can see more
clearly now that I am in robes. In monastic life, identifying with being
`equal' can be detrimental. If everybody is the same, we may not listen
to or show appropriate humility around a more senior monk, especially if
our views tell us that we know more than him. It can seem natural for
the mind to follow its own \emph{kilesas}, but there will be no
spiritual growth then. Another way to put it is that I think it is best
to recognize equality in terms of Dhamma, while realizing hierarchy is
important when it comes to communal life and the Vinaya. Hierarchy can
be a very skilful means if we use it correctly. Personally, as a monk, I
feel that I want to keep the hierarchical relationship with the Ajahn,
to keep him high, so I can do what he teaches with respect and not treat
his teaching casually.

\emph{Tan Paññāvuddho} : Very soon, Tan Jayasiri, you will be going to
Australia. What is it that interests you about practising in the West?
Why would a Thai monk leave his own country and go to practise Dhamma in
Australia?

\emph{Tan Jayasiri}: The idea came from Ajahn Ñānadhammo. He spoke to me
about this two and a half years ago. He said that the monastery there is
a good place with an excellent teacher. At that time I wanted to
continue to practise at Wat Pah Nanachat. Now the time is appropriate
and I have decided to go. I think it will be valuable for me to be
tested by a new experience. After working as the secretary at Wat Pah
Nanachat, the opportunity for more solitude in Australia will be good. I
can speed up my practice. There I will have more free time. I can set up
my practice schedule the way I like.

\emph{Tan Paññāvuddho} : In the West many people have been interested in
Dhamma practice for decades, but not many people have become monks. Most
people prefer to keep the lay practice. Conversely, young men in
Thailand may have little exposure to Dhamma practice, but they still
have an interest in going forth. Do you think that in the West in the
future this pattern will change?

\emph{Tan Jayasiri}: Yes, in the future it will change. But in becoming
a monk, one has to give up so many things. When you become a monk, the
important foundation is the discipline. Someone with many attachments
might feel it is a narrow path and might not feel comfortable. They
might think it is more convenient to practise with just the Five
Precepts. To a Westerner the monastic precepts may seem like a lot at
first. They go against our habit in the lay life of being preoccupied
with making and acquiring things as the way to realize happiness. As
monks we have few requisites. But even though we may be ordained, we can
still have the same perceptions as when we were laymen. When a monk is
ordained on the outside, it doesn't mean he is ordained on the inside.
We can also feel there are too many precepts, because we still hold the
views and attachments we had in the lay life. So we have to put a lot of
effort into the practice. After some years the results of the practice
will appear. Perceptions and feelings from the lay life will transform
naturally.

These days, with modern communications and international travel, it is
more possible for people to be exposed to the possibility of practising
the Dhamma through monastic life. In the West you can sometimes see
monks out in society and leading retreats. People can see how a monk
conducts himself and lives his life. With more exposure to this, some
people may become interested in ordination.

\emph{Tan Paññāvuddho} : Tan Joti, you will be going soon to our branch
monastery in New Zealand. What do you think about the interest of
Westerners in the monastic path?

\emph{Tan Jotimanto}: In the West now people are interested in
meditation, but without knowing much about the bigger picture of Dhamma
practice. To become a monk is a very great step to take, because one has
to relinquish a lot of attachments. But if Westerners are really
interested, they can do it. Today they can even be ordained in the West,
but in my opinion there are still advantages to training in the East.
Because the \emph{sāsana} is so rooted here, it can be very nourishing
to the practice, especially in the early stages of monastic life. But
whether or not to be ordained as a monk is something one needs to know
for oneself.

\emph{Tan Dhirapañño} : For myself, I feel it is important to add that
if someone is really seriously about the practice, I would strongly
recommend the monastic life. It makes the foundation in \emph{sīla}
solid and firm. The Buddha laid down the three-fold training of
\emph{sīla, samādhi} and \emph{pañña}, but the \emph{samādhi} that has a
strong foundation in \emph{sīla} is more fruitful and leads more
directly towards wisdom. In the lay life there is often a compromise.
The monk, on the other hand, can go all the way to the highest goal. The
only limitation is his inner effort, not the situation outside.

\emph{Tan Paññāvuddho} : For me it is very uplifting to see so many
people across Thailand who have a strong love for Dhamma. That can be a
powerful inspirational and motivational force for a young monk like
myself. Some people in the West have this love of the Dhamma, but most
people don't know anything about the Dhamma at all. How do you feel
about going to a place where many people don't know about Dhamma
practice? They might not even know who Buddhist monks are.

{[}Silence and smiles{]}

\emph{Tan Paññāvuddho} : Tan Joti, you lived in Manhattan for five
years. Can you imagine going for alms-round on Fifth Avenue?

\emph{Tan Jotimanto}: No, I can't really imagine that. But if I were in
that situation I would have to accept it if people didn't know what a
Buddhist monk was, or if they looked at me strangely. But then again,
many strange things happen in New York City. In any case, it is a good
practice to uphold the tradition, and the sight of a simple monk walking
for alms, practising mindfulness, might offer something to the people in
their busy lives. These days, however, people tend to know more about
monks. Still, if I were to go walking in New York, I would be sure to be
prepared to answer any questions that I might receive from the people I
met.

\emph{Tan Paññāvuddho} : Tan Jayasiri, how would you respond to the
question, `Who are you?' which is sometimes asked of monks in the West,
perhaps while they are on alms-round?

\emph{Tan Jayasiri}: If someone asked who we were, I would probably
smile and respond that we are alms-mendicant Buddhist monks. If he asked
me to describe our practice, I would say that we try to live a simple
life. We practise meditation to try to understand the true nature of
things in this world. If we can practise the teachings of the Buddha and
see things clearly as they truly are, we can help to lessen suffering.
The aim is to have wisdom with the thoughts, emotions and feelings that
arise: `I like this' or `I don't like this', `I want this' or `I don't
want this' -- we can let go of them. If we can keep our minds above the
worldly feelings of happiness and unhappiness, we can begin to find a
true freedom and happiness that do not change. This may be easy to
believe, but it is difficult to do. The practice is not easy. If the
person who asked me had some free time, they could try to practise
mindfulness and meditation to investigate these things on their own.

\emph{Tan Paññāvuddho} : Tan Joti, how would you respond?

\emph{Tan Jotimanto}: I would try to explain to him that I am a Buddhist
monk who does not handle or use money, and that it is our tradition each
morning to walk for alms. It might take some time to explain the
Buddha's teachings. What I would say would depend on the person's
background and the situation. If the situation were right, I would say
we practise to let go, not to attain something. We do this through
training to develop virtue, meditation and wisdom. Higher teachings on
non-self and emptiness would be difficult to explain. It would be more
appropriate to start with teachings about \emph{sīla} and \emph{mettā},
and when a friendly relationship was established, then we could talk
about something deeper. If he wanted to talk more, I could invite him to
visit the monastery to learn more about Dhamma practice and try it out
for himself.

\emph{The Interviewees}

\emph{This previously unpublished interview is from the original
material for the first edition of `Forest Path'. It was not included in
the original book, probably because of its length. On re-reading, it
seemed a real pity not to include it this time. Tan Paññavuddho (see
page} **

!! Needs to be altered after final layout.

\emph{)} \emph{the main editor of the first edition, conducted the
interview. The three monks interviewed have all in the meantime been
abroad: Tan Jotimanto went to New Zealand, Tan Jayasiri to Australia,
and Tan Dhirapañño stayed with one of his former teachers in America,
Bhante Gunaratana. All have now returned to Thailand and are at present
engaged in running their own little monasteries there, Tan Jayasiri in a
remote hermitage called Mettāgiri Forest Monastery in Chayabhumi
Province; Tan Jotimanto and Tan Dhirapañño co-abboting Wat Pah Boon Lorm
on the Moon River in Ubon Province, a half-hour drive from Wat Pah
Nanachat and closely linked to it.}

