% Title: Twain Shall Meet
% Author: Ajahn Jayasāro

From the mid-fourteenth century until its sack by the Burmese in
1767, Ayudhya was the capital of the Thai nation. Established on an
island in the Jow Phya River, it was ideally situated to act as an
entrepôt port at a time when land routes were safer than the sea, and
merchants in the Orient sought to avoid the Straits of Malacca. Within
two hundred years Ayudhya had become one of the most thriving
cosmopolitan cities in Asia. Its population of a million exceeded that
of London. Around five hundred temples, many with pagodas covered in
gold leaf, lent the city a magical, heaven-like aura that dazzled
visiting traders. 

By the mid-seventeenth century the inhabitants of Ayudhya were
accustomed to the sight of \emph{farang}.\footnote{Derived from `Frank'
  or `French', the first Westerners known to the Thais.} Communities of
traders from France, Holland, Portugal and England were housed outside
the city wall. The kings of Ayudhya often employed foreign mercenaries
as bodyguards. To the Thais these strange white beings seemed like a
species of ogre: hairy, ill-smelling, quarrelsome and coarse; lovers of
meat and strong spirits, but possessors of admirable technical skills, 
particularly in the arts of war. The ogres had a religion -- priests
and monks accompanied them -- but it was unappealing to the Thais, who
were content with their own traditions. Having long equated spirituality
with renunciation of sensual pleasures, they perceived the Western
religious as living luxurious lives. They found the way the missionaries
slandered each other in their competition for converts undignified; they
saw little agreement between their actions and words. The Ayudhyan Thais
gently rebuffed what they saw as an alien faith with politeness and
smiles. 

But the legendary Siamese tolerance was stretched to the limit during
the reign of King Narai (1656-88), when a Greek adventurer, Constantine
Faulkon, became \emph{Mahatthai}, minister for trade and foreign
affairs, second in influence to the king himself. After his conversion
to Catholicism, Faulkon became involved with the French in plots to put
a Christian prince on the throne and thus win the whole country for God
and Louis XIV. At the old king's death in 1688, however, conservative
forces prevailed; French hopes were dashed and Faulkon was executed. For
the next 150 years the Siamese looked on Westerners with fear, aversion
and suspicion. 

But as French and British power and prestige spread throughout the
region in the nineteenth century, the image of the Westerner changed. He
came to represent authority and modernity, the new world order that had
to be accommodated. As all the rest of the region fell into European
hands, Siam's independence became increasingly fragile. King Mongkut
 (1851-68) reversed the policies of previous monarchs and cultivated
friendships with Western scholars and missionaries. He believed that the
only way for a small country to survive in the colonial era was to earn
the respect of the Western powers by becoming like them. He introduced
Western styles of dress and uniform. He predicted eclipses by scientific
means, undermining the hitherto unshakeable prestige of the astrologers. 
He sought to reform popular Buddhism along more rational `scientific'
lines, to protect it from the missionaries' disdain. After King
Mongkut's death his son King Chulalongkorn sought to create a modern
centralized state and administration, relying heavily on Western
expertise. Members of the royal family and aristocracy were sent to
study in the West, particularly England. The humiliation inflicted upon
the Thais by the French annexation of their eastern territories
confirmed the superiority of the West in worldly matters. 

At the time that Ajahn Chah reached manhood, Western culture had already
attained its pre-eminent position. Among the wealthy elite, expensive
imported clothes, motor vehicles, gadgets and foods were sought-after
status symbols. The absolute monarchy was overthrown in 1932 in favour
of a Western style democracy, which was soon displaced by a more potent
import: military dictatorship. Fascism was the new vogue -- far more
appealing to the military men running the country than the messiness of
political debate, and far more accommodating to the Thai penchant for
uniforms. Field Marshal Pibulsongkram passed laws making it compulsory
for men to wear hats and kiss their wives on the cheek before leaving
for work in the morning. The country's name was changed to Thailand. 
Chauvinism was promoted in the guise of patriotism. The marginalization
of Buddhist goals and ideals, albeit coupled with official support for
Buddhist forms and rituals, became a feature of future development. In
the hamlets of Ubon, images of the West came from Hollywood. Travelling
movie companies set up their screens and loudspeakers in village wats; 
Clark Gable and Greta Garbo enchanted their audiences in homely Lao, 
dubbed live from behind the screen. Thus the first flesh-and-blood
glimpse of \emph{farangs} in Ubon, exciting though it was, came as a
shock. While the newly-ordained Ajahn Chah was studying in local village
monasteries, a group of gaunt ragged POWs were gaoled in the centre of
town. They were prisoners of the occupying Japanese forces, hostages
against Allied bombing raids. The local people smuggled them bananas. 

Then in the nineteen sixties came the Vietnam war. Ubon, closer to Hanoi
than to Bangkok, attained strategic importance once more. By the end of
the decade twenty thousand young Americans were stationed on a sprawling
airbase to the north of the town. Huge uniformed men, black, brown and
white, strode along the streets hand-in-hand with mini-skirted
prostitutes, caroused in tacky nightclubs with names like 'Playboy' and
took their minds on vacation with `Buddha sticks'. Overhead every few
minutes came the deafening sound of F4 fighters and heavily-laden
bombers taking off on missions over Laos, Cambodia and Vietnam. 

American military personnel were not, however, the only young Westerners
in Thailand at that time. It was during this period that villagers
working in the fields to the east of Wat Pah Pong became used to a
strange new sight. Tall fair-skinned young men with long hair, T-shirts
and faded blue jeans would often be seen walking along the ox-track with
a dogged, diffident stride, a large grubby backpack like a malignant
growth behind them. These young men were the first trickle of the steady
stream of Westerners who were to find their way to Ajahn Chah. They were
to become the senior members of a Western Sangha that now numbers over a
hundred and fifty monks and nuns. 

`Luang Por, only a few of your Western disciples speak Thai and you
can't speak their language. How do you teach them?' This was one of the
most common questions that Ajahn Chah faced from the early seventies
onwards, as the number of his Western disciples rapidly increased. He
would explain that he was teaching Buddhism not as a philosophy but as a
way of liberation; pointing directly to the experience of suffering and
its cause was more important than finding words to describe the process. 
Sometimes, to clarify this point, he would pour from the thermos flask
on the table beside him into a cup: 

`In Thai we call this \emph{nam rawn}, in Lao it is \emph{nam hawn} and
in English they call it hot water. These are just names. If you dip
your finger into it there is no language that can convey what that feels
like, but even so, people of all nationalities know it for themselves.'

On another occasion a visitor, seeing all the foreign monks, asked Ajahn
Chah whether he spoke English, French, German or Japanese, to which in
every case Ajahn Chah replied that no, he could not. The questioner
looked confused: how did the foreignmonkslearn anything, then? Ajahn
Chah replied characteristically with a question: `At your home do you
keep any animals? Have you got cats and dogs? Have you got any oxen or
buffalo? Yes? Well can you speak Cat language? Can you speak Dog? Can
you speak Buffalo? No? Then how do they know what you want them to do?'

He summarized: `It's not difficult. It's like training water buffaloes. 
If you just keep tugging the rope, they soon catch on.'

To Thais, water buffaloes are the epitome of dullness and stupidity. 
Comparing a human being to a buffalo would normally be considered
offensive; someone who calls anyone a \emph{kwai} to their face is
either very angry or spoiling for a fight. Given the exaggerated respect
for the intelligence of Westerners common in Thailand, Ajahn Chah's
audience would always find the buffalo comparison hilarious. 

The sight of the Western monks was a powerful one. At a time when
Western technology, material advances and expertise were being so
touted, here were educated young men who had voluntarily renounced the
things that people were being encouraged to aspire to; men who had
chosen to live austere lives in the forest as monks, not understanding
the language, eating coarse food, striving for peace and wisdom in the
same way that Thai monks had been doing for hundreds of years. It was
baffling, fascinating and, above all else, inspiring. Many Thai visitors
would leave Wat Pah Pong thinking that perhaps there was more to
Buddhism than they had thought. If the Westerners had so much faith in
it, how could it be outdated? 

Luang Por's basic technique was not, he insisted, particularly
mysterious; he led his Western disciples, he showed them what to do, he
was an example. It wasn't necessary to impart a great deal of
information: `Even though I have a lot of Western disciples living with
me, I don't give them so much formal instruction. I lead them in the
practice. If you do good, you get good results; if you do bad, you get
bad results. I give them the opportunity to see that. When they practise
sincerely they get good results, and so they develop conviction in what
they're doing. They don't just come here to read books. They really do
the practice. They abandon whatever is bad in their hearts and goodness
arises in its place.'

The Westerners came to Buddhist teachings and monastic life without the
cultural conditioning of the Thais. In one sense they had `beginner's
mind'. Ajahn Chah found their open, questioning attitude refreshing and
stimulating. As students they were free of the complacency that he
considered such a serious obstacle for his Thai disciples. But their
openness was not without drawbacks: the Westerners were often dragged
into the quagmires of doubt. Whereas the Thai monks could usually give
themselves to the training in a wholehearted way, fired by an
unquestioning faith in the teacher and the tradition, the Westerners
were often fettered by doubts. Ajahn Chah said: 

`Once you've got them to stop, these Westerners see clearly exactly how
they've done it, but in the beginning it's a bit wearing on the teacher. 
Wherever they are, whoever they're with, they ask questions all the
time. Well, why not, if they don't know the answers? They have to keep
asking until they run out of questions, until there's nothing more to
ask. Otherwise they'd just keep running -- they're hot.'

\section{The first disciple -- Ajahn Sumedho}

In 1967 a Wat Pah Pong monk called Tan Sommai returned from a
\emph{tudong} trip to the north of Isan with an old friend who literally
stood head and shoulders above him. Even the most restrained monks in
Wat Pah Pong were unable to resist at least a surreptitious glance. The
new monk was 6 foot 3 inches tall, with blonde hair, an angular nose and
bright blue eyes. His name was Sumedho. The two men had run into each
other for the first time in Korea more than ten years before, dressed in
the creased white uniforms of their nations' navies. And now, by
coincidence, they had met a second time, dressed in the yellow robe of
the monk, in the meditation monastery on the banks of the Mekong River
where Sumedho had recently been ordained. They exchanged their stories. 
Sumedho told Sommai how he had returned to college after the Korean War
and gained a Master's degree in Asian Studies from Berkeley. After
graduation he had joined the Peace Corps and taught English in Borneo, 
before moving on for a spell at Thammasat University in Bangkok. It was
after receiving meditation instruction at the nearby Wat Mahadhatu that
his interest in Buddhism, born in Korea, had ripened into the decision
to become a monk. Now, though, after months of solitary meditation in a
small hut, Sumedho was beginning to feel some frustration about the form
of his monastic life, and was feeling the need for a more rounded way of
practice. Tan Sommai's descriptions of Wat Pah Pong were opportune and
inspired him. His preceptor kindly gave permission for him to leave, and
the two monks set off to walk down to Ubon, Sumedho feeling `as if I was
being pulled by a magnet'. 

The force of attraction held. Eventually Sumedho would stay for ten
years, form the nucleus around which the Western community of monks
would coalesce and establish Wat Pah Nanachat, before moving to
England to begin the first of nine overseas branch monasteries at
Chithurst in southern England. 

Someone once asked Luang Por whether he had any special connection with
Westerners that led to so many becoming his disciples. He replied that
his acquaintance was restricted to cowboy movies he had watched before
he ordained. `It was \emph{déjà vu} -- when I was a small child I went
to see a cowboy movie with my friends and one of the characters was this
big man smoking cigarettes. He was so tall it fascinated me. What kind
of human being could have such a huge body? The image has stuck in my
mind until now. And so a lot of Westerners have come. If you're talking
about causes, there was that. When Sumedho arrived, he was just like the
cowboy in the movie. What a long nose! As soon as I saw him, I thought
to myself, ``This monk is a Westerner'', and I told him that I'd seen
him before in a movie. So there were supporting causes and conditions. 
That's why I've come to have a lot of Western kith and kin. They come
even though I can't speak English. I've tried to train them to know the
Dhamma as I see it. It doesn't matter that they don't know Thai customs. 
I don't make anything of it, that's the way things are. I just keep
helping them out -- that's the gist of it.'

When Ajahn Sumedho asked to be accepted as a student, Luang Por agreed
but made one condition, that he should fit in with the Thai monks and
not receive any special consideration. 

`At the other monasteries in Thailand where I'd lived, the fact that I'd
been a Westerner had meant that I could expect to have the best of
everything. I could also get out of the work and other mundane things
that the other monks were expected to do: ``I'm busy meditating now. I
don't have time to sweep the floor. Let someone else sweep it. I'm a
serious meditator.'' But when I arrived at Wat Pah Pong and people said,
``He's an American; he can't eat the kind of food we eat'', Luang Por
said, ``He'll have to learn''. And when I didn't like the meditation hut
I was given and asked for another that I liked better, Luang Por said,
``No''. The whole way of training was that you had to conform to the
schedule. When I asked Luang Por if I could be excused from the long
Dhamma talks which I didn't understand, he just laughed and said, ``You
have to do what everyone else does.''\thinspace'

Wat Pah Pong provided a very different monastic environment from the one
with which Ajahn Sumedho was familiar. In his previous wat he had been
living in solitude, sitting and walking in and near his hut, 
single-mindedly devoted to the development of a meditation technique. 
The only human contact had been a daily interview with his teacher. It
had been a beneficial period for him, but he had become unsure how
sustainable such a kind of monastic life would be in the long term. What
he felt he lacked was Vinaya training: 

`At Wat Pah Pong the emphasis was on communal activities, working
together, eating together, etc., with all its rules. I knew that if I
was going to live as a monkI needed the monk's training, and I hadn't
been getting that at the meditation centre I had been in before. What
Luang Por gave me was a living situation to contemplate. You developed
an awareness around the monastic tradition, and it was something that I
knew I needed. I needed restraint and containment. I was a very
impulsive person with a tremendous resistance to any kind of authority. 
I had been in the Navy for four years and had developed an aversion to
authority and rank. And then before I went to Thailand I had spent a few
years at Berkeley, California, where it was pretty much a case of doing
your own thing -- there was no sense of having to obey anybody or live
under a discipline of any sort. But at Wat Pah Pong I had to live
following a tradition that I did not always like or approve of, in a
situation where I had no authority whatsoever. I had a strong sense of
my own freedom and rights, and of asserting them, but I had no idea of
serving anyone else; being a servant was like admitting you were somehow
inferior. So I found monastic life very useful for developing a sense
for serving and supporting the monastic community. 

`What impressed me so much about Luang Por was that although he seemed
such a free spirit, an ebullient character, at the same time he was very
strict with the Vinaya. It was a fascinating contrast. In California the
idea of freedom was being spontaneous and doing what you felt like; and
the idea of moral restraint and discipline in my cultural background was
like this big ogre that's coming to squash you, with all these rules and
traditions -- you can't do this and you can't do that -- and pressing
down on you so much. 

`So my immediate reaction in a strict monastery like Wat Pah Pong was to
feel oppressed. And yet my feeling about Luang Por was that although his
actions were always within the margins of the Vinaya, he was a free
being. He wasn't coming from ideas of doing what he liked, but from
inner freedom. So in contemplating him I began to look at the Vinaya so
as to use it, not just to cut myself off or to oppress myself, but for
freedom. It was like a conundrum: how do you take a restrictive and
renunciant convention and liberate your mind through that convention? I
could see that there were no limits to Luang Por's mind. Oftentimes
attachment to rules makes you worry a lot and lack confidence, but Luang
Por was radiant. He was obviously not just someone keeping a lot of
rules, anxious about his purity. He was a living example of the freedom
that comes from practice.'

Ajahn Sumedho was impressed and reassured by Ajahn Chah's inquiries
about his meditation practice. Ajahn Chah merely acknowledged with a
grunt that the method Ajahn Sumedho was using was valid, and gave him
permission to carry on with it if he found it useful. It did not seem to
be a crucial issue. It was clear that what Ajahn Chah was teaching was
not confined to a particular meditation technique, but consisted of a
comprehensive training, the creation of a context or environment in
which any legitimate technique would bear fruit. This was exactly what
Ajahn Sumedho felt he needed: 

`You have to find someone you resonate with. I'd been in other places
and nothing had really clicked. I didn't have a fixed idea of having a
teacher either, I had a strong sense of independence. But with Luang Por
I felt a very strong gut reaction. Something worked for me with him. 

`The training at Wat Pah Pong was one of putting you in situations where
you could reflect on your reactions, objections, etc., so that you began
to see the opinions, views and prejudices and attachments that come up
naturally in those situations. Luang Por was always emphasizing the need
to reflect on the way things are. That is what I found most helpful, 
because when you're as self-centred and opinionated as I was then, you
really need to open your mind, and so I found Luang Por's way much more
clear and direct. As I was very suppressed already, I really needed a
way of looking at myself honestly and clearly, rather than just trying
to suppress my feelings and force my mind into more refined states. He
was also very aware of the individual needs of the monks, so it wasn't
like there was a blanket technique. He realized that you really have to
figure it out for yourself, and so how I saw him, how he affected me, 
was that he seemed to provide a backdrop for my life from which I could
reflect.'

Even with this kind of appreciation of the way of practice at Wat Pah
Pong, Ajahn Sumedho did not find it easy. Apart from the easily foreseen
difficulties and frustrations he experienced with the language, culture, 
climate, diet and so on, he began, ironically, to harbour misgivings
about the Vinaya. His personality had always been an idealistic one; he
was drawn to the big picture, the unifying vision, and tended to get
impatient with the nuts and bolts of everyday life. He felt a natural
antipathy to the nit-picking and cavilling over trivial matters that
seemed to him to characterize Vinaya instruction: 

`Even when I could understand the language, the Vinaya readings were
excruciatingly boring to listen to. You'd hear about how a monkwho has a
rent in his robe so many inches above the hem must have it sewn up
before dawn, and I kept thinking, ``This isn't what I ordained for!'' I
was caught up in these meticulous rules, trying to figure out whether
the hole in my robe was four inches above the hem or not, and whether I
should have to sew it up before dawn. \emph{Bhikkhus} would even become
argumentative about the borders of sitting cloths! When it came to the
pettiness of everyday life and of living with people of many different
temperaments, problems and characters, whose minds were not necessarily
as inspired as mine seemed to be at the time, I felt great depression.'

The Vinaya texts prescribe various duties to be performed towards a
teacher by his students. One of them is to wash the teacher's feet on
his return from alms-round. At Wat Pah Pong as many as twenty or thirty
monks would be waiting for Ajahn Chah at the dining hall footbath, eager
for the honour of cleaning the dirt from his feet or of having a hand on
the towel that wiped them dry. At first Ajahn Sumedho found the whole
thing ridiculous. Every day he would begin to fume as monks **started to
make their way out to the footbath. It was the kind of ritual that made
him feel alienated from the rest of the community. He would feel angry
and critical. 

`But then I started listening to myself and I thought, 
``This is really an unpleasant frame of mind to be in. Is it anything to
get so upset about? They haven't made me do it. It's all right; there's
nothing wrong with thirty men washing one man's feet. It's not immoral
or bad behaviour and maybe they enjoy it; maybe they want to do it --
maybe it's all right to do that. Perhaps I should do it.'' So the next
morning thirty-one monks ran out and washed Luang Por's feet. There was
no problem after that. It felt really good; that nasty thing in me had
stopped.'

Although the Buddha called praise and blame `worldly dhammas', not even
the most dedicated and unworldly spiritual seekers can avoid them. 
Throughout his early days at Wat Pah Pong, Ajahn Sumedho received
generous praise. In Buddhist cultures the voluntary renunciation of
sensual pleasures for spiritual training is an esteemed virtue. The
sacrifices Ajahn Sumedho had made to become a monkinspired both his
fellow monastics and the monastery's lay supporters. In leaving America
and donning the yellow robe, not only had he given up a standard of
living that Isan peasant farmers could only dream about, but he had done
so in exchange for a life in one of the strictest and most austere
forest wats in the country. The conservative Isan people, their sense of
security and well-being so bound up with the maintenance of their
traditions, were impressed at how well Ajahn Sumedho could live in exile
from the conditions he was used to, how readily he adapted to a new
climate, language and (especially) diet. They were inspired by how
diligent and dedicated he was in his practice. As the only Westerner he
stood out and was a centre of attention wherever he went, second only to
Ajahn Chah himself. 

On the other hand, the Thais have a natural, apparently almost
effortless physical grace, and the monastic techniques of developing
mindfulness by close attention to detail enhance it. For them to see
Ajahn Sumedho -- physically intimidating and with an obvious zeal for
the practice, but at the same time by their standards so awkward and
ungainly, confused them. In most it provoked a quiet but affectionate
amusement; for some that amusement was soured with a hint of fear, 
jealousy and resentment. Ajahn Sumedho, both a little paranoid at the
attention and also enjoying it, could not help but feel self-conscious: 

`They would ask, ``How old are you?'' I'd say, ``Thirty-three.'' And
they'd say, ``Really? We thought you were at least sixty.'' Then they
would criticize the way I walked, and say, ``You don't walk right. You
are not very mindful when you walk.'' And I'd take this \emph{yarm} and
I'd just dump it down, without giving it any importance. And they'd say, 
``Put your bag down right. You take it like this, fold it over, and then
you set it down beside you like that.'' The way I ate, the way I walked, 
the way I talked -- everything was criticized and made fun of; but
something made me stay on and endure through it. I actually learnt how
to conform to a tradition and a discipline -- and that took a number of
years, really, because there was always strong resistance. But I began
to understand the wisdom of the Vinaya and over the years my equanimity
grew.'

\section{Pushed}

Ajahn Chah's attitude to Ajahn Sumedho changed after a few years. Seeing
his disciple's growth in confidence and the praise he was receiving, he
began to treat him more robustly. Ajahn Sumedho remembers: 

`For the first couple of years Luang Por would compliment me a lot and
boost up my ego, which I appreciated because I tended to be
self-disparaging, and his constant very positive attitude towards me was
very helpful. Because I felt so respected and appreciated by him, I put
a lot of effort into the practice. After a few years it started to
change; he saw I was stronger and he began to be more critical. 
Sometimes he would insult me and humiliate me in public -- but by then
I was able to reflect on it. 

`There were times when Luang Por would tell the whole \emph{sala}-full
of laypeople about things I'd done that were uncouth, like my clumsy
attempts to eat with my hands. He would imitate me making a ball of
sticky rice and then making a complete mess, pushing it into my mouth
and nose. The whole \emph{sala}, monks and laypeople, would be roaring
with laughter. I'd just sit there feeling angry and embarrassed. One
time a novice picked up my outer robe by mistake and gave it to him. 
Luang Por laughed and said he knew immediately whose it was because of
the bad smell, ``the \emph{farang} stink''. When I heard Luang Por say
that, of course I felt pretty indignant; but I could endure it, and
because of the respect I felt for him I didn't show any reaction. He
asked me if I was feeling all right and I said yes, but he could see
that my ears were bright red. He had a wonderful sense of timing, and so
I could work with it, and I benefited from being able to observe my own
emotional reactions to being insulted or humiliated. If he'd done that
at the beginning I would never have stayed. There was no real system
that I could see; you just felt that he was trying to help you --
forcing you to look at your own emotional reactions -- and I always
trusted him. He had such a great sense of humour, there was always a
twinkle in his eye, always a bit of mischief, and so I just went along
with it.'

Many of Ajahn Sumedho's most powerful memories of his early years at Wat
Pah Pong are of occasions when some dark cloud or other in his mind
dissolved through a sudden insight into the desires and attachments that
conditioned it. To him Ajahn Chah's genius as a teacher seemed to lie in
creating the situations in which this process could take place --
bringing a crisis to a head, or drawing his attention most skilfully to
what was really going on in his mind. His faith in Ajahn Chah made him
open. A smile from his teacher or words of encouragement at the right
time could make hours of frustration and irritation seem ridiculous and
insubstantial; a sharp question or a rebuke could wake him up from a
long bout of self-indulgence: `He was a very practical man and so he was
using the nitty-gritty of daily life for insight. He wasn't so keen on
using special events or extreme practices as on getting you to wake up
in the ordinary flow of monastic life, and he was very good at that. He
knew that any convention can become perfunctory and deadening after a
while if you get used to it. He was aware of that, so there was always
this kind of sharpness that would startle and jolt you.'

In the early days anger was the major fuel of Ajahn Sumedho's suffering. 
He relates how exhausting the afternoon leaf-sweeping periods could be
in the hot season. One day as he toiled in the sun, his body running
with sweat, he remembers his mindfulness becoming consumed by aversion
and self-righteousness: `I don't want to do this. I came here to get
enlightened, not to sweep leaves off the ground.' Just then Ajahn Chah
approached him and said, `Where's the suffering? Is Wat Pah Pong the
suffering?'

`I suddenly realized there was something in me which was always
complaining and criticizing, and which was preventing me from ever
giving myself or offering myself to any situation. 

`Another time I had this really negative reaction to having to sit up
and practise all through the night, and I must have let it show. After
the evening chanting Luang Por reminded everyone that they should stay
and meditate right through to dawn. ``Except'', he said, ``for Sumedho, 
he can go and have a rest.'' He gave me a nice smile and I just felt so
stupid. Of course, I stayed all night. 

`There were so many moments when you were caught up in some kind of
personal thing and he could sense that. He had the timing to reach you
in that moment when you were just ripe, so that you could suddenly
realize your attachment. One night we were in the little \emph{sala}, 
where we did the Pātimokkha, and his friend Ajahn Chaluay came to visit. 
Usually, after the Pātimokkha was over we would go and have a hot drink, 
and then join the laypeople in the main \emph{sala}. But on that night
he and Ajahn Chaluay sat there telling jokes to each other for hours, 
and we had to sit there and listen. I couldn't understand what they were
talking about and I got very irritated. I was waiting for him to tell us
to go to the hall, but he just carried on. He kept looking at me. Well, 
I had a stubborn streak and I wasn't going to give up. I just got more
and more angry and irritated. It got to about midnight and they were
still going strong, laughing like schoolboys. I got very self-righteous; 
they weren't even talking seriously about practice or Vinaya or
anything! My mind kept saying, `What a waste of time. They should know
better'. I was full of my anger and resentment. He knew that I had this
stubborn, tenacious streak, and so he kept going until two in the
morning, three in the morning. At that time I just gave up the whole
thing, let go of all the anger and resistance and felt a wave of bliss
and relaxation; I felt all the pain had gone. I was in a state of bliss. 
I felt I'd be happy if he went on forever. He noticed that and told
everyone we could leave.'

\section{Dhamma Talks}

Given Ajahn Sumedho's celebrity and his steadily growing proficiency in
Thai, it was natural that Wat Pah Pong's lay supporters would be eager
to hear him give a Dhamma talk. Four years after Ajahn Sumedho's
arrival, Ajahn Chah decided that the time was ripe for his first Western
disciple to begin a new kind of training: that of expressing the Dhamma
in words. 

One night, during a visit to another monastery, Ajahn Chah caught Ajahn
Sumedho by surprise. With no prior warning, he asked him to talk to the
lay supporters who had gathered in honour of their visit. The prospect
of ascending the monastery's Dhamma seat and struggling to give an
extempore address to a large audience in a language in which he was not
particularly fluent was overwhelming. Ajahn Sumedho froze and declined
as politely but firmly as he could. But strong in his trust in Ajahn
Chah and the realization that he was merely postponing the inevitable, 
he began to reconcile himself to the idea. When Ajahn Chah `invited' him
to give a talk on the next Wan Phra, he acquiesced in silence. Ajahn
Sumedho was well aware of Ajahn Chah's view that Dhamma talks should not
be planned in advance, but he felt insecure. At the time he was reading
a book on Buddhist cosmology and reflecting on the relationship between
different realms of existence and psychological states. He made some
notes for the coming talk. 

Wan Phra soon came and Ajahn Sumedho gave the talk. Although his
vocabulary was still quite rudimentary and his accent shaky, it seemed
to go down well. He felt relieved and proud of himself. Throughout the
next day laypeople and monks came up to him to express their
appreciation of a fine talk, and he looked forward to basking in the sun
of his teacher's praise. But on paying respects to Ajahn Chah beneath
his \emph{kuti}, he met a stony frown. It sent a chill through his
heart. In a quiet voice Ajahn Chah said, `Don't ever do that again'. 
Ajahn Sumedho realized that Ajahn Chah knew he had thought the talk out
beforehand, and that in his eyes, although it had been an intelligent, 
interesting and informative discourse, it was not the Dhamma speaking; 
it was merely thoughts and cleverness. The fact that it was a `good
talk' was not the point. 

In order to develop the right attitude in giving Dhamma talks, a
monkneeds a thick skin. One night Ajahn Chah told Ajahn Sumedho to talk
for three hours. After about an hour Ajahn Sumedho had exhausted his
initial subject and then began to ramble, hunting for things to talk
about. He paused, repeated himself and embarked on long meandering
asides. He watched as members of his audience got bored and restless, 
dozed, walked out. Just a few dedicated old ladies sat there throughout, 
eyes closed, like gnarled trees on a blasted plain. Ajahn Sumedho
reflected after it was all over: 

`It was a valuable experience for me. I
began to realize that what Luang Por wanted me to do was to be able to
look at this self-consciousness, the posing, the pride, the conceit, the
grumbling, the laziness, the not-wanting-to-be-bothered, the wanting to
please, the wanting to entertain, the wanting to get approval.'

Ajahn Sumedho was the only Western monkat Wat Pah Pong for four years, 
until in 1971 two more American monks arrived to spend the Rains
Retreat. One of them, Dr.~Douglas Burns, was a psychologist based in
Bangkok who intended to be a monk for the duration of the retreat; the
other was Jack Kornfield (Phra Suñño), who after practising in
monasteries throughout Thailand and Burma was to return to lay life, and
become one of the most influential teachers in the American
\emph{Vipassanā} movement. Neither monk stayed at Wat Pah Pong very
long, but both exercised a strong influence on future developments. At
the end of his short period in the robes Dr Burns returned to Bangkok, 
where he would recommend any Westerners interested in ordaining to go to
live with Ajahn Chah. A number of the first generation of monks came to
Ubon after such a referral. In the months that Jack Kornfield was with
Ajahn Chah he made assiduous notes of the teachings that he received, 
and later printed them as the extremely popular \emph{Fragments of a
Teaching} and \emph{Notes from a Session of Questions and Answers}. 
Subsequently, as Kornfield's own reputation spread in America, his
frequent references to Ajahn Chah introduced him to a Western audience. 
This acquaintance was strengthened by \emph{Still Forest Pool}, a
collection of Ajahn Chah's teachings which Kornfield co-authored with
Paul Breiter, another ex- monk (formerly Venerable Varapañño). 

Ajahn Chah's charisma and his ability to move and inspire his Western
disciples soon became well-known. But if Ajahn Chah was the main reason
why Wat Pah Pong became the most popular Thai forest monastery for
Westerners seeking to make a long-term commitment to monastic life, 
Ajahn Sumedho's presence may often have been a deciding factor. Here was
someone who had proved it could be done, who had lived a number of years
in austere conditions with no other Western companions, and had
obviously gained much from the practice. He was both a translator, elder
brother and, more and more, although he resisted the evolution, a
teacher in his own right. Phra Varapañño arrived in Wat Pah Pong at a
time when Ajahn Chah was away for a few days. His meeting with Ajahn
Sumedho was crucial to his decision to stay: 

`Sitting up there on the porch in the peace of the forest night, I felt
that here was a place beyond the suffering and confusion of the world
-- the Vietnam War, the meaninglessness of life in America and
everywhere else, the pain and desperation of those I had met on the road
in Europe and Asia who were so sincerely looking for a better way of
life but not finding it. This man, in this place, seemed to have found
it, and it seemed entirely possible that others could as
well.'\footnote{Quoted from Paul Breiter's \emph{Venerable Father: A
  Life with Ajahn Chah} (Cosimo Books, New York City 2004)}

In 1972 the Western Sangha of monks and novices numbered six, and Ajahn
Chah decided that they should spend the Rains Retreat at Tam Saang Pet, 
a branch monastery perched on a steep-sided hill overlooking the flat
Isan countryside, about 100 kilometres away to the north. Personality
conflicts festered away from the guiding influence of Ajahn Chah, and
Ajahn Sumedho felt burned: 

`To begin with I felt a lot of resentment about taking responsibility. 
On a personal level, the last thing I wanted to do was be with other
Western monks -- I was adjusted to living with Thai monks and to
feeling at ease within this structure and culture, but an increasing
number of Westerners were coming through. Dr Burns and Jack Kornfield
had been encouraging people to come. But after the Western Sangha had
this horrendous Rains Retreat at Tam Saang Pet I ran away, spent the
rains in a monastery in the South-East and then went to India. But while
I was there I had a really powerful heart-opening experience. I kept
thinking of Luang Por and how I'd run away, and I felt a great feeling
of gratitude to him, and I decided that I would go back and serve. It
was very idealistic. ``I'll just give myself to Luang Por, anything he
wants me to do.'' We'd just opened this horrible branch monastery at
Suan Glooay down on the Cambodian border, and nobody wanted to go and
stay there. I'd gone down there for a \emph{Kathina} ceremony and been
taller than all the trees. So in India I thought I'd volunteer to go and
take over Suan Glooay. I had this romantic image of myself. But of
course, when I got back Luang Por refused to send me there, and by the
end of the year there were so many Westerners at Wat Pah Pong that he
asked me to come back to translate for them. Basically, I trusted him
because he was the one pushing me into things that I wouldn't have done
by myself.'

\section{The Author}

Tan Ajahn Jayasāro stayed on as Abbot of Wat Pah Nanachat until
2002. During that time the monastery grew in terms of monastics and Thai
laypeople keen to come and practise for short periods of time. Having
completed his five-year commitment to guide the community in Bung Wai, 
he moved to the solitude of a hermitage offered to him in the Pak Chong
district of Korat province, a couple of hours' drive north-east of
Bangkok, where he has lived ever since. He currently divides his time
between solitude at his hermitage, public teaching and an active role in
the field of Buddhist-based education, both in Thailand and abroad. He
is the spiritual director of the Panyaprateep Foundation, which is the
umbrella organization for a secondary school of the same name. 

Tan Ajahn Jayasāro has always maintained his close links with Wat
Pah Nanachat, visiting the monastery frequently. He is always available
for personal consultation with senior and junior monks alike. As the
author of Luang Por Chah's exhaustive biography, and with his vast
knowledge of Thai history and culture, he is also a precious source of
knowledge of our tradition for the current generation of Wat Pah
Nanachat residents. 

