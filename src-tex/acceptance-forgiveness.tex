
Tan Acalo

It had been three years since I'd spent any time in the country of my
birth, when I unexpectedly had the opportunity to return to Australia
with one of my teachers. Tan Ajahn Anan had been invited to Melbourne,
to visit a newly established meditation hermitage and give teachings at
the local Buddhist society. Several other monks were going and we would
be passing through the cities of Sydney and Canberra, staying in Thai
wats and then later going on to Melbourne. In Melbourne I would take
leave of my teacher and travel to Queensland to spend time with my
mother and father. As a monk, one tries to practise in all situations.
In Australia I would be close to the members of my family. I would have
to honour our own relationship and also their relationships with others.
I would have to be considerate of their lifestyles and views, yet at the
same time maintain my own loyalties. This article explores some of the
challenges, along with what to me were some of the more significant and
moving times during my visit.

An earlier incident is a good way to introduce Tan Ajahn Anan. It was
the middle of my second Rains Retreat, and I was staying for the first
time at a Wat Pah Pong branch monastery where everyone except myself and
another English-speaking monk was Thai. During one evening meditation
session I was concerned about myself. I was stuck in a negative mood
that just wouldn't move. All the other monks appeared so sweet and kind,
and I was sitting there being angry, thinking about what was wrong with
everyone and everything: `Maybe everyone else's moods arise and pass
away, but maybe mine won't! Maybe I'm just too defiled to be a monk.' I
faintly knew all these thoughts were silly, yet somehow I couldn't stop
them and it was very uncomfortable. The Rains Retreat can be a tense or
difficult period, and most monks will experience some kind of negativity
at some time or other during those three months. All this I knew, but
this particular time I couldn't help making a big deal out my negative
mood.

After the evening sitting the bell was rung routinely and we chanted in
Pali and Thai, sharing merit and taking refuge, bowing our respects to
the Buddha and our teacher. As the monks left the room I lingered a
little, wanting to be alone. When I left a few minutes later, passing
through the rear door, I noticed that Tan Ajahn Anan was seated alone on
a wooden dais in the corner of the balcony surrounding the hall.
Intuitively I lowered to my knees and respectfully approached. With a
characteristic firmness that was also caring, he looked at me, raising
his chin in acknowledgement, and asked, as if he had been waiting,
`\emph{Achalo, tam-mai mai sa-ngop ley?}' (Why are you not peaceful at
all?) Defeated, I couldn't actually articulate anything, but I knew he
was not angry at me; he was not demanding that I answer. After what
seemed like a very long but exquisitely empathic moment, he said to me
in carefully practised English, `You think a lot about Australia', a
statement and a question to which I replied, `Yes it's true.' Then he
continued, `When you think about the future you throw your mindfulness
in the dirt.' Another pause. `When mindfulness is not strong you cannot
let go of \emph{arom}.' Then we talked a little more about how I'd been
feeling and what to do about it. The Ajahn explained that the
mindfulness of a good meditator is usually a nice clean white colour,
while the mindfulness of an enlightened being is radiantly clear and
impenetrable like a diamond. But when a person is lost in some kind of
delusion, mindfulness degenerates to the colour of dirt or mud. After
allowing me some time to describe my difficulties of late, this kind and
wise teacher assured me that it was ordinary for young monks to
experience such things. He then encouraged me very gently to keep to
myself for a few days, try to eat lightly and refrain as much as
possible from thinking about the future, letting mindfulness
re-establish itself.

It was true I had been obsessing about the future, a transgression in
the Buddhist sense if ever there was one, particularly for someone who
has made a firm resolve to cultivate awareness in the present. But I'd
been caught off guard when just a month before I'd found that there was
an opportunity to go to Australia and that my family was eager to help
with the ticket. I had previously assumed that it would be a few more
years before I returned to Australia. As a young monk still learning the
ropes, I would ideally choose to stay within the most clear and
supportive of contexts. However, the opportunity to return to my country
with one of my teachers seemed a blessing too providential to refuse.
Now that there was an upcoming travel date to fuel proliferation, I had
got lost in remembering, fantasizing and planning. Certainly there were
things to think about. I would have to prepare my mind for the change of
locations and consider what rules of discipline to brush up on, so that
I could maintain my practices while travelling. I would also have to
consider my relationship with the members of my family. How should I
relate to them? Over the years I'd been fairly diligent in maintaining
correspondence with my family. Basically they had been supportive of my
choice to be a monk. My mother gave me her blessing and encouragement
before I'd even asked for it. I recollected with affection the words she
spoke to me before I last left Australia: `You seem very happy and the
monastery sounds like a safe place. I always knew you'd find what you
needed to do with your life. I never knew what you should do, that's why
I used to worry about you so much! But I knew you'd find your niche. If
you want to become a monk you have my blessing.' Indeed I had the
blessing of my parents, for they were happy that I seemed well, but at
the same time I sensed an absence of true empathic appreciation. They
appeared to be unconcerned about spiritual matters, and even though they
were supportive, there was always, perhaps reasonably enough, the
lament: `We're glad that you're happy, but we miss you and it's sad that
you have to live so far away.'

At Wat Pah Nanachat a few days before our departure I was meditating
alone in the main \emph{sala}. I had been busy getting things together
for my trip, and had decided to take a few quiet moments to collect my
mind. When I opened my eyes, a layman approached and said that he wanted
to offer me the Buddha statue placed by his side. Saying that the Abbot
would be along in ten minutes, I suggested that he wait and offer it to
him. He went away for a minute, but then came back and said that he
wished to offer it to me, so this time I received it happily.
Ironically, this small figure was in the same standing posture as the
large Buddha statue in the ceremonial hall where I'd been ordained. I
remembered the morning of my ordination in pre-dawn darkness, looking up
at an ominous black figure standing with both hands held in the posture
named `holding back the waters'. His arms and hands were straight and
taut, with palms exposed in a gesture that seemed to be making a firm
command. At that time I had seen the gesture as an emanation of
assertive compassion, compelling me to see the importance and urgency of
my opportunities. The figure I held in my hands was a more androgynous
and serene one. With both palms held out, he seemed to be saying
something more along the lines of, `Truly I come in absolute peace'.
When I showed it to Tan Ajahn Jayasāro, he suggested that I let my
parents be its new owners. As I looked at the statue's expression, I had
a feeling that my mother would adore it but I wasn't quite sure. It was
definitely a religious object, and, as my parents do not profess any
faith they might in fact find it objectionable. But I decided to trust
my feeling and take it anyway.

On the day of our departure in Bangkok, at the airport moving towards
the check-in counter, a robust short-haired woman was vigorously
employing her feet to kick and push her backpack along the ground, while
heaving another large bag in her arms. I was standing in the queue
alongside her with my teacher and three other senior monks. They were
quiet and composed in manner. The young woman then began arguing with
the delicately mannered Thai assistant behind the counter, in an English
that screamed `Australia!' `This is the bag I want to take onto the
plane.' `I'm sorry, but it's too big', came the quiet and polite reply.
The passenger continued, obstinate and confrontational, `But they let me
take it last time! They did! This same bag! They let me take it last
time!' After several more imploring but quiet pleas, the check-in
assistant gave in and allowed her to keep the large bag. As a rule Thai
people abhor public conflict. At once I felt deeply embarrassed and a
terrible sense of trepidation about our journey. The incident reminded
me of some of the defects in the character of my countrymen. As I looked
to my right I could see a row of monks who had come to bid farewell to
their teacher. Their robes were immaculate, their countenance serene;
they were sitting together in a neat row, talking quietly among
themselves. The contrast created an uneasy tension. I shouldn't have
taken it so personally. I should just have accepted things as they were,
and taken solace in the equanimity and wisdom of the other monks and my
teacher. But as the only Australian in our group, I felt somehow
responsible. We would be flying economy class on Qantas, the Australian
national carrier, and I couldn't help worrying that the plane would be
full of poorly mannered drunken people who would be rude to my fellow
monks. The ground staff were organising a first-class classification for
our luggage and a security escort onto the plane, and I was acutely
aware then of the grace, humility and kindness of so many of the Thais.

We were the last passengers to enter the plane, on stopover from London.
As we were led through the aisle to our seats, I was struck by the sheer
number of big-bodied white people. At the same time I was relieved to
see that for the most part they appeared personable and harmless enough.
As my anxiety level lowered, I forgivingly remembered that for the most
part that is the way Australians are. After sitting down I noticed this
was a fascinating realm, caught in limbo, but which in many respects
could already be considered part of Australia. The air was thick with a
familiar slouching accent and hearty laughter. The airline staff walked
swiftly, confidently and heavily up and down the aisles, stopping
occasionally to receive loud orders or deliver food and drink with an
equally enthusiastic command to enjoy. I must admit I really did enjoy
the spectacle, and was surprised by the amount of affection I felt for
everyone inside the plane.

In the aircraft my mind at last felt clear and it was easier to feel
prepared for the impending change of situation. After four or so hours
in the air, the Thai monk next to me said that he would really like to
meet the captain and have a look at the cockpit. He'd seen pictures
before but never the real thing. Seeing the eager look on his face, I
knew I would have to ask one of the cabin crew. `Why not?' I thought,
`they might just say yes', and so, feeling stupid, I asked. Twenty
minutes later there were three Buddhist monks in the cockpit and the one
by my side was positively delighted. Rolling 180 degrees before us were
scores of magical billowy cumulus clouds, illuminated by the clear blue
sky of mid-afternoon. Eager to know how the pilots were trained, the
senior monk asked many questions and we were both impressed by their
impeccable discipline. The monks drew parallels between the discipline
of the pilots and our own training. Much to my surprise, the co-pilot
then confessed that he thought he was a Buddhist. Our visit to the
cockpit was timely, as the captain announced that in a few minutes we
would come to the north-western corner of the Australian continent.
Feeling sad for the other two monks, still on the lower level, I asked
if it would be possible to send for them. My friends departed,
satisfied, and a few minutes later my teacher and another monk were with
us in the cockpit. As we approached land I was surprised by an
unfamiliar burst of patriotism, as if this really was my country.
Through the many windows the view was fantastically clear. A delicate
white strip of sand gracefully traced the coastline of the semi-arid
ochre-coloured earth, defining it as separate from the huge expanse of
ocean directly below. Before us the continent seemed endless, and once
over land I was emotional for a different reason. Years ago I had left
this country seeking a lifestyle with purpose and an authentic spiritual
training. Now I had returned physically, and as I reflected that one of
my much-revered teachers was sitting just a little to my left, I felt
exultant at such good fortune.

On arriving in Sydney, I was met by my younger by eighteen months
brother, and it was weird to look into a face so similar to mine. He was
easy in manner and happy to see me, and I felt the same. Over the
following few days I had the opportunity to spend some time with both my
younger brother and elder sister, who came to visit me at the Thai
temple and also took me on a picnic. The talk was mostly about jobs and
careers, politics and current affairs. We all felt affection and joy at
being able to be with one another after such a long time. Upon returning
to the wat however, I felt a little scattered, and was grateful to
participate in the evening chanting and meditation with the resident
monks and a few local Thais. Throughout the entire trip the visiting
monks were surrounded by the warmth and courtesy of faithful Thai
Australians, and were taken on many sight-seeing tours by their eager
hosts. In Sydney my Dhamma brothers were as impressed by the clean and
free-flowing roadways and the large suburban houses as they were by the
Opera House,and the deep harbour and its famous old bridge. Although
there were no particular teaching engagements in Sydney, the Thai
community truly relished the presence of Tan Ajahn Anan. Some asked
questions about meditation practice or inquired about his monastery and
our lineage, and several people subsequently made plans to visit and
spend some time practising back in Thailand.

From Sydney we drove to Canberra and all, including myself, were
awestruck at the vast expanses of pasture lands so sparsely inhabited.
Heavy spring showers had turned the land a gorgeous rich green, with
large patches of lilac wild flowers. Staying in a Thai monastery once
again, we were pleased to see that the Wat in Canberra was very well
supported. Each evening a good number of laypeople came to chant,
meditate and listen to either a taped Dhamma talk or a teaching by one
of the Ajahns. Ajahn Anan gave a teaching on the second evening.

Having been away from Australia for some years, and then returning
surrounded by Thais, I felt I could contemplate aspects of the country
of my birth with a greater degree of detachment, asking myself such
questions as what nationality actually meant. What did it mean to be
Australian and what were the core values of this culture? I wasn't
looking for answers so much as stimulating reflection. In Canberra, the
nation's capital, the War Museum and Parliament House were particularly
thought-provoking. In Parliament House, walking through vast rooms and
corridors, I wondering whether I was really close to the heart of
Australia, in this place that had cost over a billion (Australian)
dollars to build. I was certainly grateful to such an abundant and
well-organized society, and all the opportunities it had made available
to me. But somehow I felt unmoved. One notable comment by the other
English-speaking monk seemed to capture something of my sentiments: `The
design seems to be full of complicated patterns, shapes and angles, but
it doesn't seem to be going anywhere or pointing to anything. Like
cleverness just for the sake of cleverness.' Admittedly, sitting in the
Senate Room and House of Representatives with my teacher was certainly
fun, and there is one particular image that I will always remember.
There was a curving corridor; large panes of plate glass looked over a
perfectly tended garden lawn. Above the mirror-like granite floor hung
an impressively lofty roof, emanating soft, even golden light. A huge
hand-woven woollen rug with a bold geometric design of black with reds
and yellows lay beneath, between two large dark leather sofas. On one of
these large sofas I and a Thai layman sat, and sitting alone on the
other was my teacher. Amid such contrived beauty, cleanliness and order,
the look on my teacher's face was probably that of equanimity, yet that
word somehow doesn't capture it -- cool, relaxed, detached and unmoved.,
But the air around him was not a vacuum. He seemed in fact to be making
a pronouncement, saying something about where to place one's efforts and
attention in order to know what is most worthy of reverence.

In Melbourne we were received by members of the Buddhist Society of
Victoria. It was an inspiring sight to see a room full of practising
Buddhists of many different ethnic origins. Here in the evenings Tan
Ajahn gave several Dhamma teachings followed by questions and answers,
all of which were translated. Many practitioners were keen to ask
specific questions about their meditation practice, and Tan Ajahn was
pleased with their energy and willingness to learn. People stayed for
hours listening to the answers to questions and all the evenings ended
quite late. It was a pleasure to be practising the dhammas of listening
and meditating among sincere Australians. As most of my Dhamma
experience is associated with Thailand, these occasions helped me to
feel more the global relevance of the Dhamma, and less isolated from
where I had begun my life. Tan Ajahn instructed me that I must practise
very hard to develop my skills well, so that I could come back and truly
help these people. Although I was flattered by his faith in me, I was
more than a little daunted by the prospect.

After four days in Melbourne it was time for me to separate from my
teacher. He and the others would stay on there for a while and then fly
to our branch monastery in Perth. I would go alone to Sydney, travelling
to the Blue Mountains to spend a few days with my eldest sister and her
children. From there I would go back to Sydney to meet one of the monks
and we would fly to Brisbane together. Having a friend who was a monk as
my companion would be a valuable support, and I felt lucky that he had
been willing to meet up with me again. After performing the ceremony for
asking forgiveness, I asked Tan Ajahn Anan if he had any advice. He
simply answered that I should take care of my heart and mind. We were
drinking afternoon tea when a kind Sri Lankan doctor arrived to pick me
up. As I drank the last few mouthfuls of tea, I became aware that I was
suddenly feeling very vulnerable. Moments later I noticed a distinct
lightness in my heart. The feeling grew, my face smiled broadly as if of
its own will and I was left feeling supported and safe. I realized that
my teacher was radiating loving-kindness toward me, a highly developed
ability for which Tan Ajahn Anan is renowned.

Away from my teacher and the Thais who surrounded him like a comforting
universe, it was interesting to see that I did feel a little less safe
and confident. However, having a few days alone in Sydney was relevant
to my experience of going home. It was like going on a pilgrimage into
my past. As a young adult these streets were the backdrop of my life,
and I was keen to walk them alone, to observe my thoughts and notice if
there had been any changes. With nothing but my robe and the underground
ticket offered by a lay supporter, I walked down the main street of the
central business and shopping district. Observing people's faces, I
could remember the way I used to live my life and the thoughts I used to
think. Most people looked physically healthy but distracted and tense,
in a hurry. While walking along the street I was determined to think
thoughts of kindness and keep in mind my faith in the Buddha's teaching.
Although I was a little fearful, it was interesting to see that among
all the varied inner-city subcultures, there were very few comments,
sneers or noticeable reactions to my presence. I had wondered if people
would move away or be uncomfortable standing close to me at traffic
lights or while waiting on the train platform. Repeatedly people stopped
close by and seemed quite at ease, so much so that my faith in the human
capacity to sense a commitment to harmlessness grew. Towards the end of
my urban pilgrimage, an elderly Aboriginal woman caught my attention.
She was sitting on a park bench outside the Town Hall, by the entrance
to the underground. As I walked towards the stairs, she looked me
briefly in the eyes, then raised her joined hands to her forehead in the
traditional Buddhist gesture of respect. I felt as though I'd been
blessed by a true native elder. This incident touched me profoundly, as
it was so completely unexpected. I looked her in the eyes with
tremendous appreciation, and I'm sure she began to blush.

Visiting my eldest sister in the Blue Mountains entailed a significant
shift in modes of relating. I was glad that I'd had a few weeks to
adjust to being busier and talking more. It was also interesting that
for this leg of the journey I was not accompanied by any other monks. My
sister's ten-year-old son fulfilled the Vinaya requirement for the
presence of another male, creating a lot of space for spending very
natural time talking in her home. Before living overseas I had been
especially close to this sister. Dianne trained as a nurse and a
midwife, and had also travelled through Asia as a backpacker in her
mid-twenties. With the arrival of three delightful children she had
ceased caring for other people's babies to pay attention to her own. She
has a love of good natural food, yoga, alternative medicine and Asian
arts. We always got on well together. After living for a year or so in
Thailand, my sister fell out with her husband and a long and painful
separation ensued, complete with custody battles and ugly court cases.
That time in Thailand was difficult for me too. I wanted to be
physically and emotionally present for my sister and her children.
Feelings of love and guilt often tugged at my heart. I'm quite sure that
had the difficulties begun before my departure, I wouldn't have left
such a situation. As it happened, though, I'd relinquished all my
possessions and was already wearing the \emph{sāmanera's} robe when the
news reached me.

Seeing Dianne waiting on the train platform with her beautiful
fairy-like three-year-old daughter I was relieved to notice that she was
radiant with life. I still remember her first few words: `God look at
you! You look great\ldots{} I suppose I can't give you a hug? \ldots{}
How strange! \ldots{} Oh well, doesn't matter! \ldots{} Wow it's great
to see you \ldots{} The colour of your robe is beautiful, so earthy and
natural.' When I explained to her that the dye was handmade from
boiled-down heartwood, she was overjoyed and started telling me how she
had begun taking African drumming classes, and that the drums were
carved from the heartwood of big old mango trees. Later she admired the
handiwork of my crochet bowl cover and hand-made monk's umbrella. Things
settled into a new kind of order. My sister and her children were quite
well, indeed, all having a lot of fun. The new house they were renting
was perched on a small hill which backed directly onto an expansive area
of undulating ranges, carpeted in a reserve of native Australian bush.
The forest was a beautiful backdrop to our many long conversations. It
was a relief to be able finally to listen, to hear the many details of
the past years of struggle, to encourage her in maintaining a generous
heart and in learning to forgive, a process in which she was already
well established. She and her children performed a show of tribal
rhythms and dance on their drums, and Dianne later played some songs on
the piano, the very songs she used to rehearse in our family home and to
which I would wake up as a five-year-old all those years ago. Admittedly
these were strange activities to be participating in as a monk, yet in
such a context they seemed harmless enough.

I also visited my niece's and nephew's school classes and gave a talk
about the lifestyle of forest monks. Happily, the school children were
attracted by the forest monk uncle, asking many questions about my daily
routine, meditation and Thai culture. Some curious questions were put to
me: `Can you make yourself float up in the air?' `I know you can't eat
at night, but if you're really hungry can't you just have some crackers
or something?' They were amazed when I explained that I didn't have a
refrigerator or cupboard, or even electricity. On both occasions I left
classrooms hearing such exclamations as: `That's what I'm gonna do. When
I grow up, I'm gonna be a monk!' I had intended these talks to be a gift
to my young niece and nephew, to show them that although I lived far
away, I still cared for them. There had been some reservations in my
mind, however, as I was not sure that such exotic spirituality would be
appreciated by all. The fears were unfounded. When I asked my nephew
whether any of the kids thought I was weird, he answered, `Nuh \ldots{}
They all thought you were cool!'

A monkcompanion met me for the flight between Sydney and Brisbane, which
was probably the tensest time during my travels in Australia. When, a
person is propelled several hundred miles in a few minutes and thousands
of feet above the ground from where a life was recently being conducted,
I expect it is normal to become very circumspect. I thought a lot during
that flight. Now that after so many phone-calls I would really be
meeting my parents and again staying in their home, the nagging little
thoughts which had been lingering in the recesses of the mind came
clearly into view and expressed themselves: `I do live a long way away
\ldots{} A son being a Buddhist monk would be challenging to most
conservative parents \ldots{} Maybe they've been pretending to be
supportive out of fear of estrangement \ldots{} Maybe the truth is that
they have many reservations, and once I am again within their sphere of
influence they will be angry or possessive.' As I mused over the
possibilities, I laughed to reflect that they probably had uneasy
thoughts themselves: `He might be demanding or difficult to take care of
\ldots{} Maybe he will try to convert us! \ldots{} Maybe he'll just want
to meditate and be by himself all the time.' Remembering the small
Buddha statue, I recollected his message and reminded myself that I was
also coming in peace. Foremost in my mind was that we should all have a
pleasant and relaxed time, to be remembered happily. I had already
decided not to try to teach my parents anything about Buddhist
spirituality unless they asked out of their own interest. I hoped this
attitude would make them feel open and unthreatened. But even with such
a sure game-plan, the nervousness persisted. As we started to descend, I
decided that there was nothing to do but let go of wanting to control
and be with the anxious, gurgly feelings in the stomach. It would be how
it would be, I'd done everything I could to try to set things up right.
On landing I asked my friend to wait a minute so that we could leave
last and take things slowly. Walking down the connecting corridor
between aircraft and terminal I was embraced by an amazing sense of
familiarity. The weather in the other cities we'd visited recently had
been surprisingly cool to our seasoned tropical bodies. Brisbane, an
hour south of my parents' house, is notably subtropical. It is the
region where I spent my childhood, and though I could not say exactly
what they were, the very air seemed pungent with familiar smells.

Reaching the end of the corridor and seeing my mother and father, I
noticed that they looked older than I remembered, but I was glad to see
such big smiles and joyful light streaming from their eyes. My father
moved forward as though he were about to embrace me. Slowing a little, I
looked at him firmly --- they had promised to save their hugs for once
we got home. As I took the last few deliberate steps, he responded to my
hesitation with a hearty, `You're my son, I haven't seen you for a long
time and I'm gonna give you a hug.' In his strong embrace I felt
humbled, childlike, with a disarming kind of happiness. Concerned
thoughts whispered in the back of my mind: `He hasn't honoured my wishes
to be discreet and restrained. He's not going to respect the boundaries
that a monk needs to maintain.' Gently I pushed the thoughts aside. For
a sixty-year-old Caucasian man to publicly hug another man,
shaven-headed and clad in religious robes, seemed a gesture too
beautiful to be censured. There was also a feeling of acceptance. The
boldness of his affection signalled that he had indeed respected my
decisions. By his side, my mother smiled a little awkwardly but didn't
express any demands for physical affection. I introduced them both to
Tan Neng, the monk who had accompanied me, and then presented my mother
with the gift which I had carefully wrapped myself. She gazed at the
black and gold paper, testing the weight in her hand and I caught a
sparkle in her eye --- my mother always loved gifts! My father insisted
on helping with the bags, saying: `When you didn't come with the others,
I thought, ``Uh-oh! He's missed the plane''\ldots{} When we saw you
walking down you did look very peaceful. You look healthy but you seem
to have lost some weight.'

In the car my mother opened her gift. `She's beautiful', she exclaimed,
then kissed the statue and held it to her breast. I explained that it
was an image of the Lord Buddha, and that it would probably be best if
she didn't kiss it. My mother promised that she would find a nice place
for it in the living room. In the company of a Thai \emph{bikkhhu} I was
embarrassed by my parents' lack of familiarity with Buddhist customs,
but happy that they appreciated the gift. Things went well at my
parents' house. Although Tan Neng and I continued to eat our main meal
from our alms bowls alone on the veranda, Ajahn Jayasāro had given
special permission to have breakfast with my parents, to enjoy a casual
time for chatting. There were many simple joys to rediscover: looking at
childhood photographs, eating favourite foods specially home-cooked,
walking through the local forest reserves. And after discussing at
length what would be a most suitable and practical gift in the future, I
drew a pattern and taught my mother how to sew the monk's lower robe.

One day, after about a week or so with my parents, Tan Neng and I spent
the day at a meditation centre in the Tibetan tradition and my father
came to pick us up. Much to my astonishment, a little more than halfway
home, I caught myself speaking to my father in a sharp and angry voice.
He'd said something about some of my habits in adolescence, and an
unexpected ball of fire began flaring in my abdomen. Do parents have a
special gift for poking the most sensitive spots in their children's
hearts? I implored myself to abide with the feeling and contain it
skilfully, hoping that my father wouldn't say anything else to
exacerbate the situation, but inevitably he did and out flew my
grievances. In his eyes I had been in some ways an obnoxious teenage
son. To me he had appeared in some respects to be a distant man who
could have paid his children a little more attention. Was there any
point to discussing these things after so long? Sitting in the strained
silence afterwards, I was sure the rest of my stay would be tainted by
this unfortunate outburst and couldn't help feeling sad.

Returning to Thailand for a moment, in the forest monasteries one finds
oneself with many hours to consider things deeply, to investigate
affections and disaffections and the layers and aspects of attachments
as they present themselves. I must have asked myself a hundred times
what I missed about Australia, and was frequently surprised to find that
except for a few close relationships, I didn't really miss much. It
seems that once we've firmly established new habits and a new lifestyle,
old memories and old habits slowly fade, and unless deliberately
recollected become increasingly distant. But one memory that would
always assert itself hauntingly was that of the beach and the ocean. I
spent my entire childhood and early adult life within a short distance
of the deep blue Pacific. Along the entire east coast, from spring
through autumn the water is cool but not really cold, and when people go
to the beach it may be for a half-day or indeed an entire day. A day at
the beach is an Australian institution, and if the sun is shining there
are always as many people in the water as there are on the sand. The
surf of the east coast had received me in its waters on thousands of
occasions over many years, propelling me between waves, pummelling my
body or allowing me to float out past the breakers, stilled by the sheer
immensity below, around and beyond. The ocean seemed to possess an
awesome power, being capable of empathizing with anything that I could
feel, offering a watery cushion between myself and the demands of living
back in the shore-bound world. Once I entered the sea, hours could go by
before I returned from this watery other-world. I was never a surfer,
but for as long as I can remember I was a swimmer. Sometimes in the dry
or hot seasons in Thailand my body would lament and grieve, aching for
the temporary respite that can be had in the Pacific.

During my trip to Australia I had several opportunities to visit the
ocean. One time in particular was especially magical. It was the weekend
and I was walking along the beach with my father. The terrible feelings
from the incident in the car seemed to have evaporated. Surprisingly,
what had appeared to be a disaster had triggered a warm opening in my
father's heart. Perhaps it was wisdom coming from age, or perhaps it was
my long absence and impending departure that stimulated a kind of
urgency in him. His past habit would have been to talk little if at all
about such tension, but this time he wanted to discuss things, with an
interest in the details. What ensued were several beautiful
conversations and an opportunity to learn many unknown things about one
another. On this fantastically bright day my father was wearing his
swimming trunks, I was wearing my robe and my twin brother had lent me
his sunglasses. We walked for many minutes along an endless beach, and
then turned round to enjoy the long walk back. The smell of the salt
air, the tumbling roar of the waves and my father by my side engendered
an exquisite sense of reunion. Our discussions led us to the place where
I could lovingly acknowledge my father's honourable qualities. Walking
over the silk-like sand we discussed each other's good qualities, with
the sky above a clear and brilliant blue and the ocean beneath it
darker, rich and deep. Hearing about my father's own experiences with
his father and teachers, I felt sure he had done a better job than the
examples given him. Learning about his past opened up a space in my mind
which placed him in a larger context. Aware of my pain and his pain, I
experienced a vast sense of empathy and wondered if there were ever a
father and son who did not feel such things. As we approached our spot
on the sand, my sun-bronzed twin brother who had driven up for the
weekend waved from his beach towel, pleased to see us together.

The remaining few days went by easily. On the last day we performed a
small ceremony where my parents formally offered me a new lower robe. I
had drawn the pattern, my mother had sewn the fabric and my father had
helped with the dyeing. On the way to the airport we visited the
memorial where my grandmother's cremated remains were stored and I
recited some funeral chants. I had been unable to come to her funeral,
so this was a kind of symbolic goodbye and it felt good to be able to
offer the chanting.

Once again in an airport, another neat expensive new building of
well-considered design, sitting in large comfortable chairs arranged
around a coffee table to give the sense of lounge-room intimacy, my
family and I drank some `goodbye espresso'. Brisbane Airport is a large
cube-like open space, with veritable walls of glass and huge skylights
which nourish the fully grown palm trees that towered around us inside.
The polished marble and chrome and the sophisticated displays of the
duty-free shops gave the place the air more of a modern art gallery than
a transit centre. The monk who had accompanied me earlier had left for
Bangkok a week before. After five weeks in Australia I felt markedly out
of place sitting there. My father and eldest brother were talking about
car insurance, my twin brother was exploring the options for
rust-proofing his new four-wheel drive, while my sister-in-law was
complaining that her work was boring. Nervous and self-conscious, I
reacted to my family's discussions by feeling impatient. I was guilty of
indignation and could fairly have been accused of having judgemental and
critical thoughts. At this and other times during the preceding weeks I
had felt exasperated by the nature of some of the conversation from the
members of my family. It would have been nice had that not been so, as
that would have been most comfortable. But there were times when I was
confounded by what appeared to me to be my family's disinterest in
engaging their experience of life with integrity. I had to remind myself
several times to see things in a suitable context.

As a member of a community of dedicated contemplatives, I had begun to
take the qualities of meditation and frank investigation for granted. If
you were to define a religious life by the qualities of honesty,
morality and generosity, then the members of my family are in fact quite
religious. I had to remember simply to be grateful for the Sangha back
in Thailand, to recognize that they were people dedicated to spiritual
cultivation and to respect the members of my family for the decent
people that they are. I should add, however, that at times my critical
thoughts were not based in negativity. As an observer I could see that
some of my family's habits were bringing them more pain and danger than
happiness. When these came up in conversation I wanted to interject.
More often than not, however, I bit my tongue, as I know that people
must feel a lot of trust and be in an open, willing frame of mind before
criticisms can be skilfully received.

Now at the airport, my frustrations were running high. Within minutes I
would have to leave them once again, and somehow it seemed important to
say something of integrity. Deciding there was little to lose, I
interrupted the flow and announced that I needed to say something:
`Mother, father, thank you very much for looking after me and making my
visit comfortable and enjoyable. I really am grateful and impressed with
the way you made so much effort and took so much time for me. If there's
anything that I said or did that was inappropriate, please forgive me.
And if there was anything that I didn't say or do which I should have,
please forgive that also.' `No, it was great having you here, mate!'
said my father, while my mother looked at me in a moment of
uncharacteristically intense concentration, and said all of a sudden,
`And will you please forgive me?' The look in her eyes was surprisingly
emotional. As I expect is the case for everyone, my mother and I have
had our share of confusion and pain. As an adult and as a meditator, one
who has had ample time to notice his own shortcomings and human
fallibility more and more, I have grown to appreciate both my parents
more and more and to feel very much at peace with them. The past ten or
so days had been very pleasant and it seemed there was no particular
`thing' to forgive. But I felt I knew somehow what she was saying. When
separated from loved ones by large distances and time, it is normal to
ask oneself such questions as, `Did I do something wrong for them to
want to stay away?'; or to have worrying thoughts like, `Maybe I
shouldn't have said those things in the past', or `Maybe I should have
done more or given more.' These thoughts had occasionally been in my
mind in Thailand. Being close to each other and allowing the worries a
moment of acknowledgement seemed important. It was with earnest
sincerity and much relief that I could look into her caring eyes and
truthfully say, `I forgive you.' I could have said much more, but that
phrase seemed to say just enough.

Both my parents are over sixty years old now and they are in some
respects creatures of habit. We have discussed the possibility of their
coming to visit me in Thailand, and although they try to sound willing,
I sense that perhaps they're just not that adventurous. Sadly, it may
once again be years before I see them. Indeed, as the Buddha encourages
us to consider, one or all of us may pass away, so this might have been
our last opportunity to see each other. It may sound precious or
excessive, but it was important to give them these messages and to let
them know, `I really am okay with you people. I care for you, and when I
think of you it will be kindly.' The sudden shift to frankness and
resolution left me feeling a little bewildered. And then it was time to
go. My father followed me to the escalator and insisted upon another
hug. Alone with him at the top of the escalator, I felt something very
consciously for the first time: that I was a man just as he was a man.
Indeed, I actually stood a little taller than he. There was an uncanny
sense of mutual respect and separateness, yet at the same time I felt
closer to him than I ever had before.

Going down the escalator and through customs wearing my new lower robe,
I was quite astonished by the sense that things had gone wonderfully,
that in fact I couldn't have hoped for a more positive visit. There were
many events that I could later share with my teachers and the other
monks. Tan Ajahn Anan had been very supportive of this visit with my
family. I would tell him how my mother had decided to offer the Buddha
statue flowers fresh from her own garden whenever she was thinking about
me, so that she could feel that she was giving something instead of just
worrying.

Once a person has taken the mendicant's vows, the question of exactly
where home is becomes primary. For me, though, I suppose home is the
place that I miss most. After five weeks in Australia, my years in
Thailand had begun to feel like a short vacation I'd had years ago.
Although I'd had many pleasant experiences and met many good people, I
was missing the quiet of the forest monasteries. I missed my Dhamma
brothers and the simplicity and clarity that come from monastic routine.
After having such a pleasant time with the members of my family, I
anticipated that perhaps upon returning to Thailand I would feel even
more at ease. After coming home, I felt that I could once again go home.

\emph{About the Author}

\emph{As well as living in Thailand, Tan Acalo went on to spend time
with Tan Ajahn Pasanno at the then recently-established Abhayagiri
Monastery in California, and with Luang Por Sumedho at Amaravati
Monastery in England. Following his fifth Rains he went to live in Wat
Marb Jan, to continue his training with Ajahn Anan for a few years.
After more time in the UK and then in Melbourne, where he spent the
Vassa with Ajahn Kalyano, he returned to Thailand, where following the
2010 Rains Retreat he was invited to take up residence on a piece of
land in the beautiful hills of Petchaboon Province, starting Ānandagiri
Forest Monastery which nestles at a meeting point between central,
northern and north-east Thailand.}

