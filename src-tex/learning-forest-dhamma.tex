% Title: Learning Forest Dhamma
% Author: Paññāvuddho Bhikkhu

\section{Alone with other creatures}

\begin{quote}
\small
`As long as the monks see their own benefit in wilderness
dwellings, their growth can be expected, not their decline.'
\quoteref{D.II.77; A.IV.20%
\footnote{Bhikkhu Aparihaniya-Dhamma, translated by Ṭhānissaro Bhikkhu.}
}
\end{quote}

Emerging from meditation while nestled in an isolated spot deep within
the folds of the forest certainly makes a wonderful way to greet the
day. As I sit here on my \emph{dtieng} in Dtao Dam, the crescendo of the
birds and insects celebrating the crack of dawn has subsided, and shafts
of light begin to seep through the trees. The beams of sunshine from the
early morning sun rising over the mountainous horizon produce a dance of
light throughout the trees and imbue the forest air with a glowing
yellow-orange hue. 

Here in the lush, tropical forests of Dtao Dam I often catch myself
marvelling at the degree to which the forest brims with life. The
natural surroundings pulsate with a vibrant energy. From the gurgling
flow of water in the creek, to the bass undertone of humming bees, to
the punctuated, high-pitched cries of barking deer, the forest provides
a constant symphonic medley of sound. At dawn the energy peaks; while
nocturnal creatures return to their abodes, the rest of the fauna awaken
with the sun to embark on a search for a new day's meal. A little bird
perched just on the front ledge of my \emph{dtieng} sings a small song. 
Then with a sudden quiver of its wings it slices through the
timelessness of a moment. I note how a detached observation of nature's
rhythmic movements brings peace, ease and a sense of release to the
ceaselessly thinking mind. In my Dhamma practice the pristine wilderness
inspires me to turn within and attempt to attune to that which is true. 
The whole environment encourages me to develop a deeper awareness of the
present moment and practise letting go. 

Suddenly, while sitting on my \emph{dtieng}, I hear a commotion in the
brush several metres away. The noises are insistent in their beckoning. 
Noting a subtle ripple of sensation from my abdomen up to the crown of
my head, I anticipate the onset of some mental proliferation. I make an
effort to bring attention to the breath to re-establish mindfulness in
the present moment, but the mind doesn't cooperate. Instead the sounds
trigger a reminiscence of a similar occasion a year ago. Immediately I
feel myself reliving the scene. 

I have just completed my early morning meditations, concluding with a
chant of the \emph{Mettā Sutta}, the Buddha's discourse on
loving-kindness for all beings. I check my clock and discover that I am
a little late for the descent down the mountain to take the daily meal. 
But this is my favourite part of the day and I want to soak up a bit
more meditation in the radiant morning sun. The fresh dew-drops on the
abundant flora indicate the cool, crisp moistness of the air, and I feel
quite content to remain sitting here, comfortably tucked away in my
robes. 

I re-close my eyes to sit in meditation for a couple more minutes, but a
sudden crashing sound in the brush competes for my attention. To hear
sounds, even loud sounds, in the forest is so common that I tell myself
not to take note and to return to the meditation. But as the volume and
frequency of the fracas in the brush increase, I can't refrain from
speculation. I realize that whatever is going on is coming closer. 
Unable to control my curiosity, I open my eyes and slowly rotate my neck
to look over my left shoulder. It takes my vision a moment to focus. The
greenness of the forest all seems to melt together like an Impressionist
painting. And then I see it. The form of an animal is darting to and fro
between clusters of bamboo and through the dense underbrush.

I make out
the shape of a large sandy-beige feline-like form, about human size, 
jumping up and down and zipping left and right in an almost playful
fashion. Two butterflies flutter their wings over its head. Is it
chasing butterflies or following some ground animal through the brush? I
can't see. Everything is taking place so quickly. But whatever is
happening, this feline creature apparently doesn't see or smell me and
is on a collision course with my \emph{dtieng}. Not really thinking of
anything, I find myself making a noise by gently but audibly clearing my
throat. The large cat instantly stops jumping around, ducks low behind
some tall grass and shrubs, then shoots clear out of sight. It had been
only three or four metres from my \emph{dtieng}. 

\section{Finding resolve in the forest}

The proverbial forest monk story revolves around the encounter with a
tiger. As the fiercest predator to be found in the wild, this large
flesh-eating character unquestionably rules the jungles of Southeast
Asia. Tigers notoriously reek with the smell of death on their breath, 
usually strong from a recent kill. Yet interestingly, although there are
countless documented instances of forest monks meeting up with tigers in
recent recorded Thai history, there is not one single known case of any
forest monk being killed by one of these beings. I had just met up with
a relative of one of these regal beasts, (later I discovered the animal
I saw was probably an Asian Golden Cat, not a properly striped
full-grown Bengal Tiger), and fortunately I did not become the first
victim on the list. 

Now a year later, as I sit at the foot of a giant tree at a new spot in
the same Dtao Dam forest recollecting this encounter, I contemplate why
fear did not arise. Why do I feel so at home in this seemingly wild and
uncontrollable environment? Practising in the forest in accordance with
a forest monastic tradition dating back to the time of the Buddha, I get
a gut sense of the authenticity of this form of training, although it is
a far cry from my upbringing and education in America. For a forest monk
there are a bare-bones honesty and naked simplicity to daily life. 
Everything is a teacher. Every moment is geared toward awakening. Ajahn
Chah points the way: 

`Whether a tree, a mountain or an animal, it's all Dhamma, everything is
Dhamma. Where is this Dhamma? Speaking simply, that which is not Dhamma
doesn't exist. Dhamma is Nature. This is called the \emph{Sacca-Dhamma}, 
the True Dhamma. If one sees Nature, one sees Dhamma; if one sees
Dhamma, one sees Nature. Seeing Nature, one knows the Dhamma.'%
\footnote{Ajahn Chah: ``Dhamma Nature'', The Collected Teachings of Ajahn Chah Vol.III, page 11}

\section{And elsewhere}

`Where is the Buddha? We may think the Buddha has been and gone, but the
Buddha is the Dhamma, the Truth of the way things are. The Buddha is
still here. Regardless of whoever is born or not, whether someone knows
it or not, the Truth is still there. So we should get close to the
Buddha, we should come within and find the Dhamma. When we reach the
Dhamma we will see the Buddha and all doubts will dissolve.'%
\footnote{Ajahn Chah: ``No Abiding'', The Collected Teachings of Ajahn Chah Vol.I, page 34 (adapted)}

In the solitude of the deep forest, however, inspiration in practice is
always quickly tempered by the work at hand -- overcoming the
\emph{kilesas}. For me, battling the \emph{kilesas} involves a continual
struggle with some deeply ingrained proclivities: always catching myself
ruminating about some aspect of the past, present or future; trying to
let go of and unlearn desires that have been drummed into me by society
to become `somebody' or to achieve `something'; being mindful of the
arising and passing away of moods, emotions and unskilful habitual
tendencies; and forever uprooting and investigating deeply entrenched
perceptions. 

With nowhere to go and nothing new to see, I experience a spaciousness
and lightness of mind that allow some deep stuff to percolate up, 
flooding the mind with a deluge of memories. I'm reminded of various
accounts about the experience of the mind just before death. Here at
Dtao Dam I've managed to review what seems like my whole life, 
remembering some of the tiniest details, recollecting where I did no
more than spit on the ground. Yet when I can get beyond this recursive
thinking, there's absolutely nothing to do all day to distract me from a
full-on practice of sitting and walking meditation. Any patterns of
greed, hatred and delusion are given room to manifest in their most
subtle forms. The process of birth, sickness, ageing and death, 
internally and externally, becomes so obvious. As Ajahn Chah liked to
say, `We practise to understand just this much'. 

In similar intensive meditation situations of the past I often found
support and encouragement though practising with others and surrendering
to a retreat schedule. But at Dtao Dam there is no retreat schedule to
which to surrender. Although I have gratitude for those years of formal
retreats in America -- they now provide me with an invaluable array of
tools to help put my time to wise use -- I recognize that the level of
surrender here on a long-term retreat in solitude is of an entirely
different order. I realize that in the past I often motivated myself to
practise through working with a teacher or a group of
fellow-practitioners. Here, although the Sangha is scattered in isolated
spots throughout the surrounding Dtao Dam forest, when it comes down to
it I must muster up the gumption and resolve to maintain an impeccable
standard of discipline on my own. In solitude a more honest and natural
kind of effort replaces any determination fuelled by hubris. 

The anchor for my practice of awareness is \emph{ānāpānasati}, 
mindfulness of breathing. I learn to come back to the breath in the here
and now, again and again and again. With one-pointed awareness of the
breath in the present moment, I practise quieting the mind, cutting off
the incessant internal chatter and, as Carlos Castaneda says, `stopping
the world'. As the practice moves towards a balanced sense of serenity
and tranquillity, I note how investigative energy begins to transform
the very base of conditioned consciousness. Instead of relating to
nature by dividing experience into dichotomous fictions of `self' and
other, as my mind becomes more silent I see the possibility of
experiencing things simply and truly as they are. 

Recollecting family and spiritual companions, teachers and students, and
wishing I could offer them a realization of peace and truth, elevates my
mind and gives it motivation. But over time I witness my mind
oscillating between inspiration and a more humble recognition of how
much there is to do.

Slowly I learn to see these passing emotions as
just more mind states. Could it be any other way? I keep the goals of
the Buddha's path clear in my mind, but the art is learning how to
relate to these goals in a skilful way. The sincere desire for true
freedom from the compulsions of craving is usually the most direct way
to give rise to right effort. Over the weeks and months at Dtao Dam, I
gradually learn how to exert an effort that is wholehearted and
rigorous, while at the same time balanced, measured and at ease with
letting go. Ajahn Chah remarks: 

`The worldly way is to do things for a reason, to get some return, but
in Buddhism we do things without any gaining idea \ldots{} If we don't
want anything at all, what will we get? We don't get anything! Whatever
you get is just a cause for suffering, so we practise not getting
anything \ldots{} This kind of understanding which comes from
[practising Dhamma] leads to surrender, to giving up. Until there is
complete surrender, we persevere, we persist in our contemplation. If
desires or anger and dislike arise in our mind, we aren't indifferent to
them. We don't just leave them out but rather take them and investigate
to see how and from where they arise. We see them clearly and understand
the difficulties which we cause ourselves by believing and following
these moods. This kind of understanding is not found anywhere other than
in our own pure mind.'

\section{The challenge to live in harmony with nature}

In the forest monk life there are various themes which undergird our
practice, to give us a form in which to surrender. One main theme is
that of \emph{nekkhamma}, simplicity and renunciation. The Buddha and
our teachers urge us to eat little, sleep little, talk little and
practise a lot. We're far away from any distraction. The nearest Thai
village is over forty kilometres away, three hours by a difficult
four-wheel drive journey. When it rains the road is easily washed out. 
So here at Dtao Dam the sense of \emph{viveka} -- solitude, quietude
and detachment from the world -- is real. Furthermore, we undertake a
number of \emph{dhutaṅga} practices to cultivate a spirit of simplicity
and renunciation in relation to our four requisites of food, shelter, 
clothing and medicine. We live at the foot of trees on small open-air
bamboo platforms, take just one meal a day in one bowl, wear and sleep
with our robes and get by with a modest supply of communal medicines. 

At the same time, the natural habitat and wildlife make me feel deeply
enmeshed in nature. Biologists and botanists who visit speak with great
enthusiasm about the ecological diversity of the surroundings. The place
is a tropical paradise. Exotic funky-looking palms and ferns abound. 
Ancient hardwood trees tower over dense thickets of bamboo. It takes
seventeen people with arms stretched to form a ring around the base of
the tree where I sit this year. The tree-top provides a home for a
cornucopia of life-forms. There are dozens of bee colonies, whose
beehives at this time of the year drop every so often like grenades from
the branches over a hundred feet above. Ancient ferns sprout out from
the hardwood branches. A family of hornbills makes its nest in the tree
as well. Indeed, looking up at the top I find a whole ecosystem. In
several recent visits to Dtao Dam forest a group of birdwatchers spotted
and catalogued over two hundred bird species, some of which were thought
to be extinct in Thailand. And I can only speculate about how many
\emph{devas} make their homes here. 

Throughout the forest, water flows everywhere. From the cusp of the
mountain ridge to the depth of the valley several hundred metres below, 
the creek cascades into a meandering staircase of waterfalls which
furnish the community with invigorating showers. Small pools at their
bases offer fresh-water baths -- that is, if we can withstand the
curious nibbling on the skin from schools of colourful fish which dart
about. We have agreed not to use soap products in or near the water when
bathing, washing by gently scrubbing with sand or taking a bucket of
water to a spot away from the creek if we use soap. Ajahn Chah, having
spent many years living in the forest, would teach his monks how to live
off the forest in harmony, while keeping the strict precepts of the
Vinaya. He would describe the different trees and plants, indicating
which ones could be used for medicines or allowable food. So although we
have hardly any possessions beyond a few simple requisites, we rarely
feel a sense of lack in such a natural environment. The whole experience
creates an attitude of mind that easily learns to let go and live in
harmony with the surroundings. 

While living in such an environment may sound quite idyllic from a
romantic standpoint, on a practical front a number of difficulties
exist. The hour-long daily climb up the mountain after the meal is
always a hot and sweaty affair. Any water drawn from the creek used for
drinking must be filtered and then boiled to prevent sickness. And
perhaps the most incessant challenge comes from the impressive array of
insects and creepy-crawlies that have to be reckoned with: ticks, biting
horse flies, bees, termites, ants, mosquitoes, spiders, snakes, 
scorpions, rats and centipedes. Bloodsucking leeches, albeit harmless
creatures, can also cause quite a mess of blood. If not bitten by a
leech, my body always manages to get cut, scraped, bloodied and bruised
in some manner. 

It requires constant effort to keep my few possessions dry from the rain
and free from the creatures that ascend the \emph{dtieng}. The nights
are cold and damp. When I awaken in the night on the \emph{dtieng}, I
often have the sense that I am open prey for any large flesh-eating
creature. It is not uncommon in the middle of the night to hear the
footsteps of animals, or even the breathing next to my \emph{dtieng} of
some confused animal such as a bear or wild boar, seemingly trying to
determine what it has bumped into. Invariably these animals smell who I
am and leave me alone. On more rare occasions, monks have come across
deer being attacked by wild dogs or a panther. Other monks have seen
tigers while doing walking meditation. Some locals have even reported
running into wild rhinoceroses. One recent night a monk walked down the
path from his \emph{dtieng} to investigate a curious sound of bamboo
being munched upon, only to find, to his astonishment, a herd of wild
elephants. The earth shook, rumbling as if there had been a small
earthquake, as the elephants fled in surprise. The one creature in the
forest that does offer a serious health hazard is the malarial mosquito. 
This year five members of our community of about twenty have contracted
the malarial parasite from a mosquito bite and have had to be taken out
to a hospital for treatment. Mosquito nets offer some protection, but
the insects can apparently bite at any time of the day, not just dawn
and dusk. 

All in all I experience a renouncing of many comforts and securities I
didn't even realize I had. The tenuous, uncertain nature of the body
really draws me within, in an urgent search for a peace unconditioned by
these external phenomena. If I complicate my daily life by holding on to
any attachments or acting in an unwholesome way, the suffering and
negative \emph{kamma-vipāka} seem almost instant. 

Lastly, the sense of urgency in practice is heightened by the fact that
the forested Dtao Dam area, which has undoubtedly taken centuries to
grow, could be gone or nearly destroyed within the next few years. The
hardwood trees fetch a good price for loggers in the timber industry and
the wild animals are prized by hunters. National Park officials have
been known to burn the forest to get reforestation funds. I can vividly
remember sitting in meditation at the upper \emph{sāla} late one night, 
with a clear vista of the forest for miles, and observing lines of fire
apparently set by arsonists blazing from mountain to mountain. Now
commercial interests want to cut a road through the heart of this
pristine, virgin forest to expedite the transfer of goods from a port in
Burma to Bangkok. First-hand accounts from fellow monks who have done
\emph{tudong} throughout Thailand indicate that forests like this, which
one generation ago covered this country, are now almost non-existent. 
When I think about this it makes me want to practise even harder. 

\section{Cultivating the Sublime Mind}

A second theme penetrating many aspects of our practice is the diligent
development of \emph{sīla, mettā} and wholesome, radiant states of mind. 
For Buddhist monks, the primary precept guiding our relation to the
world is harmlessness. Expressed in a positive way, this means the
cherishing of all life. Interestingly, it also functions as our greatest
protection when living in a wild environment. Giving great importance to
our \emph{sīla} and actively practising the \emph{brahma-vihāras} of
\emph{mettā}, \emph{karuṇā}, \emph{muditā} and \emph{upekkhā} gives us
a skilful method to work with fear. Dealing with the wild animals is not
a matter of bravado or machismo. Asserting a self against nature only
gives rise to conflict. Instead we learn through the cultivation of
\emph{mettā} to emanate a kindness that gives no footing for fear to
arise. Indeed, in the forest we can study how fear and love are like
darkness and light -- the presence of one drives out the other. In the
forest we are instructed not to go out and look for wild animals, nor to
shun them. We just attempt to look at whatever comes our way with
equanimity. When we face wild animals with \emph{mettā} and succeed in
letting go of fear, we can unearth a deep Dhamma treasure buried beneath
the fear. This can open us up to a new taste of freedom and ease. It is
a common theme in the poems written by monks and nuns at the time of the
Buddha: 

\begin{quote}
`I am friend to all, companion to all,\\
sympathetic to all beings,\\
and I develop a heart full of mettā,\\
always delighting in non-harming.'
\quoteref{Thag. 648.%
\footnote{Adapted from K.R. Norman's translation (Pāli Text Society Translation Series No 38, Oxford 1990)}}%
\end{quote}

Holding to the principles of non-violence and harmlessness, a monk trains
not to lift a finger, even in self-defence, to harm another sentient
being. Many of the 227 major training precepts in the Vinaya cultivate a
respect for animal and plant life in the most refined ways. Monks are
prohibited from digging the earth, trimming foliage or uprooting plants. 
When drawing drinking or bathing water from the creek, we must carefully
check the water for any beings visible to the eye. If there is even a
tiny mosquito larva the water cannot be used, or it must be filtered and
any living creatures returned to the water. 

These details of the monk's discipline might appear excessive, but they
create a new attitude of mind in relation to nature. We endeavour to
look upon everything in nature as worthy of care and respect. Rather
than being a source of material for use and consumption, nature is
understood as a process that incorporates our very life. I am aware that
the \emph{dtieng} upon which I meditate and sleep is constructed from
dead bamboo, and when I move away the bamboo rots in the torrential
rains and becomes a natural part of the forest carpet, as it would
anyway. As alms mendicants, Sangha members depend upon lay support for
the building of these simple structures in the first place, so we learn
to relinquish any sense of ownership.

The practice of living in harmony
with nature also extends to the method we use to wash our robes. 
Laypeople offer a piece of heartwood from a jackfruit tree, which the
monks in turn chop into small chips and boil in water, making a
\emph{gaen-kanun} concoction for washing. The \emph{gaen-kanun} has a
marvellous disinfectant and deodorizing effect that lasts for days. If a
robe washed in it becomes sweaty, hanging it in the sunlight gives it a
natural freshness in minutes. In understanding that we are an aspect of
nature, the emphasis is upon living in harmony, attuned to nature's
processes. The fortnightly recitation of the Pāṭimokkha is scheduled
according to the lunar cycle, occurring every new and full moon. By
forgetting the date and month of the worldly calendar and just living
according to the patterns of the sun and the moon, we create a sense of
timelessness. There's no time to practise awakening except the present
moment. 

This shift in attitude generates some positive results. To begin with, 
wild animals respond differently. They can intuitively sense
harmlessness and any accompanying fearlessness. When we encounter wild
animals in the forest, they seem simply to mirror what they sense. Many
people also respect the strict ethical standards of forest monks. Lay
Buddhists consider it auspicious to have forest monks around, and as a
subsequent effect the monastery protects not only the forest in its own
immediate area, but the whole forest around it. Lay Buddhists in Asia
find inspiration if their monks are putting forth a lot of effort in
practice. Although few people make it into Dtao Dam, the ones who do
come from throughout Thailand, and range from businessmen to ecologists
to military historians. 

Another interesting development in Thailand is that a tradition of
ordaining trees has been introduced to protect the forest from logging
--, tying a semblance of the \emph{gaen-kanun}-coloured forest monk's
robe around the trunks of trees. Even the most callous Thai logger will
think twice before killing a tree with a monk's robe tied round it. To
be sure, there are glaring exceptions to this tendency to respect the
trees and the wildlife in and around the forest monastery, but generally
the presence of monks has a strong deterrent effect on deforestation, 
inhibits the hunting of wild animals and engenders an increased
appreciation and love of nature. As we are forest monks, our
teachers encourage us to reflect consciously on the value of the forest, 
and bring to mind that our presence and sincerity of practice are
intended to be a force for preservation. At Dtao Dam Ajahn Jayasāro
spurs us on with an analogy: 

`Living in a forest threatened with extinction is like encountering a
human being on the side of the road, injured or with an illness, 
possibly dying. One doesn't worry about the person's previous behaviour, 
inquire about their nationality or ethnicity, wonder whether the human
being is rich or poor, young or old, famous or not. In any case, human
life is sacrosanct. You attend to the threat on the person's life by
taking them to a hospital or doing everything possible to try to save
them. Similarly, just as human life has intrinsic value, so does the
life of the forest, with the multitudes of life within it. Who knows, 
the forest may contain a rare plant species that leads to a cure for
cancer. Or maybe not -- one way or the other, a forest supports the
lives of countless beings, and if it is subject to destruction, that
merits an immediate and appropriate response towards protection and
preservation.'

\section{The intrinsic power of mind}

A third major theme of the forest Dhamma practice, in many ways the
point of retreating to the forests, is to develop \emph{sati, 
sampajañña} and \emph{samādhi}. Ajahn Ṭhānissaro (abbot of Wat Mettā in
California) once challenged my enthusiasm for practising at a place like
Dtao Dam, perhaps sensing my enchantment with the exoticism of the
retreat setting, by remarking, `It's a good thing as long as it helps
you with your meditation'. This echoes a relevant exhortation by the
Buddha in the Saṃyutta Nikāya of the Pāli Canon:

\begin{quote}
`Monks, the establishing of mindfulness is to be practised with the thought, ``I'll watch after myself.'' The establishing of mindfulness is to be practised with the thought, ``I'll watch after others.'' When watching after yourself, you watch after others. When watching after others, you watch after yourself.

`And how do you watch after others when watching after yourself? Through cultivating [the practice], through developing it, through pursuing it. This is how you watch after others when watching after yourself.

`And how do you watch after yourself when watching after others? Through endurance, through harmlessness, through a mind of goodwill, \& through sympathy. This is how you watch after yourself when watching after others.'

\quoteref{S.V.168\footnote{\href{http://www.accesstoinsight.org/tipitaka/sn/sn47/sn47.019.than.html}{http://www.accesstoinsight.org/tipitaka/sn/sn47/sn47.019.than.html}}}
\end{quote}

With the simplicity of the \emph{viveka} environment, all the energies
of the day can be focused resolutely on cultivating the Four Foundations
of Mindfulness. \emph{Sati} functions as a fulcrum for every aspect of
our practice. It is the \emph{sine qua non} of the spiritual life. In
whatever posture one finds oneself, there is the determination to give
rise to the \emph{sati} and \emph{sampajañña} of the situation, knowing
the body as body, feelings as feelings, the mind as mind and dhammas as
just dhammas. We practise to know things just as they are, impermanent
and empty of an inherent self. \emph{Sati} and \emph{sampajañña} both
bolster the strength of the \emph{samādhi} cultivated in the formal
sitting practice, and operate as extensions of it. Ideally, \emph{sati}
and \emph{sampajañña} form a seamless continuum of awareness and
investigation throughout the entire day and night. They are the presence
of mind that is life itself; without them there is heedlessness. When we
are heedless, as the Buddha said, it is as if we are dead. 

Over an extended period of diligent practice, I begin to experience how
the power and knowledge of \emph{sati} and \emph{sampajañña} grow
organically and build upon themselves as they develop. The power and
knowledge intrinsic to these qualities of mind are not derived from
force or coercion. Once in motion, with continual effort, they naturally
deepen. By their inherent nature, the faculties of \emph{sati} and
\emph{sampajañña} are ready for development in all human beings. And
although they can be aimed toward a variety of ends, our task in Dhamma
practice is to use \emph{sati} and \emph{sampajañña} as tools to be
awakened by all things. We're not trying to concoct beautiful theories
and ideologies, but to develop a penetrative clarity in the study of
moment-to-moment experience. We learn from Dhamma, manifest in the
nature around us, through \emph{opanayiko}, turning inwards. We
endeavour to continue to take the practice yet further, looking deeply
at the nature of the mind. Indeed, by getting in touch with the
pristine, natural state of the mind, we let the rigid separation between
inside and outside become deliquescent, giving rise to a more unified
awareness. 

The breath, in constant flow between the inner and outer natures, is an
ideal \emph{samatha} meditation object for myself and many others. As I
learn to let the breath breathe itself naturally, accompanied by a
suffused and unwavering awareness, I try to let the doer of the
meditation process disappear. Only when the doer steps out of the way
and the knower of the breath lets go of any attachment and
identification with the process does, \emph{samādhi} mature. But as soon
as my mind proliferates -- `How can I give impetus to the \emph{samādhi
nimitta}?' or, `Will this lead to \emph{jhāna}?' -- a taint of craving, 
a glimmer of grasping and a subtle sense of lack all obfuscate the
knowing, and my mind becomes distracted. With distraction there is no
possibility to cultivate awareness of phenomena just as they are. With
even a subtle sense of contrivance, luminosity fades. But fortunately, 
there's always the next breath, another mind moment, to begin anew. 

\section{The wisdom to let go}

According to the Buddha's teaching, \emph{sīla} and \emph{samādhi} are
the ground for \emph{paññā} to arise. With a mind brought to
malleability, sharpness, clarity and deep peace through \emph{samādhi}, 
the defilements are at least temporarily held at bay and the
investigative faculty of \emph{paññā} is tractable for making some
decent headway. Again from Ajahn Chah: 

`With right \emph{samādhi}, no matter what level of calm is reached, 
there is awareness. There is \emph{sati} and \emph{sampajañña}. This is
the \emph{samādhi} which can give rise to \emph{paññā}, one cannot get
lost in it \ldots{} Don't think that when you have gained some peace of
mind living here in the quiet forest, that's enough. Don't settle for
just that! Remember that we have come to cultivate and grow the seeds of
\emph{paññā}.'

Reading the Suttas, I am struck by the plethora of passages in which a
monk or a nun, given the teachings by the Buddha, retreats to meditative
solitude at the foot of a tree and in no long time `does what is to be
done'. That is, the monk or nun sees the five \emph{khandhas} as
impermanent and empty, puts an end to \emph{dukkha}, liberates the mind
from the \emph{saṃsāric} treadmill of birth and death and realizes
complete Awakening. It sounds utterly straightforward. Yet in my
practice, without constant heedfulness it can be natural for mindfulness
to lose touch with the present moment and allow the \emph{kilesas} to
slip in.

Memories from the past, coupled with creative imagination about
the future, perpetually enchant and fascinate. Or conversely, my mind
can feel immured by the extremes of languor and impetuousness. When
practising in America I never considered myself to be interested in
sense desires. Now though, living the renunciant life of a monk in the
meditative solitude of the forest, I am more acutely aware of sense
impingement and the lure of sensuality. The ostensible comfort of the
familiar and the secure and the ensuing entanglements of the worldly
life can seem more alluring than ever before. Such is the pathos of
\emph{dukkha}! 

If I do get a bit of sustained success in the practice, I notice that a
subtle sense of pride in living a pure and good life can enter the mind. 
As I attempt to live a simple, selfless life, the sense of self
unwittingly tends to emerge in new and unforeseen forms. It wants to
claim ownership of any goodness and wisdom that might arise. Here Ajahn
Chah continues the encouragement: 

`For the ultimate in the practice of Buddhist Meditation, the Buddha
taught the practice of Letting go. Don't carry anything around! Detach! 
If you see goodness, let it go. If you see rightness, let it go. These
words, ``Let go'', do not mean we don't have to practise. They mean that
we have to practise the method of letting go itself. The Buddha taught
us to contemplate all of the dhammas, to develop the Path through
contemplating our own body and heart. The Dhamma isn't anywhere else. 
It's right here, not somewhere far away! It's right here in this very
body and heart of ours.'

On still other occasions I find the sense of self assert itself through
doubt. My mind wonders, `Would it be better to go out and help others? 
What use am I to the world sitting at the foot of this tree? What if I
don't awaken to the unconditioned truth of Nibbāna? Isn't it a bit
presumptuous for me to think that I can realize the ultimate truth?' The
Theravāda forest masters' arousing exhortations in this respect are
echoed in Patrul Rinpoche's classic introduction to Tibetan Buddhism, 
`The Words of My Perfect Teacher': 

`Until you have overcome wanting anything for yourself, it would be
better not to rush into altruistic activities\ldots{} The ancient
[practitioners] had these four goals: Base your mind on the Dhamma, 
base your Dhamma on a humble life, base your humble life on the thought
of death, base your death on a lonely cave. Nowadays we think we can
practise Dhamma alongside our worldly activities, without the need for
bold determination, courage, and difficult practices, all the while
enjoying comfort, well-being and popularity\ldots{} But how could there
be a way to marry Dhamma and worldly life? Those who claim to be doing
so are likely to be leading a good worldly life, but you may be sure
that they are not practising pure Dhamma. To say that you can practise
Dhamma and worldly life at the same time is like saying that you can sew
with a double pointed needle, put fire and water in the same container
or ride two horses (simultaneously) in opposite directions. All these
things are simply impossible. Could any ordinary person ever surpass
Sākyamuni Buddha? Yet even he found no way of practising Dhamma and
worldly life side by side.'

Another reflection that sits powerfully in my mind is the story of a Zen
Buddhist monk who, as I remember the story, was asked, `What would you
do if you were told that you had twenty-four hours to live?' He
responded, `Sit straight zazen \emph{samādhi}' -- that is concentrate
the mind in meditation in the sitting posture. The questioner persisted, 
`What about your vow to liberate all sentient beings?' The monk
retorted, `That is the most direct, complete way to liberate all
sentient beings'. 

Our teachers remind us that the mind absorbed with the bliss of
\emph{samādhi} is far from intoxication. With a pliant and supple mind
primed for the work of investigation from one-pointed concentration, 
it's natural for insight into the tenuous, transient nature of
conditioned existence to deepen. In the forest I notice that when my
mind is in a peaceful state, ordinary discursive consciousness
dissatisfies; thoughts, even astute ones, are of their nature ephemeral
and capricious. To indulge in the thinking mind gives me a feeling akin
to not having taken a shower or brushed my teeth for days. 

When I realize a dispassion towards mental proliferation the practice
takes on a greater immediacy for me. \emph{Dukkha} is ubiquitous. It is
imperative to understand the source of the incessant torrent of
suffering, and to tread the path to realize its cessation. Resorting to
belief systems, philosophical explanations of ultimate truth or
supplication to an external being only takes me further away from peace. 
Any approach that is bound up with the five \emph{khandhas} is still
within the realm of \emph{dukkha}. But how do I penetrate or transcend
the conditionality of the five \emph{khandhas} while still operating
from within their realm? I find this to be the heart of the
investigation. 

In his book \emph{Heartwood of the Bodhi Tree}, Ajahn Buddhadāsa
mentions that in the \emph{Majjhima Nikāya} the Buddha is asked to
summarize his teachings in one statement. To this the Buddha responds, 
`\emph{Sabbe dhammā nālaṃ abhinivesaya}' -- nothing whatsoever should
be clung to. Implicit in this utterance is the teaching of \emph{anattā}
-- not-self, voidness of a separate self, emptiness of a soul-entity. So
in the Dhamma practice I attempt to see though the seemingly \emph{a
priori} concepts of `I', `me' and `mine'. Ajahn Chah explains: 

`So we practise not getting anything. Just this is called ``making the
mind empty''. It's empty, but there is still doing. This emptiness is
something people don't usually understand, but those who reach it see
the use of knowing it. It's not the emptiness of not having anything, 
it's emptiness within the things that are here.'

\section{Investigation of birth and death}

In my Dhamma practice, the steady contemplation of birth and death
brings the practice right to the heart. The forest affords numerous
opportunities to confront matters of life and death. Walking along the
forest paths, one frequently comes across a snake eating a frog or
lizard, swallowing the animal whole and head first. The process takes
maybe twenty minutes and the snake, having once overcome his victim, 
rarely stops the process because of a curious onlooker. 

In the investigation of birth and death I make a repeated effort to come
to peace with the fact that my body -- this bag of flesh and bones --
although not likely prey, will certainly die and is not really mine. Its
constituent parts, comprised of the four elements, are constantly
returning to their nature. When I look closely, I can begin to see this
process of birth and death at every moment. We have all known many
people who have died. I find the practice most honest and powerful, 
however, when I am able to weigh the fact of death not just for others, 
but for my own body. 

So contemplation of my death helps brings the essence of the practice
into the present moment. Ajahn Chah would ask newcomers to his
monastery, `Did you come here to die?' (Interestingly, the word `die' is
the same in Thai and English). This kind of vital question catapults us
from the dogmatic slumbers of our everyday existence into awakeness. 
Making an investigation into death with continuity and sincerity
breathes a heedful clarity into our daily life. The Buddha exhorts his
monks to recollect frequently: `Has my practice borne fruit with freedom
or insight, so that at the end of my life I need not feel ashamed when
questioned by my spiritual companions?' By frequently reflecting
on death, I find my understanding of life and death gradually takes on
an earthy honesty. Death can become just like an old friend. Connecting
this with the fact that death spurs urgency, I then ask, `Within just
this, what is it that does not die?'

\section{Everything or nothing?}

So as a community of forest monks, our endeavour at Dtao Dam is to make
progress on the noble path to Nibbāna. Naturally, in this process our
gift to the forest is our very practice of \emph{sīla}, \emph{samādhi}
and \emph{paññā}. Our aim is to tilt the balance of virtuous and evil
forces in the world decisively towards goodness. And if a positive
evolutionary change is to take place in the world in any significant and
fundamental way, from a Buddhist perspective it must grow from an
enlightening shift or awakened transformation in consciousness. Without
such a change, any attempts to heal the world are just band-aid remedies. 
Although perhaps well intentioned and important, these efforts are not
enough. The Buddha has indicated that the human being has the potential
to go beyond the \emph{saṃsāric} realm of \emph{dukkha} altogether, and
settling for anything less would be to sell ourselves short. 

But, to see beyond the clutches of conditioned existence necessarily
entails experiential knowledge of the unconditioned. In other words, to
paraphrase Einstein, the most significant problems we face cannot be
solved by the same level of thinking which created them. Similarly, in
the practice of Dhamma we see that \emph{dukkha} is a problem which
cannot be extinguished by the same mindset that fabricates it in the
first place. Hence the imperative to develop the path to go beyond our
conditioned perspective. And the Buddha and the \emph{arahants} show us
that the goal of realizing the unconditioned is achievable in this life, 
and worth any sacrifice. 

The teachings of the Buddha also indicate that the effects of the
profound transformation of Awakening reverberate far and wide throughout
the web of life, although perhaps in ways imperceptible to the
unenlightened eye. We're more interconnected than we think. Systems
theorists make a parallel point when they contend that a seemingly small
input at the beginning of a process can have huge ramifications in the
big picture. And quantum physicists concur that the effect of something
as small as one electron making contact with another electron might not
bear any fruition until over a thousand years later. From a Buddhist
perspective, it's the accumulation of the moment-to-moment efforts to
practise \emph{sīla, samādhi} and \emph{paññā} and tread the path to
Nibbāna over the long haul that count. 

So given that every small action can carry significant \emph{kammic}
weight, before we (as monks) presume to know what is the best way to
help others, and before we become too engaged in resolving worldly
matters, we need to be solid in our realization of Dhamma. Again, we can
draw a parallel with contemporary science. David Bohm, the eminent
quantum physicist, also holds that the process of changing the world
occurs first through transforming the mind: `A change of meaning
[within the human mind] is necessary to change this world
politically, economically and socially. But that change must begin with
the individual; it must change for him\ldots{} if meaning is a key part
of reality, then, once society, the individual and relationships are
seen to mean something different a fundamental change has taken place.'

In any case, for the meditator who has success in the practice, all
these explanations are superfluous. From reading the Suttas and hearing
the words of contemporary masters, we see that the enlightened, 
liberated mind validates itself and inherently knows what is the best
thing to do in order truly to help others. Even the teachings and the
practice become just the raft to the other shore. They are the finger
pointing at the moon, not Awakening itself. As Ajahn Chah indicates: 

`The Buddha laid down \emph{sīla}, \emph{samādhi} and \emph{paññā} as
the Path to peace, the way to enlightenment. But in truth these things
are not the essence of Buddhism. They are merely the Path \ldots{} The
essence of Buddhism is peace and that peace arises from truly knowing
the nature of things \ldots{} Regardless of time and place, the whole
practice of Dhamma comes to completion at the place where there is
nothing. It's a place of surrender, of emptiness, of laying down the
burden \ldots{}'

So here at Dtao Dam, through various aspects of the practice of Forest
Dhamma, I have come to find that many of the Buddha's discourses in the
Suttas which once seemed recondite or beyond me are now clearer and easy
to apply to my life. Pāli words such as \emph{nekkhamma}, \emph{viveka, 
sīla, mettā, sati, sampajañña, samādhi} and \emph{paññā} become the
vernacular of everyday situations. Far from being ossified or esoteric
teachings from two and a half millennia ago, the Suttas become living
teachings. They are urgent reminders and vivid pointers to the way it is
with the body and mind right here and now. I ask myself, can my practice
with heartfelt dedication make progress towards the goal of awakening, 
dropping the burden of \emph{dukkha} and at last transcending the
vicious cycle of birth and death? 

My mind returns once more to the time of the cat encounter. It is
forty-eight hours later, and I have just spent another pleasant morning
of meditation at my \emph{dtieng}. I pick up my robe and carry-bag and
set out for the morning descent down the mountain for the meal. Having
taken only a few steps on the path, I suddenly see a large black animal
up ahead, but I am not fully sure what it is. It has apparently just
stepped off the path about twenty metres ahead of me and has hunched
itself right behind a large cluster of brush. By sight I can't make out
the type of animal for certain from its general physical form, but I am
instantly reminded of the large black panther (three and a half metres
in length!) sighted several times wandering around the Dtao Dam forest
in recent weeks. (Several nights earlier one of the novices had been
circled by the black cat while walking the thirty-minute walk from the
upper to the inner \emph{sāla} late at night with only his candle
lantern. The workers at the mine also saw it one morning recently). In
any case, the animal is far too large to walk up to on the path and it
seems to have its gaze fixed dead centre on me. 

Spontaneously, and perhaps with a bit of over-confidence and
fearlessness from my previous cat encounter, I clear my throat as if to
signal an indication of my presence and wish to walk by. The animal does
not budge. Then I feel a wave of energy coming from this impressive cat
sweep over me. It has a very strong and fierce feeling-tone. If I had to
put it into words, it would be something like: `Who in the world do you
think you are to be telling me to get off this path? I am in charge
here!'

This feeling that I get from the stealthy animal on the path is unlike
anything I have ever experienced in the forest in daylight hours. To be
sure, at night on countless previous occasions I have heard the
footsteps of a large animal padding though the brush, gently crushing
the bamboo leaves as it moves. On those occasions fear sometimes would
arise, but I have never actually seen any animal that I recognized as a
tiger. This morning, however, I am looking at a giant black cat. There
is uncertainty and I feel a profound presence of death. I collect my
\emph{sati} and slowly take a couple of steps backwards without turning
round. My hands instinctively go into \emph{añjalī}, palms joined in
reverential salutation, and I start to gently chant the \emph{Mettā}
\emph{Sutta}, the Buddha's words of loving-kindness which I had chanted
moments before at my \emph{dtieng}. I close my eyes and tap into the
energy I had felt so strongly that morning. And with the mind imbued
with \emph{mettā}, I practise letting go. 

After finishing the chant about three minutes later, I open my eyes and
can no longer make out any large black animal in the brush. Keeping my
hands in \emph{añjalī}, I chant homage to the Buddha (\emph{Namo tassa
bhagavato arahato sammā-sambuddhassa}) and proceed down the path. 
Whatever was previously there has silently walked away. While walking
down the mountain that morning, my mind naturally turns to the
contemplation of death and the empty and impersonal nature of the five
\emph{khandhas}. The energy from the encounter gives me a definite
penetrative push. During these moments I feel a deep and meaningful
trust in the Buddha and the path of Dhamma. To the extent that I have
been able to devote my life to the cultivation of the path of
\emph{sīla, samādhi} and \emph{paññā}, my heart feels good and true. In
the face of what appeared to me as death, or at least a purveyor of it, 
the question of what really matters hit home. I experience an
appreciation for my life that resonates deeply in my bones. 

Yet while contemplating the empty nature of the five \emph{khandhas} and
feeling immeasurable gratitude for the Buddha's teaching, there is a
wonderful and ineffable mystery to it all, a kind of `Don't know' mind. 
The experience is light and peaceful. Although my heart feels full of
\emph{mettā} and connected to the beauty of everything around it, 
\emph{paññā} insists that everything is completely empty, with nothing
really there at all. And for once I don't experience the pressing need
to reconcile this seeming contradiction. How to realize these two
insights as not separate but one is the practice of Forest Dhamma that I
continue to learn. 

\dividerRule

\section{The Author}
\label{pannyavuddho-desc}

Tan Paññāvuddho carried on his training in Wat Pah Nanachat and
branch monasteries in Thailand before joining the community at Buddha
Bodhivana Monastery, north-east of Melbourne, Australia. Having returned
to Thailand after his sixth Rains Retreat as a monk, he spent many
months in solitude in a cave in the Pak Chong district of Korat
province. In 2005, the year of his seventh Rains Retreat, Tan
Paññāvuddho travelled to Wat Boonyawad in Chonburi, where he entered the
Rains Retreat practising under Tan Ajahn Dtun Thiracitto's guidance. A
few days into the Rains Retreat he failed to come out for the alms-round
or the meal, which was unusual for him. A check in his \emph{kuṭī} led to the
very sad news, which spread through Thailand and very quickly all round
the world, that this much-loved Sangha member had died in a simple
accident in his bathroom. Subject for many years to fatigue caused by
low blood pressure and dehydration, he had fainted early in the morning. 
His forehead had struck a corner post as he fell, forcing his head
sharply back with tremendous force and breaking his neck. His death, the
doctors concluded, would have been instantaneous and painless. His
cremation, carried out at Wat Pah Nanachat, was attended by monks and
laypeople from all over Thailand, Ajahn Kalyāṇo, the abbot of Buddha
Bodhivana Monastery, Melbourne, and Tan Paññāvuddho's brother from the
US.

As a Sangha member Tan Paññāvuddho was known for his keen
intellect, sensitivity and kindness in community situations, and most of
all for his absolute dedication to training in the way of the Buddhist
monk. This publication, `Forest Path', is one example of his creative
skills: as well as writing this chapter, he was one of the main
collaborators on the whole project. His love of the Thai forest
tradition was very deep, despite his earlier background in the Zen
school and a strong appreciation of Mahāyāna and Vajrayāna teachings. 
When he died it was a certain source of comfort that he was truly
practising in the way he loved, and living with a teacher, Tan Ajahn
Dtun, for whom he had much devotion and respect.

