% Title: Two months at Wat Pah Baan Taad
% Author: Tan Yātiko

Mosquito repellent? Check. Ground sheet? Check. Road map of Isan? Check.
Spare flashlight batteries? Check. Water filter? Check. Candle lantern?
Check.

So I was packed and ready for my upcoming \emph{tudong}. I had decided
that I would start from Wat Pah Baan Taad, the famous monastery of Ajahn
Mahā Boowa, and from there walk through much of the province of Udorn
Thani in the north-east of Thailand. My plan was pretty loose, if you
wanted to call it a plan. I would keep my ears open for tips on good
teachers and monasteries in the area that monks would recommend. It
would be my first real \emph{tudong} on my own, and I was looking
forward to the adventure and the uncertainty that \emph{tudong} would be
sure to offer. 

The \emph{tudong}, however, never even began. 

I remember going to Ajahn Paññāvaddho's \emph{kuṭī} the day after I
arrived at Wat Pah Baan Taad to introduce myself and ask permission to
stay. Ajahn Paññāvaddho is a senior English monk of over forty rains and
is in charge of the practical running of the monastery, while Luang Por
(Venerable Grandfather) Mahā Boowa occupies his time in receiving guests
and teaching. Ajahn Paññāvaddho's \emph{kuṭī} is a simple wooden
structure in the forest, built on concrete posts about four feet high. 
Inside it is perhaps some ten by ten feet, and it has a rectangular
porch out front, large enough to seat four or five people. Just in front
of this old well-worn \emph{kuṭī} is a sandy meditation path that
stretches some eighteen paces, shaded by the shadow cast by a tall
wooden pole. Ajahn Paññāvaddho's solar panel, which he uses to recharge
batteries and run his water heater, is secured atop the pole. I coughed
as I entered the sandy yard, and when he invited me up I ascended the
\emph{kuṭī} steps and saw him sitting on the step between the
\emph{kuṭī} door and the porch, with knees up near his chest and his
elbows on his knees. 

I felt immediate affection for Ajahn Paññāvaddho. His gentle old eyes
glowed at me as I bowed to pay my respects, and his informal greeting, 
`So, how's it going?' quickly made me feel at ease. I explained to him
that I wanted to spend a few nights at the monastery, pay respects to
Ajahn Mahā Boowa and then head off on my \emph{tudong} trip through the
province of Udorn Thani. Ajahn Mahā Boowa, he told me, had gone away to
Bangkok until New Year, still two weeks away, but I would be welcome to
stay until he came back. Well, it wasn't what I had planned, but I
thought that this would be an ideal opportunity to spend some time at
the monastery and receive some teachings from Ajahn Paññāvaddho. 
Besides, after Ajahn Mahā Boowa came back I could just carry on with my
\emph{tudong} as I had planned. So I decided to stay on till Ajahn Mahā
Boowa returned. 

That afternoon Ajahn Paññāvaddho took me personally around the
monastery. It's a fairly small place, maybe some 200 acres in area, and
with a forbidding chicken population one's first impression is that it's
a rather unlikely place to come and calm your nerves. Within the
monastery confines there are about forty little shelters, mostly
open-air wooden platforms with iron roofing, and a handful of larger
\emph{kuṭīs}. He took me to one of the \emph{kuṭīs} which had recently
been evacuated by one of the senior monks, and suggested I sweep around
the \emph{kuṭī} and spend the day meditating. With the monastery being
about as peaceful as a chicken-farm, and with Ajahn Mahā Boowa's current
fund-raising programme to help the country out of its financial crisis
 (he had then raised over a billion Thai Baht as an offering to the
National Reserve), I wasn't expecting much out of my meditation. I
prefer the silence of caves to the clucking of chickens and I have
always been sensitive to noise when I sit. But quite unexpectedly, I
found that as I sat in my hut my mind felt brighter and stronger than
usual. `Must be a fluke,' I thought to myself. But no, there was no
denying it, after a few days I found that my meditations had improved; I
was feeling much more focused. I mentioned this to Ajahn Paññāvaddho, 
and he said it was the power of living in Ajahn Mahā Boowa's monastery. 
`He's a special man', he said with an air of understatement, `and that's
why your meditation is going so well here. Living under a teacher like
Luang Por is a critical part of the practise.'

Well, I'm not so sure that
just being there had any magical influence, but it's certainly true that
I took to reading Ajahn Mahā Boowa's talks like never before, and I felt
a confidence in the tradition and in the man that surprised even me. 
Being in the presence of a realized being helps one to know that this is
not a path `I' am walking on, and one for which `I' must have the
qualifications to succeed, but rather that the path is a simple cause
and effect process. Just as any one of these realized masters was once
as deluded and self-centred as I am now, so too can this set of five
\emph{khandhas} one day be as free of suffering and delusion as theirs. 

Afternoon tea would take place at the dyeing shed shortly after 1 p.m.,
and then I would have the chance to ask questions about Dhamma. During
my stay there I had many inspiring conversations with Ajahn Paññāvaddho
over a hot drink. A small handful of Westerners would gather round
outside the shed and squat down on tiny five inch-high wooden stools. 
One of the resident monks would make Ajahn Paññāvaddho's usual, a cup of
black unsweetened Earl Grey, and would make me a cup of black coffee. 
I'd set my stool on the dusty soil to the left of Ajahn Paññāvaddho, 
while a dachshund, a poodle and a few cheerful dalmatians would
playfully compete for our attention around the shed. As the
conversations evolved, almost invariably the talk would turn to or at
least encompass Dhamma. 

Often when talking Dhamma with someone it's not so much the content of
what is said that educates, but rather the way a point is made, 
fashioned exactly to illuminate an aspect of practise you may have
overlooked or forgotten. Ajahn Paññāvaddho had that valuable knack of
being able to offer a pointful reflection on any given topic of Dhamma
that one might raise. I described to him once how my meditation was
losing some of its sense of direction and my mind was beginning to
wander again. Ajahn Paññāvaddho's response was almost uncannily fitting. 
He told me that whatever we do in meditation, the main point is to
undermine the defilements. That's it. Simple words, nothing I couldn't
have thought of myself, but their simplicity and directness gave me a
new outlook on meditation and I found that my sittings improved. I
noticed that I had been getting bogged down with techniques and
strategies, without focusing on the actual purpose. A good teacher is
one who can notice these subtleties in a student and point them out. 

That being the case however, the most educative thing about being with
Ajahn Paññāvaddho was just his relaxed and mild presence. He was a man
with nothing to prove to anyone, and he was not out to convince me of
anything. One point he would often make was that a meditator has to have
\emph{samādhi} before he can get very far with \emph{vipassanā}. 
`Without \emph{samādhi}', he'd say with a gentle authority, `it's
unlikely that one will have the emotional maturity to deal skilfully
with profound insights. When deep insights occur, one can be left with
the sense that there is no place for the mind to stand, and this can be
very unsettling. When one has \emph{samādhi}, one has a comfortable
abiding and one has the freedom to ease into one's insights with skill.'
I then asked, `What about the need for \emph{jhāna} on the path? Do you
feel it's necessary to have \emph{jhāna}?' A reserved look appeared on
his wrinkled face, and in his soft voice he explained, `The thing is
that people will often talk about \emph{jhāna} without knowing what it
really is. Ajahn Mahā Boowa prefers to use the word \emph{samādhi}. We
need \emph{samādhi}, and the more we have the better, but we must not
neglect our investigation.'

One of the practise techniques that is often encouraged by the forest
Ajahns is the investigation of the body. This practise will involve
visualizing a part of the body, focusing on it and studying it, so as to
achieve a soothing and dispassionate calm and counterbalance an
underlying infatuation with the body that lies within the unenlightened
mind. It is a practise which for those unacquainted with Buddhism is
often unpalatable and hard to understand. Ever since I ordained I had
had a chronic avoidance of this technique and I asked the Ajahn about
it. He said that the fact that I avoided it probably showed there was
much for me to learn in that area. `That's your front line of
investigation.' He went on to say, `Why do you think you are so hesitant
to pick up this practice? I'm not saying it is necessarily an
appropriate meditation theme for you to work with simply because you are
resistant to it, only that you should at the very least understand the
nature of that resistance before you just go ahead and believe in it.'
So I looked at the resistance deeply and found it hard to see exactly
what it was all about. After a few weeks of being in the monastery and
taking up a considerable amount of contemplation of the body as my
meditation theme, I began to feel a certain weariness with it, and felt
a kind of depression coming up. I told this to Ajahn Paññāvaddho. He
smiled knowingly and shook his head as if to say, `Yup, that makes
sense', and explained, `Yes, this can happen to some people if they are
doing a lot of body contemplation. It can feel disappointing if you
suddenly start to see the reality of the body's unattractiveness and
impermanence. After all, most of us have been running around glorifying
it for lifetimes.'

I sat and listened to what the Ajahn was saying, somehow sensing that
the truths that he was talking about weren't sinking in. I tried to
force myself to believe what he was saying, but found that I couldn't. 
And yet I trusted the man and I trusted in the truths of which he was
speaking. In the end I found myself stuck between a reality I couldn't
accept and a trust I couldn't deny. I asked him how I should proceed if
I found this kind of meditation left me feeling a bit down. His answer
was reassuring. He said I shouldn't force it and that it was important
to balance this meditation technique with other techniques that I found
uplifting. For years one of my major meditation themes had been
contemplation of the Buddha, something that always brought me great joy. 
I told this to Ajahn Paññāvaddho and he said that it would be an
excellent counterpart to body contemplation. 

While at Wat Pah Baan Taad I actually had very little contact with Ajahn
Mahā Boowa himself. He kept quite aloof from the monks and gave more
attention to the laity. I remember that a few days after he had returned
from Bangkok, word went round that he would be giving a talk at the
dyeing shed that night at eight o'clock. `That must be nice', I thought
to myself. I imagined the group of monks gathering around Luang Por
in the cool of a chilly Isan night, warmed by a fire lit in the wood
stove, listening to stirring personal Dhamma from the master himself. 
Not quite. I approached the dyeing shed around five minutes to eight and
saw Ajahn Mahā Boowa studying some newspaper clippings that had been
posted on the wall of the shed. There wasn't a soul around and I thought
to myself, `What an opportunity to approach Ajahn Mahā Boowa and say a
few words!' So I somewhat meekly approached Luang Por, holding my hands
up in the traditional gesture of respect, when suddenly from out of the
shadows of the night appeared a nervous young monk, tugging at my robes
and, whispering in Thai, `Come here! Come here!' gesturing me to steal
away behind a pile of buckets and squat down. As we squatted there in
the silence with eyes glued to the Ajahn, I noticed that to my left
behind the water tank another shadowy figure was squatting, and then to
my right, closer to the wood stove, was another one. This one had a pair
of headphones on, and some electronic gadgetry with little red LED
indicators that danced up and down to the crackle of the fire. He
reminded me of some FBI agent or private eye. In fact the whole scene
was a kind of \emph{déjà vu} of some half-forgotten memory from high
school. I was both amused and confused, and it took more than a few
seconds to realize what was going on. Ajahn Mahā Boowa wasn't giving a
talk to the monks, he was giving a talk to a certain layperson, and the
resident monks were gathering around in the dark to hear and even tape
the conversation. 

It was strange. I didn't understand. Why was Ajahn Mahā Boowa so aloof
from the monks? He was known for his uncompromising strictness and his
readiness to scold, and I found it hard to understand to what purpose. 
Was this really coming from a heart of \emph{mettā}? As with any
situation one wants to understand, one has to understand it from within
its own context. It wasn't until I had stayed there for several weeks
that I began to get a feeling for how Ajahn Mahā Boowa motivates his
monks. It isn't through playing the charming, soft, lovable role of a
guru. He does not become informal with any of his disciples, and this
serves to establish a very specific role in their relationship which is
felt and understood by those who practise under him, but may not be felt
and understood by newcomers. It was obvious that Ajahn Mahā Boowa cares
for the monks, he cares for the world, he cares for the religion, and he
sees the sincere practise of monks and nuns as an immeasurable
contribution to the virtue in the world. He wants to protect the high
standards of Dhamma and his life is dedicated to that valuable end: 

`Don't waste your time letting any job become an obstacle, because for
the most part exterior work is work that destroys your work at mental
development \ldots{} These sorts of things clutter up the religion and
the lives of the monks, so I ask that you not think of getting involved
in them.'

For Ajahn Mahā Boowa the role of monastics is clearly defined. He says
it is those who have gone forth who are the `\ldots{} important factors
that can make the religion prosper and serve as witnesses [of the
Dhamma] to the people who become involved with it, for the sake of all
things meritorious and auspicious \ldots{}' It is the monastics who
preserve the actual teachings through their practise, and it is the
teaching which will help promote goodness and virtues in all those who
come in contact with it and take an interest in it. 

It seems that one of the dangers he sees for monks is over-association
with laypeople. Particularly in a country like Thailand, where so many
of the population have great faith and respect, I think he sees that
respect as a possible source of corruption for monks. Maybe that is
partly why he doesn't seem to over-value impressing laypeople with
politeness. He shows no interest in receiving respect for mere social
niceties, and when he teaches the monks he warns them of the danger: `No
matter how many people come to respect us, that's their business. In
practising the Dhamma we should be aware of that sort of thing, because
it is a concern and a distraction, an inconvenience in the practise. We
shouldn't get involved in anything except the contact between the heart
and the Dhamma at all times. That's what is appropriate for us\ldots{}'

In one of his talks he tells a story about a childhood experience. It
seems his father was a short-tempered man, and one day during the
evening meal his father began to turn on him and his brothers: `You're
all a bunch of cow manure. All of you. I don't see a single one of you
who is going to ordain as a monk and make something of yourself\ldots{}
none of you except maybe\ldots{} Boowa over there. He might have what
it takes\ldots{} but other than him, you're all a bunch of cow manure, 
you'll never come to anything.' The scrawny little Boowa was so shaken
by this first and only `compliment' from his father that he got up from
the meal and ran out of the house with tears welling up in his eyes. He
leant up against one of the water tanks with his head on his arm, trying
to collect his emotions and thoughts. There was no denying the feelings
rushing through him. His fate had been sealed. `It's decided,' he felt
in the pit of his stomach, `I'll be a monk, and if I'm going to be a
monk, I'll do it right. I'll spend however long it takes to pass the
third grade of Pāli studies and then I'll go off to the mountains and
spend my life meditating.'

And that's exactly what he did. It's perhaps this determined nature that
comes across most clearly in his talks, and it is what he seeks to
cultivate in his disciples. He makes frequent references to the need for
whole-heartedness in one's practise and candidly recounts his own
attitude as a young monk: `From the very start of my practise, I was
really in earnest, because that's the sort of person I was. I wouldn't
just play around. Wherever I would take my stance, that's how it would
have to be. When I set out to practise, I had only one book, the
Pāṭimokkha, in my shoulder-bag. Now I was going for the full path and
full results. I was going to give it my all, give it my life. I was
going to hope for nothing but release from suffering.'

He is uncompromising in the teaching and devotes a large part of its
contents to the dangers of being consumed by the changes of modern
society and all its trappings. I have heard the occasional person
comment that Ajahn Mahā Boowa is old-fashioned. I consider the comment
misses the point of Dhamma practise. Yes, he keeps to most of the old
traditions and may not be in touch with the younger generation and where
they are, but that means we must use our own wisdom to unearth where he
is and learn from that. There is a well-spring of wisdom in the forest
tradition which people are liable to overlook because it doesn't answer
their questions in the way they hope. Ajahn Mahā Boowa is not interested
in dressing up the Dhamma to make it tastier medicine to swallow, and
this may make him distasteful to the modern spiritual seeker. But what
he does teach is powerful and transformative for those who are willing
to commit themselves to it. 

One of his constant refrains to monks is to seek out the forests, the
hills, the lonely places, and meditate. Distractions not only interfere
with our cultivation of meditation; more seriously, they can delude us
into taking the worthwhile for worthless and the worthless for
worthwhile. Computers, books, worldly conversation -- it's not so much
that he sees them as bad in themselves, but he does see them as great
dangers to meditating \emph{samaṇas} and puts them in their proper
perspective. Regarding material society, Ajahn Mahā Boowa says: 

`We've gone way out of bounds. We say we've progressed, that we're
advanced and civilized, but if we get so reckless and carried away with
the world that we don't give a thought to what's reasonable, noble, or
right, then the material progress of the world will simply become a fire
with which we burn one another and we won't have a world left to live
in.'

And elsewhere he suggests:

`The teachings of the religion are an important means for putting
ourselves in order as good people living in happiness and peace. If you
lack moral virtue, then even if you search for happiness until the day
you die, you'll never find it. Instead you'll find nothing but suffering
and discontent.'

While there is almost certainly no forest monastery in Thailand that
could bring in anywhere near as much money as Wat Pah Baan Taad, Ajahn
Mahā Boowa still lives in a simple wooden hut. As he approaches his late
eighties he still eats at only one sitting. His monastery still has no
electricity running into it (though generators are used for certain
things), and thus the monks clean the Dhamma-hall by the light of candle
lanterns. Apparently the King of Thailand has repeatedly offered to
build a new modern Dhamma-hall for the monastery, but Ajahn Mahā Boowa
has insisted on keeping the original old wooden structure in use. So in
a sense he is a bit old-fashioned; he's certainly not intoxicated with
progress and development, because he sees virtue and insight as the most
significant factors in peace and happiness. 

When I first arrived at the monastery I was not without some critical
feelings. I wondered if there might not be a better system of running a
monastery than having hundreds of laypeople donating food to a monk who
was in the midst of receiving his fifteenth bowl-full that morning. But
when I looked closer, when I looked from within, I could see that Ajahn
Mahā Boowa is operating from a fundamental premise that nothing has more
value than the Dhamma, and indeed that its value is immeasurable. That's
what the fault-finding mind tends to forget. I would become frustrated
with the speed with which the monks there ate or their style of going on
alms-round (they walk very fast), because I failed to remember that
behind all this, in the backdrop, is an enlightened mind which is
creating the conditions for profound transformation within individual
minds. In one of his Dhamma talks Ajahn Mahā Boowa says the happiness of
the world is like the happiness of a prisoner, and the happiness that
comes from Dhamma is like the happiness of freedom: 

`[As we practise] the happiness that comes from the outside world
-- in other words, from the current of the Dhamma seeping into our
heart -- we begin to see, step by step, enough to make comparisons. We
see the outside world, the inside world, their benefits and drawbacks. 
When we take them and compare them, we gain an even greater
understanding -- plus greater persistence, greater stamina \ldots{} The
more peace we obtain, the greater the exertion we make. Mindfulness and
wisdom gradually appear. We see the harm of the tyranny and the
oppressions imposed by the defilements in the heart. We see the value of
the Dhamma, which is a means of liberation. The more it frees us, the
more ease we feel in the heart. Respite. Relief.'

That's what it all comes down to. He wants his monks and lay supporters
to practise. And when I read one of his books or listen to one of his
talks I can sense the intelligence that goes into them, and the will
that is doing part of the work for me by inspiring me and instructing
me. 

When the time came for me to leave Ajahn Mahā Boowa's monastery, I felt
richer and more connected to the forest tradition. For three or four
years I had had a strong appreciation of the Suttas, but not having
known Ajahn Chah in his teaching days, I had not had the good fortune of
training under a fully enlightened master. My visit to Ajahn Mahā
Boowa's monastery and subsequent visit to the monastery of one of his
great disciples, Ajahn Wanchai, opened my eyes to the Sangha, the
enlightened Sangha, as a living force in this world, shaping and
moulding the understanding and right intentions of those who come into
contact with it. 

\clearpage

\section{The Author}

Tan Yātiko did finally go on \emph{tudong}, spending some of his years as
a junior monk in the Dhammayut forest
monasteries of Udorn Thani, most notably with Tan Ajahn Wanchai. He
returned to Wat Pah Nanachat to offer his assistance to the community in
2003. For the next four years he was the senior monk at Dtao Dam Forest
Hermitage in Kanchanaburi Province. Following a Rains Retreat in his
home country of Canada in 2008, he joined the community at Abhayagiri
Monastery in California where he is still a resident monk, assisting Tan
Ajahn Pasanno in the running of the monastery and the training of the
monks. Over the years he has maintained the tudong spirit of an
unsupported alms mendicant, walking in Thailand, India and, most
recently, California.

\section{Note}

Venerable Ajahn Mahā Boowa passed away on January 30, 2011 in his
\emph{kuṭī} in Wat Pah Baan Taad.

