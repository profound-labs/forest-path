% Title: Faith in the Quest
% Author: Ajahn Jayasāro

As a child I was fascinated by quests: I loved reading about Greek
heroes facing tests of their endurance and ingenuity as they sought some
great treasure, and of the knights of the Round Table searching for the
Holy Grail. In my teens I discovered the Buddhist vision of life as a
spiritual quest.

At a time when I was finding the values of the world in
which I lived hollow and inane, the Buddha's declaration of the search
for freedom from ignorance and attachments as the truly Noble Quest
seemed irrefutable to me, and it still does today. I became convinced
that any human endeavour alienated from the Noble Quest, no matter how
conventionally worthy it may be, was ultimately trivial. 

As monastics we take on the form of the Buddhist monk with its
discipline and regulations, not as a new identity, but out of a
recognition that in our chosen quest we need a certain degree of
structure and support. Any quest worthy of its name involves facing up
to demons, avoiding quicksands and disregarding sirens. With the backing
of Sangha and faith in the value of the quest we can achieve this. 

Faith has been an unpopular word in some Western Buddhist circles, 
especially with those people who have felt bitter about their theistic
upbringing and seen in Buddhism something more `scientific'. For myself
I like the word, and find `confidence', the other popular choice as a
translation for `\emph{saddhā}', too mundane. Anyway, however this term
is rendered into English, I think we must first acknowledge that we
can't do without it. Nobody can prove that there is such a thing as
enlightenment, but if we don't have faith that there is our practise is
unlikely to go very far. Faith clarifies the goal, focuses our efforts
and fills us with energy. Ultimately it is wisdom rather than faith that
moves mountains, but it is faith that impels us to move them in the
first place, and faith that sustains us through the inevitable
frustrations that dog our efforts. 

There are so many things that human beings can search for in life: 
security, wealth, power, fame, respect, love, immortality and even
annihilation. Many people spend their life searching without ever
clarifying exactly what it is they want; all they know is that whatever
it is, they haven't found it yet. Some give up their quest, some turn to
drink or drugs, others become bitter and cynical. Many people search for
the ultimate experience, doing outlandish things merely because nobody
has ever done them before. They want `the challenge', they crave the
adrenaline. They want to stand out from the crowd. Everyone is afraid
that their life doesn't mean anything. But all experience, from cleaning
out a dingy drain filled with human hair to sexual bliss in Shangri-la, 
lies within this very narrow circumscribed realm which the Buddha called
the \emph{āyatanas}. We see that none of the exotic, mind-boggling
things that people get up to in the world ever transcend the sense
spheres. No matter how much money people have and how well-endowed they
are with worldly blessings, wherever they go they are still stuck fast
within the realm of the sense bases. No matter how exalted the aesthetic
experience, nobody is ever going to be able to see with their eyes a
form which is anything other than a form. Form is just form. It arises
and then it passes away. That's all it knows how to do -- quite pathetic
when you think about it. \emph{Saṃsāra} just doesn't live up to the
hype. When we understand a form as a form, the content of that form
starts to lose its power to enthral or enslave us, depress or enrage us. 
Sounds are only ever just sounds. No matter what sound it might be, the
nature of a sound is always the same -- it's just this much, unstable
and inconstant. Sound is just sound. Odours are just odours. Flavours
are just flavours. Physical sensations are just physical sensations. 
Thoughts, moods, emotions are just that, and they can't ever be more or
less than just that -- impermanent, empty and ownerless. Not one of the
\emph{āyatanas} can be sustained and enjoyed for an indefinite length of
time. 

This, I feel, is where the whole concept and idea of renunciation starts
to become so compelling, uncommon sense. Racing around, struggling and
striving, just to be able to experience more forms, more sounds, more
odours, more flavours, more physical sensations, more emotions, thoughts
and ideas, starts to appear tiresome. Is it really a satisfactory way to
spend a human existence? Prince Siddhārtha and countless men and women
after him have started their spiritual quests with the conviction that
there must be something more to life than this. 

But in the world, pursuing sense desires is the norm. Enjoying certain
kinds of feeling and avoiding others is the aim. The more experiences
you have, the fuller your life is considered to be. Intensity and
passion are seen as ends in themselves. From an early age we absorb the
idea that romantic love is the supreme fulfilment -- poetry, novels, 
television, movies all tell us the same thing. Freedom is usually
considered to lie in consumption; the more money we have, the greater
the choice of hamburger fillings available, the freer, it seems, we are
to consider ourselves. Even financial markets are called free. In the
monastic life, however, we are willing to look into everything carefully
in order to see what is truly what. Our role in the world is to step
back a little from the pace and pressures of the world, to investigate
and penetrate the nature of existence. We do this with the understanding
that the quest is not exclusively intellectual. Success will depend a
great deal on our moral integrity and emotional maturity. In the
uncovering of truth, our effort is not to try to make all our ideas
conform to what's written in the Buddhist books, but to use the
teachings as working hypotheses. Do they explain human experience
adequately, completely? With the power of \emph{sati, samādhi} and
\emph{paññā}, we learn to see the truth, just as it is. 

What we're offered by the Buddha is a teaching that opens our eyes to
look at life and at experience. We investigate: `What is this?' `What am
I?' `What is this life \ldots{} this body \ldots{} this mind?' And
humble acknowledgement that we don't know is the motor for the search. 
We cherish a faith in the ultimate value of the search for a direct
experience of truth. Ajahn Chah once spoke of a bird waking up and
realizing it's in a cage. No matter what the cage is like, even if it's
a most beautiful ornate cage with gold bars, once the bird clearly
understands what a cage is and that liberation lies outside it, it can
never be content with its old life again. We undertake the spiritual
life to seek freedom from the confines of the cage. 

The path we follow, the way we live our life as \emph{samaṇas}, only
becomes really sensible and meaningful in light of the aspiration for
transcendence. We must believe in the vision of freedom outside the
bonds and attachments founded on a sense of self. That freedom comes
from fully understanding, moment by moment, the nature and mechanics of
bondage. We practise to understand the nature of the five
\emph{khandhas} and the six \emph{āyatanas}. The reason why attachment
to forms, sounds, odours and all the \emph{āyatanas} is so treacherous
is, quite simply, because they don't last. Today I went to visit our lay
supporter Mae Jorm in the hospital. Her cancer is now at the stage where
she doesn't even have enough energy to swallow water. Her throat was dry
and she desperately wanted some water, but when it was poured into her
mouth she couldn't swallow it. The water started to drip from the side
of her mouth. It was a heartbreaking thing to see. Her body and sense
organs are shutting down. I have known this woman since I was a novice
and soon she will be dead. 

Our eyes start to go, our ears start to go, and even if the forms are
still there, they are not for us any more. Sounds were there before we
were born and will be there after we have died. No one has ever become
free through looking at beautiful things. No one has ever become free, 
liberated from the vicious round of birth and death, through listening
to beautiful music or hearing beautiful sounds. We might have become
quite peaceful for a short while, but there was no wisdom -- we were
merely fine-tuning the quality of our distraction. 

With basic faith and confidence in the teachings of the Buddha, we get a
foretaste that there is such a thing as freedom, and that it is
realizable. It is possible. There is hope. There is a path. But as long
as there is still attachment to the five \emph{khandhas}, as long as
there is still clinging to delight in the physical body, to delight in
feeling, perception, thoughts, ideas, emotions and sense-consciousness, 
there is still delight in \emph{dukkha}. The Buddha made this very
clear. The attachment to the five \emph{khandhas} is attachment to
\emph{dukkha}. One who is attached to \emph{dukkha} cannot be free from
\emph{dukkha}. 

So there is really nowhere to go -- nowhere new, anyway. Wherever we
go, we're always going to be in exactly the same place. We'll be in a
place where there are forms, sounds, odours, tastes, physical sensations
and mental events. This is it. We've already got the whole package. No
matter whether we're trekking through a pristine Himalayan valley or
struggling through a crowd in a frowning city, we have the same work to
do. We can see the truth of things wherever we are. Of course, certain
environments are more conducive to the work than others -- that is why
the Buddha established monasteries and a monastic order -- but even so, 
wherever we are, in whatever posture we find ourselves, we can do the
work of developing awareness, turning the light within. We seek to learn
from whatever it is, learning to see things as dhammas rather than as
`this person', `that person' and `this and that'. We apply ourselves
steadily to the process of de-conditioning and re-education. Stopping
the rot. Making a fresh start, again and again. Infinitely patient. 
Until the work is done. 

With the faith that the Buddha was fully enlightened, with the trust
that the teachings which he shared with human beings and \emph{devas}
for forty-five years are true, and with the conviction that the
\emph{ariya-sāvakas} truly penetrated those teachings, it follows that
each one of us, wherever we're from, wherever we were born, whatever
language we speak, man, woman, old or young; we all bear within us this
capacity to realize the truth. Human beings can attain Awakening, can
realize Nibbāna, because we're fish in the water. Why shouldn't fish be
able to understand what water is? It's all around us, it's all within
us. All we have to do is learn how to open our eyes. 

It's common amongst Buddhist practitioners, however, to realize that
their strong sense of \emph{saddhā} or faith in the Buddha, Dhamma and
Sangha, a profound trust and confidence in the truth of the Buddha's
teachings, is not matched by faith in their capacity to realize that
truth. But without this faith in ourselves the five \emph{indriyas} have
no opportunity to mature. This lack of faith in our potential for
enlightenment is crippling and unwise. The doubt is based on a mistaken
way of looking at ourselves. Swallowing the whole myth of the
independent `I' gives us spiritual indigestion. We can't force ourselves
to have faith, and we don't need to. We merely have to remove the wrong
thinking that prevents faith from arising, and start paying more
attention to our experience. 

Our tradition makes an important distinction between two levels of
truth: the conventional and the absolute. The term `conventional truth'
refers to the conditioned, phenomenal or relative sphere. In this sphere
it is valid to talk about `self', `human beings', `monasteries' and
`monastic orders'. `Absolute truth' refers to the way things are, 
unmediated by concepts and bias; in this sphere language and thought are
transcended. The wise person uses conventional truths in order to
communicate, but he is not fooled by them. When this way of exposition
is understood, certain Buddhist teachings which would otherwise remain
quite puzzling become clear, in particular those regarding `self'. 

A number of references in the Suttas to `self' -- for example, the
famous saying that the self should be the refuge of the self and the
instructions on various kinds of self-development -- are expressions on
the conventional level. They do not clash with the `absolute' truth of
\emph{anattā}. The teaching of \emph{anattā} does not mean that the
Buddha is refuting the self on the conventional level; he is simply
reminding us not to confuse a useful social fiction with ultimate
reality. There is no independent, timeless `I', no unmoving centre of
experience, no soul-entity, no separate ego-identity, as we assume. 
However closely we look we cannot find `one who acts', `one who thinks', 
`one who does' and `one who wanders from life to life'. But there is a
conventional `I'. One teacher puts it well: `There is a self; it's just
not permanent'. 

Our discouragement in the practise frequently comes from trying to
imagine how this limited `I' could possibly realize the unlimited. How
could this bounded `I' realize the unbounded? Having posed a question
based on false premises (that the `I' is real) we naturally conclude
with a false answer, `No, I can't see how realizing Nibbāna could ever
happen; it just doesn't seem possible'. In other words, how could little
old me ever realize something so marvellous? The gap seems too wide. How
can this person realize the truth? 

Well, that's exactly the point, isn't it? It can't. This `person'
doesn't realize the truth. Rather, it's through understanding what this
`person' is, that truth is revealed. The realization of the Third Noble
Truth leads to the unveiling and manifestation of Nibbāna. In the words
of the Suttas, it involves `upturning something that has been
overturned'. It is a `shining of light in the darkness'. Nothing new is
created; what occurs is a radical re-appreciation of experience and a
recognition of something which has always already existed. The deathless
element is also a birthless element. It is not something that is brought
into existence. Instead, those things which conceal or envelop it are
removed. If we can grasp this point, we can feel a new surge of energy. 
We see that any sense of inadequacy we might feel is founded on
attachment to the conventional self as being ultimately real. At this
point our effort and energy, our persistence in practise are greatly
strengthened, and the nagging doubt about our capacity to follow the
path to its end may even disappear in a flash. We start to give what it
takes. 

If doubts arise in practise, investigate the discouragement, the
uncertainty and hesitancy as mental states. Watch when questions like, 
`Could I possibly reach the same level as Ajahn Chah and Ajahn Mun and
all those great teachers?' arise. If we dwell on `me' and `my' personal
history, `my' foibles and idiosyncrasies, it seems ridiculous even to
imagine that we could ever possibly attain the same level as monks like
that. But the essence of the practise is not -- what a relief! -- the
gradual perfection of character and personality; it is the understanding
of character and personality as conditioned phenomena. Certainly, 
unwholesome features like selfishness, jealousy, anxiety, etc. 
inevitably abate through practise, but the idea is not to mould our self
into a new, more `spiritual' being. Character and personality are not
and never have been who we are. They are not self. They are not anything
ultimately real. So we have to learn to stand back from the idea of
becoming an enlightened being. Otherwise, when we contemplate, `Is
enlightenment really possible for me?' and we are unsure -- is this
hubris, worldly ambition, spiritual materialism? -- we may decide, `No, 
not for me' and then falsely dignify that wrong view by calling it
humility. 

\emph{Asmimāna}, the subtle conceit `I am', is the crux of the problem, 
the spoiler, the fly in the cosmic soup. It's the most difficult thing
to see through, because the self-assumption is the foundation on which
unenlightened human beings build their whole world-view. The existence
of an independent self-existent `I' seems obvious; everyone takes it for
granted, it's common sense. That's why it's so hard to face up to the
realities of life -- birth, old age, sickness and death -- we see them
as things that happen to `me'. It's `my' dilemma, it's `my' problem. `I'
was born, `I'm' getting old and `I'm' going to die. In Thai the word
\emph{samkan} means `important'. But sometimes the word is used as a
verb, to \emph{samkan tua}, which means to give importance to self. And
we give importance to self in so many ways, not just in our arrogance
and pride, but also in our humility. Or in being anything at all. We see
this clearly when we compare ourselves with others, considering
ourselves as better, as equal, as worse and so on. This is where we
`\emph{samkan tua}' This is where we most prominently uphold the myth of
self. 

The Buddha said that the practise which most directly opposes or
undermines \emph{asmimāna} is \emph{aniccā} \emph{saññā}, the
contemplation or constant recollection of transience and change. 
Investigation of the impermanence and inconstancy of phenomena enables
us to see that those things we've always assumed to be solid are in fact
not solid; that which we think is permanent is not permanent at all. 
This solid `I' who does things, has experienced things, has highs and
lows, ups and downs, is not a coherent entity at all. If you take a
light, a candle, a torch, and you wave it around in a circle fast
enough, you get the illusion of a fixed circle of light. But in fact
there is no such thing. The same principle applies to our investigation
of the mind and the five \emph{khandhas}. Through the practise of being
fully awake and alert in the present moment, the truth of change becomes
manifest. Mindfulness slows things down, at least subjectively. Suddenly
we have time. There are gaps. There is a sense of things not moving so
fast any more. And when there is that penetrative awareness and presence
of mind, there is the opportunity for circumspection and for the
recognition of things arising and passing away. The arising and passing
away of the five \emph{khandhas} may be seen as a simple impersonal
truth. We know consciousness as just that, without having to add
anything to it. With mindfulness and wisdom we don't make a big story
out of things any more. We experience the episodes of our life more as
haikus than as scenes from a fat and portentous autobiographical novel. 

Faith is what keeps us going through the difficult times. Faith and
endurance. Human beings in extreme conditions show an incredible
capacity for endurance -- the prisoners of war on the Death Railway in
Kanchanaburi during the Second World War are a good example -- but
whether people survive ill-treatment and deprivation seems often to lie
more in whether they want to endure, rather than whether they can. Those
who don't see the purpose or the value of endurance are alienated from
their innermost resources and die. They lose their will to live, or we
might say their faith. In spiritual life, our capacity to endure through
the ups and downs, the dark nights and deserts and sloughs of despond, 
is dependent on our wanting to do so. And if we want to, it's because we
believe it to be worthwhile. This is faith. 

Blind dogmatic faith was sharply criticized by the Buddha. He taught a
faith that welcomes the critical faculty and does not claim to be more
than it is. He pointed out that it is possible to have a strong faith in
something and be completely mistaken. The strength of the feeling is not
a proof. He taught us to take the teachings as working hypotheses and
then put them to the test of experience. Observe yourself and the world
about you. Many years ago I experienced a small epiphany, one that
greatly increased my faith in the capacity for radical change in my
life. What I experienced wasn't an intellectual proof of that capacity, 
but it had a deep emotional significance for me which has not faded. 

I was travelling in a bus through a huge desert. The journey was to take
fifteen to twenty hours and there was almost nothing to be seen on
either side of the bus, just sand and rock. At the time I was in my late
teenage years and at a pretty low ebb in my life. I'd been in India
practising meditation and just starting to feel that I was making some
progress, but then I had to leave because my money ran out. During the
many adventures I had as I travelled westwards, there was this
underlying feeling that I had squandered a marvellous opportunity. 
Something had been lost. So I was travelling through a seemingly endless
expanse of desert. Looking out of the window, all I could see was just
sand and rock everywhere. I remember thinking, `That's me, just sand and
rock, as far as the eye can see \ldots{}' Every time I looked out of the
window this thought just kept coming up: `That's me, sand and rock.'
Then I must have dozed off. During the night quite an unusual thing
happened; there was a rain storm in the desert. As I came round I could
feel straight away that it was cooler and fresher. I looked out through
the window and couldn't believe my eyes. Throughout the desert and on
the rocky outcrops were these beautiful flowers, a profusion of the most
vibrant yellow, mauve and turquoise blooms! It struck me as a miracle. 
How could these flowers exist in such a place? Where did they come from? 
Just a few hours ago there were endless stretches of sand and rock. Now
there were beautiful wild flowers everywhere! The flowers were not big
flowers, just tiny little flowers, but they sprang up in such a short
time. And as I was already in a metaphorical frame of mind, the beauty
and surprise of the experience made me think, `I've got all those little
flowers in my heart, they're dormant in my mind, and all they need is
just a little bit of rain on them'. And so with that thought a big smile
came over my face and I felt, `Yes, I can do it'. 

Even the desert can sprout flowers. Even when our mind is feeling dry, 
lifeless and dull, if we just keep at the practise, continue the
development of the five \emph{indriyas}, sprinkling the water of Dhamma, 
of mindfulness, clear comprehension and sincere effort, skilfully
applying all the Buddha's wonderful teachings that we've learnt, then we
can create freshness and beauty in the mind. There's always a way
forward. There's always a way to peace. This is the hope that the Buddha
held out to us. All mental conditions are just that: they're conditions. 
They change. And we can influence the nature of that change through
seeing life just as it is, by doing something wise about it through our
study and practise of Dhamma. 

So with \emph{saddhā} in the path, in the quest for truth, seeing its
value, there arises a \emph{viriya} independent of all the passing
feelings of inspiration, depression, like, dislike and pleasure --
those are all part of it, they're not something outside of it. Practice
is developing this right, wise attitude to practise and not taking all
those feelings so seriously -- not taking the person who seems to
experience them so seriously. 

Practically speaking, it is \emph{sati} and \emph{samādhi} that enable
us to see what is what. The practise of \emph{samatha} meditation, 
concentration on an object, is in fact a kind of mindfulness practise. 
It's training our awareness to maintain an uninterrupted conjunction
with an object, for example, the breath. Mindfulness of breathing has
been called the king or the crown jewel of meditation objects, because
it may be used both as a means of calming the mind and also for the
direct penetration of \emph{aniccā, dukkha} and \emph{anattā}. Arousing
the feeling that the breath is more important, more interesting, more
fascinating than anything else in the world makes the practise progress. 
We generate the deep faith that mindfulness of breathing can take us to
liberation. This faith energizes the mind. 

When the mind becomes calm, notice how your attitudes and values change. 
There is a recognition that stuffing the mind full of thoughts and
fantasies is pointless, that dwelling in even subtle forms of anger, 
ill-will or greed is painful and a waste of time; that searching the
universe for pleasant sensory experiences is demeaning and irrelevant. 
You wonder that you never thought about getting out of these traps
before. \emph{Samādhi}, the deep peace and happiness of mind, brings
forth a very different kind of logic from that of the busy mind. 
Suddenly there is a sense of sadness for the time that you've allowed
the mind to hang out with the hindrances, all the time that has been
squandered! You think, `How could I have been so foolish?' To the
peaceful mind, only peace makes sense. 

The mind stabilized by \emph{samādhi} loses its habitual reaction to
objects, which is to rush towards the pleasant and away from the
unpleasant. Without \emph{samādhi} the mind has no home. It has no
dwelling-place, it has nowhere it really wants to be. And so when faced
with objects, the mind rushes around, moving towards the pleasant and
away from the unpleasant, and hovering around the neutral, not quite
sure whether to move towards it or away from it. But peace of mind has a
stabilizing effect on this process. Suddenly it's almost as if the mind
is too content; it just can't be bothered to make a fuss about things
any more. The mind in \emph{samādhi} is quite happy to be where it is, 
at home. 

But the mind is not peaceful all the time. Awareness of the value of
\emph{samādhi} can be lost again. When we're in the hindrance-mode, then
\emph{samādhi} seems so far away. All the teachings about peace of mind
seem like pious platitudes and the practise doesn't really gel. We may
even find ourselves trying to avoid meditation, though we still aspire
to its fruits. But if we are willing to go against the grain, once the
mind starts to become calm and \emph{sati} and \emph{sampajañña}
increase, that kind of negative thinking appears foolish once more. The
pacification and clarification of the mind's intrinsic power seem so
obviously the most intelligent thing that we could be doing. We see how
state-specific are our thoughts about life. 

If the mind takes joy in its object, chooses it wholeheartedly, then
what starts to become clear is the inherently peaceful nature of the
mind. The meditator experiences clarity, transparency, brightness and
purity; he connects with the strength, resolution and firmness of the
concentrated mind. At the same time, with \emph{samādhi}, we are aware
of a flexibility, suppleness and malleability in the mind. Put into
words, that sounds self contradictory, doesn't it? How is that possible? 
How can the mind be both firm, resolute and rock-solid, and yet at the
same time flexible and pliable? Well, why not? It's not a logical
theorem. It is `\emph{paccattaṃ}', to be realized by each person for
themselves. 

With the practise of \emph{samādhi} the meditator samples the initial
wonders of the inner world. He reaches the gates to the marvellous, 
something few human beings ever experience. Here is where the mind
begins to intuit its full power and potential, and is exhilarated by
that. The meditator sees how unsatisfactory and superficial ordinary
sense-consciousness is -- it's as if human beings are just skating
around on dirty ice looking for water, never aware of the beautiful, 
cool flow beneath their feet. 

As the mind becomes imbued with \emph{sati} and \emph{samādhi}, the
powers of this penetrative awareness can be applied. In accordance with
its nature, the mind will move and flow towards the objects of
investigation and contemplation. The mind emerging from \emph{samādhi}
is naturally ripe for the emergence of \emph{paññā}. With \emph{paññā}, 
what becomes most clear to us is that every aspect of our experience, 
everything that we can perceive and conceive, has the same value. We
enter a calm egalitarian land. Everything does exactly the same thing: 
it arises and then passes away. For the first time the nature of
experience far outweighs the significance of its content. We make a
radical switch or revolution, from obsession with the contents of
experience to the cool, clear-eyed appreciation of the process or
contour of experience, this rising and passing away. With insight and
understanding of the process of rising and passing away, it's here that
\emph{asmimāna}, the upholding of the idea of self, raising the flag of
`me' and `mine', starts to be undermined. 

So as we progress down the path, we come to understand that part of our
development as human beings is the gradual maturing of our understanding
of happiness. The increasingly subtle and profound forms of human
happiness developed through the five \emph{indriyas} of \emph{saddhā, 
viriya, sati, samādhi} and \emph{paññā} are sometimes invisible to
others, especially those who do not practise. They're not objects of
possession and they're not founded upon the \emph{āyatanas}, and yet
these qualities truly sustain the human heart. But if we find ourselves
trapped in that no-man's land where we have given up some of the coarser
pleasures based on gratification of sense desire, but do not yet have
any real access to the higher, more subtle and refined pleasures, 
enjoyments and happiness of the path, then we need to be very patient
and dwell in faith in the Buddha. 

Even though I don't think my critical faculty is lacking in any way, 
after devoting most of my life to studying and practising the Buddha's
teachings to the best of my abilities, I have yet to find a single
teaching that I have been able to disprove. This gives me a great deal
of faith in those aspects of the Dhamma that I have not yet verified. 
It's like a map. If you have found it to be trustworthy in one area of
the landscape, you find it unlikely to be at fault in another. The
Buddha teaches that the practise of Dhamma brings happiness to the human
heart. We trust the Buddha's teachings not by dismissing doubts, but by
putting our life on the line. Faith does not entail the mere acceptance
of a philosophy. Buddhist faith is the faith to do. It is a trust in our
capacity, a belief in our own potential; something we can put to the
test. The daily practise may sometimes feel little more than a stumble
or crawl, but through faith our underlying effort and sincerity is
unwavering. Eventually, attainment of the goal is assured. 

