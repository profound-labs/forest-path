
Wat Pah Nanachat (The International Forest Monastery) is situated in a
small forest in the north-east of Thailand, about fifteen kilometers
from the city of Ubon Ratchathani. In 1975 Ajahn Chah established it in
order to give foreigners who do not know the Thai language and culture
the possibility of a traditional monastic training. English serves as
the primary language of communication and instruction. Our community
consists of monks, novices and postulants from a wide range of
nationalities. There is no permanent nuns' community at Wat Pah
Nanachat. Women interested in a monastic commitment are invited to
contact our affiliated nuns' community at Amaravati Buddhist Monastery,
Great Gaddesden, Hemel Hempstead, Hertfordshire HP1 3BZ, Great Britain.

Although Wat Pah Nanachat is not a meditation centre, there are
facilities for a limited number of male and female guests to stay at the
monastery and practise with the resident monastic community. We like our
guests to follow the daily routines of the monks as much as possible,
and join in all communal meetings and work activities. As the teachers
of the forest tradition stress, in monastic life qualities like
cooperation, respect and self-sacrifice facilitate both communal harmony
and individual growth in the practice. Generally the training at Wat Pah
Nanachat aims to follow the Dhamma-Vinaya, the teachings and code of
monastic discipline laid down by the Buddha, respecting both their
letter and their spirit. The monastic life encourages the development of
simplicity, renunciation and quietude. It is a deliberate commitment to
this way of life that creates a community environment where people of
varied backgrounds, personalities and temperaments can cooperate in the
effort to practise and realize the Buddha's path to
liberation.

\section{Practice Schedule}

In general guests have many
hours a day for study and meditation practice, so to make the best use
of the situation it is advisable to have had previous meditation
experience in a retreat setting and exposure to Buddhist teachings.
Below is an outline of the standard daily routine, which varies from
time to time.

%{[}h{]}
%
%\ctable[pos = H, center, botcap]{ll}
%
%{% rows
%\FL
%3:00 AM & Morning wake-up bell
%\ML
%3:30 AM & Morning meeting: chanting and meditation
%\\\noalign{\medskip}
%Dawn & Monks go out to surrounding villages on alms-round. Lay guests
%sweep the monastery or help in the kitchen.
%\\\noalign{\medskip}
%8:00 AM & The meal (followed by short instructions and teachings by the
%Abbot)
%\\\noalign{\medskip}
%9:30 AM & Chores period
%\\\noalign{\medskip}
%4:00 PM & Afternoon drink
%\\\noalign{\medskip}
%6:15 PM & Evening meeting: chanting and meditation
%\LL
%}

The schedule may be supplemented by periods of group practice,
communal work or Dhamma instructions according to the needs of the
community. After the meal the Abbot or a senior monk is available to
receive visitors and resident guests and answer questions. Four times in
a lunar month, on Wan Phra (the Buddhist holy day), the community
observes a late-night vigil, during which there is the opportunity to
discuss aspects of Dhamma practice with one of the senior monks.

Much of the day is reserved for private practice, using the time for
sitting and walking meditation in either one's private hut in the forest
or one of the meditation halls. Regarding meditation instructions at Wat
Pah Nanachat, rather than utilizing only one particular technique, we
aim to have our practice include all aspects of daily life, however
simple and ordinary, as opportunities to develop mindfulness and other
spiritual qualities such as diligent effort, joy, contentment, patience
and faith. In time, the virtuous qualities that grow out of such a
training gather strength and contribute towards deeper peace and
concentration leading to insight and the growth of liberating
wisdom.\\\\\textbf{The Buddhist lay training guidelines (precepts)}

Lay guests who stay at Wat Pah Nanachat are expected to abide by the
traditional eight Buddhist precepts. The first five form the basic
guidelines for conduct leading to harmony and self-respect. The other
three precepts encourage a spirit of renunciation and simplicity and are
among the fundamental principles of monastic practice.\\\\\textbf{The
five training precepts}:

Harmlessness: to refrain from intentionally taking the life of any
living creature.

Trustworthiness: to refrain from taking anything that is not given.

Chastity: to refrain from all sexual activity.

Right Speech: to refrain from false, abusive, malicious or disharmonious
speech and

worldly gossip.

Sobriety: to refrain from taking intoxicating drinks or drugs (smoking
is prohibited

at all times at the monastery).

The three renunciation precepts:

To refrain from eating after midday. The monastery practice is to eat
one meal a day in one bowl at one sitting. This frees time for
meditation and enhances simplicity of life.

To refrain from using entertainment such as music, dance, playing games
and beautifying or adorning the body with jewellery or makeup. This
assists in focusing the mind's attention inwards towards Dhamma.

To refrain from using high or luxurious beds or seats and from indulging
in sleep. This develops the qualities of wakefulness, mindfulness and
clear awareness in all postures and all activities throughout the day.\\
These training precepts are guidelines for good conduct in body and
speech and provide a necessary foundation for the development of
mindfulness, clear comprehension and meditation in our endeavour to
cultivate the Noble Eightfold Path. The precepts serve to promote
harmony within the community through restraining unwholesome speech and
action. These fundamental principles of training cultivate the
self-discipline necessary for spiritual development and are taken up as
an act of deliberate personal choice and initiative.

\textbf{Staying as a guest}\\\\If you wish to come and stay at Wat Pah
Nanachat, you need to write in advance to the guest monk and allow
several weeks in which to receive a written response. We only have
limited space for guests and are often booked up, so it is good to write
well in advance. Please understand that it is the wish of our community
to not be publicly available by telephone or email. Guests are accepted
initially for three days. If they wish to stay longer, they can consult
the guest monk or the Abbot. The best time to arrive is before 8:00 am,
in order to take part in the meal and meet with the guest monk.

\\Resident lay guests in Wat Pah Nanachat wear traditional Thai lay
monastic attire: loose white and long trousers with a white shirt for
men, and a white blouse and long black skirt for women. Men staying
longer than one week are asked to shave their heads, beards and
eyebrows. Guests are advised to be in good physical and mental health
and to have health coverage or travel insurance. If you have previously
had any serious mental illnesses, please inform us openly about them, so
we can be sure that your stay in the monastery won't give rise to major
problems for you and the community. There is no malaria at Wat Pah
Nanachat.

\\While the monastery provides bedding and a mosquito net, guests are
expected to supply other requisites (e.g.~a good flashlight/torch, an
alarm clock, flip-flop sandals, candles, mosquito repellent and
toiletries). A padlock for locking away personal valuables is very
useful. The monks are happy to share food and drinks that are offered to
them with the lay guests each morning, but as it is part of the
renunciant tradition to accept whatever is offered, they are unable to
arrange any special diets for the guests or residents. Please either do
not bring electronic gadgets like mobile phones, portable computers,
cameras, etc. with you, or lock them away in the monastery safe. These
things create a worldly atmosphere which impinges on the simple
meditative lifestyle in the monastery. Also, this is a strictly
non-smoking monastery. Please note that the financial expenses of the
monastery are completely covered by donations out of faith and free will
from our lay community, whether local or international.

If you would like to visit and stay at Wat Pah Nanachat, please write a
letter (suggesting possible dates) to:

The Guest Monk\\Wat Pah Nanachat\\Bahn Bung Wai\\Warin Chamrab\\Ubon
Rachathani 34310\\THAILAND

Some more information about life at Wat Pah Nanachat can be found on the
monastery's website:

www.watpahnanachat.org
