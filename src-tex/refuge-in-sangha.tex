
\setChapterNote{Two local lay supporters talk about their relationship to the wat.}

\chapter{Refuge in Sangha}
\markright{\chapterAuthor}

\emph{Many people in Thailand have ties to Wat Pah Nanachat. The
villagers of Ban Bung Wai, even those who rarely go to the wat except
for funerals, see the monastery as `theirs'. They feel a sense of pride
about it, and a sense of responsibility. Then there are the people from
surrounding villages, the local town of Warin and the city of Ubon, who
regularly come to make merit, keep the Eight Precepts on Wan Phra days
or practise in the monastery for longer periods. Lay supporters from
Bangkok and other provinces may also come up to stay in the wat during
their holidays. This section consists of the words of Mae Samlee, a
village woman living in a house in the fields outside the monastery, and
Por Khroo, a primary school teacher from Ubon.}

Mae Samlee

`The pain's not been so bad really. My husband ordained as a monk for
fifteen days to make merit for me {[}she smiles at him warmly{]}, and
I've been feeling better ever since. It's just the past three or four
days that have been a bit more difficult.'

Mae Samlee is 55. She has cancer of the spleen which is metastizing. She
has spent many months in hospital over the past year and had two
operations. Now she is back at home in her house among the rice fields
between the monastery and the main road.

`I've been going to monasteries for as long as I can remember. When I
was a young girl my mother would always take me with her when she went
to make merit. After I got married I used to go to Wat Pah Pong on
Observance Days. I loved it so much: making food for the monks,
listening to Dhamma talks, meditating. Then we moved to Kanchanaburi and
stayed there for six years. It was a rough place. My husband became the
village headman and everywhere he went he had to carry a rifle and a
pistol, he said one weapon wasn't enough. Then he read a talk by Luang
Por Chah and we decided to come back to Ubon. My brother-in-law lives in
Bung Wai. He wanted us to come here and said he'd look around for some
land for us. I said,''I don't care how expensive it is, please find us
land close to the monastery, so that I will be able to go every day,
even when I'm old.'' Everything worked out: we got this plot of land
right in front of the wat, we built a house on it and now {[}she smiles
widely{]} it looks like I'm not going to have an old age after all. I
must admit that sometimes I wish I had accumulated more merit in my
life.

`I meditate whenever I can, whenever the pain is not so bad. I chant in
the morning and evenings. Actually, these days I often do the evening
service at three o'clock in the afternoon! The pain usually comes on in
the evening, you see, and I'm afraid it will stop me from chanting. But
today, I've been so excited all day waiting for Ajahn Jayasāro to come
to visit that I've felt fine all day.

`I felt homesick when I was in the hospital in Bangkok having my
operation. It all took such a long time. On the days when I could sit
up, I did the morning and evening chanting normally. When I couldn't sit
up, I chanted as best I could lying down. Then that first time Ajahn
Jayasāro walked into the ward I felt so happy! It was such a wonderful
surprise. And he brought me a little picture of Luang Por Chah, too, to
put on the table by my bed. After he'd gone the other patients were
really curious. ``Who is that Western monk? Where does he come from? How
do you know him?'' I felt much better after he came. Things didn't seem
so bad. I remembered the things he taught me and they were a refuge to
me. I always kept in my mind the virtues of the Buddha, Dhamma and
Sangha.

`Next Wan Phra, if I'm feeling any better, I hope I will be able to go
to the wat. I don't want to miss the morning talk. It's another few days
yet. I hope I'll feel a bit better by then.'

Por Kroo

Por Kroo is a male primary school teacher aged 57 and a member of the
monastery's lay committee.

`I've been going to monasteries since I was a child, when my father used
to take me. I first went to Wat Pah Pong way back in 1959. But it was so
difficult to get there that I didn't keep it up. I was disillusioned by
the monasteries near my home in Ubon. One day I went to invite some
monks from the local monastery to come and eat in my house. I happened
to see them acting in ways completely unfitting for monks. I lost all my
faith. I didn't go near monks again for three or four years. I didn't
even put food in monks' bowls on alms-round. During that time I felt as
if I'd lost my sense of purpose, that I had no refuge and was just
drifting through my life. Then I thought of Wat Pah Nanachat, and it
happened that that year --- it was 1983 or 1984, I think --- the
ministry made it compulsory for teachers to go to listen to Dhamma
teachings in monasteries. Our group came to Wat Pah Nanachat. I walked
into the \emph{sala}, and the first thing I saw was Ajahn Pasanno and
Ajahn Jayasāro sitting on the \emph{āsana} talking together. It was such
a moving sight for me to see two Westerners so restrained and composed
in their bearing; I felt a new wave of inspiration and I started coming
regularly. I was impressed by the dedication of the monks. I came to
know for the first time that monks are not supposed to use money, and I
began to realize the way monks are supposed to live.

`The result of coming to the wat over the years that I see most clearly
is that I'm a lot more calm and patient than I used to be. I 'm
basically quite a forceful, headstrong kind of person. Listening to the
Dhamma and the teachings of the Buddha and trying to put them into
practice, I've seen my mind cool down and become more peaceful. I don't
lose my temper or get angry when I'm provoked in the way that I used to.
My temper has improved a lot. Also, I feel more mindful in my daily
life. I always tell people that I've been to many monasteries, but it's
here that I've received the most beneficial teachings.

`I've had some difficult times at work over the past few years. When I
see something wrong I can't always keep quiet. On occasions when I've
spoken up against corruption, I've been slandered and victimized by my
seniors. At home too it's been hard. My wife was in a motorcycle crash
last year. Worse than that, a few years ago my son died. One day he
returned from work in Bangkok in his boss's car. His boss had driven him
up himself. He said my son had an inoperable brain tumour and didn't
have long to live. That night I spent many hours with my son, teaching
him the Dhamma reflections that I had learned from my teachers. The
following morning I went to the wat, and when I came home in the late
morning I found him lying dead on the couch. It was a terrible shock for
my wife and it took her months to get over it. I've had a lot to endure.
If it wasn't for the Dhamma and the advice and support of the Sangha, I
don't know how I would have coped.

