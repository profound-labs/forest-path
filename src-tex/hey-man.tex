% Title: Hey Man, Don't Give Up Your Music
% Author: Samanera Gunavuddho

Barely a month in the robes, I am a newly ordained \emph{sāmanera} who
is still trying to understand exactly what has happened to his life.
Sometimes I wake up from sleep in a moment of disorientation and ask,
`Where am I and why am I dressed like this?' I thought my goal in life
was to be a jazz recording artist, but somehow I have made the
transition into the Theravāda  monastic lifestyle. I have a desire to
understand the transition better; and I admit that it is only now,
through writing this piece, that I am able to start investigating the
deeper reasons why I am willing not to play music again. I wish to make
the piece an exploration of my experiences in music practice and
monastic practice, in the hope of understanding better what has
happened.

When I look deep in my heart and ask why I practise Dhamma, I see that
the answer comes forth with great energy. I practise to learn the truth
of how nature works and to do what is good. After the heart has spoken,
I feel a heating-up of the body with an increased flow of blood; my back
straightens up nobly, my mind becomes quiet and my gaze softens. I am
also told by my heart that the same goal was the powerful current that
carried me through all those years of music practice. When this is
revealed I see that I have not given up what I find truly important, and
that there was a natural flow to the recent transition I've made into
the monastic life. Like a raft, music practice was able to take me
part-way across the river, but I have now switched to the raft of
Buddhist monasticism, which I believe has the ability to take me right
across to the shore of liberation. When looking at the similarities
between my music practice and this monastic practice, I am able to feel
a deep sense of gratitude for my past musical experience, while
investigating the differences which mean the monastic practice goes
further towards my goal. I have experienced the role of devotion,
sacrifice, the teacher, solitude, awareness, creativity, effort and
challenge in both my past music practice, and the monastic practice of
the Thai forest tradition.

Devotion

My music practice started with devotion. I was born into a family
devoted to music, where all the conditions for setting me on the path of
music practice were ripe. My family would go to church occasionally, but
music was the religion I practised at home, and I could trust it and use
it to relate to the world. My grandfather had a music store and taught
my father music. My father taught me, and I eventually taught others
after many years of practice. Playing a musical instrument almost seemed
to be a prerequisite for being in the family. A visit to my
grandparents' house always included an offering of a musical performance
by someone. I remember my father would play music on the fishing boat as
we sat meditating on our floating bobbers, in the house while we cleaned
it, and even in the bathroom as he showered. Our house always had many
different instruments to experiment with: drums, saxophones, keyboards,
trumpets and others. Teaching music and selling instruments were what
put food on the table and enabled us to continue our music practice. My
father and I would talk for hours about music; this was one of our main
ways of bonding. I would also carry his instruments to his public
performances in order to watch him play, which he did in a way that was
inspiring and magical for myself and others. He taught the values of
music to me through example from the very beginning of my life, and
after seeing my deep desire to follow in the path of music, he started
giving me formal piano lessons when I was five. I grew up with small
ceramic clowns holding instruments and Christmas ornaments that played
music. Our house was also filled with paraphernalia like mugs, posters,
belts, scarves, T-shirts, neckties and stationery which were all
decorated with musical notes.

A bit obsessive, you may think, but believe it or not, it all made
perfect sense at the time. Upon reflection I think that always seeing
those musical trinkets in my environment strengthened my identity as a
musician. Likewise, someone can strengthen the identity of being a
Buddhist by merely owning a statue or image of the Buddha, but I think
it is more beneficial to use the Buddha statue as a devotional tool to
open the heart and the mind. When I bow to the statue of the Buddha I
have the opportunity to recollect the Buddha's virtuous qualities and
his teaching, and to bring my mind back to the practice. I feel that I
am fortunate in having such an object of devotion as the Buddha, and
there is a feeling that every ounce of effort I give is returned one
hundredfold. To me the goodness that can result from the practice of
Buddhism seems limitless, so it is worthy of limitless devotion.

Sacrifice

When one has deep devotion, one is willing to sacrifice just about
anything. Music always came first in my life, and I would sacrifice many
things without question. Sleep would be sacrificed by waking up early in
the morning or staying up into the late hours of the night, practising
the piano or listening to jazz performances in clubs. After-school
activities, sports and dances didn't seem nearly as important as going
home to practise music. My parents spent a lot of money on music
lessons, instruments and recordings to support the cultivation of my
abilities, and put in many hours behind the wheel of the car to drive me
to lessons and performances. At home my family sacrificed their outer
silence to let me practise, and my sister can testify to the hours and
hours of piano-playing on the other side of her bedroom wall.

Now that I am ordained I have given up my worldly possessions. Bit by
bit, as my instruments, recordings and other tools of the trade made
their exit from my life by being sold or given away, my identity as a
musician started to fade. Having left my position as a teacher and
bandleader in the field where I had experience, I have now traded that
status to become a beginner at the bottom of the line. After being
ordained I found simple actions such as getting dressed hard to do. I
thought I was shown how to dress myself when I was a little boy, but the
familiar pants are gone and the robes are hanging in their place. So
many actions I thought I already knew, such as walking, eating and
sleeping, are now challenged by the monastic training. Even the name
I've used since birth has changed. I once believed there were
possessions and experiences that were necessary for survival, but as I
experiment with living very simply I experience some joy from learning
that my safety doesn't hinge on having those things after all. I would
say that the main thing I thought I knew was that I was a musician, but
where has that gone?

Realizing that I have personally felt a fair bit of thrashing around in
my life in an attempt to `be somebody', I hope to resist the temptation
merely to trade my identity as a musician for that of a monk. Trading
for the identity of an ex-musician is probably another trap. At some
level I understand that attaching to the identity of being a monk is
entirely different from actually being present to the way things are,
through the monastic practice as taught by the Buddha. Sacrificing the
need to attach to an identity is extremely hard to do, but I believe
that the invaluable guidance provided by an experienced teacher on the
spiritual path can make it possible.

Role of the teacher

Sometimes the teacher will select a practice which is not one we would
select ourselves, but out of faith and trust we follow the guidance we
have been given. Great experience gives the teacher the wisdom to see
where the student is weak in the practice, and the ability to protect
the student from training in a harmful way. In both music practice and
monastic practice, the teacher is one to whom we can sacrifice our ego.
I would admit what I did not know to my music teachers, but they could
already hear it in my playing. There was nowhere to hide and it felt
great to be seen.

I think that this attitude of observing and exposing weakness in a
relationship of trust is a central part of the Holy Life. I find that in
the monastic practice, due to the safety of the Dhamma teachings and the
morality practised in the environment, there is much more of an
opportunity to humble oneself and give oneself fully to the teacher. We
can bow with reverence to the teacher and allow this relationship to
develop as a spiritual tool in order to dissolve egotism. There are
opportunities to tend to the teacher by carrying his bag, washing his
feet, washing his clothes, bringing him something to drink, cleaning his
lodging and cleaning his bowl. Even though this man can do these things
for himself and has done for many years, the opportunity to think of
someone besides oneself, even for a moment, can be very powerful.

With my music teachers the relationship was usually only musically and
financially based. I always felt there was a cut-off after I paid for
the lesson and was out the door. If I did not pay for the lessons, the
relationship would end. That's the way it works in the business. Some
teachers would take more time to talk after the lesson, go to the store
and help me buy recordings or invite me to come to their performances,
but that was as deep as the relationship would go in the context of my
life as a whole. The only opportunities to give of myself were to pay
for the lesson, practise what I was assigned, provide supportive energy
in the audience at a performance and maybe give a gift at Christmas. It
would be quite strange to go early to a piano lesson to tidy up the
teacher's studio and bring him a drink on my knees. There just wasn't
room for that.

Solitude

Both musicians and monks spend a great deal of time in solitude in order
to deepen their practice. I remember the small piano rooms at college
dedicated to solitary practice. Each was insulated with white-walled
foam for soundproofing, in order to create an environment with minimal
distraction. Restraining the non-listening senses was an aid to
concentration. It was in those rooms that most of the daily sweat was
released. I spent many timeless hours in these practice rooms, and
sometimes I would emerge surprised to see that it was already dark
outside, or that a snowfall had blanketed the city without my knowing
it. So after years of such experiences, I don't mind practising in the
\emph{kuti}. Having space to oneself in the forest functions as
insulation from many worldly dhammas. I've searched for solitude in both
practices, and I am just starting to understand that true solitude is a
state of mind. Whether from amplified long-haired guitarists, mango
pickers, aeroplanes or loud insects, there is always outside noise. The
silence must come from inside. Just as the silence or `rests' in music
give the listener space to appreciate all the notes, our mental silence
similarly brings awareness and meaning to our actions and the actions of
those with whom we are in contact.

Working with the sangha and helping others

As Buddhists, a central part of our practice is to work with the
community of monks, nuns, laymen and laywomen who make up the Sangha.
Similarly, musicians work with their bandmates to give to and receive
from an audience. In each public performance there is an opportunity for
the band members to give a teaching in the virtues of diligent practice,
by providing a good example through the way they play their instruments
and how they act onstage or offstage. Each composition focuses on
different aspects of life, and if the composition touches something
inside the listener or the performer, the experience can be used as a
point of reference or inspiration for understanding better what is truly
important in their own lives. If the music provides an environment for
releasing blocked energies such as stress and anxiety, there is an
opportunity for healing and insight to occur.

In the jazz world it is completely appropriate for the listener to
respond to the music during the song. Listeners have an opportunity to
be an active part of the improvisation and their response becomes a part
of the music. Some responses may be in the form of shouting out,
`A-MEN!', or `That's right!', or `!'. Clapping and whistling after a
musician has played a solo improvisation are also common responses. If
the band is really `burnin'' or a player is playing so well that they
may be described as being `on fire', the audience's response will
reflect exactly that and be more animated and lively. I see this as a
sharing of energy. The musician directs energy through the instrument,
and the audience transmits energy through verbal and physical responses.
It is this sharing of energy that makes the performance experience so
alive and special.

When people come to hear the Dhamma teaching with sincere interest and
faith, they reciprocate by transmitting the positive energy of the
inspiring truths they hear back to the teacher. The teacher may respond
by touching the heart with humour, which will be met with wholesome
laughter from the listeners. The fearsome demons of negative mental
states can be transformed into little puppies during these special
moments of a powerful Dhamma teaching. A release of fear and anxiety can
bring a sigh of relief and a moment of silence from the listener.
Breathing may become deep and slow. Others may feel open and safe enough
to respond by asking questions that touch tender areas of their lives.
As with jazz improvisation, if the teaching is done naturally in the
moment, the product is much better than anything one might pre-plan.

The development and communication of the most profound insights in music
occur during rehearsals, performances, instruction and jam-sessions.
Through these forms we have the opportunity to help others with the
challenges of life. Personal issues will come up during rehearsals, and
the group contact can help bring awareness of certain issues. If a
musician only plays alone in the practice room, he will never have the
possibility of experiencing the contrasts that expose his habits through
the group. But when a musician plays with a group he can investigate
whether he plays too loud, too soft, too slow, too busily or in balance
with the others.

Similarly, in Buddhist practice the Sangha is the community of disciples
who work together through the traditional form of daily group
activities, ceremonies, instruction and discussion. We constantly
compare ourselves in relation to the way the group practises as a whole.
Do I walk too slowly or too fast? Do I eat more slowly or more quickly
than the others in the group? In relation to the group, how do I speak,
sleep, do chores or put forth effort? Being aware of these habitual
energies is a preliminary step, but in the monastic practice awareness
is encouraged in order to transform unwholesome habits into the
wholesome ones which lead to liberation, as taught by the Buddha. On one
occasion a community member spoke with me about my tendency to chant
louder than others in the group. It was only through chanting with the
Sangha that I had the opportunity to have this tendency brought forth
into my awareness. The old tendency to play louder than some bandmates
also used to crop up in my jazz playing at times. Training with the
Sangha helps me to bring my habits into the light of mindfulness, so
that I can begin to change them for the good of all.

Awareness

In the practice of both music and monasticism repetitive actions are
used to cultivate mindfulness. The scales, finger techniques and
following music theory rules are to the daily life of a musician as
bowing, meditation techniques and following the Vinaya rules are to the
daily life of a monk. Walking, speaking, washing things, moving things,
entering and leaving the \emph{sala} or \emph{kuti} and getting on the
\emph{āsana} are all areas which, if approached with great care and
awareness, become beautiful and magical. How many times do we take off
our bowl lid each day? Even in the smallest of acts there are
opportunities to act with dignity and allow the sacred to enter our
lives. How does one ring the bell? If we listen closely to the way it is
rung, we can learn about the one who is ringing it. Is there a sense of
urgency, frustration or restlessness? Can one hear the focus of the act,
or is there an irregularity? In general, does the execution of our
actions both support the mind in abiding peacefully and inspire others?
For myself I find it very inspirational to witness masterful expression
through the living examples of experienced practitioners.

I was introduced to the Theravāda tradition at Abhayagiri Monastery near
my home in California. I remember how the everyday tea conversation at
Abhayagiri was turned into an art. It was a safe place for people to ask
questions about the practice. Ajahn Pasanno would answer questions from
the laity with ease, and then skilfully allow for a space of silence. It
was a moment of being with things as they were. At this time one could
be conscious of the breath and let what had been said be observed fully.
If the person had another question, they had the opportunity to ask it.
There was a feeling that people could ask all their questions until
there were none left. After that period had passed there was another
time of silence. Since at that time I was an enthusiastic layman with
many questions, I remember feeling very satisfied to be listened to. The
whole approach was peaceful and allowed for the release of anxious
energy.

Usually in life I experience conversations that move quickly, jumping
around from one topic to another. People rarely listen to what has
already been said and might even interrupt each other. Interruption is
not a personal attack but a reflection of the restless mind. When
conversing in the language of music, it is the same restless mind which
interrupts others on the bandstand or interrupts one's own thread of
continuity during solo improvisation. I have noticed that many of these
unsatisfactory conversations tend to be oriented towards trying to be
understood, instead of trying to understand what others are
experiencing. We may have to embrace an uncomfortable feeling if we wish
to truly understand the suffering another is expressing. If we are not
aware that we are uncomfortable with an issue brought up in a
conversation or that we are interrupting others, we will always be
blinded by the restless mind, which selfishly only wants to be
understood. It helps to use creative means to keep the mind malleable
and receptive, in order to listen with understanding to whatever comes
to us in life.

Creativity

Sometimes `life just happens' on the bandstand, and in that moment some
playful creativity may work better than letting it all fall apart. In
jazz improvisation a mistake is worked with on the spot in live time.
For example, if the saxophone player accidentally makes an irritating
high-pitched squeak on his reed, this mistake can be transformed by
deliberately inviting it into the piece. If the player makes the mistake
twice in a row, another band member could notice the rhythmic pattern of
the squeak and incorporate that rhythm in some form of a response.
Sometimes the reed may be difficult to work with, but many times the
squeaking is the result of a player experiencing fear. Think how
someone's voice may crack or squeak if they are excited; it can be
similar to that. The process of befriending the fear happens all in the
moment. There isn't time for thinking, and it takes practice in
creativity to pull this off.

Creativity is a necessity in both practices. Ajahns use creative means
in relating the teaching. One of my favourite devices is humour, and as
a musician I would use humour in compositions or improvisation. I have
noticed that the Ajahns in this lineage use wholesome humour as a
skilful device, to make things light enough for the mind to be malleable
and receptive to the teaching. I bet the Buddha had a good sense of
humour, but unfortunately I don't hear much about this personality
trait. It takes creative means such as this to really reach people. If
they feel bored or just talked at they may get up and leave the Dhamma
teaching behind. Metaphor and analogy are creative devices that the
Buddha used and his disciples continue to use today. The ability to find
creative ways to relate the Dhamma in simple terms that people can
understand is the mark of a masterful teacher. All Ajahn Chah's
teachings that I've heard are of this style. The teaching is clear,
simple and goes straight to the heart.

Effort

In the music world, challenges such as anxiety, fear, absent band mates,
miscommunication, drug and alcohol use, bad attitudes and different
ability levels are just some of the reasons why proficiency can decrease
during a performance. My piano teacher would recommend putting forth
effort through pushing up tempos and practising long hours, in order to
have the strength and endurance needed to persevere in such challenging
situations.

In the monastic training, effort is put into pushing up tempos and
practising long hours in order to cultivate the focus of mind needed to
greet the raw conditions of life exactly as they are in the given
moment. Doing group activities briskly, walking barefoot over long
alms-round routes and sitting in meditation for vigils and retreats are
all ways in which we put forth effort in the monastery. When one puts
effort into making things neat and tidy, there is an opportunity to see
the result of a job well done, which provides a space for future good
actions to arise. I remember seeing a monk working with great vigour as
he shovelled sand on a workday, and that inspirational display of effort
wholesomely affected many areas of my practice long after the workday
was over.

When cultivating music, I found that good results were more likely to
arise if I put effort into maintaining a daily practice. As momentum
built up and less effort was spent on simply getting myself to sit on
the piano bench, I could redirect my energy towards the more refined
aspects of the music. Just as I would wake up early before school as a
child to practise piano, or stay up late at night enduring fatigue in
order to fit in the daily practice which had not happened yet that day,
I now put in that same kind of effort to go to the meditation hall in an
attempt to cultivate good qualities. Even though there are times when I
don't feel like practising, there is also deep joy, and a desire to do
the practice that seems to summon the effort needed for the occasion.

Challenges

When I was introduced to Abhayagiri Monastery, I was given a photocopy
of what visitors needed to know before staying at the monastery. I
remember reading the Eight Precepts and laughing out loud when I got to
what I now affectionately refer to as `number seven'. I didn't
understand why refraining from playing or listening to music would even
make the list. I remember showing this Precept to a few people close to
me and saying, `I can do it for a week, but that's it!'

The one week stay gradually turned into a three-month lay residency.
That period of time was fraught with `number seven' questions. I
remember Ajahn Pasanno gently answering my questions relating to the
monastic practice. I would ask, `Ajahn, can monks give music instruction
if they don't touch any instruments? Can monks at least accompany the
chanting with guitar?' My mind wrestled with absolutes. I wondered
whether music was wrong action. Was I a bad person for playing music? I
noticed from the questions that friends and family members asked that
they also looked at the Precepts more as commandments than training
guidelines. If I looked at `number seven' as `Thou Shall Not\ldots{}', I
could only feel guilt and remorse. Through the instruction at
Abhayagiri, I feel that I was able both to appreciate what my music
practice had given to me and see how the unwholesome environments and
mental states often associated with music practice can be a hindrance to
the Holy Life and true peace of mind. I was also glad to see that by
giving up my music for the practice of the Thai forest tradition, I was
not giving up learning from challenging situations. In this Thai forest
tradition there is a warrior spirit that is willing to go against the
grain by marching fearlessly into the middle of the battle to endure
whatever comes, even if it is uncomfortable. One is encouraged to give
100\% to the training with fierce determination. I find it particularly
challenging at this point in my development to reach a balance between
knowing when it is time to push and time not to push, but I feel that
through the practice of the Thai forest tradition I have an opportunity
to find the middle way by seeing where the boundaries lie.

With my music practice I was constantly looking for challenges in order
to sharpen my skills. Some of them included moving far away from home to
study at one of the world's most intensive music colleges; taking almost
any opportunity to put my practice on the line in a public performance
situation; sharing my most personal original compositions with others;
enduring the discomfort of playing with musicians who were much more
advanced instead of trying to be in the superior position; seeking out
the most qualified and disciplined teachers available; and playing
demanding music that was outside the comfort zone of the traditional
jazz I specialized in.

Similarly, I looked for challenges in the monastic life such as moving
far away from home in order to participate in the intense practice at
Wat Pah Nanachat, led by some of the most disciplined and experienced
practitioners, who keep a strict standard of Vinaya. Through seeking the
opportunity to put my practice on the line by working out in the open
with the Sangha, I expose my most tender moments in the context of a
community where it is impossible to have a secret. Training as a
beginner at the bottom of the line, I am constantly in a position to
learn about all the do's and don'ts of monastery standards, instead of
comfortably sticking to what I learned years ago in the music field. I
feel that living this monastic life is the greatest challenge I can use
to meet my goal of learning the truth of how nature works. The restraint
of the senses feels like a pressure-cooker at times. We don't eat
whenever we wish, there's no flipping on the television for a little
distraction, and there's no sexual activity whatsoever. The rules of
training ask one to give up many means of outer control, and put one in
a position where there is much less room to run away and hide from
long-avoided fears. By going against the grain of these habitual
tendencies, we have the opportunity to learn the ways of the mind.

The monastic life is difficult enough as it is, but I find that being in
Thailand adds another layer of challenge to my practice. The forest
environment in Thailand is constantly changing, and has an aggressive
quality to it that never gives one a chance to relax completely. There
are intense weather conditions of heat and rain. Either mosquitoes and
ants are biting you or you know that they could bite at any time. There
are other insects, scorpions, centipedes, poisonous snakes and wild
animals to be mindful of here in Thailand as well. Recently thousands of
termites took over my \emph{kuti}, swarming all over the walls and door
just inches away from me. I walked through the night with the belongings
that I thought the termites would consider edible and tasty, and found
refuge in the monastery's sewing room. In the mornings on almsround, I
am challenged by walking barefoot over sharp stones, while dodging
occasional shards of glass and ridged bottle caps. Even though there may
be cuts on my feet, walking through the widespread buffalo dung smeared
on the road by the village traffic is sometimes unavoidable. Compared to
the USA, in Thailand I am challenged by the different sanitary
standards, increased threat of disease, different cultural values, the
different language, and being apart from family and the familiar. But
though the challenges I've mentioned have been difficult, they aren't as
bad in the moment as my mind would like to tell me. I also realize that
I am supported through these challenges by generous lay supporters,
Sangha members, family members and, in general, by practising in the
context of a Buddhist country.

So how do I blend music with the monastic life now? I like to appreciate
the sound of dissonance from out-of-tune Sangha members during the
chanting. Other times I enjoy the sound of someone mistakenly hitting
their bowl or spittoon, appreciating the different timbres each
produces. I enjoy the chance to ring the monastery bell and to listen to
others briefly express the causes and conditions of nature through
ringing the bell themselves. And of course, it may not be a surprise to
anyone that I enjoy chanting.

`Hey man, don't give up your music!', counselled my jazz piano-player
friend with a tone of urgency and disappointment, as he saw me move off
the path of music practice that we had once shared. I've been
questioning what I have actually given up and what is happening in my
life, but after the exploration of writing this piece I feel that the
question of `Who am I?' is closer to my heart. There are many layers to
my recent transition from music practice into the monastic life, but
ordaining as a \emph{sāmanera} has been predominantly marked by an
intense questioning of my identity and sense of self. Uncomfortable with
not knowing who I am, I feel that holding the question is more important
than trying to wrap it up neatly with a bow on top, tucking it away as
if I've found a conclusive answer. So in the midst of uncertainty I will
practise on.

\emph{ The Author}

\emph{Sāmanera} \emph{Gunavuddho went on to take full ordination a year
later, with Luang Por Liem, the abbot of Wat Pah Pong, as his preceptor.
He trained in Wat Pah Nanachat and subsequently at Wat Marb Jan in
Rayong Province. In 2007 and 2008 he undertook to help the current
abbot, Ajahn Kevali, to look after Wat Pah Nananachat and Poo Jom Gom
Monastery. Following his tenth Rains Retreat, he joined the community of
Abhayagiri Monastery in California. He has helped spread the Dhamma in
the San Francisco Bay Area, and played an active role in the Buddhist
Global Relief Foundation, a charity devoted to the goal of working to
eliminate global poverty and its associated problems. He has since
decided after 14 years in the robes to live as a lay Buddhist in
Thailand by his name David De Young, and continues spreading the Dhamma
in English in Bangkok.}

