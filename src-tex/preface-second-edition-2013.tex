
`Forest Path' was first printed in 1999 and originally planned as the
first edition of a Wat Pah Nanachat newsletter. Over-enthusiasm and
considerable proliferation resulted in a one-off book publication which
more or less coincided with the monastery's twenty-fifth anniversary.
Since then we have been surprised by the number of requests to reprint
this collection of little essays, talks and anecdotes about life in Wat
Pah Nanachat as it was at that time. Apart from the formal Dhamma talks
the book contains, we were hesitant at first to reprint its other
contents, the `old stories' and personal accounts by then younger
authors. But it was those other parts which in fact added much of the
book's authentic flavour and made so many people find it beneficial and
joyful to read.

Hence it appears that our hesitation over reprinting this book was due
to a concern that the snapshots of life in the monastery offered by
those individuals at that time had become dated. In fact, though, on
recently going through the various contributions again, we found that
many of those snapshots could still be written today -- there would be a
new cast of players, but the atmosphere experienced then, and
communicated so vividly in the old `outdated' accounts, is still very
similar.

So we are happy to realize that when we take the wholehearted
present-day Dhamma approach of genuinely experiencing what is happening
at a particular \emph{saṃsāric} moment in life, we also embark on
something timeless. Beyond the specific details, all our little hopes
and sorrows around daily life in the monastery and the higher values and
principles of our life -- the Buddha's core teachings -- become apparent
in these snapshots. Better still, the details of an individual's unique
experiences are in fact exemplary: they are transferable and thus pass
the test of time. Thinking in these terms gives the monks and novices at
Wat Pah Nanachat extra encouragement, as in many ways it makes the
limitations and the suffering inevitably entailed by each one's specific
experience worthwhile.

Once we became aware of all the `good examples' already to be found in
the original `Forest Path', the option of rewriting some of the material
in a 2012 setting as a `Forest Path II' suddenly seemed pointless. With
\emph{saṃsāra} essentially repeating the same old drama endlessly
anyway, we thought that simply revisiting the old setting once more with
a simple reissue of the original would be a much more effective and
honest choice than reworking the text and suggesting it was `new'. It is
also true tradition to go back a little into the past, with the hope of
taking the opportunity to realize some timeless truths.

So the material in this new edition of `Forest-Path' is an almost exact
reprint of the old 1999 version, although with the editorial assistance
we have taken the opportunity to correct some of the punctuation and
grammar. We have also created a few links to the present-day situation
 (which seems fair enough, considering that a big, impersonal monastic
community only becomes a reality when it is embodied in some specific
individuals). For us now, the present dwellers in the monastery, who in
most cases did not arrive there in time to meet the great example Luang
Por Chah in person, it seems most affirming that the principles of
monastic life as his disciples still permeate the scenes that each new
generation of monks, novices, \emph{anāgārikas} and visitors has been
continuously experiencing at Wat Pah Nanachat during the past
thirty-five years.

So please come and see our monastery for yourself, and get your free
real-time update on our community by practising the present-moment
Dhamma with us. In the meantime, we hope you will enjoy viewing these
historical snapshots again.

\bigskip
{\par\raggedleft
Yours in the Dhamma,\\
On behalf of the Sangha of Wat Pah Nanachat,

Kevali Bhikkhu\\
Abbot, 2013
\par}

\vfill

\section{Note}

Please be aware that the authors' monastic titles when the original
essays were written have been kept (e.g. `Sāmaṇera' or `Tan').  You will
see in the short summaries added after each essay, which explain what
has happened since then, that almost all of them are now senior monks
whom we would usually honour by calling them `Ajahns', teachers.

