
\setChapterAuthor{Tan Saññamo}
\setChapterNote{A young monk's reflections on working with discomfort.}

\chapter{Mindfulness with Mosquitoes}
\markright{\chapterAuthor}

I find it a challenge to be able to translate many of the Pali words
found in the Suttas into day-to-day experience. Sometimes even the most
common words such as \emph{sati} or \emph{saddhā} can remain at best
loosely defined concepts in our minds; how we personally experience
these concepts is not clearly understood. Is it not worthwhile to take
the time to investigate our use of these terms for ourselves? I have
thought so, and at times have been surprised by what a little discursive
thought can dredge up. When these terms are not clear I have noticed
that doubt tends to infiltrate through this vagueness. By defining them
more clearly and connecting them to our experience, not only do we patch
up obscurities, but we can identify the presence of the qualities to
which these terms refer and their nature of rising and falling. The
following is an extract from my journal relating to an incident that has
since redefined and clarified experientially what I take to be
mindfulness in action.

Today, during our evening meeting at the outside meditation hall, clouds
of bloodthirsty mosquitoes descended upon our vulnerable and defenceless
sangha. Lately I have been struggling to understand how to work
skilfully with the unpleasant situations that inevitably crop up in
life. This was a fortuitous occasion to explore the possibilities. A
Dhamma talk was offered this evening. When it ended I was feeling
unusually content, with no real motivation to pursue a particular
meditation object. Instead I was satisfied with watching the momentum of
my thoughts naturally become still and settle. Inspired by such a
peaceful mind-state, I generously offered my exposed right arm and
shoulder to the mosquitoes. The response was overwhelming, so much so
that my arm started twitching involuntarily from the strain of hosting
such a banquet. Frustration began to grow, and the din in my mind that
was telling me nothing was going on and everything was okay was not very
convincing. Basically, I was being eaten alive. I didn't want to admit,
`This is unpleasant'. Instead, thoughts like, `If you just get
concentrated, you won't feel it' or, `Develop compassion' arose in my
mind.

In unpleasant situations I find I habitually try to convince myself of a
solution, rather than looking directly at the matter at hand. The
thoughts play the part of `the one who knows' and supply answers for
`the one who doesn't'. Though these suggestions and advice are not
necessarily wrong, they don't give space for a real understanding or
acceptance of the situation to arise. Eventually, however, my capacity
for abuse reached its limit. With patience and goodwill exhausted, I
withdrew my offering back into the folds of my robe. Not a minute later,
a mosquito landed on the back of my head. Because both my arms were
bound beneath my robes, I was helpless. How exasperating! What happened,
though, was interesting. I was being mindful of the sensations. I noted
the mind and the object of the mind. From looking more deeply, their
separate natures became increasingly distinct, as did their connections
and the way they influence each other.

When pain arises in formal practice, I often can't bring myself to have
an honest look at it. However, tonight the mind admitted, `This body and
mind are agitated! The entire sensory experience is unpleasant. Telling
myself how I should feel only further obscures how I do feel.' I noticed
that admitting these feelings did not increase the pain. Instead, a
courage born of this detached acceptance began to strengthen, giving me
confidence to look even more closely at what

was going on. The pain continued but no longer seemed to afflict the
mind. At that point the sensations were more interesting than
unpleasant; no aversion towards them was present, and so there was no
desire to end the sitting. Mindfulness and concentration were growing
together. This was a wonderful surprise; despite all the violent
sensations, I found the mind to be quite concentrated. `Interesting', I
thought, `even if the pain is stabbing and coarse, with enough courage
and resolution to observe the sensations, one can just as well attend to
them as to any other mind object.'

Perhaps, though, the more important insight which arose was on observing
the relationship between mindfulness and concentration (at least as I
experience them). What I noticed was that mindfulness indicates the
object of our attention, while concentration is a measure of the purity
of the mindfulness itself. When we speak in terms of concentration, we
are really speaking about the quality of our attention. In this sense,
mindfulness and concentration go hand in hand.

Suppose we look at a flower garden. A casual glance will probably give
us general impressions,

such as the variety of colours and the general layout, but we will
probably not be able to identify any of the flowers. If we focus on a
specific area we may see the flowers more clearly, and their petals and
shapes may become distinctive. By looking closer still, their stamen,
anther, and pollen may come into view. In this example, it can be seen
how mindfulness responds to an increased quality of attention. With more
concentration, the object of which we are mindful comes more into focus.

In our day-to-day life, our attention is mostly diversified. The many
sense impressions we receive in consciousness dilute the quality of our
attention. When we accommodate this variety, the quality of our
attention becomes spread out, less unified and less focused.
Alternatively, the more unified our attention becomes, the greater the
degree of discernment. The more capable of discernment we are, the more
do increasingly subtle mind objects become accessible for investigation.

As concentration developed, I felt the rise and fall of sensations
become clearer. I saw that the continuity of events we mechanically
string together in everyday consciousness hides the immediate presence
of a single point or object. For the unconcentrated mind, it is
difficult to discern the constant rising and falling of phenomena. As
the mind grew more concentrated, I was able to get closer to seeing
things arise and cease. The series of events became more distinct. As a
result, the illusion of continuity began to waver. I watched the
process. It was then I felt I had a choice, to fix my attention on a
single object or contemplate the continuous flux of change. However,
being unable to sustain this concentration, I lapsed back into a more
normal state of awareness. Once again, these sensations appeared to me
as a continuity.

We all hold on to views and opinions about life and about meditation,
and part of the task of meditation is learning how to let go of our
views. As I reflected upon this evening, what struck me was that the
night's struggle was not so much about how I related to the meditation
object, but more about the difficulty of recognizing and addressing the
personal views to which I attach. Letting go of views is difficult; they
define our relationship to certain aspects of experience. Relinquishing
them leaves us insecure, without a strategy for dealing with the
unknown. When we do manage to let go, this space provides the
opportunity for wisdom to arise. With more clarity around the basic
nature of the mind, I found that certain assumptions and tendencies I
held about meditation began to surface. For instance, I observed how I
have been sabotaging the more contemplative elements in meditation
practice by attempting to force the process, rather than letting it
unfold organically.

Contemplation involves a great deal of receptivity. When we attach and
limit ourselves to our preconceived ideas, our receptivity is
compromised. This is an obstacle. Sometimes I give undue attention to
phenomena which may arise as a by-product of my concentration. By
attending to these

new and often fragile objects, I end up abandoning my original
meditation object. As a result, everything falls apart. This approach
has never been successful, but oddly enough, I had never seemed to
notice \ldots{}

The Author

\emph{Tan Saññamo, who at the time of writing the article was a
newly-ordained monk, stayed on at Wat Pah Nanachat, completing his
five-year training in Thailand. The following year he went to live with
a much-respected disciple of Luang Por Chah living in Rayong Province,
Ajahn Anan Akincano. Tan Saññamo trained under him at his monastery, Wat
Marb Jan, for a further five years, before visiting home in Canada for a
while. In 2011 Ajahn Saññamo moved to Abhayagiri Monastery in California
to live with the community there and further his training under Tan
Ajahn Pasanno.}

