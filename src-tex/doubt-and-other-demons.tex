% Title: Doubt And Other Demons
% Author: Ajahn Jayasāro

Doubt is of two main kinds. Firstly, there is the doubt born of a lack
of sufficient information or knowledge to perform the task in hand. We
may, for instance, doubt the Buddhist teachings on a particular subject. 
We may doubt which is the best route to take to a new destination. The
Buddha recognized such doubts as legitimate and did not consider them an
obstacle to spiritual growth. On the contrary, he praised a healthy
scepticism and a questioning mind: `Good, O Kalamas, you are doubting
that which should be doubted.' The fifth hindrance to meditation, 
\emph{vicikicchā}, usually rendered in English as `sceptical doubt', is
not the mere awareness of a lack of information, but rather
unwillingness or hesitation to act upon it. The person afflicted by
\emph{vicikicchā} is paralyzed by his inability to be sure that he is
following the best course of action. In other words, he must have proof
of the truth of a proposition before seeking to verify it. The Buddha
compared this to travelling in a wilderness. The commentary explains: 

`A man travels through a desert, and being aware that travellers may be
plundered or killed by robbers, he will, on the mere sound of a twig or
a bird, get anxious and fearful, thinking ``The robbers have come!'' He
will go a few steps, and then out of fear he will stop, and continue in
such a manner on his way; or he may even turn back. Stopping more
frequently than walking, only with toil and difficulty will he reach a
place of safety. Or he may even not reach it. 

It is similar with those in whom doubt has arisen with regard to one of
the eight objects of doubt. Doubting whether the Master is an
Enlightened One, he cannot accept this as a matter of trust. Unable to
do so, he does not attain to the Paths and Fruits of Sanctity. Thus, 
like the traveller in the desert who is uncertain whether robbers are
there or not, he produces in his mind, again and again, a state of
wavering and vacillation, a lack of decision, a state of anxiety; and
thus he creates in himself an obstacle to reaching the safe ground of
Sanctity. In that way sceptical doubt is like travelling in the desert.'

Modern education teaches us to think, to compare, to analyze, to use
logic -- `left-brain' abilities of great value in our daily lives. A
mind that is aware of many different ways of looking at things is also
usually a tolerant one. But without a strong conviction in his chosen
path, a meditator may often lack the ability to stick with that path
when the going gets tough. On the purely rational level there are always
reasonable objections to making the sacrifices that spiritual life
demands, there are always more comfortable alternatives. When the
emotional assent provided by faith is absent, reason can make Hamlets of
us all. This hindrance of doubt particularly affects those meditators
who have been successful in the conventional education system; it is the
dark side of an enquiring mind. A lot of learning can also be a
dangerous thing. The particular form of doubt varies. A practitioner may
harbour doubts about the efficacy of the technique or its suitability to
his character; he may be unsure of the teacher or agonize about his
ability to practise. \emph{Vicikicchā} is the most disabling of the
hindrances, because unlike lust or anger, for example, it is often not
perceived as being a defilement. The element of indulgence tends to be
concealed. In the early days of Wat Pah Pong, the majority of the monks
and lay supporters had strong faith in Ajahn Chah and little formal
education; crippling doubt was never a major problem. In later years, 
with more middle-class city dwellers arriving and a growing number of
Western disciples, it became more of an issue. Ajahn Chah's response to
the chronic doubters was always to point out that `Doubts don't stop
because of someone else's words. They come to an end through your own
practice.'

A suppression of doubts through belief in the words of an authority
figure must always be fragile. Blind faith makes the mind rigid and
narrow. Ajahn Chah's view was that the only way to go beyond doubts was
through understanding their nature as impermanent, conditioned mental
states. On one occasion he explained why he didn't conduct daily
interviews with the monks, as is the practice in many meditation
centres: 

`If I answer your every little question, you will never understand the
process of doubt in your own mind. It is essential that you learn to
examine yourself, to interview yourself. Listen carefully to the Dhamma
talk every few days, then use the teaching to compare with your own
practice. Is it the same? Is it different? How do doubts arise? Who is
it that doubts? Only through self-examination will you understand. If
you doubt everything, then you're going to be totally miserable, you
won't be able to sleep and you'll be off your food, just chasing after
this view and that the whole time. What you must remember is that your
mind is a liar. Take hold of it and look. Mental states are just that
way; they don't last. Don't run around with them. Just know them with
equanimity. One doubt passes away and then a new one arises. Be aware of
that process for what it is. Then you'll be at ease. If you run after
your doubts, then you won't just be unhappy, but your doubts will
increase as well. That is why the Buddha said not to attach to things.'

Some practitioners reach a certain point in their practice and then
doubt what they have attained, or what state they are in as they are
meditating. Ajahn Chah would say there were no signposts in the mind
like there are on highways: `Suppose that you were to give me a fruit. I
might be aware of the sweetness of the fruit and its fragrance, know
everything about it except for one thing: its name. It's the same with
meditation. It's not necessary to know what things are called. If you
know the name of the fruit, that doesn't make it any sweeter. So be
aware of the relevant causal conditions of that state, but if you don't
know the name it doesn't matter. You know the flavour. You've grabbed
both its legs, let it struggle all it wants. The name isn't so
important. If someone tells you, then take note of it but if they don't
there's no need to get upset.'

On another occasion Ajahn Chah comforted a Western disciple: `Doubting
is natural. Everyone starts out with doubts. You can learn a great deal
from them. What is important is that you don't identify with your
doubts; that is, don't get caught up in them. This will spin your mind
in endless circles. Instead, watch the whole process of doubting, of
wondering. See who it is that doubts. See how doubts come and go. Then
you will no longer be victimized by your doubts. You will step outside
them and your mind will be quiet. You can see how all things come and
go. Just let go of what you are attached to. Let go of your doubts and
simply watch. This is how to end doubting.'

Views and opinions

Concentration is that clear stable tranquillity which forms the basis
for the wisdom that knows things `as they are' and uproots the delusions
that generate suffering. In other words, wisdom is both the beginning
and end of the path of practice. Initially the meditator must
`straighten his views', develop a sound understanding of the value and
purpose of meditation. If he doesn't, much sincere effort may be wrongly
directed and thus wasted. An important element of Ajahn Chah's
meditation teaching involved pointing out the fallacy of wrong views and
opinions held by his disciples, and giving authentic reasons for the
practice of the correct path and encouragement in its practice. 

Impatience

The untrained human attention span seems to get shorter and shorter. We
have come to expect and often demand quick results at the press of a
button or key. Our underlying assumption is usually that speed and
convenience are good per se. But in spiritual life this does not always
apply, as there are no shortcuts waiting to be discovered. It is, the
Buddha declared, a gradual path, one that depends on gradual maturing. 
If we are in a hurry, our inability to speed things up can feel highly
frustrating. Once Ajahn Chah taught an impatient disciple: 

`Meditating in order to realize peace is not the same as pressing a
switch or putting on an electric light and expecting everything to be
immediately flooded with light. In the sentence of effort you can't miss
out any words or phrases. All dhammas arise from causes. When causes
cease, then so do their results. You must keep doing it steadily, 
practising steadily. You're not going to attain or see anything in one
or two days. The day before yesterday a university student came to
consult me about his practice. When he meditates his mind is not at
ease, it's not peaceful. He came to ask me to charge his batteries for
him {[}laugh{]}. You must try to put forth a constant effort. You can't
comprehend this through someone else's words. You have to discover it
for yourself. You don't have to meditate a lot, you can do just a
little, but do it every day. And do walking meditation every day as
well. 

`Irrespective of whether you do a lot or a little, do it every day. Be
sparing with your speech and watch your mind the whole time. Just refute
whatever arises in your mind, whether it's pleasure or pain. None of it
lasts; it's all deceptive. With some people who've never practised
before, when a couple of days have passed and they're still not peaceful
they start to think they can't do it. If that happens, you should ask
yourself whether you received any teachings before you were born. In
this life, have you ever tried to pacify your mind? You've just let it
go its own way for a long time. You've never trained your mind. You come
and practise for a certain time, wanting to be peaceful. But the causes
are not sufficient and so the results fail to appear. It's inevitable. 
If you're going to be liberated, you must be patient. Patient endurance
is the leading principle in practice. The Buddha taught us not to go too
slowly and not to go too fast, but to make the mind `just right'. 
There's no need to get worked up about it all. If you are, then you
should reflect that practice is like planting a tree. You dig a hole and
place the tree in it. After that it's your job to fill in the earth
around it, to put fertilizer on it, to water the tree and to protect it
from pests. That's your duty; it's what orchard owners have to do. But
whether the tree grows fast or slow is its own business, it's nothing to
do with you. If you don't know the limits of your own responsibilities, 
you'll end up trying to do the work of the tree as well and you'll
suffer. All you have to do is see to the fertilizer, the watering and
keeping the insects away. The speed of growth of the tree is the tree's
business. If you know what is and what is not your responsibility, then
your meditation will be smooth and relaxed, not stressed and fretful. 

`When your sitting is calm, then watch the calmness. When it's not calm, 
then watch that -- if there's calm, there's calm, if there's not, 
there's not. You mustn't let yourself suffer when your mind's not calm. 
It's wrong practice to exult when your mind is calm or to mope when it's
not. Would you let yourself suffer about a tree? About the sunshine or
the rain? Things are what they are, and if you understand that, your
meditation will go well. So keep travelling along the path, keep
practising, keep attending to your duties, and meditating at the
appropriate times. As for what you get from it, what you attain, what
calmness you achieve, that will depend on the potency of the virtue you
have accumulated. Just as the orchard owner who knows the extent of his
responsibilities towards the tree keeps in good humour, so when the
practitioner understands his duties in his practice, then
`just-rightness' spontaneously establishes itself.'

Ambition

Ajahn Chah would constantly encourage his disciples to cultivate the
spirit of renunciation, to see practice as a gradual process of letting
go of attachments rather than as one of gaining attainments. Practice
fuelled by the desire to get or become is more likely to lead to new
realms of existence rather than liberation: 

`Sometimes in meditation practice people make determinations that are
too extreme. Sometimes they light incense, bow and make a vow: ``While
this incense has not burned down I will not get up from the sitting
posture under any circumstances. Whether I faint or die, whatever
happens, I'll die right here.'' As soon as they've made the solemn
declaration they start to sit, and then within moments the Māras attack
them from all sides. They open their eyes to glance at the incense
sticks. ``Oh dear! There's still loads left.'' They grit their teeth and
start again. Their minds are hot and bothered and in turmoil. They're at
their wit's end. They've had enough and they look at the incense sticks
again, as surely they must be at an end. ``Oh no, not even half-way!''
This happens three or four times and then they give up. They sit and
blame themselves for being hopeless: ``Oh, why am I such an idiot, it's
so humiliating'', and so on. They sit there suffering about being
insincere and bad, all kinds of things, until they're in an utter mess, 
and then the hindrances arise. If this kind of effort doesn't lead to
ill-will towards others, it leads to ill-will towards yourself. Why is
that? Because of craving. Actually, you don't have to take resolutions
that far \ldots{} You don't have to make the resolution to tie yourself
up like that. Just make the resolution to let go.'

The desire to know and see

The goal of meditation is to understand the nature of all experience, 
rather than to attain any one particular experience, however exalted. 
Many who take up the practice of meditation are dismayed to discover
just how much agitation and defilement there is in their minds, and may
come to believe that the unpleasant things they see are caused by
meditation rather than exposed by it. Many start to crave for some
special kind of experience to validate their efforts. If a particular
experience is agreed to be `special', then its experiencer or owner must
be even more so, and the feelings of rapture that accompany such
experiences seem to confirm their significance. We tend to believe that
the more intense an experience is, the more real it is. Ajahn Chah's
unbending insistence that all experiences are ultimately of the same
value, and equally able to cause suffering to one who delights in them, 
was often hard for his disciples to appreciate. Meditators want some
return for all the work they put in. On one occasion a monk came to ask
Ajahn Chah why it was that despite putting great efforts into his
meditation, he had still never seen the lights and colours that others
said they saw. Ajahn Chah replied: 

`See light? What do you want to see light for? What good do you think it
would do you? If you want to see light, go and look at that fluorescent
lamp. That's what light looks like.' After the laughter had died down, 
Ajahn Chah continued: `The majority of meditators are like that. They
want to see light and colours. They want to see deities, heaven and hell
realms, all those kinds of things. Don't get caught up with that.'

Only the posture changes

A constantly recurring theme in Ajahn Chah's teachings is the emphasis
on continuity of mindfulness. On one occasion he instructed the Sangha: 
`Meditation isn't bound to either standing or walking or sitting or
lying down, but as we can't live our lives completely motionless and
inactive, we have to incorporate all these four postures into our
practice. And the guiding principle to be relied on in each of them is
the generation of wisdom and rightness. `Rightness' means right view and
is another word for wisdom. Wisdom can arise at any time, in any one of
the four postures. In each posture you can think evil thoughts or good
thoughts, mistaken thoughts or correct thoughts. Disciples of the Buddha
are capable of realizing the Dhamma whether standing, walking, sitting
or lying down. So where does this practice which is carried out in the
four postures find its focal point? It finds it in the generation of
right view, because once there is right view, then there come to be
right aspiration, right speech and the rest of the Eightfold Path. 

`It would then be better to change our way of speaking. Instead of
saying that we come out of \emph{samādhi}, we should say merely that we
change our posture. \emph{Samādhi} means firmness of mind. When you
emerge from \emph{samādhi}, maintain that firmness in your mindfulness
and self-awareness, in your object, in your actions, all of the time. 
It's incorrect to think that you've finished work at the end of a
meditation session. Put forth a constant effort. It is through
maintaining constancy of effort in your work, in your actions and in
your mindfulness and self-awareness, that your meditation will develop.'

Slightly better than a dog

At a certain stage in practice the `defilements of insight' may arise. 
This means that such wholesome qualities as illumination, knowledge, 
rapture, bliss, strong mindfulness and equanimity arise and mislead the
meditator into a belief that he or she is enlightened: 

`Don't stick your nose up in the air because of your practice. Don't
make too much out of your experiences. Let things proceed in peace. You
don't have to be ambitious and want to get or to become anything at all. 
After they've been practising for a while, some people have a few
experiences and take them to mean that they've really attained or become
something. That's incorrect. Once at Luang Por Pow's monastery a nun
went to see him and said, ``Luang Por, I've become a stream-enterer!''
He replied, ``Errr. Bit better than a dog.'' As soon as he said that the
``stream-enterer'' screwed up her face and stormed out. That's what
happens, you see: people go right off the track. 

`In the practice, don't ever allow yourself to get puffed up. Whatever
you become, don't make anything of it. If you become a stream-enterer, 
leave it at that. If you become an \emph{arahant}, leave it at that. 
Live simply, keep performing beneficial deeds, and wherever you are
you'll be able to live a normal life. There's no need to go boasting to
anybody that you've attained this or become that. These days when people
become \emph{arahants} they can't sit still. They think, ``I'm an
\emph{arahant}'', and have to keep telling everyone else the good news. 
In the end there's nowhere they can live. In the Buddha's time
\emph{arahants} didn't make any problems. Not like the
``\emph{arahants}'' today.'

The ability to distinguish between genuine insight and the more subtle
kinds of delusion in another person is the prerogative of the
enlightened. Ajahn Chah used to tell the story of the inexperienced
teacher who sanctioned the realization of a precocious novice, only to
become aware when the novice's body was found hanging from a tree that
he was in fact mentally disturbed, Even if, as in the case of Ajahn
Chah, a teacher has the ability to tell someone's state of mind
straightaway, this does not ensure that he will be believed. Powerful
experiences in meditation can engender an unshakeable self-confidence in
the meditator. The disciple will tend to interpret the teacher's refusal
to accept the validity of his enlightenment as a misjudgement, or
perhaps as jealousy. Strong measures may be needed in such a case, and a
short, sharp shock is usually recommended. In the scriptures there are
stories of enlightened monks disabusing others of their delusions by
creating authentic hologram-like images of elephants in rut or alluring
women. Caught by surprise, the monk who had thought himself free from
fear and lust is suddenly made painfully aware that the defilements have
only been suppressed and have merely been lying latent in his mind. On
one occasion a nun at Wat Pah Pong also thought that she had attained a
stage of enlightenment. She asked for permission to see Ajahn Chah and, 
doing her best to curb her excitement, informed him of her great
realization. He listened to her silently and then, with his face a stern
mask, his voice as cold as ice, said: `Liar'. 
