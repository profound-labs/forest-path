\documentclass[
  pagePreset=largepage,
]{aruno-anecdote}

\title{Forest Path}

\usepackage{mylayout}

\begin{document}

%
\vspace*{5em}

{\centering

{\Large\chapterTitleFont\scshape\thetitle}
\medskip

{\itshape
A collection of talks, essays, and accounts\\
from the community at Wat Pah Nanachat}
\bigskip

}


%
%\clearpage
%
For Free Distribution

Sabbadānam dhammadānam jināti

The gift of the Dhamma surpasses all other gifts.

Inquiries concerning this book may be addressed to:

The Abbot\\
Wat Pah Nanachat\\
Bahn Bung Wai\\
Amper Warin Chumrab\\
Ubon 34310\\
THAILAND

\copyright\ Wat Pah Nanachat 2012

www.watpahnanachat.org


%
%\clearpage
%
Dedication of the first edition:

In commemoration of Wat Pah Nanachat's twenty-fifth Rains Retreat, we offer this publication as a dedication to Luang Por Chah and to the first Abbot of Wat Pah Nanachat, Luang Por Sumedho.

No them, no this.

Additional dedication of the second edition:

We would like to offer the good \emph{kamma} that arises from this
publication to the editor of the first edition, Paññāvuddho Bhikkhu, our
companion in the Holy Life, who passed away in 2005.

\newpage

Abbots of Wat Pah Nanachat

Ajahn Sumedho 1975--1977

Ajahn Pabhākaro 1977--1979

Ajahn Jāgaro 1979--1982

Ajahn Pasanno 1982--1996

Ajahn Jayasāro 1996--2001

Ajahn Ñānadhammo 2001-2007

Ajahn Kevali 2007- present


%
%\clearpage
%\tableofcontents
%
%\chapter{Preface to Second Edition, 2012}
%
`Forest Path' was first printed in 1999 and originally planned as the
first edition of a Wat Pah Nanachat newsletter. Over-enthusiasm and
considerable proliferation resulted in a one-off book publication which
more or less coincided with the monastery's twenty-fifth anniversary. 
Since then we have been surprised by the number of requests to reprint
this collection of little essays, talks and anecdotes about life in Wat
Pah Nanachat as it was at that time. Apart from the formal Dhamma talks
the book contains, we were hesitant at first to reprint its other
contents, the `old stories' and personal accounts by then younger
authors. But it was those other parts which in fact added much of the
book's authentic flavour and made so many people find it beneficial and
joyful to read. 

Hence it appears that our hesitation over reprinting this book was due
to a concern that the snapshots of life in the monastery offered by
those individuals at that time had become dated. In fact, though, on
recently going through the various contributions again, we found that
many of those snapshots could still be written today -- there would be a
new cast of players, but the atmosphere experienced then, and
communicated so vividly in the old `outdated' accounts, is still very
similar. 

So we are happy to realize that when we take the wholehearted
present-day Dhamma approach of genuinely experiencing what is happening
at a particular \emph{samsāric} moment in life, we also embark on
something timeless. Beyond the specific details, all our little hopes
and sorrows around daily life in the monastery and the higher values and
principles of our life -- the Buddha's core teachings -- become apparent
in these snapshots. Better still, the details of an individual's unique
experiences are in fact exemplary: they are transferable and thus pass
the test of time. Thinking in these terms gives the monks and novices at
Wat Pah Nanachat extra encouragement, as in many ways it makes the
limitations and the suffering inevitably entailed by each one's specific
experience worthwhile. 

Once we became aware of all the `good examples' already to be found in
the original `Forest Path', the option of rewriting some of the material
in a 2012 setting as a `Forest Path II' suddenly seemed pointless. With
\emph{samsāra} essentially repeating the same old drama endlessly
anyway, we thought that simply revisiting the old setting once more with
a simple reissue of the original would be a much more effective and
honest choice than reworking the text and suggesting it was `new'. It is
also true tradition to go back a little into the past, with the hope of
taking the opportunity to realize some timeless truths. 

So the material in this new edition of `Forest-Path' is an almost exact
reprint of the old 1999 version, although with the editorial assistance
we have taken the opportunity to correct some of the punctuation and
grammar. We have also created a few links to the present-day situation
 (which seems fair enough, considering that a big, impersonal monastic
community only becomes a reality when it is embodied in some specific
individuals). For us now, the present dwellers in the monastery, who in
most cases did not arrive there in time to meet the great example Luang
Por Chah in person, it seems most affirming that the principles of
monastic life as his disciples still permeate the scenes that each new
generation of monks, novices, \emph{anāgārikas} and visitors has been
continuously experiencing at Wat Pah Nanachat during the past
thirty-five years. 

So please come and see our monastery for yourself, and get your free
real-time update on our community by practising the present-moment
Dhamma with us. In the meantime, we hope you will enjoy viewing these
historical snapshots again, 

Yours in the Dhamma,\\
On behalf of the Sangha of Wat Pah Nanachat,\\
Kevali Bhikkhu,\\
(Abbot; Version 2012.) 


%
%\chapter{Note}
%Please be aware that the authors' monastic titles when the
%original essays were written have been kept (e.g. `Sāmaṇera' or `Tan').
%You will see in the short summaries added after each essay, which
%explain what has happened since then, that almost all of them are now
%senior monks whom we would usually honour by calling them `Ajahns',
%teachers.
%
%\chapter{Luang Por}
%
\begin{verse}
Luang Por

You were a fountain of cool stream water\\
in the square of a dusty town,\\
and you were the source of that stream\\
on a high unseen peak.\\
You were, Luang Por, that mountain itself,\\
unmoved but variously seen.

Luang Por, you were never one person,\\
you were always the same.\\
You were the child laughing\\
at the Emperor's new clothes,\\
and ours.\\
You were a demand to be awake,\\
the mirror of our faults, ruthlessly kind.

Luang Por, you were the essence of our texts,\\
the leader of our practice,\\
the proof of its results.\\
You were a blazing bonfire\\
on a windy bone-chilled night,\\
how we miss you!

Luang Por, you were the sturdy stone bridge\\
we had dreamed of.\\
You were as at ease\\
in the present\\
as if it were your own ancestral land.

Luang Por, you were the bright full moon\\
that we sometimes obscured with clouds.\\
You were as kind as only you could be.\\
You were hard as granite,\\
as tough as nails,\\
as soft as butter\\
and as sharp as a razor.

Luang Por, you were a freshly dripping lotus\\
in a world of plastic flowers.\\
Not once did you lead us astray.\\
You were a lighthouse for our flimsy rafts\\
on the heaving sea.

Luang Por,\\
you are beyond my words of praise\\
and all description.\\
Humbly I place my head beneath your feet.

Buang
\end{verse}



\setChapterAuthor{Ajahn Jayasāro}
\setChapterNote{An excerpt from the biography of Ajahn Chah dealing with the arrival of Ajahn Sumedho at Wat Pah Pong.}
\chapter{Twain Shall Meet}
\markright{\chapterAuthor}

\setChapterAuthor{Ajahn Jayasāro}
\setChapterNote{An excerpt from the forthcoming biography of Ajahn Chah dealing with the arrival of Ajahn Sumedho at Wat Pah Pong.}

\chapter{Twain Shall Meet}
\markright{\chapterAuthor}

From the mid-fourteenth century until its sack by the Burmese in 2310
(1767), Ayudhya was the capital of the Thai nation. Established on an
island in the Jow Phya River, it was ideally situated to act as an
entrepôt port at a time when land routes were safer than the sea, and
merchants in the Orient sought to avoid the Straits of Malacca. Within
two hundred years Ayudhya had become one of the most thriving
cosmopolitan cities in Asia. Its population of a million exceeded that
of London. Around five hundred temples, many with pagodas covered in
gold leaf, lent the city a magical, heaven-like aura that dazzled
visiting traders.

By the mid-seventeenth century the inhabitants of Ayudhya were
accustomed to the sight of \emph{farang.}\footnote{Derived from `Frank'
  or `French', the first Westerners known to the Thais.} Communities of
traders from France, Holland, Portugal and England were housed outside
the city wall. The kings of Ayudhya often employed foreign mercenaries
as bodyguards. To the Thais these strange white beings seemed like a
species of ogre: hairy, ill-smelling, quarrelsome and coarse; lovers of
meat and strong spirits, but possessors of admirable technical skills,
particularly in the arts of war. The ogres had a religion --- priests
and monks accompanied them --- but it was unappealing to the Thais, who
were content with their own traditions. Having long equated spirituality
with renunciation of sensual pleasures, they perceived the Western
religious as living luxurious lives. They found the way the missionaries
slandered each other in their competition for converts undignified; they
saw little agreement between their actions and words. The Ayudhyan Thais
gently rebuffed what they saw as an alien faith with politeness and
smiles.

But the legendary Siamese tolerance was stretched to the limit during
the reign of King Narai (1656--88), when a Greek adventurer, Constantine
Faulkon, became \emph{Mahatthai}, minister for trade and foreign
affairs, second in influence to the king himself. After his conversion
to Catholicism, Faulkon became involved with the French in plots to put
a Christian prince on the throne and thus win the whole country for God
and Louis XIV. At the old king's death in 1688, however, conservative
forces prevailed; French hopes were dashed and Faulkon was executed. For
the next 150 years the Siamese looked on Westerners with fear, aversion
and suspicion.

But as French and British power and prestige spread throughout the
region in the nineteenth century, the image of the Westerner changed. He
came to represent authority and modernity, the new world order that had
to be accommodated. As all the rest of the region fell into European
hands, Siam's independence became increasingly fragile. King Mongkut
(1851--68) reversed the policies of previous monarchs and cultivated
friendships with Western scholars and missionaries. He believed that the
only way for a small country to survive in the colonial era was to earn
the respect of the Western powers by becoming like them. He introduced
Western styles of dress and uniform. He predicted eclipses by scientific
means, undermining the hitherto unshakeable prestige of the astrologers.
He sought to reform popular Buddhism along more rational `scientific'
lines, to protect it from the missionaries' disdain. After King
Mongkut's death his son King Chulalongkorn sought to create a modern
centralized state and administration, relying heavily on Western
expertise. Members of the royal family and aristocracy were sent to
study in the West, particularly England. The humiliation inflicted upon
the Thais by the French annexation of their eastern territories
confirmed the superiority of the West in worldly matters.

At the time that Ajahn Chah reached manhood, Western culture had already
attained its pre-eminent position. Among the wealthy elite, expensive
imported clothes, motor vehicles, gadgets and foods were sought-after
status symbols. The absolute monarchy was overthrown in 1932 in favour
of a Western style democracy, which was soon displaced by a more potent
import: military dictatorship. Fascism was the new vogue--- far more
appealing to the military men running the country than the messiness of
political debate, and far more accommodating to the Thai penchant for
uniforms. Field Marshal Pibulsongkram passed laws making it compulsory
for men to wear hats and kiss their wives on the cheek before leaving
for work in the morning. The country's name was changed to Thailand.
Chauvinism was promoted in the guise of patriotism. The marginalization
of Buddhist goals and ideals, albeit coupled with official support for
Buddhist forms and rituals, became a feature of future development. In
the hamlets of Ubon, images of the West came from Hollywood. Travelling
movie companies set up their screens and loudspeakers in village wats;
Clark Gable and Greta Garbo enchanted their audiences in homely Lao,
dubbed live from behind the screen. Thus the first flesh-and-blood
glimpse of \emph{farangs} in Ubon, exciting though it was, came as a
shock. While the newly-ordained Ajahn Chah was studying in local village
monasteries, a group of gaunt ragged POWs were gaoled in the centre of
town. They were prisoners of the occupying Japanese forces, hostages
against Allied bombing raids. The local people smuggled them bananas.

Then in the nineteen sixties came the Vietnam war. Ubon, closer to Hanoi
than to Bangkok, attained strategic importance once more. By the end of
the decade twenty thousand young Americans were stationed on a sprawling
airbase to the north of the town. Huge uniformed men, black, brown and
white, strode along the streets hand-in-hand with mini-skirted
prostitutes, caroused in tacky nightclubs with names like 'Playboy' and
took their minds on vacation with `Buddha sticks'. Overhead every few
minutes came the deafening sound of F4 fighters and heavily-laden
bombers taking off on missions over Laos, Cambodia and Vietnam.

American military personnel were not, however, the only young Westerners
in Thailand at that time. It was during this period that villagers
working in the fields to the east of Wat Pah Pong became used to a
strange new sight. Tall fair-skinned young men with long hair, T-shirts
and faded blue jeans would often be seen walking along the ox-track with
a dogged, diffident stride, a large grubby backpack like a malignant
growth behind them. These young men were the first trickle of the steady
stream of Westerners who were to find their way to Ajahn Chah. They were
to become the senior members of a Western Sangha that now numbers over a
hundred and fifty monks and nuns.

`Luang Por, only a few of your Western disciples speak Thai and you
can't speak their language. How do you teach them?' This was one of the
most common questions that Ajahn Chah faced from the early seventies
onwards, as the number of his Western disciples rapidly increased. He
would explain that he was teaching Buddhism not as a philosophy but as a
way of liberation; pointing directly to the experience of suffering and
its cause was more important than finding words to describe the process.
Sometimes, to clarify this point, he would pour from the thermos flask
on the table beside him into a cup:

`In Thai we call this \emph{nam rawn} , in Lao it is \emph{nam hawn} and
in English they call it hot water . These are just names. If you dip
your finger into it there is no language that can convey what that feels
like, but even so, people of all nationalities know it for themselves.'

On another occasion a visitor, seeing all the foreign monks, asked Ajahn
Chah whether he spoke English, French, German or Japanese, to which in
every case Ajahn Chah replied that no, he could not. The questioner
looked confused: how did the foreignmonkslearn anything, then? Ajahn
Chah replied characteristically with a question: `At your home do you
keep any animals? Have you got cats and dogs? Have you got any oxen or
buffalo? Yes? Well can you speak Cat language? Can you speak Dog? Can
you speak Buffalo? No? Then how do they know what you want them to do?'

He summarized: `It's not difficult. It's like training water buffaloes.
If you just keep tugging the rope, they soon catch on.'

To Thais, water buffaloes are the epitome of dullness and stupidity.
Comparing a human being to a buffalo would normally be considered
offensive; someone who calls anyone a \emph{kwai} to their face is
either very angry or spoiling for a fight. Given the exaggerated respect
for the intelligence of Westerners common in Thailand, Ajahn Chah's
audience would always find the buffalo comparison hilarious.

The sight of the Western monks was a powerful one. At a time when
Western technology, material advances and expertise were being so
touted, here were educated young men who had voluntarily renounced the
things that people were being encouraged to aspire to; men who had
chosen to live austere lives in the forest as monks, not understanding
the language, eating coarse food, striving for peace and wisdom in the
same way that Thai monks had been doing for hundreds of years. It was
baffling, fascinating and, above all else, inspiring. Many Thai visitors
would leave Wat Pah Pong thinking that perhaps there was more to
Buddhism than they had thought. If the Westerners had so much faith in
it, how could it be outdated?

Luang Por's basic technique was not, he insisted, particularly
mysterious; he led his Western disciples, he showed them what to do, he
was an example. It wasn't necessary to impart a great deal of
information: `Even though I have a lot of Western disciples living with
me, I don't give them so much formal instruction. I lead them in the
practice. If you do good, you get good results; if you do bad, you get
bad results. I give them the opportunity to see that. When they practise
sincerely they get good results, and so they develop conviction in what
they're doing. They don't just come here to read books. They really do
the practice. They abandon whatever is bad in their hearts and goodness
arises in its place.'

The Westerners came to Buddhist teachings and monastic life without the
cultural conditioning of the Thais. In one sense they had `beginner's
mind'. Ajahn Chah found their open, questioning attitude refreshing and
stimulating. As students they were free of the complacency that he
considered such a serious obstacle for his Thai disciples. But their
openness was not without drawbacks: the Westerners were often dragged
into the quagmires of doubt. Whereas the Thai monks could usually give
themselves to the training in a wholehearted way, fired by an
unquestioning faith in the teacher and the tradition, the Westerners
were often fettered by doubts. Ajahn Chah said:

`Once you've got them to stop, these Westerners see clearly exactly how
they've done it, but in the beginning it's a bit wearing on the teacher.
Wherever they are, whoever they're with, they ask questions all the
time. Well, why not, if they don't know the answers? They have to keep
asking until they run out of questions, until there's nothing more to
ask. Otherwise they'd just keep running --- they're hot.'

\section{The first disciple -- Ajahn Sumedho}

In 1967 a Wat Pah Pong monk called Tan Sommai returned from a
\emph{tudong} trip to the north of Isan with an old friend who literally
stood head and shoulders above him. Even the most restrained monks in
Wat Pah Pong were unable to resist at least a surreptitious glance. The
new monk was 6 foot 3 inches tall, with blonde hair, an angular nose and
bright blue eyes. His name was Sumedho. The two men had run into each
other for the first time in Korea more than ten years before, dressed in
the creased white uniforms of their nations' navies. And now, by
coincidence, they had met a second time, dressed in the yellow robe of
the monk, in the meditation monastery on the banks of the Mekong River
where Sumedho had recently been ordained. They exchanged their stories.
Sumedho told Sommai how he had returned to college after the Korean War
and gained a Master's degree in Asian Studies from Berkeley. After
graduation he had joined the Peace Corps and taught English in Borneo,
before moving on for a spell at Thammasat University in Bangkok. It was
after receiving meditation instruction at the nearby Wat Mahadhatu that
his interest in Buddhism, born in Korea, had ripened into the decision
to become a monk. Now, though, after months of solitary meditation in a
small hut, Sumedho was beginning to feel some frustration about the form
of his monastic life, and was feeling the need for a more rounded way of
practice. Tan Sommai's descriptions of Wat Pah Pong were opportune and
inspired him. His preceptor kindly gave permission for him to leave, and
the two monks set off to walk down to Ubon, Sumedho feeling `as if I was
being pulled by a magnet'.

The force of attraction held. Eventually Sumedho would stay for ten
years, form the nucleus around which the Western community of monks
would coalesce and establish Wat Pah Nanachat, before moving to
England to begin the first of nine overseas branch monasteries at
Chithurst in southern England.

Someone once asked Luang Por whether he had any special connection with
Westerners that led to so many becoming his disciples. He replied that
his acquaintance was restricted to cowboy movies he had watched before
he ordained. `It was \emph{déjà vu} --- when I was a small child I went
to see a cowboy movie with my friends and one of the characters was this
big man smoking cigarettes. He was so tall it fascinated me. What kind
of human being could have such a huge body? The image has stuck in my
mind until now. And so a lot of Westerners have come. If you're talking
about causes, there was that. When Sumedho arrived, he was just like the
cowboy in the movie. What a long nose! As soon as I saw him, I thought
to myself, ``This monk is a Westerner'', and I told him that I'd seen
him before in a movie. So there were supporting causes and conditions.
That's why I've come to have a lot of Western kith and kin. They come
even though I can't speak English. I've tried to train them to know the
Dhamma as I see it. It doesn't matter that they don't know Thai customs.
I don't make anything of it, that's the way things are. I just keep
helping them out --- that's the gist of it.'

When Ajahn Sumedho asked to be accepted as a student, Luang Por agreed
but made one condition, that he should fit in with the Thai monks and
not receive any special consideration.

`At the other monasteries in Thailand where I'd lived, the fact that I'd
been a Westerner had meant that I could expect to have the best of
everything. I could also get out of the work and other mundane things
that the other monks were expected to do: ''I'm busy meditating now. I
don't have time to sweep the floor. Let someone else sweep it. I'm a
serious meditator.'' But when I arrived at Wat Pah Pong and people said,
''He's an American; he can't eat the kind of food we eat'', Luang Por
said, ''He'll have to learn''. And when I didn't like the meditation hut
I was given and asked for another that I liked better, Luang Por said,
''No''. The whole way of training was that you had to conform to the
schedule. When I asked Luang Por if I could be excused from the long
Dhamma talks which I didn't understand, he just laughed and said, ''You
have to do what everyone else does.''

Wat Pah Pong provided a very different monastic environment from the one
with which Ajahn Sumedho was familiar. In his previous wat he had been
living in solitude, sitting and walking in and near his hut,
single-mindedly devoted to the development of a meditation technique.
The only human contact had been a daily interview with his teacher. It
had been a beneficial period for him, but he had become unsure how
sustainable such a kind of monastic life would be in the long term. What
he felt he lacked was Vinaya training:

`At Wat Pah Pong the emphasis was on communal activities, working
together, eating together, etc., with all its rules. I knew that if I
was going to live as a monkI needed the monk's training, and I hadn't
been getting that at the meditation centre I had been in before. What
Luang Por gave me was a living situation to contemplate. You developed
an awareness around the monastic tradition, and it was something that I
knew I needed. I needed restraint and containment. I was a very
impulsive person with a tremendous resistance to any kind of authority.
I had been in the Navy for four years and had developed an aversion to
authority and rank. And then before I went to Thailand I had spent a few
years at Berkeley, California, where it was pretty much a case of doing
your own thing -- there was no sense of having to obey anybody or live
under a discipline of any sort. But at Wat Pah Pong I had to live
following a tradition that I did not always like or approve of, in a
situation where I had no authority whatsoever. I had a strong sense of
my own freedom and rights, and of asserting them, but I had no idea of
serving anyone else; being a servant was like admitting you were somehow
inferior. So I found monastic life very useful for developing a sense
for serving and supporting the monastic community.

`What impressed me so much about Luang Por was that although he seemed
such a free spirit, an ebullient character, at the same time he was very
strict with the Vinaya. It was a fascinating contrast. In California the
idea of freedom was being spontaneous and doing what you felt like; and
the idea of moral restraint and discipline in my cultural background was
like this big ogre that's coming to squash you, with all these rules and
traditions --- you can't do this and you can't do that --- and pressing
down on you so much.

`So my immediate reaction in a strict monastery like Wat Pah Pong was to
feel oppressed. And yet my feeling about Luang Por was that although his
actions were always within the margins of the Vinaya, he was a free
being. He wasn't coming from ideas of doing what he liked, but from
inner freedom. So in contemplating him I began to look at the Vinaya so
as to use it, not just to cut myself off or to oppress myself, but for
freedom. It was like a conundrum: how do you take a restrictive and
renunciant convention and liberate your mind through that convention? I
could see that there were no limits to Luang Por's mind. Oftentimes
attachment to rules makes you worry a lot and lack confidence, but Luang
Por was radiant. He was obviously not just someone keeping a lot of
rules, anxious about his purity. He was a living example of the freedom
that comes from practice.'

Ajahn Sumedho was impressed and reassured by Ajahn Chah's inquiries
about his meditation practice. Ajahn Chah merely acknowledged with a
grunt that the method Ajahn Sumedho was using was valid, and gave him
permission to carry on with it if he found it useful. It did not seem to
be a crucial issue. It was clear that what Ajahn Chah was teaching was
not confined to a particular meditation technique, but consisted of a
comprehensive training, the creation of a context or environment in
which any legitimate technique would bear fruit. This was exactly what
Ajahn Sumedho felt he needed:

`You have to find someone you resonate with. I'd been in other places
and nothing had really clicked. I didn't have a fixed idea of having a
teacher either, I had a strong sense of independence. But with Luang Por
I felt a very strong gut reaction. Something worked for me with him.

`The training at Wat Pah Pong was one of putting you in situations where
you could reflect on your reactions, objections, etc., so that you began
to see the opinions, views and prejudices and attachments that come up
naturally in those situations. Luang Por was always emphasizing the need
to reflect on the way things are. That is what I found most helpful,
because when you're as self-centred and opinionated as I was then, you
really need to open your mind, and so I found Luang Por's way much more
clear and direct. As I was very suppressed already, I really needed a
way of looking at myself honestly and clearly, rather than just trying
to suppress my feelings and force my mind into more refined states. He
was also very aware of the individual needs of the monks, so it wasn't
like there was a blanket technique. He realized that you really have to
figure it out for yourself, and so how I saw him, how he affected me,
was that he seemed to provide a backdrop for my life from which I could
reflect.'

Even with this kind of appreciation of the way of practice at Wat Pah
Pong, Ajahn Sumedho did not find it easy. Apart from the easily foreseen
difficulties and frustrations he experienced with the language, culture,
climate, diet and so on, he began, ironically, to harbour misgivings
about the Vinaya. His personality had always been an idealistic one; he
was drawn to the big picture, the unifying vision, and tended to get
impatient with the nuts and bolts of everyday life. He felt a natural
antipathy to the nit-picking and cavilling over trivial matters that
seemed to him to characterize Vinaya instruction:

`Even when I could understand the language, the Vinaya readings were
excruciatingly boring to listen to. You'd hear about how a monkwho has a
rent in his robe so many inches above the hem must have it sewn up
before dawn, and I kept thinking, ``This isn't what I ordained for!'' I
was caught up in these meticulous rules, trying to figure out whether
the hole in my robe was four inches above the hem or not, and whether I
should have to sew it up before dawn. \emph{Bhikkhus} would even become
argumentative about the borders of sitting cloths! When it came to the
pettiness of everyday life and of living with people of many different
temperaments, problems and characters, whose minds were not necessarily
as inspired as mine seemed to be at the time, I felt great depression.'

The Vinaya texts prescribe various duties to be performed towards a
teacher by his students. One of them is to wash the teacher's feet on
his return from alms-round. At Wat Pah Pong as many as twenty or thirty
monks would be waiting for Ajahn Chah at the dining hall footbath, eager
for the honour of cleaning the dirt from his feet or of having a hand on
the towel that wiped them dry. At first Ajahn Sumedho found the whole
thing ridiculous. Every day he would begin to fume as monks **started to
make their way out to the footbath. It was the kind of ritual that made
him feel alienated from the rest of the community. He would feel angry
and critical.

`But then I started listening to myself and I thought,
``This is really an unpleasant frame of mind to be in. Is it anything to
get so upset about? They haven't made me do it. It's all right; there's
nothing wrong with thirty men washing one man's feet. It's not immoral
or bad behaviour and maybe they enjoy it; maybe they want to do it ---
maybe it's all right to do that. Perhaps I should do it.'' So the next
morning thirty-one monks ran out and washed Luang Por's feet. There was
no problem after that. It felt really good; that nasty thing in me had
stopped.'

Although the Buddha called praise and blame `worldly dhammas', not even
the most dedicated and unworldly spiritual seekers can avoid them.
Throughout his early days at Wat Pah Pong, Ajahn Sumedho received
generous praise. In Buddhist cultures the voluntary renunciation of
sensual pleasures for spiritual training is an esteemed virtue. The
sacrifices Ajahn Sumedho had made to become a monkinspired both his
fellow monastics and the monastery's lay supporters. In leaving America
and donning the yellow robe, not only had he given up a standard of
living that Isan peasant farmers could only dream about, but he had done
so in exchange for a life in one of the strictest and most austere
forest wats in the country. The conservative Isan people, their sense of
security and well-being so bound up with the maintenance of their
traditions, were impressed at how well Ajahn Sumedho could live in exile
from the conditions he was used to, how readily he adapted to a new
climate, language and (especially) diet. They were inspired by how
diligent and dedicated he was in his practice. As the only Westerner he
stood out and was a centre of attention wherever he went, second only to
Ajahn Chah himself.

On the other hand, the Thais have a natural, apparently almost
effortless physical grace, and the monastic techniques of developing
mindfulness by close attention to detail enhance it. For them to see
Ajahn Sumedho --- physically intimidating and with an obvious zeal for
the practice, but at the same time by their standards so awkward and
ungainly, confused them. In most it provoked a quiet but affectionate
amusement; for some that amusement was soured with a hint of fear,
jealousy and resentment. Ajahn Sumedho, both a little paranoid at the
attention and also enjoying it, could not help but feel self-conscious:

`They would ask, ``How old are you?'' I'd say, ``Thirty-three.'' And
they'd say, ``Really? We thought you were at least sixty.'' Then they
would criticize the way I walked, and say, ``You don't walk right. You
are not very mindful when you walk.'' And I'd take this \emph{yarm} and
I'd just dump it down, without giving it any importance. And they'd say,
``Put your bag down right. You take it like this, fold it over, and then
you set it down beside you like that.'' The way I ate, the way I walked,
the way I talked --- everything was criticized and made fun of; but
something made me stay on and endure through it. I actually learnt how
to conform to a tradition and a discipline --- and that took a number of
years, really, because there was always strong resistance. But I began
to understand the wisdom of the Vinaya and over the years my equanimity
grew.'

\section{Pushed}

Ajahn Chah's attitude to Ajahn Sumedho changed after a few years. Seeing
his disciple's growth in confidence and the praise he was receiving, he
began to treat him more robustly. Ajahn Sumedho remembers:

`For the first couple of years Luang Por would compliment me a lot and
boost up my ego, which I appreciated because I tended to be
self-disparaging, and his constant very positive attitude towards me was
very helpful. Because I felt so respected and appreciated by him, I put
a lot of effort into the practice. After a few years it started to
change; he saw I was stronger and he began to be more critical.
Sometimes he would insult me and humiliate me in public --- but by then
I was able to reflect on it.

`There were times when Luang Por would tell the whole \emph{sala}-full
of laypeople about things I'd done that were uncouth, like my clumsy
attempts to eat with my hands. He would imitate me making a ball of
sticky rice and then making a complete mess, pushing it into my mouth
and nose. The whole \emph{sala}, monks and laypeople, would be roaring
with laughter. I'd just sit there feeling angry and embarrassed. One
time a novice picked up my outer robe by mistake and gave it to him.
Luang Por laughed and said he knew immediately whose it was because of
the bad smell, ``the \emph{farang} stink''. When I heard Luang Por say
that, of course I felt pretty indignant; but I could endure it, and
because of the respect I felt for him I didn't show any reaction. He
asked me if I was feeling all right and I said yes, but he could see
that my ears were bright red. He had a wonderful sense of timing, and so
I could work with it, and I benefited from being able to observe my own
emotional reactions to being insulted or humiliated. If he'd done that
at the beginning I would never have stayed. There was no real system
that I could see; you just felt that he was trying to help you ---
forcing you to look at your own emotional reactions --- and I always
trusted him. He had such a great sense of humour, there was always a
twinkle in his eye, always a bit of mischief, and so I just went along
with it.'

Many of Ajahn Sumedho's most powerful memories of his early years at Wat
Pah Pong are of occasions when some dark cloud or other in his mind
dissolved through a sudden insight into the desires and attachments that
conditioned it. To him Ajahn Chah's genius as a teacher seemed to lie in
creating the situations in which this process could take place ---
bringing a crisis to a head, or drawing his attention most skilfully to
what was really going on in his mind. His faith in Ajahn Chah made him
open. A smile from his teacher or words of encouragement at the right
time could make hours of frustration and irritation seem ridiculous and
insubstantial; a sharp question or a rebuke could wake him up from a
long bout of self-indulgence: `He was a very practical man and so he was
using the nitty-gritty of daily life for insight. He wasn't so keen on
using special events or extreme practices as on getting you to wake up
in the ordinary flow of monastic life, and he was very good at that. He
knew that any convention can become perfunctory and deadening after a
while if you get used to it. He was aware of that, so there was always
this kind of sharpness that would startle and jolt you.'

In the early days anger was the major fuel of Ajahn Sumedho's suffering.
He relates how exhausting the afternoon leaf-sweeping periods could be
in the hot season. One day as he toiled in the sun, his body running
with sweat, he remembers his mindfulness becoming consumed by aversion
and self-righteousness: `I don't want to do this. I came here to get
enlightened, not to sweep leaves off the ground.' Just then Ajahn Chah
approached him and said, `Where's the suffering? Is Wat Pah Pong the
suffering?'

`I suddenly realized there was something in me which was always
complaining and criticizing, and which was preventing me from ever
giving myself or offering myself to any situation.

`Another time I had this really negative reaction to having to sit up
and practise all through the night, and I must have let it show. After
the evening chanting Luang Por reminded everyone that they should stay
and meditate right through to dawn. ``Except'', he said, ``for Sumedho,
he can go and have a rest.'' He gave me a nice smile and I just felt so
stupid. Of course, I stayed all night.

`There were so many moments when you were caught up in some kind of
personal thing and he could sense that. He had the timing to reach you
in that moment when you were just ripe, so that you could suddenly
realize your attachment. One night we were in the little \emph{sala},
where we did the Pātimokkha, and his friend Ajahn Chaluay came to visit.
Usually, after the Pātimokkha was over we would go and have a hot drink,
and then join the laypeople in the main \emph{sala}. But on that night
he and Ajahn Chaluay sat there telling jokes to each other for hours,
and we had to sit there and listen. I couldn't understand what they were
talking about and I got very irritated. I was waiting for him to tell us
to go to the hall, but he just carried on. He kept looking at me. Well,
I had a stubborn streak and I wasn't going to give up. I just got more
and more angry and irritated. It got to about midnight and they were
still going strong, laughing like schoolboys. I got very self-righteous;
they weren't even talking seriously about practice or Vinaya or
anything! My mind kept saying, `What a waste of time. They should know
better'. I was full of my anger and resentment. He knew that I had this
stubborn, tenacious streak, and so he kept going until two in the
morning, three in the morning. At that time I just gave up the whole
thing, let go of all the anger and resistance and felt a wave of bliss
and relaxation; I felt all the pain had gone. I was in a state of bliss.
I felt I'd be happy if he went on forever. He noticed that and told
everyone we could leave.'

\section{Dhamma Talks}

Given Ajahn Sumedho's celebrity and his steadily growing proficiency in
Thai, it was natural that Wat Pah Pong's lay supporters would be eager
to hear him give a Dhamma talk. Four years after Ajahn Sumedho's
arrival, Ajahn Chah decided that the time was ripe for his first Western
disciple to begin a new kind of training: that of expressing the Dhamma
in words.

One night, during a visit to another monastery, Ajahn Chah caught Ajahn
Sumedho by surprise. With no prior warning, he asked him to talk to the
lay supporters who had gathered in honour of their visit. The prospect
of ascending the monastery's Dhamma seat and struggling to give an
extempore address to a large audience in a language in which he was not
particularly fluent was overwhelming. Ajahn Sumedho froze and declined
as politely but firmly as he could. But strong in his trust in Ajahn
Chah and the realization that he was merely postponing the inevitable,
he began to reconcile himself to the idea. When Ajahn Chah `invited' him
to give a talk on the next Wan Phra, he acquiesced in silence. Ajahn
Sumedho was well aware of Ajahn Chah's view that Dhamma talks should not
be planned in advance, but he felt insecure. At the time he was reading
a book on Buddhist cosmology and reflecting on the relationship between
different realms of existence and psychological states. He made some
notes for the coming talk.

Wan Phra soon came and Ajahn Sumedho gave the talk. Although his
vocabulary was still quite rudimentary and his accent shaky, it seemed
to go down well. He felt relieved and proud of himself. Throughout the
next day laypeople and monks came up to him to express their
appreciation of a fine talk, and he looked forward to basking in the sun
of his teacher's praise. But on paying respects to Ajahn Chah beneath
his \emph{kuti}, he met a stony frown. It sent a chill through his
heart. In a quiet voice Ajahn Chah said, `Don't ever do that again'.
Ajahn Sumedho realized that Ajahn Chah knew he had thought the talk out
beforehand, and that in his eyes, although it had been an intelligent,
interesting and informative discourse, it was not the Dhamma speaking;
it was merely thoughts and cleverness. The fact that it was a `good
talk' was not the point.

In order to develop the right attitude in giving Dhamma talks, a
monkneeds a thick skin. One night Ajahn Chah told Ajahn Sumedho to talk
for three hours. After about an hour Ajahn Sumedho had exhausted his
initial subject and then began to ramble, hunting for things to talk
about. He paused, repeated himself and embarked on long meandering
asides. He watched as members of his audience got bored and restless,
dozed, walked out. Just a few dedicated old ladies sat there throughout,
eyes closed, like gnarled trees on a blasted plain. Ajahn Sumedho
reflected after it was all over:

`It was a valuable experience for me. I
began to realize that what Luang Por wanted me to do was to be able to
look at this self-consciousness, the posing, the pride, the conceit, the
grumbling, the laziness, the not-wanting-to-be-bothered, the wanting to
please, the wanting to entertain, the wanting to get approval.'

Ajahn Sumedho was the only Western monkat Wat Pah Pong for four years,
until in 1971 two more American monks arrived to spend the Rains
Retreat. One of them, Dr.~Douglas Burns, was a psychologist based in
Bangkok who intended to be a monk for the duration of the retreat; the
other was Jack Kornfield (Phra Suñño), who after practising in
monasteries throughout Thailand and Burma was to return to lay life, and
become one of the most influential teachers in the American
\emph{Vipassanā} movement. Neither monk stayed at Wat Pah Pong very
long, but both exercised a strong influence on future developments. At
the end of his short period in the robes Dr Burns returned to Bangkok,
where he would recommend any Westerners interested in ordaining to go to
live with Ajahn Chah. A number of the first generation of monks came to
Ubon after such a referral. In the months that Jack Kornfield was with
Ajahn Chah he made assiduous notes of the teachings that he received,
and later printed them as the extremely popular \emph{Fragments of a
Teaching} and \emph{Notes from a Session of Questions and Answers}.
Subsequently, as Kornfield's own reputation spread in America, his
frequent references to Ajahn Chah introduced him to a Western audience.
This acquaintance was strengthened by \emph{Still Forest Pool}, a
collection of Ajahn Chah's teachings which Kornfield co-authored with
Paul Breiter, another ex- monk (formerly Venerable Varapañño).

Ajahn Chah's charisma and his ability to move and inspire his Western
disciples soon became well-known. But if Ajahn Chah was the main reason
why Wat Pah Pong became the most popular Thai forest monastery for
Westerners seeking to make a long-term commitment to monastic life,
Ajahn Sumedho's presence may often have been a deciding factor. Here was
someone who had proved it could be done, who had lived a number of years
in austere conditions with no other Western companions, and had
obviously gained much from the practice. He was both a translator, elder
brother and, more and more, although he resisted the evolution, a
teacher in his own right. Phra Varapañño arrived in Wat Pah Pong at a
time when Ajahn Chah was away for a few days. His meeting with Ajahn
Sumedho was crucial to his decision to stay:

`Sitting up there on the porch in the peace of the forest night, I felt
that here was a place beyond the suffering and confusion of the world
--- the Vietnam War, the meaninglessness of life in America and
everywhere else, the pain and desperation of those I had met on the road
in Europe and Asia who were so sincerely looking for a better way of
life but not finding it. This man, in this place, seemed to have found
it, and it seemed entirely possible that others could as
well.'\footnote{Quoted from Paul Breiter's \emph{Venerable Father: A
  Life with Ajahn Chah} (Cosimo Books, New York City 2004)}

In 1972 the Western Sangha of monks and novices numbered six, and Ajahn
Chah decided that they should spend the Rains Retreat at Tam Saang Pet,
a branch monastery perched on a steep-sided hill overlooking the flat
Isan countryside, about 100 kilometres away to the north. Personality
conflicts festered away from the guiding influence of Ajahn Chah, and
Ajahn Sumedho felt burned:

`To begin with I felt a lot of resentment about taking responsibility.
On a personal level, the last thing I wanted to do was be with other
Western monks --- I was adjusted to living with Thai monks and to
feeling at ease within this structure and culture, but an increasing
number of Westerners were coming through. Dr Burns and Jack Kornfield
had been encouraging people to come. But after the Western Sangha had
this horrendous Rains Retreat at Tam Saang Pet I ran away, spent the
rains in a monastery in the South-East and then went to India. But while
I was there I had a really powerful heart-opening experience. I kept
thinking of Luang Por and how I'd run away, and I felt a great feeling
of gratitude to him, and I decided that I would go back and serve. It
was very idealistic. ``I'll just give myself to Luang Por, anything he
wants me to do.'' We'd just opened this horrible branch monastery at
Suan Glooay down on the Cambodian border, and nobody wanted to go and
stay there. I'd gone down there for a \emph{Kathina} ceremony and been
taller than all the trees. So in India I thought I'd volunteer to go and
take over Suan Glooay. I had this romantic image of myself. But of
course, when I got back Luang Por refused to send me there, and by the
end of the year there were so many Westerners at Wat Pah Pong that he
asked me to come back to translate for them. Basically, I trusted him
because he was the one pushing me into things that I wouldn't have done
by myself.'

\section{The Author}

Tan Ajahn Jayasāro stayed on as Abbot of Wat Pah Nanachat until
2002. During that time the monastery grew in terms of monastics and Thai
laypeople keen to come and practise for short periods of time. Having
completed his five-year commitment to guide the community in Bung Wai,
he moved to the solitude of a hermitage offered to him in the Pak Chong
district of Korat province, a couple of hours' drive north-east of
Bangkok, where he has lived ever since. He currently divides his time
between solitude at his hermitage, public teaching and an active role in
the field of Buddhist-based education, both in Thailand and abroad. He
is the spiritual director of the Panyaprateep Foundation, which is the
umbrella organization for a secondary school of the same name.

Tan Ajahn Jayasāro has always maintained his close links with Wat
Pah Nanachat, visiting the monastery frequently. He is always available
for personal consultation with senior and junior monks alike. As the
author of Luang Por Chah's exhaustive biography, and with his vast
knowledge of Thai history and culture, he is also a precious source of
knowledge of our tradition for the current generation of Wat Pah
Nanachat residents.



\setChapterAuthor{Ajahn Jayasāro}
\setChapterNote{An excerpt from the biography of Ajahn Chah containing giving advice for meditators.}
\chapter{Doubt and Other Demons}
\markright{\chapterAuthor}
% Title: Doubt And Other Demons
% Author: Ajahn Jayasāro

Doubt is of two main kinds. Firstly, there is the doubt born of a lack
of sufficient information or knowledge to perform the task in hand. We
may, for instance, doubt the Buddhist teachings on a particular subject.
We may doubt which is the best route to take to a new destination.

The Buddha recognized such doubts as legitimate and did not consider them an
obstacle to spiritual growth. On the contrary, he praised a healthy
scepticism and a questioning mind: `Good, O Kālāmas, you are doubting
that which should be doubted.' The fifth hindrance to meditation, 
\emph{vicikicchā}, usually rendered in English as `sceptical doubt', is
not the mere awareness of a lack of information, but rather
unwillingness or hesitation to act upon it. The person afflicted by
\emph{vicikicchā} is paralyzed by his inability to be sure that he is
following the best course of action. In other words, he must have proof
of the truth of a proposition before seeking to verify it. The Buddha
compared this to travelling in a wilderness. The commentary explains: 

`A man travels through a desert, and being aware that travellers may be
plundered or killed by robbers, he will, on the mere sound of a twig or
a bird, get anxious and fearful, thinking ``The robbers have come!'' He
will go a few steps, and then out of fear he will stop, and continue in
such a manner on his way; or he may even turn back. Stopping more
frequently than walking, only with toil and difficulty will he reach a
place of safety. Or he may even not reach it. 

`It is similar with those in whom doubt has arisen with regard to one of
the eight objects of doubt. Doubting whether the Master is an
Enlightened One, he cannot accept this as a matter of trust. Unable to
do so, he does not attain to the Paths and Fruits of Sanctity. Thus, 
like the traveller in the desert who is uncertain whether robbers are
there or not, he produces in his mind, again and again, a state of
wavering and vacillation, a lack of decision, a state of anxiety; and
thus he creates in himself an obstacle to reaching the safe ground of
Sanctity. In that way sceptical doubt is like travelling in the desert.'

Modern education teaches us to think, to compare, to analyze, to use
logic -- `left-brain' abilities of great value in our daily lives. A
mind that is aware of many different ways of looking at things is also
usually a tolerant one. But without a strong conviction in his chosen
path, a meditator may often lack the ability to stick with that path
when the going gets tough. On the purely rational level there are always
reasonable objections to making the sacrifices that spiritual life
demands, there are always more comfortable alternatives. When the
emotional assent provided by faith is absent, reason can make Hamlets of
us all. This hindrance of doubt particularly affects those meditators
who have been successful in the conventional education system; it is the
dark side of an enquiring mind. A lot of learning can also be a
dangerous thing. The particular form of doubt varies. A practitioner may
harbour doubts about the efficacy of the technique or its suitability to
his character; he may be unsure of the teacher or agonize about his
ability to practise. \emph{Vicikicchā} is the most disabling of the
hindrances, because unlike lust or anger, for example, it is often not
perceived as being a defilement. The element of indulgence tends to be
concealed. In the early days of Wat Pah Pong, the majority of the monks
and lay supporters had strong faith in Ajahn Chah and little formal
education; crippling doubt was never a major problem. In later years, 
with more middle-class city dwellers arriving and a growing number of
Western disciples, it became more of an issue. Ajahn Chah's response to
the chronic doubters was always to point out that `Doubts don't stop
because of someone else's words. They come to an end through your own
practice.'

A suppression of doubts through belief in the words of an authority
figure must always be fragile. Blind faith makes the mind rigid and
narrow. Ajahn Chah's view was that the only way to go beyond doubts was
through understanding their nature as impermanent, conditioned mental
states. On one occasion he explained why he didn't conduct daily
interviews with the monks, as is the practice in many meditation
centres: 

`If I answer your every little question, you will never understand the
process of doubt in your own mind. It is essential that you learn to
examine yourself, to interview yourself. Listen carefully to the Dhamma
talk every few days, then use the teaching to compare with your own
practice. Is it the same? Is it different? How do doubts arise? Who is
it that doubts? Only through self-examination will you understand. If
you doubt everything, then you're going to be totally miserable, you
won't be able to sleep and you'll be off your food, just chasing after
this view and that the whole time. What you must remember is that your
mind is a liar. Take hold of it and look. Mental states are just that
way; they don't last. Don't run around with them. Just know them with
equanimity. One doubt passes away and then a new one arises. Be aware of
that process for what it is. Then you'll be at ease. If you run after
your doubts, then you won't just be unhappy, but your doubts will
increase as well. That is why the Buddha said not to attach to things.'

Some practitioners reach a certain point in their practice and then
doubt what they have attained, or what state they are in as they are
meditating. Ajahn Chah would say there were no signposts in the mind
like there are on highways: `Suppose that you were to give me a fruit. I
might be aware of the sweetness of the fruit and its fragrance, know
everything about it except for one thing: its name. It's the same with
meditation. It's not necessary to know what things are called. If you
know the name of the fruit, that doesn't make it any sweeter. So be
aware of the relevant causal conditions of that state, but if you don't
know the name it doesn't matter. You know the flavour. You've grabbed
both its legs, let it struggle all it wants. The name isn't so
important. If someone tells you, then take note of it but if they don't
there's no need to get upset.'

On another occasion Ajahn Chah comforted a Western disciple: `Doubting
is natural. Everyone starts out with doubts. You can learn a great deal
from them. What is important is that you don't identify with your
doubts; that is, don't get caught up in them. This will spin your mind
in endless circles. Instead, watch the whole process of doubting, of
wondering. See who it is that doubts. See how doubts come and go. Then
you will no longer be victimized by your doubts. You will step outside
them and your mind will be quiet. You can see how all things come and
go. Just let go of what you are attached to. Let go of your doubts and
simply watch. This is how to end doubting.'

\section{Views and opinions}

Concentration is that clear stable tranquillity which forms the basis
for the wisdom that knows things `as they are' and uproots the delusions
that generate suffering. In other words, wisdom is both the beginning
and end of the path of practice. Initially the meditator must
`straighten his views', develop a sound understanding of the value and
purpose of meditation. If he doesn't, much sincere effort may be wrongly
directed and thus wasted. An important element of Ajahn Chah's
meditation teaching involved pointing out the fallacy of wrong views and
opinions held by his disciples, and giving authentic reasons for the
practice of the correct path and encouragement in its practice. 

\section{Impatience}

The untrained human attention span seems to get shorter and shorter. We
have come to expect and often demand quick results at the press of a
button or key. Our underlying assumption is usually that speed and
convenience are good per se. But in spiritual life this does not always
apply, as there are no shortcuts waiting to be discovered. It is, the
Buddha declared, a gradual path, one that depends on gradual maturing. 
If we are in a hurry, our inability to speed things up can feel highly
frustrating. Once Ajahn Chah taught an impatient disciple: 

`Meditating in order to realize peace is not the same as pressing a
switch or putting on an electric light and expecting everything to be
immediately flooded with light. In the sentence of effort you can't miss
out any words or phrases. All dhammas arise from causes. When causes
cease, then so do their results. You must keep doing it steadily, 
practising steadily. You're not going to attain or see anything in one
or two days. The day before yesterday a university student came to
consult me about his practice. When he meditates his mind is not at
ease, it's not peaceful. He came to ask me to charge his batteries for
him [laugh]. You must try to put forth a constant effort. You can't
comprehend this through someone else's words. You have to discover it
for yourself. You don't have to meditate a lot, you can do just a
little, but do it every day. And do walking meditation every day as
well. 

`Irrespective of whether you do a lot or a little, do it every day. Be
sparing with your speech and watch your mind the whole time. Just refute
whatever arises in your mind, whether it's pleasure or pain. None of it
lasts; it's all deceptive. With some people who've never practised
before, when a couple of days have passed and they're still not peaceful
they start to think they can't do it. If that happens, you should ask
yourself whether you received any teachings before you were born. In
this life, have you ever tried to pacify your mind? You've just let it
go its own way for a long time. You've never trained your mind. You come
and practise for a certain time, wanting to be peaceful. But the causes
are not sufficient and so the results fail to appear. It's inevitable. 
If you're going to be liberated, you must be patient. Patient endurance
is the leading principle in practice. The Buddha taught us not to go too
slowly and not to go too fast, but to make the mind ``just right''. 
There's no need to get worked up about it all. If you are, then you
should reflect that practice is like planting a tree. You dig a hole and
place the tree in it. After that it's your job to fill in the earth
around it, to put fertilizer on it, to water the tree and to protect it
from pests. That's your duty; it's what orchard owners have to do. But
whether the tree grows fast or slow is its own business, it's nothing to
do with you. If you don't know the limits of your own responsibilities, 
you'll end up trying to do the work of the tree as well and you'll
suffer. All you have to do is see to the fertilizer, the watering and
keeping the insects away. The speed of growth of the tree is the tree's
business. If you know what is and what is not your responsibility, then
your meditation will be smooth and relaxed, not stressed and fretful. 

`When your sitting is calm, then watch the calmness. When it's not calm, 
then watch that -- if there's calm, there's calm, if there's not, 
there's not. You mustn't let yourself suffer when your mind's not calm. 
It's wrong practice to exult when your mind is calm or to mope when it's
not. Would you let yourself suffer about a tree? About the sunshine or
the rain? Things are what they are, and if you understand that, your
meditation will go well. So keep travelling along the path, keep
practising, keep attending to your duties, and meditating at the
appropriate times. As for what you get from it, what you attain, what
calmness you achieve, that will depend on the potency of the virtue you
have accumulated. Just as the orchard owner who knows the extent of his
responsibilities towards the tree keeps in good humour, so when the
practitioner understands his duties in his practice, then
``just-rightness'' spontaneously establishes itself.'

\section{Ambition}

Ajahn Chah would constantly encourage his disciples to cultivate the
spirit of renunciation, to see practice as a gradual process of letting
go of attachments rather than as gaining attainments. Practice
fuelled by the desire to get or become is more likely to lead to new
realms of existence rather than liberation: 

`Sometimes in meditation practice people make determinations that are
too extreme. Sometimes they light incense, bow and make a vow: ``While
this incense has not burned down I will not get up from the sitting
posture under any circumstances. Whether I faint or die, whatever
happens, I'll die right here.'' As soon as they've made the solemn
declaration they start to sit, and then within moments the Māras attack
them from all sides. They open their eyes to glance at the incense
sticks. ``Oh dear! There's still loads left.'' They grit their teeth and
start again. Their minds are hot and bothered and in turmoil. They're at
their wit's end. They've had enough and they look at the incense sticks
again, as surely they must be at an end. ``Oh no, not even half-way!''
This happens three or four times and then they give up. They sit and
blame themselves for being hopeless: ``Oh, why am I such an idiot, it's
so humiliating'', and so on. They sit there suffering about being
insincere and bad, all kinds of things, until they're in an utter mess, 
and then the hindrances arise. If this kind of effort doesn't lead to
ill-will towards others, it leads to ill-will towards yourself. Why is
that? Because of craving. Actually, you don't have to take resolutions
that far \ldots{} You don't have to make the resolution to tie yourself
up like that. Just make the resolution to let go.'

\section{The desire to know and see}

The goal of meditation is to understand the nature of all experience, 
rather than to attain any one particular experience, however exalted. 
Many who take up the practice of meditation are dismayed to discover
just how much agitation and defilement there is in their minds, and may
come to believe that the unpleasant things they see are caused by
meditation rather than exposed by it. Many start to crave for some
special kind of experience to validate their efforts. If a particular
experience is agreed to be `special', then its experiencer or owner must
be even more so, and the feelings of rapture that accompany such
experiences seem to confirm their significance. We tend to believe that
the more intense an experience is, the more real it is. Ajahn Chah's
unbending insistence that all experiences are ultimately of the same
value, and equally able to cause suffering to one who delights in them, 
was often hard for his disciples to appreciate. Meditators want some
return for all the work they put in. On one occasion a monk came to ask
Ajahn Chah why it was that despite putting great efforts into his
meditation, he had still never seen the lights and colours that others
said they saw. Ajahn Chah replied: 

`See light? What do you want to see light for? What good do you think it
would do you? If you want to see light, go and look at that fluorescent
lamp. That's what light looks like.' After the laughter had died down, 
Ajahn Chah continued: `The majority of meditators are like that. They
want to see light and colours. They want to see deities, heaven and hell
realms, all those kinds of things. Don't get caught up with that.'

\section{Only the posture changes}

A constantly recurring theme in Ajahn Chah's teachings is the emphasis
on continuity of mindfulness. On one occasion he instructed the Sangha: 
`Meditation isn't bound to either standing or walking or sitting or
lying down, but as we can't live our lives completely motionless and
inactive, we have to incorporate all these four postures into our
practice. And the guiding principle to be relied on in each of them is
the generation of wisdom and rightness. ``Rightness'' means right view and
is another word for wisdom. Wisdom can arise at any time, in any one of
the four postures. In each posture you can think evil thoughts or good
thoughts, mistaken thoughts or correct thoughts. Disciples of the Buddha
are capable of realizing the Dhamma whether standing, walking, sitting
or lying down. So where does this practice which is carried out in the
four postures find its focal point? It finds it in the generation of
right view, because once there is right view, then there come to be
right aspiration, right speech and the rest of the Eightfold Path. 

`It would then be better to change our way of speaking. Instead of
saying that we come out of \emph{samādhi}, we should say merely that we
change our posture. \emph{Samādhi} means firmness of mind. When you
emerge from \emph{samādhi}, maintain that firmness in your mindfulness
and self-awareness, in your object, in your actions, all of the time. 
It's incorrect to think that you've finished work at the end of a
meditation session. Put forth a constant effort. It is through
maintaining constancy of effort in your work, in your actions and in
your mindfulness and self-awareness, that your meditation will develop.'

\section{Slightly better than a dog}

At a certain stage in practice the `defilements of insight' may arise. 
This means that such wholesome qualities as illumination, knowledge, 
rapture, bliss, strong mindfulness and equanimity arise and mislead the
meditator into a belief that he or she is enlightened: 

`Don't stick your nose up in the air because of your practice. Don't
make too much out of your experiences. Let things proceed in peace. You
don't have to be ambitious and want to get or to become anything at all. 
After they've been practising for a while, some people have a few
experiences and take them to mean that they've really attained or become
something. That's incorrect. Once at Luang Por Pow's monastery a nun
went to see him and said, ``Luang Por, I've become a stream-enterer!''
He replied, ``Errr. Bit better than a dog.'' As soon as he said that the
``stream-enterer'' screwed up her face and stormed out. That's what
happens, you see: people go right off the track. 

`In the practice, don't ever allow yourself to get puffed up. Whatever
you become, don't make anything of it. If you become a stream-enterer, 
leave it at that. If you become an \emph{arahant}, leave it at that. 
Live simply, keep performing beneficial deeds, and wherever you are
you'll be able to live a normal life. There's no need to go boasting to
anybody that you've attained this or become that. These days when people
become \emph{arahants} they can't sit still. They think, ``I'm an
\emph{arahant}'', and have to keep telling everyone else the good news. 
In the end there's nowhere they can live. In the Buddha's time
\emph{arahants} didn't make any problems. Not like the
``\emph{arahants}'' today.'

The ability to distinguish between genuine insight and the more subtle
kinds of delusion in another person is the prerogative of the
enlightened. Ajahn Chah used to tell the story of the inexperienced
teacher who sanctioned the realization of a precocious novice, only to
become aware when the novice's body was found hanging from a tree that
he was in fact mentally disturbed, Even if, as in the case of Ajahn
Chah, a teacher has the ability to tell someone's state of mind
straight away, this does not ensure that he will be believed. Powerful
experiences in meditation can engender an unshakeable self-confidence in
the meditator. The disciple will tend to interpret the teacher's refusal
to accept the validity of his enlightenment as a misjudgement, or
perhaps as jealousy. Strong measures may be needed in such a case, and a
short, sharp shock is usually recommended. In the scriptures there are
stories of enlightened monks disabusing others of their delusions by
creating authentic hologram-like images of elephants in rut or alluring
women. Caught by surprise, the monk who had thought himself free from
fear and lust is suddenly made painfully aware that the defilements have
only been suppressed and have merely been lying latent in his mind. On
one occasion a nun at Wat Pah Pong also thought that she had attained a
stage of enlightenment. She asked for permission to see Ajahn Chah and, 
doing her best to curb her excitement, informed him of her great
realization. He listened to her silently and then, with his face a stern
mask, his voice as cold as ice, said: `Liar'.



\setChapterAuthor{Ajahn Pasanno}
\setChapterNote{An edited version of a Dhamma talk given at Wat Pah Nanachat on 18 November 1995.}
\chapter{Keep it Simple}
\markright{\chapterAuthor}

\setChapterAuthor{Ajahn Pasanno}
\setChapterNote{An edited version of a Dhamma talk given at Wat Pah Nanachat on 18 November 2538 (1995).}

\chapter{Keep it Simple}
\markright{\chapterAuthor}

When considering the Dhamma, if you look at it in one way it is quite
complicated, quite complex; there is a lot to know, a lot to figure out,
a lot of information to digest. Looked at another way, it is quite
straightforward --- it is just a matter of following it, of doing. There
is a certain element, particularly, it seems, in the Western
temperament, which makes us believe that the more information we have,
the better we will get to know about something, and so the more
information we have, the better we should be at practising the Dhamma.
This is actually not true.

A lot of importance should be given to patience: to being able to be
patient with one's experience, observing oneself, observing the world
around one and learning to trust the observer, the watcher, the ability
of the human mind to pay attention to itself. When we talk about
liberation or enlightenment, we are actually just talking about paying
attention, what the attention is directed towards. So it means learning
to observe oneself, one's experience, to recognize the quality of the
mind.

The Buddha particularly emphasized the quality of suffering, of
unsatisfactoriness. The Four Noble Truths are based on observing this
quality of unsatisfactoriness. It is something to be known.
Understanding unsatisfactoriness is a duty to oneself. The problem is
how we relate to the world around us. The way we relate to each other
means that we tend either to create or experience unsatisfactoriness.
Then we hold on to it, cling to it, judge it, try to avoid it; we create
incredible scenarios around it, we look for someone to blame because of
it, or we feel sorry for ourselves. So we create a whole range of
reaction around \emph{dukkha}. But the Buddha says that all we have to
do is just know it. This quality of knowing is to be turned to, to be
focused on our experience, and then we learn to recognize that this
knowing is a point of balance: not affirmation or rejection, not wanting
or not wanting. It is the balancing of the faculties of the mind. The
body and the mind are the tools we have for experiencing the world. We
revolve around the sense faculties of the body and the faculties of the
mind: the ability to create and experience emotional tones of happiness,
suffering or neutrality, the ability to remember, to conceptualize, to
put labels on things through perception, the ability to act in a
volitional way, to initiate thought processes and be conscious of the
world around us. These are the tools we have. The practice of the Dhamma
is learning the quality of knowing: knowing the world around us, both
the material world of the physical body and the sense spheres of the
mind, the faculties of the mind; just knowing, not reacting to the
proliferation around them, but just being with the knowing.

But although this practice is just knowing, it seems complicated because
the simplest things are difficult to sustain. So we need to develop
certain tools, certain qualities. The Dhamma provides a theoretical
framework that may look complicated but facilitates this knowing. It
requires us to come back to the human heart, which is capable of
knowing, capable of peace, capable of creating hell round us and capable
of creating celestial worlds. We have to see this point of clarity and
stillness within us in order to stop creating worlds around us. Once
Ajahn Chah and a group of his disciples went to visit a well-known
disciple of Ajahn Mun, Luang Por Khao. Ajahn Sumedho was one of the
group. They listened to a Dhamma teaching, and when they were leaving,
Ajahn Sumedho as the most junior monk in the group was the last to leave
the room. Just as he was leaving, Luang Por Khaorose quickly and came up
to him, and since Ajahn Sumedho did not know much Thai at the time,
Luang Por Khaopointed at his heart and said, `It's all here, it's all
here.' All the talking, the explanations, come back to the heart, we
have to see this clearly and pay attention to the mind, to the heart.

This is the reason the Buddha gave the teaching of the Four Noble
Truths, as it is the heart, the mind that motivates us. All sentient
beings prefer happiness to suffering, so we are motivated to try to free
ourselves from suffering. Often, however, our attempts to do so are
either superficial or misguided, and only lead to a temporary
appeasement of suffering. We put off really dealing with it to the
immediate or distant future. The Buddha said that it should be dealt
with by understanding its causes, because we can only understand
something when we understand its cause. He pointed out that often our
misapprehension of the truth, of reality, is due to \emph{avijjā},
non-knowledge. \emph{Avijjā} is often translated as ignorance, but it is
really the lack of true knowledge. Through this lack of true knowledge,
different kinds of desire are created: desire to seek out sensual
gratification; desire for the affirmation of self, for becoming; desire
for self-negation, for annihilation, the pushing away of experience -
not wanting to experience things is also a desire. So this pushing and
pulling, this grasping after experience is the real cause of our
suffering.

And so it is the relinquishing of desire that brings about the cessation
of suffering; we do not relinquish suffering itself. But if we try to
push away its immediate cause and find something more satisfactory, this
is not dealing with the real causes, which have to be seen for what they
are and relinquished. Letting go is something that one needs to feel
consciously; letting go of holding in our hearts the emotional reactions
to experiences and relationships and judgements, how things should or
should not be. It is relinquishing the whole of this, letting go of all
of it, that is summed up in a short teaching the Buddha gave when he
said, `All dhammas are not to be clung to.' This is like the core of his
teaching --- everything has to be relinquished, given up.

The nature of desire is to hoard, to cling, to attach to things, to hang
on. We have to establish attention to this tendency in our practice and
try to go against it, to let all of it go. When you actually see
suffering, you want to let go of it. The more clearly you see suffering,
the more willing you are to let go. It is somewhat similar to the method
of trapping monkeys. A small hole is cut in a coconut; it should be just
big enough for a monkey to put his paw in. A piece of some hard fruit is
put inside the coconut. When a monkey comes, being very curious, he puts
his paw in, and finding the fruit, he grabs it. Then he is stuck, as the
hole is too small to get his fist out with the fruit in it. When the
hunter comes, the monkey keeps pulling at his fist but will not let go
of the fruit. His desire for gratification is stronger than the
recognition of the suffering that will follow when the hunter grabs him
by the scruff of his neck. If the monkey could really see the suffering,
it would be easy for him to let go of the fruit and get away.

We do the same thing. Suffering is there all the time, but we do not
relinquish it because we do not see it clearly enough. As soon as we see
it, we should let it go. But we do not recognize suffering---aversion,
ill-will, anger --- and carry it around with us for long periods of
time: minutes, hours, days, weeks, months, years, because we can justify
it in some way. We are also able to suffer tremendously over things
which we perceive could give us pleasure --- and which may even be
pleasurable on a certain level, but suffering is inherent in them. The
clear recognition of suffering is therefore related to the ability to
let it go. And the ability to let go is clearly related to the degree of
awareness and mindfulness, the stability of knowing. So we come back
again to this quality of knowing: to the establishment of awareness, the
establishment of mindfulness.

The purpose of this path, the whole point of our practice, is to
facilitate this quality of clarity. Mindfulness or awareness is not
passive; there is a sense of moral responsibility within it, a sense of
patience and endurance, the ability to bring up effort. Our path lies in
developing virtue or \emph{sīla} to see our actions and speech clearly
and take responsibility for them in a moral sense. And we need to
develop the quality of renunciation and the quality of wisdom in our
practice, to question, to investigate, to reflect.

At the level of mental training we train to bring forth effort, training
to recognize ways of cultivating the wholesome and letting go of the
unwholesome. This is developing stability of mind, concentration,
steadiness of mind. The steadiness of mind that has to be developed is
an emotional steadiness in the sense of the heart and mind, not in the
sense of the analytical mind. It is the ability not to be drawn by our
habitual preferences, our wanting and not wanting, but to establish
stability. Concentration sometimes has a sense of focusing, of
exclusion. Exclusion, the blinkering of the mind, does not lead to a
really stable and still mind. There has to be an openness, not reacting
to likes and dislikes, an ability to observe, staying with the knowing.
So the steadiness pertains to the ability of recognition, the ability to
observe without a sense of focusing in an exclusive way.

So we need to develop the qualities of investigation. And the Buddha has
given the parameters, the boundaries of investigation, what to
investigate and the tools for investigating experience. The structures
of the Four Noble Truths, the five \emph{khandhas} or aggregates of
being and the six sense spheres are tools for the delineation of our
experience. They enable us to recognize the patterns of our mind, the
patterns of our experience.

So when we sit in meditation, it is very important to have a structure,
a framework, to guide us in investigating our experience. If we just sit
and watch the breath going in and out, pretty soon either the mind
starts wandering and gets hooked to something or other, or else it
becomes bored and collapses on itself, and you sit in a state of
dullness. When the mind is in \emph{samādhi} it is ready for work, the
work of a meditator, which is to investigate one's experience, to
investigate what it is that motivates one, what causes the mind to
proliferate, what it is that creates suffering. What brings a point of
balance to the mind? These are questions that need to be investigated
when we are engaged in meditation. Sometimes we sit and wait for an
illumination to descend upon us and free us from all confusion, but
that's not how the mind works. In order to understand the mind and
ourselves more clearly, we have to apply the mind to look at and
investigate the actual problems we keep running up against in our
experience. This is where we develop the exercise of coming back to the
breath to encourage mindfulness, using the in-breath and the out-breath
to clarify the movement of the mind. We use the breath as it keeps going
in and out as a framework to see where the mind is moving, to clarify
its movement. This close observation within the framework of the
meditation object clarifies and makes us understand the nature of the
mind.

Calmness of mind is not obtained by shutting things out or forcing the
mind to a point of stillness. The more you force the mind, the more
tense it becomes. What is needed is application of the mind, using the
tools set out for us by the Buddha. With the application of effort the
practice takes on all kinds of meanings, and one finds that gaps in
one's understanding are filled in and doubts and misunderstandings
overcome. So this aspect of wisdom is not just passive knowledge or a
piece of information that you get from a book or a teacher. It is
arrived at through applying one's mind, investigating one's mind
honestly. Often our minds create distractions for themselves, creating
stories around ourselves. Unless you see the mind for what it is, you
will keep buying into these stories, into all that proliferation. So to
let go we have to develop clear understanding, relinquish the mind's
creations and proliferation, let it all cease.

As we become more and more familiar with knowing, we are able to find a
quality of relinquishment, a point of stillness within that knowing.
Ajahn Chah used a lovely image to describe the proliferations of the
mind. He compares them to the wheels of an oxcart, which create deep
tracks that seem to go on endlessly. The wheels are not all that big,
but the tracks are very long. The purpose of our training and our
practice is to stop that oxcart, to let it come to rest. And this is
where our practice should be going, in the direction of a point of rest.



The Author

\emph{Tan Ajahn Pasanno was the Abbot of Wat Pah Nanachat for seventeen
years, between 1980 and 1997. During that period the monastery's
reputation as a training centre for monastics in the Ajahn Chah
tradition grew both in Thailand and abroad. In 1996 plans for the
beginnings of a new branch monastery were under way in California, under
the guidance of Ajahn Amaro. When Ajahn Pasanno joined the project with
a view to sharing the leadership of the new monastery as co-Abbot,
wonderful as this news was for California, it meant that the Wat Pah
Nanachat community and the Ubon laity would be saying goodbye to their
much-loved teacher.}

Ajahn Pasanno nevertheless left Thailand to begin Abhayagiri Buddhist
Monastery in Mendecino County, California, which is now a thriving
forest monastery with a recently-opened branch near Portland, Oregon. In
2010 Ajahn Amaro was invited to return to Amaravati Monastery in England
to take up the post of Abbot there, leaving Ajahn Pasanno as the sole
Abbot at Abhayagiri.

\emph{Luang Por Pasanno continues to inspire men and women to practise
Dhamma and lead the holy life. He combines the many facets of the modern
Western Buddhist monastic culture with maintaining close ties with the
Thai Sangha. On his annual visits to Thailand he always finds time to
stay for a while at Wat Pah Nanachat, where the whole community can
still continue to benefit from his teachings.}



\setChapterAuthor{Ajahn Vipassi}
\chapter[Facets of Life at Wat Pah Nanachat]{Facets of Life at\newline Wat Pah Nanachat}
\markright{\chapterAuthor}
% Title: Facets of Life in WPN
% Author: Ajahn Vipassi

The north-east of Thailand is flat -- the once thick forests are long
gone -- and when one drives along the long straight roads one passes
through mile after mile of flat, scrubby land given over to cultivation, 
mainly of rice. There are trees, but just here and there in the open
spaces, occasionally providing a bit of shade; there's no hint of the
majestic and almost impenetrable forest that once dominated the
north-eastern region. Then villages would have been linked by rough
jungle tracks, human beings cutting back the undergrowth here and there
to grow their crops; and they would have had to keep cutting lest nature
return to reclaim these clearings for herself. These days nature here is
firmly under man's thumb. 

Driving along the main Si Saket Road out of Warin, the first distant
sight of Wat Pah Nanachat is a long, high white wall, behind which is a
forest. The trees are tall and the growth is thick, a noticeable
contrast to the surrounding terrain. Arriving at the wat on a hot
afternoon, one's first impression on being put down at the gate is of
being about to enter a different world. The view up the drive is like
looking up a tunnel, a tunnel of trees. On venturing up the drive one
immediately feels the cool of the shade -- the forest canopy is thick
and the sun can only glint through the trees, finding an opening here
and there down which to pour a pool of fierce light. The wide, swept
concrete drive opens out after a hundred metres or so into a circle, as
one comes to a long low building on the right, the kitchen, and further
on the large unadorned \emph{sāla}. In the centre of the circle, around
which cars can turn but beyond which they cannot proceed, there is a
strange rectangular brick structure, loosely covered with a few scraps
of corrugated iron. In a few days there might be a large crowd of people
gathered here, for this is the place where the villagers cremate their
dead, something which was going on here long before Wat Pah Nanachat was
ever thought of. 

The wat came into being twenty-five years ago in a rather unlikely way. 
Ajahn Sumedho, who had already been training with Luang Por Chah at Wat
Pah Pong for years, and a group of other Western monks were wanting to
fire some alms-bowls. This is a process whereby a rustproof coating is
baked onto an iron bowl, and it requires heating the bowls in an intense
fire for several hours. In the forest at Wat Pah Pong it was difficult
to come across sufficient quantities of firewood -- there were so many
monks and firewood was needed all the time for dyeing and washing robes. 
So Ajahn Sumedho was recommended to go with the other monks to the
forest at the nearby village of Bung Wai, where there were plentiful
supplies of fallen branches and dry bamboo. The group of monks came to
the forest, put up their \emph{glots} (large umbrellas) and mosquito
nets and began their work. This soon attracted the attention of the
local villagers, who were impressed that these \emph{farang} monks had
the courage to pitch their umbrellas and camp out there, for this was
their cremation forest -- a place haunted by ghosts and spirits, and so
feared by the locals that it was left unused. 

As often happens on such occasions, when the monks were ready to move on
the local villagers begged them to stay. And it just so happened that
Luang Por Chah had already decided it would be good to start a branch of
Wat Pah Pong specifically for the \emph{farangs}. Ajahn Sumedho, who had
been with Luang Por Chah for eight years, would be the teacher, and the
\emph{farangs} could train in their own language. So, as has happened so
many times in Thailand, the simple act of a monk hanging his umbrella
from a tree was the seed that sprouted and grew into a flourishing
monastery. 

During the twelve years of my monastic life in England I heard many
things about Wat Pah Nanachat, and met and lived with many of the monks
who began their monastic careers there. Several of the Wat Pah Nanachat
monks who came to live with us in England ended up disrobing. Several of
our monks who went out to stay there did the same. It appeared that each
of our situations represented a last chance for the other: `If it won't
work in Thailand, at least try it in England before calling it quits', 
and vice versa. It was impossible not to form impressions based upon
what I'd heard, but I knew from experience that things are never quite
like you imagine -- no matter how good somebody's description, the
actuality is always far richer and more multifaceted than can be
conveyed by words. So having spent more than ten years in a non-Buddhist
country where one is part of a small group of monastics trying somehow
to model Theravāda monasticism for the surrounding culture, and where
there is often a feeling of learning to be a monk somewhat `at second
hand', I decided to go East and experience Theravāda Buddhism in some of
its native settings. Wat Pah Nanachat seemed the natural terminus for my
journey, but I was in no hurry. 

I arrived one hot evening in July 1997, the journey having taken a year
and a half, much of that spent staying in several different locations in
Sri Lanka, followed by a stint in a Wat Pah Pong branch monastery in
central Thailand, attempting to get a foothold in the Thai language. The
first impression was of the size and scale of things, the large rather
gloomy \emph{sāla} and the numbers of people. On my first morning I
watched as two large coaches pulled up and disgorged their contents, a
posse of faithful Thai people come to make offerings before the start of
the Rains Retreat. The \emph{sāla} was full -- maybe 150 people
gathered to hear Ajahn Jayasāro convey some words of wisdom. `This is
impossible!' I thought, `How can the monks here survive if these kinds
of numbers of people are descending upon the place?' However, I later
realized that this impression lacked a context. I was just seeing things
in terms of the situation in England, where people may often be coming
to the monastery for the first time and have a full bag of questions to
ask. They may also be carrying a large number of inaccurate
preconceptions about Buddhist teachings -- 'Is the Buddha a God?'; `You
Buddhists believe that life is suffering, don't you, and you're trying
to take the easy way out' -- which have to be slowly and patiently
dismantled so that sufficient openness appears for the teaching to begin
to penetrate. Not so here, where the people already have faith, where we
are only one among thousands of monasteries and this group of monks is
not solely responsible for presenting and modelling Buddhism for the
culture. 

In fact there are not so many tour bus parties, but things do build up
just before the Rains Retreat period as people go off on pilgrimage for
a few days, each day perhaps visiting six monasteries (and making two
shopping trips). After they've been to Wat Pah Pong, they simply must
come and have a look at where the \emph{farangs} live. Receiving these
visitors can be quite straightforward. Usually they have just come for a
quick look and often don't expect much teaching to speak of; it can be
enough to see foreigners with shaven heads and robes for them to be
profoundly affected. However, this kind of superficial interest is
changing somewhat as Thai laypeople seem to be getting more involved in
practising the Dhamma. Although the conversations may begin with a few
apparently innocent and superficial questions -- `How many monks are
there here?' or `Do you eat once a day?' -- it is more and more
noticeable that the conversation will move on to questions about
meditation and how to practise Dhamma in daily life. 

On that first morning, as so often happens, the wave of visitors receded
as quickly as it had flooded into the monastery, and there prevailed a
humid stillness soaked with the high-pitched sounds of cicadas. So to
pick up the question, how many monks do we have? These days it is
getting on for twenty monks and novices of about twelve nationalities. 
The number of monks who began here and still live in Thailand is
considerably greater. At any one time we will have four or five junior
monks placed at some of the Wat Pah Pong branch monasteries, having been
sent there to learn the ropes of living with a Thai community and speak
the language. After spending his first five years training under
guidance at Wat Pah Nanachat, the monk is usually `freed from
dependence' and from then on it is up to him. Some monks go off walking
on \emph{tudong}, visiting other teachers and regions. Some settle in
other places and some go abroad, but people still keep in touch and
usually regard Wat Pah Nanachat as some kind of home base, coming back
to check in once in a while. This means that at certain times of the
year there is a lot of coming and going -- in fact, the population of
the monastery can sometimes fluctuate from week to week. Thai monks also
happen by, usually on \emph{tudong}, and more often than not when we
really get down to it they are interested in learning English. This is
not enough of a reason to stay beyond three nights, says the Abbot, and
off they go. We usually do have two or three Thai monks here, but they
already speak English and have some prior Dhamma connection with Ajahn
Jayasāro or the community. For instance, one Thai monk here at the
moment was working as a doctor in America when he met Ajahn Jayasāro, 
and his faith arose there upon hearing the Ajahn teach.\footnote{He
  describes this event himself in `No Thai No Farang' on page \pageref{no-thai-no-farang}.}

The monastery serves several different and quite distinct groups of
people, and for the Abbot this is quite a balancing act. There are the
many guests from all over the world who, for many different reasons, 
spend time here developing their understanding and practice of Buddhism
through experience of monastic life. Long-term and loyal support, of
course, come from the local Bung Wai villagers, about a dozen of whom
come every day to cook and help out, and many regular supporters come to
the monastery from the local towns of Warin and Ubon. There is a
sizeable following of Bangkok people who come and stay when they can --
one group of air hostesses even arrange their schedules so that they can
fly up to Ubon on the evening flight, spend all Wan Phra night
meditating and then fly back down to Bangkok on the morning flight. In
fact, on the weekly Wan Phra observance days it is common for some
hundred people to be observing the Eight Precepts and staying to
practise and hear the Dhamma in the monastery until the following dawn. 
On these observance days the Abbot and the second monk divide their
attention between the various groups, talks being given simultaneously
in Thai and English in different locations. 

A steady stream of non-Thai visitors come and stay for varying lengths
of time. Usually the initial period is limited to three days, but in
most cases this can be extended, depending upon the availability of
accommodation. We require people to write beforehand and will only take
those who turn up unannounced if there is space. Demands on
accommodation are getting tighter these days, so quite often we have to
ask people to come back at a later date. Interest in the monastic life
can be sparked off through coming to stay at the monastery. Men are
asked to wear white and shave their heads after three days, while women
wear a white blouse and black skirt but keep their hair. These gestures
give them a chance to feel they are part of the monastic community for
the time being, and indeed they are perceived as such by the local
people. For many the level of renunciation required is quite demanding
-- living according to the Eight Precepts, eating just one meal a day, 
following a routine which requires getting up at 3.00 a.m., and having
many hours of the day with no form or structure. All this can be quite a
challenge. 

For men who wish to go further, the next step is to request to become an
\emph{anāgārika}, someone who formally joins the community in a ceremony
where he is given the Eight Precepts in front of the Sangha. 
\emph{Anāgārikas} wear a white sarong and white sash, and begin their
training in the rudiments of monastic life under the guidance of the
senior monks. There are no equivalent facilities for women to train
here, but on occasions committed women who can manage to fit into what
is undoubtedly a male-oriented atmosphere have been granted permission
to stay for periods of time. 

When people have been with us for some time as \emph{anāgārikas} and
wish to make a deeper commitment, we consider arranging for them to
become \emph{sāmaṇeras} (novices) -- taking the brown robe and looking
and behaving to all intents and purposes like the \emph{bhikkhus}, 
except that their code of discipline is less demanding. The have
alms-bowls and go on alms-round with the monks, are given a Pāli name
and are expected to commit themselves to training for one year. Those
wishing to take higher ordination can request to do so, and on taking
full ordination are expected to stay for five years as \emph{bhikkhus}
under the guidance of the Abbot. 

Community members and, as far as is possible, lay guests are each
assigned a \emph{kuṭī}, a simple wooden hut on stilts, about thirty of
which are scattered around in the forest (about 150 acres, 48 hectares
or 300 \emph{rai}). The first letters of Pāli names are derived from the
day of the week on which the individual is born, and their meaning
usually provides an ideal to which to aspire. Accommodation is basic --
there is no electricity in any but a few \emph{kuṭīs}, and a trip to the
toilet can mean a walk through the forest. At night it is not uncommon
to encounter snakes and other creepy-crawlies. Life at Wat Pah Nanachat
was once described to me as being `total insect attack'; this is an
exaggeration, but it does convey something of the flavour of the
experience. From time to time people are forced to evacuate their
\emph{kuṭī} as ants or termites invade their living space, which they
have usually already had to reckon on sharing with geckos (lizards about
20 cm long which punctuate the stillness of the night with a loud
`gekk-kko' call), bats, spiders and sometimes the odd snake which
decides to coil itself round the rafters. Rats also compete for the
space and help themselves to anything which can be eaten. 

The daily routine varies according to the season. Usually there is a
period of morning chanting and meditation at 3.30 a.m. in a large open
\emph{sāla} on the edge of the forest, followed by a leaf-sweeping
period for the lay guests while the monks go out at dawn on their
alms-round. The meal is taken at 8.00 a.m. and is followed by a period
of cleaning chores. From then until mid-afternoon there is free time, 
and besides spending that time in meditation people will make use of the
well-equipped library to read and study. At 4.30 p.m. the community
gathers for tea, which is an informal affair where questions can be
raised and things discussed in a good-humoured spirit. A couple of days
a week are kept as silent days, one when all formal meetings are
cancelled and another on which the community follows a structured
practice routine together. On these evenings a formal talk is given. 

The atmosphere of the monastery also varies according to the season. 
During the three months of the Rains Retreat the community is quite
stable, as Sangha members are not allowed to travel away for more than
six days during this period. It is a time of focused practice and study; 
in particular, study of the monastic discipline (the \emph{Vinaya}) is
undertaken during these three months. At the end of the Rains Retreat
comes the \emph{Kaṭhina}, the ceremonial presentation by the laity of
cloth which is collectively sewn into a robe by the members of the
community who spent the Rains Retreat together. This is one of the
biggest festivals of the year, and draws the community together before
monks move on to other monasteries or return from other places to live
here. There is also a tradition amongst the branch monasteries of Wat
Pah Pong to attend one another's \emph{Kaṭhina} ceremonies, and so it is
a month of travelling here and there, listening all through the night to
Dhamma talks, and trying to stay awake and centred amid the swirling
changes going on around. For new monks who are just starting to find
their feet in their first Rains Retreat, this time can be quite
disorienting. 

When the wind swings round to blow from the north, the local people say
that this marks the beginning of the cold season. As the rain stops and
the weather turns cooler, people fly kites in the almost continuous
breeze, flying them high over the rice fields. They attach a device to
them which plays a low, melancholy kind of tune over and over, and this
characterizes the atmosphere of the cold season. This is really the most
pleasant time of the year here, and it is common for senior monks from
England to come visiting during this period from late October until
February. During the last cold season we had visits from Bhante
Guṇaratana from Virginia in October, Ajahn Munindo in November, Ajahns
Pasanno and Viradhammo in December and Ajahns Sumedho and Attapemo in
January. Luang Por Sumedho comes to Thailand annually for the
commemorative celebrations for Luang Por Chah, which are held at Wat Pah
Pong in the week leading up to the anniversary of his death and
cremation (which was one year later) on 16 January. 

The cold season is also a time when frequent trips are made to the
nearest of our small hermitages. Poo Jom Gom, which means `little
pointed hill', is situated on the Laotian border, about 150 km from
here, and is set in a large area of national parkland. Four or five
monks stay there most of the time, living spread out over an area of
about two square miles, some in caves, others in simple thatched
\emph{kuṭīs}. Some of these dwellings look out over the great Mekong
river that forms the border between Thailand and Laos and flows south
from China, touching Burma, Thailand, Laos and Cambodia, before reaching
the sea in Southern Vietnam. It's one of the world's great rivers, 
comparable to the Amazon, the Nile and the Mississippi in length. At the
moment just a few little motorized canoes ply back and forth linking the
villages on either bank, which share a language and set of customs. In a
few years this area will probably develop and be much like the rest of
Thailand, but at present it is still quite remote and is touched only
lightly by the hand of modern culture. 

At the end of February almost the entire community travels across the
country to our other hermitage, Dtao Dam, on the Burmese border in
Saiyok National Park beyond Kanchanaburi. This leaves just a skeleton
crew minding the monastery, and so things quieten down as the hot season
begins. Wat Pah Nanachat remains quiet for two months, until the Sangha
returns at the beginning of May. During the following months leading up
to the Rains Retreat there are more comings and goings, people returning
to Wat Pah Nanachat to spend the Rains Retreat here, and young monks
being sent off to Thai branch monasteries to spend a year away. By the
time when Luang Por Chah's birthday is celebrated at Wat Pah Pong on 17
June, it is usually clear who is going to be where for the next four
months or so, and the monastery starts to take on a much more settled
and stable atmosphere. 

It was this situation I encountered when I first arrived here, and for
the first few months the impressions I formed were based on this
background feeling of stability in the community. It was some surprise
to see what happened here in the months after the Rains Retreat, when
all of a sudden there was a lot of coming and going. This is quite
difficult and challenging for people who are still fairly young in
training. As a young monk spending your first Rains Retreat here, you're
just starting to get your bearings and settle into the training with the
group of companions you've been living with over the last three months. 
Then suddenly the Rains Retreat is over and two people have disrobed, 
three people have shown up from other places, three more have left for
other monasteries and the character of the Sangha has completely
changed. This is quite a contrast to monastic life in England, where
there simply aren't the opportunities to leave and go elsewhere. You can
go to Aruṇa Ratanagiri or Chithurst if you've been at Amaravati for a
long time and are feeling in need of a change, but that's about it apart
from going abroad. Hence the atmosphere in the communities there is
often more stable, and I think it is easier there in some ways to stay
put and endure the difficulties you might have. And from that you can
learn and gain strength. 

The wealth of different monastic opportunities here in Thailand is both
a blessing and a curse. One is surrounded by a culture that still
carries the monastic form with considerable confidence, and this can be
tremendously uplifting for a monk who has grown up in the UK. For me, 
simply going on alms-round every day in the traditional way has felt like
a shot in the arm after thirteen years of not having had the
opportunity. The faith of the laypeople in Thailand is an unending
source of support, and there are opportunities here to meet and live
with monks of wide experience and great wisdom. On the negative side, it
can be difficult to settle. There are always people coming and going
with tales of this place or that, this Ajahn or that, and for someone
who has a lot of restlessness or discontent, the temptation to go off
and explore new pastures is indeed great. 

Reflecting on this, I feel quite grateful to have spent my first
thirteen years or so as a monk living in just three places. When you
stick with one thing, you see how moods and atmospheres in a place
change on their own. When things are difficult, uprooting and going
somewhere else is not always the answer. As Luang Por Chah once said of
a monk who was forever going off in search of a better place to
practise: `He's got dog-shit in his bag. He gets to a place and thinks, 
``Oh, this looks promising, nice and peaceful, good teacher, good
community, I should be OK here'', and he puts his bag down and settles
in. Then after a while, ``Hmmm, what's that bad smell? I can't stand for
that, the place seems to be full of it. Oh well, better try and find
somewhere else'', and he picks up his bag and off he goes.'

So when just staying put in one place one has to be willing to roll with
the changes, which can help to develop some internal stability. One has
to investigate things, reflecting again and again that this and that is
not going to last, and just letting go, letting go and not creating
problems over how things change. It's a relief to realize that one does
not have to fix things or try to hold them steady in order to feel at
ease. The problem comes from being convinced that they should be
otherwise, when, perennially, it is `just this way'. 

One factor that has brought an increased sense of stability to Wat Pah
Nanachat is the decision by Ajahn Jayasāro to stay put here for five
years as Abbot. In the past Ajahn Jayasāro and Ajahn Pasanno would take
it in turns to administer the monastery for a year at a time, which
allowed each of them to have a period of retreat every other year. 
Looking back, however, I think that Ajahn Jayasāro wonders how good this
was for the community. An additional thing that has made being Abbot
more workable is the new Abbot's \emph{kuṭī}. The previous one was
virtually open on all sides, like living on a platform, and only a
stone's throw from the \emph{sāla}, which meant that visitors could seek
the Abbot out at any time of the day or night. No wonder it was
stressful -- the Abbot had very little privacy there. I thought it a
healthy sign, then, when I saw that Ajahn Jayasāro was having a new
Abbot's \emph{kuṭī} built at quite a distance from the \emph{sāla}, in a
less conspicuous location and with a much greater feeling of privacy to
it. `That's significant', I remember thinking. `If the Abbot knows how
to look after himself, can take space and find some recuperative
solitude here, he won't feel the need to escape to get some time on his
own. That seems like a healthy direction.'

Ajahn Jayasāro has commented that there is a more harmonious atmosphere
here these days than he can ever remember. Whereas in the old days monks
used to look forward to getting past the five Rains Retreat mark so that
they could go off on their own, there is less of this kind of talk now, 
and monks who have grown up here in the last five years seem to regard
Wat Pah Nanachat as home. When the community is harmonious the Abbot is
better supported, and he is more effective at what he does. So it
becomes a more attractive prospect to stay here. 

Here, then, are just a few fleeting impressions of this mysterious, 
multifaceted place. One of the things I've heard Ajahn Jayasāro comment
upon more than once is how he feels when people talk about what Wat Pah
Nanachat is -- `Oh, you don't want to go to Wat Pah Nanachat. It's like
this or that', or `Wat Pah Nanachat is a really great place.' He says
that he's been around Wat Pah Nanachat for over twenty years and the
place is constantly changing. You can't say for certain what it is, even
though people try. They take away a snapshot of how it might have been
at a particular time when they visited or lived there, and then they
tell people, `Wat Pah Nanachat is like this', grinding out the same old
stale impressions year after year when in fact it has long since
changed. Well, if the Abbot himself declares that he doesn't really know
what Wat Pah Nanachat is like, who are the rest of us to presume to say? 

\dividerRule

\section{The Author}

Ajahn Vipassi left Thailand a short time after this piece was
written. He stayed in various monasteries in Europe, then in 2000
decided to return to lay life. He has since returned to live in the UK
and built up a computer support business.



\setChapterAuthor{Ajahn Jayasāro}
\setChapterNote{An edited version of a Dhamma talk given to the Sangha during the 1998 Rains Retreat.}
\chapter{The Beauty of Sīla}
\markright{\chapterAuthor}
% Title: The Beauty of Sīla
% Author: Ajahn Jayasāro

Ajahn Chah taught us to bear constantly in mind that we are
\emph{samaṇas}; we have left behind the household life for an existence
single-mindedly devoted to peace and awakening. He would say that now we
must die to our old worldly habits, behaviour and values, and surrender
to a new higher standard. 

But how exactly do we follow the way of the \emph{samaṇa}? In the
Ovāda Pāṭimokkha\footnote{``Not to do evil / to do good / to purify
the mind / this is the teaching of the Buddhas.'' (Dhp 183).} the Buddha laid
down the most basic and important guidelines for the \emph{samaṇa}'s
path, and there we find that harmlessness is the principle he most
emphasized. Through our way of life as \emph{samaṇas} we offer the gift
of harmlessness to the world. People may be inspired by how we live our
lives, they may be indifferent, or they may even be contemptuous of us, 
but whatever the various reactions people might have to a Buddhist monk, 
fear is highly unlikely to count among them. People see a Buddhist monk
and they know that he is not dangerous to them. Animals see a Buddhist
monk and they sense that he is no threat to them. This is a singular
thing. 

It's very unusual to be so scrupulous and so caring for even the
smallest kind of creature -- not just human beings, not just the cuddly
lovable kinds of creatures like Shetland ponies and fluffy cats, but
even poisonous centipedes, geckos and biting ants. You find that after
you've been keeping the Vinaya precepts sincerely for a while, the idea
of depriving even a venomous snake or a small poisonous insect of its
life becomes almost inconceivable. With the cultivation of \emph{sīla}
and \emph{mettā}-\emph{bhāvanā} it's just no longer an option. Through
our practice as \emph{samaṇas} we are able to observe how closely the
devotion to moral precepts is connected to being truly benevolent and
altruistic. The best austerity is patient endurance. The Buddhas say
\emph{Nibbāna} is supreme. One who has gone forth does not harm others; 
one who harms others is not a \emph{samaṇa}. 

Not to revile, not to do any harm, to practice the Pāṭimokkha restraint, 
to be moderate in taking food, to dwell in a secluded place, to devote
oneself to the higher mental training -- this is the teaching of the
Buddhas, benevolent and altruistic. If we continue to harm other beings
by body and speech, our expressions of \emph{mettā} remain hollow and
cannot lead us to peace. At the same time, if we attempt to uphold a
strict level of \emph{sīla} without a spirit of goodwill and compassion, 
without a kind and forgiving heart, we can easily fall into the traps of
self-righteousness, a false sense of superiority and contempt for the
unvirtuous. This is what is called losing the plot. 

Our practice of \emph{sīla} and \emph{mettā} starts to mature when we
don't consider that our life and our comfort have any more ultimate
significance than those of even a housefly or a mosquito. Why should our
life be any more valuable than that of a malarial mosquito? I can't
think of any logical reason myself. The Buddha said that as all living
beings desire to be happy and fear pain just as we do, we should abstain
from all actions that deprive beings of happiness or increase their
pain. \emph{Sīla} is an offering of \emph{dāna}, a gift of fearlessness
and protection to all sentient beings. To live our lives within the
boundaries defined by the Precepts, mindfulness of our commitment has to
be constantly maintained; sensitivity and skill are continually called
for. With wisdom and understanding of the law of \emph{kamma}, we
abstain from killing, harming, and hurting any sentient being through
our actions and speech. Gradually, our good intentions unbetrayed by our
actions, we are able to tame our unruly minds. 

`Not-doing' or refraining is a kind of creativity. I very much admire
Chinese brush paintings. In these works of art only a very small portion
of the canvas is painted on; the effect and power of the picture are
conveyed by the relationship between the painted form or the painted
area, and that which is not painted. In fact, the large blank area of
the white canvas is what gives the black brush strokes their power and
beauty. So if you were to say to a Chinese landscape painter, `What a
waste of good paper, there's a big white area there that you haven't
painted on at all', he would probably snort with derision. But where
human behaviour is concerned, sometimes we don't see that. I think it's
rarely appreciated that certain things we do have weight, beauty, 
integrity, nobility, precisely because of other things that we don't do. 
And that skilful abstention from actions, from certain kinds of speech, 
or from certain kinds of proliferation or imagination -- that is the
creativity. 

Artists and writers mention this often. They tend to agree that the art
lies in the editing -- in what is left out. Many writers say that it is
much more difficult to write in a simple style than in an ornate, 
complicated one. Simplicity is a skill to be learned; it does not come
easily. And this is another aspect of our life; making simplicity a
standard to return to. We must seek to refrain not only from the immoral
but also from the needlessly disturbing. We can measure our practice by
how simple our life is. We can ask ourselves, `Is my life getting more
complicated?' If it is, maybe we need to re-establish our attention on
the basics. Pictures need frames. We need wise limits for our actions. 
Otherwise our lives become cluttered and our energies dissipated. 

Appreciating the austere beauty of the simple, taking joy in simplicity, 
leads the mind to peace. What could be more simple than the
\emph{samatha} object in meditation? Whether it is the breathing process
or the word `Buddho', the experience of unifying the mind in meditation
goes against the whole tendency to mental and emotional proliferation
 (\emph{papañca}). Through meditation we acquire the taste for simplicity
in every aspect of our lives. In the external sphere, in our
relationships with others and the physical world, we rely on certain
abiding principles that support the simplicity we seek. The most
important of these is non-oppression of oneself and others. As
\emph{samaṇas} we seek to imbue our actions with a reverence for life, a
spirit of kind-heartedness, benevolence and altruism. And we learn to
make that reverence for life unqualified. The sanctity of life, and the
potential of all beings for awakening form the basis for the 227
precepts of the Pāṭimokkha.

When Ajahn Chah asked Ajahn Mun about the
discipline and voiced his fears that there were just too many rules to
make it a practical guide for conduct, Ajahn Mun pointed to
\emph{hiri}\footnote{A sense of shame regarding unwholesome
actions.} and \emph{ottappa}\footnote{An intelligent fear of
the consequences of unwholesome actions.} as the heart of Vinaya. 
Develop these two things, he said, and your practice of the Vinaya will
be impeccable. The commentaries state that these two dhammas are based
respectively on self-respect and respect for others. Respect for life, 
our own and others, is the foundation of noble conduct. So we train to
strengthen our devotion to harmlessness -- harmlessness to others, 
harmlessness to oneself -- always to bear the welfare of self and
others in mind. The more you open up to the pervasive nature of
suffering, the more compassion arises and the more care you take about
the quality of your actions. You realize that whenever you are not part
of the solution, you're bound to be part of the problem. 

In fact, the welfare of self and the welfare of others are
complementary. If we truly understand what our own welfare is, we don't
neglect the welfare of others, because in helping others we grow in
virtue. If we really understand what the welfare of others is, we don't
neglect our own welfare, because the more peaceful and wise we are, the
more we are able truly to benefit others. When there seems to be a
conflict between our welfare and that of others, it is usually a sign of
confusion about the nature of welfare. 

A second fundamental principle underlying our lives as \emph{samaṇas} is
that of contentment. We are taught to cultivate gratitude and
appreciation for the robes, alms-food, lodgings and medicines that we
receive, whatever their quality. We go against the worldly desire for
the biggest, the finest and best. We're willing to make do with second
best or third best. We find we can be happy with the worst, the things
that nobody else wants. That is a wonderful discovery. Whatever we are
given, we remind ourselves, is good enough -- beggars should not be
choosers. Even the coarsest requisites that we use have been offered
freely with faith, and have been purified by the benevolence of the
donor. It is our responsibility to make use of the requisites that are
given to us with mindfulness and wisdom. The Buddha said that the merit
gained by the donor is directly affected by the purity of mind with which
we receive and make use of the gift. Thus, even in solitude, our life is
always being affected by and affecting others. 

To be content means that we don't waste our time scheming about getting
things that we don't have or don't have a right to. It frees the body
for more wholesome activities and frees the mind for more wholesome
thoughts. As \emph{samaṇas} we do not covet the \emph{borikan} of other
monks with narrow beady eyes. We don't even touch the possessions of
others unless we have been invited to do so. 

The Vinaya lays down many detailed rules concerning our behaviour
towards the material world. In the forest tradition we're taught that
the second expulsion offence can be incurred by theft of even the
smallest object, something the value of one baht.
In the formal announcements which are part of the ordination
ceremony, the preceptor instructs the new monk to take nothing
whatsoever that does not belong to him, not even as much as a blade of
grass. 

To take on that standard -- a single blade of grass -- is the essence
of `leaving home'. It entails a radical shift of perspective from lay
attitudes. Such a standard differs not only from that of criminals and
thieves, but also from that of most `law-abiding' citizens. There are
few people who would not take advantage of some kind of little loophole
in the law if they were absolutely sure they could get away with it --
`Everyone does it, I'd be a fool not to.' Moral rectitude is not unknown
outside the walls of monasteries, of course -- I'm not by any means
suggesting we have the monopoly on honesty -- but for a whole community
to abide scrupulously by such principles is extremely rare. 

The essence of our daily life as \emph{samaṇas} consists of putting
forth effort to abandon defilements and develop wholesome qualities
through meditation practice. We spend hours a day sitting cross-legged
and walking on our \emph{jongrom} paths. Even if we may not always be
satisfied with the results of our efforts, we can at least take heart
from the fact that we've done something practical to purify our minds. 
By comparison, the training in \emph{sīla} seems nondescript and its
effects intangible. 

To maintain our devotion to precepts and \emph{kor wat}, we need to
remember that spiritual life is not just about doing, it's also about
not doing. Abstaining from things is neither immediately inspiring nor
dramatic. We don't see sudden progress in non-harming or in
non-acquisitiveness, or in not coveting things which are not ours, in
the same way as we might from a good meditation sitting or a good
retreat. But there is movement, even if it is like that of the hour hand
of a clock. And \emph{sīla} is a treasure. It is merit, it is
\emph{pāramī}. How wonderful it is that through living this life
sincerely, \emph{sīla} is steadily accumulating and maturing in our
heart. The Buddha said that \emph{sīla} is the most beautiful adornment
for a human being; it's the only fragrance that is all-pervasive. But
the skill is to remember it, to recollect the beauty of virtue, bringing
it up to refresh and give joy to your heart and mind. 

The third principle underlying the life of a \emph{samaṇa} deals with
integrity, restraint and chastity in matters related to the sexual
instincts of the body. That a group of young men -- most monks here are
young -- are able to lead a completely celibate life is almost
unbelievable to many people in the world. They assume that we have some
kind of sexual release, that we must have homosexual relationships or
else that we masturbate. They don't think it's possible to live this
way. People these days can hardly credit the idea that a community of
men can live in a completely chaste way and not be utterly screwed-up, 
repressed or misogynist. Maybe we are! -- if we were repressed we
wouldn't realize it, would we? But I don't think so. I think our
community is living the `Holy Life' in a resolute and intelligent way. 
And though it's not difficult all the time, for almost everyone there
are periods when it is definitely challenging. It's a struggle, and it
is fitting to feel a sense of wholesome pride in the fact that you can
do it. 

It's only through adopting this impeccable standard that we can begin to
understand the whole nature of sexuality. We begin to see its
conditioned nature, how it arises and passes away. We begin to see the
suffering inherent in any attachment to it, how impersonal it is, what
feeds it, what gives power to it -- whether physical conditions, food, 
lack of sense restraint or indulgence in imagination. We begin to see it
as a conditioned phenomenon. But we can only have a distance from it, be
able to reflect on it and see it for what it is by refraining from its
physical and verbal expressions. 

There is an important point about defilements here, that we have to pin
them down on the mental level before we can let go of them. And the way
we pin something down on the mental plane is by consciously refraining
from, or enduring through, the intention to express it physically or
verbally. This is where the relationship between \emph{sīla}, 
\emph{samādhi} and \emph{paññā} becomes very clear. As long as we're
still expressing sexual feelings physically or indulging in lascivious
or careless speech about sexual matters, we can never isolate them. 
They're moving, They're still receiving energy. We're still keeping them
in motion, we're still feeding the flames. So we seek to counter the
stream of craving, and to do that successfully we must aspire to
transcend sexuality altogether. It is that aspiration as much as the
actual restraint which distinguishes the \emph{samaṇa} from the
layperson. 

So as celibate monks we take a whole new stance toward our sexual
feelings, towards women -- half of the human race. We practice looking
on women who are older than us as mothers, and as older sisters if
they're just a few years older than us, or as younger sisters if they're
a few years younger than us. We substitute wholesome perceptions of
women for sensual ones. This is a beautiful gift that we can give women. 
An attractive woman comes into the monastery and we refrain from
indulging in sexual perceptions, sexual thoughts about that woman, 
replacing them with wholesome reflections, whether by consciously trying
to perceive the woman as a sister, or wishing her freedom from
suffering. Practising \emph{mettā}, we reflect, `May she be well'. We
offer women the gift of a wholesome response to them as human beings, 
rather than following the instinctive attraction to or obsession with their
bodies or some aspect of their physical appearance. Through that
intention we experience an immediate elevation from the blind, 
instinctive level of our being to the uniquely human. It is a movement
from the coarse to the refined. Indeed, the Pāli word
\emph{brahmacariya}, which we translate into English as `the Holy Life'
literally means `the way of the gods'. In other words, within the human
realm, a chaste life led voluntarily and with contentment is the most
refined, sublime and happy form of existence. 

As a fourth principle in the Dhamma, we have a love of truth. We
endeavour to uphold integrity and honesty in every aspect of our life. 
Honesty includes non-deceit, non-trickiness, non-hypocrisy, not trying
to appear in a way that is not a true reflection of how we are. This
includes not trying to hide our faults or exaggerate our good points. 
The goal is to develop clarity and straightness. This may also be seen
as a two-way process. The more honest we are with ourselves about what
we're feeling and thinking, the easier we find it to be honest with
other people. Similarly, the more we train ourselves to be honest in our
external dealings, the easier it becomes to be honest about what's going
on within us. 

An important obstacle to honesty is the sense of self. We often attach
to an idea of how we are or how we would like to be. We find it
difficult to own up to those things that don't fit the picture of
ourselves with which we identify, we feel resistance. We feel
embarrassed, ashamed, we find good reasons to dissemble. Everybody likes
to be liked and respected. Nobody likes to lose face. Integrity demands
courage. An unflinching gaze and a devotion to truth -- these are
powers, strengths to be developed. 

Truth is power. In many lifetimes the \emph{Bodhisatta} harnesses the
power of his speech through an \emph{adhiṭṭhāna}, a resolute
determination: `I have never in my life done this or done that. By the
power of these words may such and such happen.' And it happens. In some
profound inexplicable way, the truth exerts a tangible impact on the
physical world. It can affect events in the most marvellous kind of way. 
When one has built up that power of truth, one can draw on that power of
integrity with a sincere, solemn declaration. 

So in the path of awakening we take a joy and wholesome pride in caring
for the truth. We contemplate a word the Buddha would use, 
\emph{saccanurak}, having fidelity to the truth, loving the truth, being
devoted to truth, and being careful to be honest about what we really
know. It means having the clarity, when we speak, only to utter words
that we know to be accurate; being open to receiving others' viewpoints
and not thinking that what we know now is timeless absolute truth; 
learning to distinguish between what we know, what we believe, what we
think and what we perceive, and not confusing them. Often when people
say they know something, they mean they believe it. Religious people may
often consider their strong faith to be direct knowledge. The Buddha
said that we care for the truth by being very scrupulous in
distinguishing what we know as a direct experience from what we believe
to be the truth. 

Lastly, the fifth principle and precept is devotion to sobriety. The
word `sobriety' doesn't have such a pleasant ring to it. In my mind it
used to bring up an image of thin pinched people in tight clothes, 
sitting on the edge of chairs in rooms with flowered wallpaper, sipping
tea and talking about the weather. Carlos Castaneda's use of the word
rescued it for me. Now I've come to like its hard edges. Here I am using
sobriety to signify that clarity and sharpness of mind that is so
infinitely superior to confused, dull or altered states of
consciousness. 

After my travels and adventures in the East as a layman, on my way back
to England I stopped off to visit a friend in the mountains in Austria. 
She was away for a while and I was in her house by myself. Flicking
through some of her books, I came across a pamphlet called
\emph{Questions and Answers with Lama Govinda}, a transcript of a
session he'd had with some Westerners in Darjeeling. I was particularly
taken by one of his replies. Somebody asked him, `What do you think of
mind-expanding drugs?' and he said, `Well, if you've got an ignorant
mind, all you get is expanded ignorance'. That was it for me: game, set
and match to sobriety. 

Mind-expanding, mind-altering drugs and altered states of consciousness
are all still within the sphere of darkness. This is still playing
around with different modes of ignorance. Even if you experience
different dimensions of reality, without the wisdom and discernment of
\emph{paññā} you can't benefit from them. You may transcend one
particular room of ignorance, but you're still in the same building of
unknowing. When you're not out of the building, you're still in the
prison. So this sobriety means turning away from the whole razzmatazz of
abnormal experiences, visions, and physiological and mental states that
are available through liquids, fumes, powders and pills. It means
grounding yourself in the simple down-to-earth clarity of awareness --
of the eyes seeing forms, ears hearing sounds, nose contacting odours, 
tongue tasting flavours, body experiencing sensations and mind cognizing
ideas. Seeing the true nature of these things. Being with these things
as they are. And taking joy in that. Coming more and more to focus on
`the one who knows', the
knower and the knowing. This is the great mystery of our life. We don't
want to fuzz and confuse that. We want to clarify it. As we start to
clear away all the garbage of the mind, the sense of knowing becomes
clearer and sharper. 

Ajahn Tate, one of Ajahn Mun's senior disciples, stresses the sense of
knowing. He talks about the \emph{jit} and the \emph{jy}. By `\emph{jy}'
he means the sense of equanimity, the clarity of knowing; `\emph{jit}'
refers to thinking, feeling, perceiving. This is his way of talking. And
he gives a very simple means of understanding what he's talking about. 
He advises holding your breath for a few moments. Your thinking stops. 
That's \emph{jy}. You start breathing again, and as the thinking
reappears, that's \emph{jit}. And he talks about getting more and more
in contact with \emph{jy}, as the mind becomes calm in meditation. He
doesn't talk about a \emph{samādhi} \emph{nimitta} or a mental
counterpart to the breath, he talks about turning towards the one who
knows the breath. So as the breath becomes more and more refined, the
sense of knowing the breath becomes more and more prominent. He says you
should then turn away from the breath and go into the one who knows the
breath. That will take you to \emph{appaṇā-samādhi}. 

So this is sharpening this sense of knowing, knowing the one who knows, 
and that's what will take you to peace. But this ability to go from
obsession with the content of experience back to the state of
experiencing and that which is experiencing is simplifying, bringing the
mind more and more together. The mind becomes more and more composed, 
and more and more one-pointed. 

So I invite you to contemplate these principles which give grace, beauty
and meaning to life. Recognize the extent to which they are already a
part of your life, and continue to cultivate them consciously: 
principles of harmlessness, honesty, integrity, chastity, love of and
devotion to truth, sobriety and, above all, the constant clarification
and sharpening of the sense of knowing. 



\setChapterAuthor{Tan Saññamo}
\setChapterNote{A young monk's reflections on working with discomfort.}
\chapter{Mindfulness with Mosquitoes}
\markright{\chapterAuthor}
% Title: Mindfulness with Mosquitoes
% Author: Tan Saññamo

I find it a challenge to be able to translate many of the Pāli words
found in the \emph{Suttas} into day-to-day experience. Sometimes even the most
common words such as \emph{sati} or \emph{saddhā} can remain at best
loosely defined concepts in our minds; how we personally experience
these concepts is not clearly understood. Is it not worthwhile to take
the time to investigate our use of these terms for ourselves? I have
thought so, and at times have been surprised by what a little discursive
thought can dredge up. When these terms are not clear I have noticed
that doubt tends to infiltrate through this vagueness. By defining them
more clearly and connecting them to our experience, not only do we patch
up obscurities, but we can identify the presence of the qualities to
which these terms refer and their nature of rising and falling. The
following is an extract from my journal relating to an incident that has
since redefined and clarified experientially what I take to be
mindfulness in action. 

Today, during our evening meeting at the outside meditation hall, clouds
of bloodthirsty mosquitoes descended upon our vulnerable and defenceless
Sangha. Lately I have been struggling to understand how to work
skilfully with the unpleasant situations that inevitably crop up in
life. This was a fortuitous occasion to explore the possibilities. A
Dhamma talk was offered this evening. When it ended I was feeling
unusually content, with no real motivation to pursue a particular
meditation object. Instead I was satisfied with watching the momentum of
my thoughts naturally become still and settle. Inspired by such a
peaceful mind-state, I generously offered my exposed right arm and
shoulder to the mosquitoes. The response was overwhelming, so much so
that my arm started twitching involuntarily from the strain of hosting
such a banquet. Frustration began to grow, and the din in my mind that
was telling me nothing was going on and everything was okay was not very
convincing. Basically, I was being eaten alive. I didn't want to admit, 
`This is unpleasant'. Instead, thoughts like, `If you just get
concentrated, you won't feel it' or, `Develop compassion' arose in my
mind. 

In unpleasant situations I find I habitually try to convince myself of a
solution, rather than looking directly at the matter at hand. The
thoughts play the part of `the one who knows' and supply answers for
`the one who doesn't'. Though these suggestions and advice are not
necessarily wrong, they don't give space for a real understanding or
acceptance of the situation to arise. Eventually, however, my capacity
for abuse reached its limit. With patience and goodwill exhausted, I
withdrew my offering back into the folds of my robe. Not a minute later, 
a mosquito landed on the back of my head. Because both my arms were
bound beneath my robes, I was helpless. How exasperating! What happened, 
though, was interesting. I was being mindful of the sensations. I noted
the mind and the object of the mind. From looking more deeply, their
separate natures became increasingly distinct, as did their connections
and the way they influence each other. 

When pain arises in formal practice, I often can't bring myself to have
an honest look at it. However, tonight the mind admitted, `This body and
mind are agitated! The entire sensory experience is unpleasant. Telling
myself how I should feel only further obscures how I do feel.' I noticed
that admitting these feelings did not increase the pain. Instead, a
courage born of this detached acceptance began to strengthen, giving me
confidence to look even more closely at what
was going on.

The pain continued but no longer seemed to afflict the
mind. At that point the sensations were more interesting than
unpleasant; no aversion towards them was present, and so there was no
desire to end the sitting. Mindfulness and concentration were growing
together. This was a wonderful surprise; despite all the violent
sensations, I found the mind to be quite concentrated. `Interesting', I
thought, `even if the pain is stabbing and coarse, with enough courage
and resolution to observe the sensations, one can just as well attend to
them as to any other mind object.'

Perhaps, though, the more important insight which arose was on observing
the relationship between mindfulness and concentration (at least as I
experience them). What I noticed was that mindfulness indicates the
object of our attention, while concentration is a measure of the purity
of the mindfulness itself. When we speak in terms of concentration, we
are really speaking about the quality of our attention. In this sense, 
mindfulness and concentration go hand in hand. 

Suppose we look at a flower garden. A casual glance will probably give
us general impressions, 
such as the variety of colours and the general layout, but we will
probably not be able to identify any of the flowers. If we focus on a
specific area we may see the flowers more clearly, and their petals and
shapes may become distinctive. By looking closer still, their stamen, 
anther, and pollen may come into view. In this example, it can be seen
how mindfulness responds to an increased quality of attention. With more
concentration, the object of which we are mindful comes more into focus. 

In our day-to-day life, our attention is mostly diversified. The many
sense impressions we receive in consciousness dilute the quality of our
attention. When we accommodate this variety, the quality of our
attention becomes spread out, less unified and less focused. 
Alternatively, the more unified our attention becomes, the greater the
degree of discernment. The more capable of discernment we are, the more
do increasingly subtle mind objects become accessible for investigation. 

As concentration developed, I felt the rise and fall of sensations
become clearer. I saw that the continuity of events we mechanically
string together in everyday consciousness hides the immediate presence
of a single point or object. For the unconcentrated mind, it is
difficult to discern the constant rising and falling of phenomena. As
the mind grew more concentrated, I was able to get closer to seeing
things arise and cease. The series of events became more distinct. As a
result, the illusion of continuity began to waver. I watched the
process. It was then I felt I had a choice, to fix my attention on a
single object or contemplate the continuous flux of change. However, 
being unable to sustain this concentration, I lapsed back into a more
normal state of awareness. Once again, these sensations appeared to me
as a continuity. 

We all hold on to views and opinions about life and about meditation, 
and part of the task of meditation is learning how to let go of our
views. As I reflected upon this evening, what struck me was that the
night's struggle was not so much about how I related to the meditation
object, but more about the difficulty of recognizing and addressing the
personal views to which I attach. Letting go of views is difficult; they
define our relationship to certain aspects of experience. Relinquishing
them leaves us insecure, without a strategy for dealing with the
unknown. When we do manage to let go, this space provides the
opportunity for wisdom to arise. With more clarity around the basic
nature of the mind, I found that certain assumptions and tendencies I
held about meditation began to surface. For instance, I observed how I
had been sabotaging the more contemplative elements in meditation
practice by attempting to force the process, rather than letting it
unfold organically. 

Contemplation involves a great deal of receptivity. When we attach and
limit ourselves to our preconceived ideas, our receptivity is
compromised. This is an obstacle. Sometimes I give undue attention to
phenomena which may arise as a by-product of my concentration. By
attending to these
new and often fragile objects, I end up abandoning my original
meditation object. As a result, everything falls apart. This approach
has never been successful, but oddly enough, I had never seemed to
notice\ldots{}

\dividerRule

\section{The Author}

Tan Saññamo, who at the time of writing the article was a
newly-ordained monk, stayed on at Wat Pah Nanachat, completing his
five-year training in Thailand. The following year he went to live with
a much-respected disciple of Luang Por Chah living in Rayong Province, 
Ajahn Anan Akincano. Tan Saññamo trained under him at his monastery, Wat
Marb Jan, for a further five years, before visiting home in Canada for a
while. In 2011 Ajahn Saññamo moved to Abhayagiri Monastery in California
to live with the community there and further his training under Tan
Ajahn Pasanno.



\setChapterAuthor{Sāmaṇera Khemavāro}
\setChapterNote{A novice considers the changes in his attitudes to life following ordination.}
\chapter{Progress without Movement}
\markright{\chapterAuthor}
% Title: Progress without Movement
% Author: Samanera Khemavāro

My first encounter with the Dhamma was about eight months ago, at the
end of 1998. A friend of mine, Al, was undergoing short-term ordination
as a monk in a temple near Bangkok, and invited me to come along to the
ceremony and spend a week at the monastery. Al's teacher was an English
monk, Phra Peter. When I first heard some of the Buddha's teachings from
Phra Peter, even though the concepts were all new to me, something
resonated in my head. It was as if he was expressing something deep
inside but inchoate in my consciousness. The teachings on \emph{kamma}
and \emph{sīla} -- if you do good, then good things will happen to you, 
and your goodness will protect you from harm -- these were teachings
that I felt and have always tried to live by. 

\section{New experience yet familiar}

Though I believed in morality and ethics, I remained sceptical due to
their association with Christian puritanism and self-righteousness. The
Buddhist approach seemed to have a different tone. I found the teachings
on keeping precepts quite attractive. I was taught that this is
something you offer to the world and not something that is demanded from
above. Another appealing aspect of Buddhism for me is that we are
responsible for our own enlightenment. The Buddha discovered the path to
Nibbāna, but it is up to each one of us to make the effort and walk down
that path. The first time Phra Peter told me about the goal of Buddhism, 
which is to do good, refrain from doing evil and purify oneself, it felt
so natural and familiar that I thought to myself, `If I were to
verbalize the criteria or goal of my life, this would be it.'

That whole week at the monastery in Nakorn Sawan, I felt a bit odd. 
While everything was new to me, there was something vaguely familiar
about the monastic setting. I felt really at ease. For example, the
daily devotional chant, even though it was in Pāli and I couldn't
understand a word, gave me great inspiration, and so I went to all
morning and evening sessions. Phra Peter also introduced me to
meditation. Growing up in Los Angeles, a place of never-ending New
Age/spiritual fads and fashion, I was prejudiced against meditation. 
Rather hastily, I lumped it with all that trendy newfangled stuff. My
initial impression of meditation and yoga was that they were something
for bored corporate wives with little to do, the type who would only
drink hyper-hygienic sparkling water with just a twist of lime, 
organically grown by a politically correct commune, and do some yoga or
meditation before visiting their spiritual guru to have their auras
examined. 

Phra Peter felt my keen interest and offered to be my teacher if I
wanted to pursue this religious path further. While I was grateful for
the offer, I felt that a major component of being a monk was discipline, 
and, unfortunately this temple was a city temple which was somewhat lax
in its interpretation of the Vinaya rules. For example, one monk owned a
car and drove it around on the temple grounds. So Phra Peter told me
about Wat Pah Nanachat -- he had spent some time at Amaravati, an
affiliate branch monastery in England -- and what he told me interested
me, so I decided to visit it. 

\section{Past life, fast -- work hard, play hard}

I still remember the powerful surges of conflicting emotions during my
first few days at Wat Pah Nanachat. The monastery is only an hour's
flight away from Bangkok, but my lifestyle there as a
stockbroker seemed worlds apart from the lifestyle of the monastery. As
a stockbroker, my life revolved around information, a constant flow
 (sometimes a whirlwind) of information. A large part of the job is to be
able to sift through the stream of information and determine which piece
of news will have an impact on the stock market. Hence there is a
relentless search for the most updated news and `new' news. By nine
o'clock in the morning I would have read four newspapers (two local, one
regional, and one international); then I would continue to scan for
further news updates from international news services (Reuters and
Bloomberg) and check with the research department regarding recent
developments regarding companies, as well as broad economic and
political trends. 

Working in such a fast-paced environment, one tends to maintain the
momentum throughout the day. After work I would rush to the gym for a
quick workout, then meet up with friends and colleagues for drinks and
then dinner. I would be out until about 10:30 - 11:00 p.m. two or three
times during the week. The weekends would be filled with brunches and
lunches, dinners and clubs. Sometimes I would be literally running from
one appointment to the next. Rarely would I be home before midnight. And
then there would be weekend trips to Phuket, Chiang Mai, Hong Kong or
Singapore. It would not be uncommon for me to go to the airport from
work on Friday night and come back to the office on Monday morning
straight from the airport, having spent the weekend in Hong Kong or
Singapore. I was brought up with the motto `work hard, play hard`. 
Unfortunately, nobody told me about contentment. So in spite of all the
sensory diversions and options available to me, during the past couple
of years I had felt bored and disenchanted with life. Things started to
slow down at work due to the economic recession. Regarding my personal
life, I began to notice that no matter where I was or what I was doing, 
there was an undercurrent of boredom and existential anxiety. I would be
at some `fabulous party' or the `in' club; and then such feelings would
come over me. I would look around and realize that everybody looked as
lost as I was, and seemed to be trying to fill up their lives with the
same type of material possessions, clothes and cars; and sensory
diversions like going to restaurants and clubs and travelling to strange
and exotic places, or self-annihilation through drugs and alcohol. 

\section{Life in the monastery -- paradoxes and parables}

Coming from such a fast-pace and sensory-driven world to Wat Pah
Nanachat, where it seemed that the only sounds were from the swaying
bamboo bushes and leaves falling, was a bit of a shock. It was like
running on the treadmill with headphones on, listening to music with the
volume on high, and suddenly having someone come up and pull out the
plug. Coming from such a sensual world, I remember feeling quite lost
and ill at ease at times with the calm and stillness of monastery life. 
I can recall experiencing many mood swings during my first week. Yet
overall I was quite attracted to the simple and peaceful life in the
monastery, and the structured environment of having scheduled activities
throughout the day really appealed to me. 

In contrast to the myriad choices in the outside world, this structured
and simple monastic lifestyle may seem stifling and monotonous. But
nothing could be further from my experience so far. As a layperson I was
rushing from one sensory experience to another, constantly planning
where I should be next, but never really being present wherever I was. 
If I was eating dinner, my mind would be planning where to go
afterwards: `Should I go to so and so's party or hit the bars, or both?'
And then there was the dilemma of tomorrow: with whom should I go to
lunch? And that would lead to where we should go for lunch, and the same
for dinner, and then bars and clubs, and on and on. The irony in such a
go-go and `glamorous' life, was that everything ended in boredom or, 
worse, oblivion. Most of the time I could not remember what I did
yesterday. I remember thinking: `I got cheated somehow! I have done
everything I am supposed to do. They all told me that if I worked hard, 
followed all the rules, and paid my dues, success would come, and with
it everlasting happiness. By all accounts I am a poster-child of
success. I am barely in my thirties, bringing home six-figure pay
cheques, dining at the best restaurants, taking holidays anywhere in the
world and buying whatever I want. Yet I feel so bored and discontented. 
This is utterly, utterly unfair!' In my fast-paced life there were
endless variations, yet my habitual way of reacting limited my world. 

A good image to illustrate this is a small circle flying through space: 
although the space surrounding the circle is infinite, my habitual way
of responding and seeing things limited my vision to just that small
circle. In a similar way, the Wat Pah Nanachat logo of a lotus in a
square is for me a meaningful image for monastic life. Although the
lotus is contained in the square, it has endless variety in terms of its
positions. Here at the monastery the slight variations and nuances in
monastic life fascinate me. Every day I am excited to wake up to the
endless permutations of life in the monastery. How mindful will I be
during the meal? Walking on alms-round? In my interactions with
different members of the Sangha? The scenery of the sunrise over the
rice fields outside the gate is a constant source of surprise and
delight. Mundane things like the texture of the gravel road I walk on
during the alms-round attract my interest. Is it soft and muddy from the
rain last night, or is it hard from being baked by the hot sun
yesterday? What is the sensation on the soles of my feet? Why do they
hurt more today than yesterday? Am I mindful of my steps or am I off
somewhere plotting a revolution? And then there are my mind states in
the morning. Am I happy and relaxed? Or a bit anxious and irritated? And
what is the cause of these different feelings? And am I mindful of them
as feelings, or do I get caught in them? 

I am still quite mystified by the paradox of how rich and diverse life
at the monastery can be. On one level it can seem quite repetitious and
regimented. With few exceptions, the same things take place every day. 
We go on alms-round at 5:30 in the morning and have our one meal at
about 8:00; after that an hour of chores, followed by more work or
meditation, and then it is tea-time at 4:30 p.m. Yet within that
regulated environment there are countless variations and permutations in
the surroundings, and in myself as well.

It is through this repetitive
and structured environment that I learn about myself, how I perceive and
react to my surroundings. There are many levels to the practice. For
example, at mealtime, how mindful am I when walking down the line to
collect the alms-food? Did I exercise self-restraint and take only a few
pieces of mango, and leave some for the people behind me in the line? Or
did my defilements overwhelm me, so I filled half my bowl full of
mangos? Am I exercising sense restraint in terms of keeping to myself, 
or am I anxiously looking at the front of the line to see which monk is
taking more than his share of the mangos and feeling ill-will towards
him? 

\section{The practice -- walking the walk}

Two areas of the monastic practice that I find interesting in its
contrast to my lay life are the practice of meditating throughout the
night on Observance Day (Wan Phra) and eating once a day. Before being a
stockbroker I was an investment analyst, which entailed writing research
reports about companies listed on the stock market. Working for ING
Barings, one of the top international brokerages in Thailand, I had a
heavy workload and strict publishing deadlines. The company's mantra was
`Publish or Perish'; hence it was not uncommon to work through the night
to meet a particular deadline. As a matter of fact, I had to `pull an
all-nighter' about once a month to get a certain report published by the
deadline. During these all-night sessions we had lots of help to keep
the adrenaline going. There would be a group of people at the office to
help finish the report, and then there were TV and radio, and pizza and
beer. There was much talking and running around to complete the final
details. 

\section{Altered states, altered egos}

At Wat Pah Nanachat we are encouraged to stay up all night and meditate
on Wan Phra, which falls about once every week. But instead of all the
sensory stimuli to help keep the adrenaline going and
the body awake, the only help in that area is a cup of coffee at
midnight. Other than that, one is supposed to meditate quietly by
sitting or walking. Needless to say, staying awake all night is more
challenging without the aid of external stimuli such as TV or radio, but
working with my mind states from 2:00 to 4:00 a.m. has been quite
revealing. One moment I can be feeling dull and sleepy, the next
restless and resentful. The following is the type of internal chatter
that took place during the last Wan Phra at 3:15 a.m.

`I should be in bed. This is a silly practice, staying up all night. 
It's a dead ritual without any rhyme or reason. What am I trying to
prove, anyway? How much \emph{samādhi} could I get in this current state
of stupor? Where is everybody, especially the monks? Why aren't they up
meditating? And why is that senior monk nodding off? He has been doing
this for a while, you would think he'd have got over this problem. He
doesn't seem very developed anyway, and looks as if he hasn't much to
show for all those years. Maybe it's not him, maybe it's the practice. 
Maybe it does not work after all. And why doesn't that stupid clock move
any faster? My knees are sore, my back hurts, and I hate this place.'

This is hardly the picture I have of myself as a calm, collected and
compassionate person. But the beauty of the teaching is that we are
taught to accept things for what they are, being open to all the aspects
of our personality, the good, the bad, and the ugly. The practice has
been helpful for me in recognizing and dealing with my weaknesses and
shortcomings. It is liberating to realize that I have these unwholesome
mind states, but also that they are just mind states, and to be aware of
them as such and not get caught in them or identify with them. 

One of the biggest challenges I have faced so far has been not eating
after noon. Part of the forest tradition discipline is that monastics
only eat one meal a day. With few `medicinal' exceptions such as dark
chocolate, sugar and butter, no solid food should be consumed after
midday. Before coming to Wat Pah Nanachat, food was not an issue for me. 
I have never had a weight problem and can eat pretty much whatever I
want, but always in moderation. However, in the monastery, with so few
outlets for my desires to express themselves, food has taken on a
disproportionate role. I constantly think of things I can eat or
reminisce about all the nice dining experiences I have had. As an
\emph{anāgārika}, one of my responsibilities is to prepare the afternoon
drinks for the Sangha. This entails being in the kitchen and around
food, which has been a challenge for keeping the food precept. Part of
my problem in dealing with the food issue has been that I do not see the
logic in being able to eat dark chocolate in the afternoon, but not a
banana. However, after several discussions with senior monks, I am
beginning to realize that the purpose of the ascetic practices of one
meal a day and not indulging in any worldly behaviours is to calm the
mind, which is conducive to achieving \emph{samādhi} in meditation. 

As a beginner meditator, here again I had that feeling of \emph{déjà
vu}. Not that I was entering the \emph{jhānas} in my first week of
meditation, but at the end of most sessions I have a sense of calmness
and centredness that I find quite refreshing. There is not so much
restlessness, and the preoccupation with food is not so gripping. My
perception of my surroundings seems to be enveloped in a mist of
goodwill and gentility. The irony about just sitting is that in contrast
to my blind pursuit of happiness and excitement in my lay life, which
ended up in boredom and desperation, by just sitting in my \emph{kuṭī}
and counting my breaths, I am finding enthusiasm and contentment. 

\section{Conclusion}

While I have had my share of frustrations and disappointments at Wat Pah
Nanachat when dealing with my own defilements, overall I am finding the
experience fascinating and delightful. And though my monastic life has
been somewhat short, only six months, I am finding joy. There is
excitement, but it is a different kind of excitement from what I found
when working on the stock market. 

I was ordained as a \emph{sāmaṇera} before the Rains Retreat and plan to
remain one for one year. I feel quite fortunate in having had the
opportunity to be living as a monastic so soon after my introduction to
the Dhamma, but I also realize that there is much work to be done. While
I have a strong sense of responsibility to be diligent in putting forth
the efforts required of a monastic, there is also a sense of thrill and
anticipation on this journey of self-discovery. 

%\dividerRule

\clearpage

\section{The Author}

Soon after this piece was published Sāmaṇera Khemavāro took monk
ordination and spent a further couple of years in Thailand. He
subsequently moved to Australia, taking up residence initially at
Bodhiñāṇa Monastery, where he lived for seven years. For the last three
years he has been the Abbot of Wat Buddha Dhamma, a very secluded forest
monastery a couple of hours' drive from Sydney.



\setChapterNote{An interview with three Thai monks practising at Wat Pah Nanachat.}
\chapter{No Thai, No Farang}
\markright{\chapterAuthor}
% Title: No Thai, No Farang
% Author: n/a

\label{no-thai-no-farang}

In addition to the monks from abroad, a number of Thai monks also
come to Wat Pah Nanachat to live and practice. Tan Jayasiri, Tan
Jotimanto and Tan Dhīrapañño are three such monks. Each has lived at Wat
Pah Nanachat for a number of years and has served the community greatly
by acting as secretary to the Abbot. This interview was conducted by Tan
Paññāvuddho.

\emph{Tan Paññāvuddho}: All three of you grew up in Thailand. I'd be
curious to know what were the first impressions you can remember from
your childhood when you saw Buddhist monks. 

\emph{Tan Jayasiri}: There was a branch of Wat Pah Pong (Ajahn Chah's
main monastery) near my home in the countryside outside Si Saket. When I
was a little boy I went to the monastery every day. It was clear to me
from an early age that I wanted to become a monk in the future. I liked
the way the monks shaved their heads and wore their robes, not dressing
like laypeople. When I first went to the monastery I was four years old. 
I went to the `Sunday school' there. 

\emph{Tan Paññāvuddho}: What do you remember about the first time that
you saw the monks in meditation? What feelings arose for you? 

\emph{Tan Jayasiri}: I felt special. It was so different from what
worldly people did. I thought I would like to be like them in the
future. I also noticed that the monks did not live together, but in the
forest in their own \emph{kuṭīs}. That interested me. At the forest
monastery I observed that each monk would put all his food in one alms
bowl (curries and sweets together), and eat it at one sitting in the
morning. When I was young I had the opportunity to eat the food left
over from the monks' bowls, as is common for laypeople at the monastery. 
It smelled so different from normal food. I thought it smelled like the
scent of heaven, very strange for me. While I went to school during the
week, my grandmother went to the wat. At midday she would bring the
monks' food back from the wat, so I would go home from school during the
lunch break to eat it. If she didn't have any food that day, I felt
upset. Years later, on the first day I became a monk, I tried to smell
the food in my bowl during the meal. It smelled normal, nothing special
at all. Maybe the smell was just bait to get me here! 

\emph{Tan Paññāvuddho}: When you were little, did you ever have the
opportunity to meet any of the great \emph{Kruba Ajahns} like Ajahn
Chah? 

\emph{Tan Jayasiri}: Sure. When I was a young boy about seven years old, 
I went to see Luang Por Chah at Wat Pah Pong. At that time the monastery
was still very simple. Although many monks lived there, after the meal
the place looked deserted because all the monks went back to their
\emph{kuṭīs} to practice. Luang Por Chah, however, always received many
visitors. He smiled all the time. When I looked at him I felt something
very special. 

\emph{Tan Paññāvuddho}: Why do you think you had so much faith at an
early age? 

\emph{Tan Jayasiri}: I don't know. It was just normal for me. When I was
little and saw suffering in the world, I would think about the monks, 
their robes and their bowls. Ajahn Chah's smile is always in my memory, 
it has never gone. Even today it is very clear. It is a very special
feeling for me to be a monk in his tradition. I have never had any
doubts about Luang Por Chah and the \emph{Kruba Ajahns}.

\emph{Tan Paññāvuddho}: Tan Jotimanto, can you recall your experience
when you first became aware of Buddhist monks? 

\emph{Tan Jotimanto}: Actually, my grandmother was my main influence. 
When I was young I went to the monastery quite often with her. There was
a forest monastery nearby where Luang Por Poot Thaniyo was the abbot. My
family was very close to Luang Por Poot. From an early age I was taught
to pay great respect to the monks. During the Rains Retreat my
grandmother always went to the monastery on the Wan Phra days, spending
all day and night observing the Eight Precepts. Sometimes I would go and
stay with her. 

\emph{Tan Paññāvuddho}: When you observed the monks in the forest
monastery environment, what did you experience? 

\emph{Tan Jotimanto}: At that time I thought the monks were very special
people, and that they must have psychic powers because they taught us
about heavens and hells. Actually, as is the case with many Thai
children, I was afraid of ghosts and I thought the monks could help. I
felt that since the monks had good \emph{sīla}, the ghosts would be
afraid of them. 

\emph{Tan Paññāvuddho}: Tan Dhīrapañño, how about yourself? What were
your first impressions of the \emph{Buddha-Sāsana}?

\emph{Tan Dhīrapañño}: In my case they were similar to Tan Joti. When I
was young I grew up with my grandmother in Chonburi, out in the
countryside. She used to go and \emph{sai baht}\footnote{\emph{sai baht}
(Thai): to give alms-food (literally: to put food in a monk's bowl)}
every morning. On Wan Phra she would wake up extra early in the morning
to make special food for the monks. You see, as a little boy I always
slept in the same bed with my grandmother, so on the days when I woke up
and did not see her beside me, I knew that day had to be Wan Phra. 

\emph{Tan Paññāvuddho}: Can you remember when you were young and your
grandmother taught you to \emph{sai baht}? 

\emph{Tan Dhīrapañño}: I remember her telling me to kneel down, put my
hands in \emph{añjalī}, be quiet and carefully put food into the monks'
bowls without touching the brim. That was in the morning and it was not
that difficult. On certain special days she would take my cousins and me
to the monastery. That was the hardest part because I didn't understand
most of what the monks were teaching, and I had to sit with my legs
politely folded behind me in the \emph{papiap} position. I would sit
with my grandmother in the \emph{sāla} while my younger cousins played
loudly outside. Anyhow, my grandmother always seemed to be very happy on
that day, and being around her, somehow I felt very happy too. 

\emph{Tan Paññāvuddho}: Tan Jayasiri, what gave you the inspiration to
be ordained as a monk? 

\emph{Tan Jayasiri}: Actually, I always had a strong ambition to become
a monk. I was thinking about it when I was very young, 

\emph{Tan Paññāvuddho}: Did you ever have any doubt, like when you had
a girlfriend, maybe? 

\emph{Tan Jayasiri}: It wasn't my nature to think like that. I was
fortunate to come from a happy home. Still, I saw that family life
involved a lot of suffering. I valued the ideals of simplicity and
renunciation, and I was always drawn to the monk's life, dedicated to
contemplation and peace. From the time I turned fifteen, of course I
would be around a lot of girls at school. But for some reason my mind
would always turn to the monks. When it came to having a girlfriend, I
wasn't averse to it, but I always thought that I should practice the
path of the Buddha first. When I turned twenty I asked my mother's
permission to become a monk immediately. I asked her several times, and
finally she said `OK.' She said she would be happy for me to be a
monk, but not for too long, maybe a few years. She felt that a short
while would be enough for me and then I should return to her and the
family. 

\emph{Tan Paññāvuddho}: What does she say now that you have been a monk
for almost ten years? 

\emph{Tan Jayasiri}: She is quite happy now. Not like in the first few
years, though, when she would always wish for me to go back to lay life. 

\emph{Tan Paññāvuddho}: Tan Joti, what gave you the inspiration to go
forth as a monk? 

\emph{Tan Jotimanto}: It also happened when I was young. I had the
chance to see Luang Por Poot frequently with my grandmother. Once at his
monastery a senior monk from Luang Por Mun's monastery pointed to me and
said, `You should become a monk.' This statement stuck in my mind all
through the years. When I was a teenager I went to Wat Pah Pong and had
the chance to see Luang Por Chah. He was still healthy at that time. 
During those days I had heard Ajahn Chah was very strict and fierce, and
you know, when you are young, you are not that interested in the strict
monks. 

Later my cousin introduced me to Wat Pah Nanachat. At that time I was
working as a lawyer in the nearby city, Ubon. That was quite a stressful
period in my life. When I visited Wat Pah Nanachat I was very impressed
with the peaceful quietude of the forest, and the mindfulness and
kindness of the monks. My mind started to calm down. Ajahn Pasanno and
Ajahn Jayasāro were the senior monks at that time. 

\emph{Tan Paññāvuddho}: Was that before you went to live in New York
City to help run a large restaurant? 

\emph{Tan Jotimanto}: Yes, it was. After my first visit, I came and went
to and from Wat Pah Nanachat for two or three days at a time as a layman
during one and a half years. Then I moved to the States. I worked hard
at the restaurant in Manhattan and earned enough money to support the
family back home. I had the financial independence to have almost
anything I wanted, but I was not happy. I did not know what the point
was. One day I went shopping with my friends. They asked me what I
wanted to buy and I stopped. I felt a profound sense of boredom with
worldly things and experiences. I had had enough. It was a very free and
liberating kind of feeling. Reflecting back on my life, I realized that
what I really aspired to was to practice the Dhamma and be a monk. But I
had to prepare for that and it took time. 

\emph{Tan Paññāvuddho}: How did your parents feel when you said you
wanted to become a monk? 

\emph{Tan Jotimanto}: They were happy that I would become a monk for
just one Rains Retreat. It is the tradition for Thai men to be ordained
for one rainy season, and my two other brothers had already done so. But
I did not want to ordain for only one Rains Retreat. I wanted to stay a
monk as long as I felt it was meaningful, and disrobe if I did not like
it. It was quite difficult for my parents at that time. They had to
think about it, and finally they gave me permission. At that time both
of my parents were elderly and not in good health. Even though it was
very difficult for them to allow me to ordain, they wanted what was best
for me and were generous enough to make the sacrifice. This willingness
on their part showed me how much they really loved me. Once they saw me
as a monk, especially as a forest monk, they liked it and were inspired
themselves to practice. Through Dhamma practice they really changed the
way they lived their lives. After I had been a monk for just one year, 
they said, `Don't ever disrobe.' [Laughter]

\emph{Tan Paññāvuddho}: That was quick.

\emph{Tan Jotimanto}: They were very happy. I felt very lucky to have
the chance to repay my debt of gratitude to my parents, who passed away
soon after, by helping to teach them more about the Dhamma. In the last
few years of their life we discussed the Dhamma many times, and I shared
with them some books and tapes of Luang Por Chah such as `Our Real
Home'. I noticed my parents became much more contented and at ease with
life. They deepened their understanding of impermanence and reflected on
the inevitable separation at death. They started to practice every day. 
My father died first, but my mother continued to be strong in spirit due
to her Dhamma practice, despite her poor health. She died four months
later, and at the time that she died she was about to make an offering
to the Sangha. The day she died it was a blessing to know that she was
happy and at peace. 

\emph{Tan Paññāvuddho}: Tan Dhīra, what about you? You also were
working in America before coming to be ordained, weren't you? What made
you change direction to become a monk? 

\emph{Tan Dhīrapañño}: After I finished medical school in Thailand, I
had to decide what to do. First I wanted to be a pediatrician like my
mother, and I wanted to do my internship in the West. In those days, 
however, it was difficult to train in America. Finally I was accepted
for the programme in Michigan. During one of the first teaching rounds
in the hospital, the attending doctor asked the residents where they
came from originally. When the doctor found out that I came from
Thailand, he smiled broadly and asked me, `So you're from Thailand, are
you? What is Nirvana?' I was shocked. I did not expect this kind of
question in the West. I don't remember what I answered, but it made me
think hard. I had gone all the way to America in search of knowledge, 
but I wondered if what I was looking for was in my own country. 

\emph{Tan Paññāvuddho}: So how did you end up at Wat Pah Nanachat? 

\emph{Tan Dhīrapañño}: In Detroit the local Thai community would have a
pot-luck meal at my aunt's house every weekend. They would talk and then
sit in meditation together. I joined the group once in a while. The
benefit of attending was the plentiful and delicious food, and my aunt
always insisted that I take the leftover food, because she knew that I
lived by myself and was not good at cooking. I have to admit that this
was the original inspiration for my meditation practice. One day several
years later I found a post on the Thai student internet that Ajahn
Jayasāro was coming to teach a meditation retreat in Maine. This
`backpack retreat' took place in Acadia National Park, and it turned out
to be a life-changing experience for me. In the morning and the evening
we chanted a \emph{pūjā} together in Pāli and Thai. My grandmother had
taught me to chant every night before I went to bed, and I had been
doing that since I was a child, but I knew very little about what it
meant.

During the retreat I was very moved to learn the deep meaning of
what I had been chanting all along. During the day we hiked on different
trails. When we got tired we sat in meditation. During the questions and
answers session each day, I was very impressed by Ajahn Jayasāro's
wisdom. On the last day of the retreat we had the ceremony of asking for
forgiveness from the Ajahn. It was a beautiful tradition. While we were
walking to the park for this ceremony, Ajahn Jayasāro handed me his
bowl. That was the first time that I had carried a monk's bowl and I
felt very happy. I don't know how to describe it, but it was a special
feeling that I will never forget. Tears filled my eyes. Time seemed to
stand still. At that moment I felt that I too could become a monk and
strive to be like the great \emph{Arahants}. That was the moment I
decided to become a monk. 

\emph{Tan Paññāvuddho}: Your family had put you through medical school
and supported your training as a doctor in America -- was this a shock
to them? 

\emph{Tan Dhīrapañño}: I would say yes. The year that I met Ajahn
Jayasāro and decided to become a monk, I told them of my decision, but
they did not believe it. They thought I was broken-hearted or something
[laughter]. But I persisted. For the next two years, while I was
waiting to finish my training, I tried to keep the Five Precepts on a
regular basis and the Eight Precepts once a week. My friends started to
think that I was weird, but I found I was more and more peaceful. I
would spend more time on meditation retreats, where I felt very much at
home. When I finished my fellowship I came back to Thailand and found my
way to Wat Pah Nanachat. I asked Ajahn Jayasāro to ordain me so I could
continue on the path. Of course, my parents had high expectations for
me. But I think I did not let them down, although at the time they might
have thought differently. For me, becoming a monk is more challenging. 
There are already many doctors in my family, but there is not a single
monk. There are many meaningful and worthwhile things to develop and
cultivate in this holy life. 

\emph{Tan Paññāvuddho}: Have they grown to accept your decision by now, 
or they still hoping that you will come back? 

\emph{Tan Dhīrapañño}: As time passes, they have begun to appreciate
what I am doing as a monk. To be honest, my parents' generous, loving
support throughout my life gave me the emotional strength to feel ready
to go forth as a monk. For me the monastic life is not so much a
rejection of the family life as a way to evolve one step further. If I
weren't a monk in this life, I would feel in my heart that I had not
fulfilled my responsibility to the family in a higher sense. 

\emph{Tan Paññāvuddho}: Tan Jayasiri, what was it that originally
brought you to Wat Pah Nanachat? How did you decide to come here to
train when there also are many other good Thai forest monasteries? 

\emph{Tan Jayasiri}: Just before I ordained I went to Wat Pah Pong and
met Luang Por Sumedho and Ajahn Jayasāro. From that moment I wanted to
be with them. I was ordained in Ayudhya, and after spending the first
five years there I came to Wat Pah Nanachat. 

\emph{Tan Paññāvuddho}: What was it that attracted you to Luang Por
Sumedho and Ajahn Jayasāro? 

\emph{Tan Jayasiri}: It's hard to put into words -- I just wanted to be
with them. Their presence was peaceful and inspiring. I felt that to
practice under their guidance would be beneficial. 

\emph{Tan Paññāvuddho}: Did you feel that they were different from the
Thai Ajahns? 

\emph{Tan Jayasiri}: No, similar. 

\emph{Tan Paññāvuddho}: Having come to practice at Wat Pah Nanachat, 
how do you find practising with Western monks, compared to the Thai
Sangha? 

\emph{Tan Jayasiri}: Actually, in my practice, I never feel that I'm
practising with special people. Everywhere I stay I always just keep
Ajahn Chah's practice. 

\emph{Tan Paññāvuddho}: Which part of Luang Por Chah's teaching do you
find most important in your practice? 

\emph{Tan Jayasiri}: To watch and investigate feelings and the mind. 
Seeing impermanence. 

\emph{Tan Paññāvuddho}: Why do you think that in Thailand, where there
are many great \emph{Kruba Ajahns}, so many of the \emph{farang} monks
have been ordained with Luang Por Chah or in his specific lineage? 

\emph{Tan Jayasiri}: Yes, there are many great \emph{Kruba Ajahns}
in Thailand, but Luang Por Chah was a special monk. He had
his own style of explaining the Dhamma in a simple yet profound way. It
was easy to understand his teachings. His presence made an impact. 
People were happy to see him. Everyone liked to be around him. He set up
a style of training that is helpful and suitable for many different
kinds of people. 

\clearpage

\emph{Tan Paññāvuddho}: Tan Joti, what made you decide to be ordained
at Wat Pah Nanachat as opposed to another Thai forest monastery? 

\emph{Tan Jotimanto}: As I had known about Wat Pah Nanachat for some
time, I knew that I liked the way of life and I felt the monks were very
sincere and diligent in their practice. Still, I went to visit Tan Ajahn
Dtun before ordination. I liked it there too, but the monastery was very
close to my family's house in Chonburi. I wanted to ordain away from my
home, so I came here to Ubon. Also, I felt inspired to ordain with Ajahn
Jayasāro. 

\emph{Tan Paññāvuddho}: How would you compare living with Western monks
to living with Thai monks? 

\emph{Tan Jotimanto}: In many ways things are the same, but there can be
some cultural differences. People at Wat Pah Nanachat come from many
different places. Although the community is usually very harmonious, 
from time to time difficulties or misunderstandings can arise. As Thais
we are taught to keep things inside ourselves. We don't know how to
express ourselves. But when I came here it seemed everybody knew how to
express himself. I have been trying to learn to do this skilfully, but
it is still difficult. I think that if people in a monastic community
express their feelings too much, it can create some problems. For
example, people will not learn to cultivate patience. On the other hand, 
I think that open communication is usually valuable because it increases
understanding between people. For Thai people, if there are problems we
just don't talk about them. The ideal is to find the middle way. 

\emph{Tan Paññāvuddho}: Tan Dhīra, how about you? Was it your
connection with Ajahn Jayasāro that decided you to be ordained here as
opposed to a Thai monastery? 

\emph{Tan Dhīrapañño}: At that time I didn't know much about other
monasteries in Thailand. I did retreats at Ajahn Buddhadāsa's monastery, 
Suan Mokkh, and a couple of other places. But I thought my weak point
was the Vinaya, the discipline. So I tried to find a place where there
was a strong sense of community and strict monastic training. I first
came to Wat Pah Nanachat mainly because of my connection with Ajahn
Jayasāro, and I liked it right away. I particularly loved the feeling of
the forest. The paths were well swept and there was a good environment
for meditation. And perhaps most importantly, I immediately connected
with the Sangha. I felt at home here. 

\emph{Tan Paññāvuddho}: Now I have a question for any of you. Many
people think that in Thailand today there is a crisis with the
\emph{sāsana}. There have been scandals with monks not keeping the
Vinaya, and some Western-educated lay Buddhists are critical of a Sangha
that they feel is stuck in old-fashioned ways. What do you see as the
future of the \emph{sāsana} in Thailand? 

\emph{Tan Jotimanto}: I think people are more interested in the
\emph{sāsana} now. They understand what is going on more. Before people
held the \emph{sāsana} in very high esteem, maybe too high and it became
out of touch. Now lay Buddhists are freer to speak their views and to be
heard. 

\emph{Tan Paññāvuddho}: Do you think this is due to Western influence? 

\emph{Tan Jotimanto}: It could be. 

\emph{Tan Paññāvuddho}: How do you think this new attitude will affect
the monks' practice? Some contend that the more worldly orientation and
values of lay life can be counter-productive when brought into the
sphere of monasticism. Any opinions? 

\emph{Tan Jayasiri}, \emph{Tan Jotimanto}, \emph{Tan Dhīrapañño}:
[Long pause, then laughter.] It is a difficult balance.

\emph{Tan Paññāvuddho}: In the West there is a stress on the value of
equality, but the Buddha noted that the view `I am equal' is still a
form of attachment to self-identity view. Some monks contend that is
important not to sacrifice \emph{samaṇa saññā} (the perception of being
a peaceful renunciant) at the expense of conforming to current worldly
norms. 

\emph{Tan Jotimanto}: One example of this can be the way we use
hierarchy. Thais maintain very high respect for the Ajahn. When Thais
live in a Western community we still put the Ajahn in a high position. 
In effect there is a sense of formality and respectful distance. 
However, I've noticed that Westerners are more informal and relaxed
around the teacher. For them, their relationship with the Ajahn can be
like with a friend. For Thais this would be uncomfortable. I wouldn't
allow myself to play the role of a friend to the Ajahn, even if I wanted
to. Thai monks can be very close to the teacher, but there is always a
formality in the relationship. 

\emph{Tan Paññāvuddho}: That's very interesting. It seems common in
various Buddhist traditions to keep the teacher in a formal position, so
as to maintain proper respect. If one doesn't have that respect it is
very difficult to learn, to have pure communication from the teacher to
the student. Do you think Westerners may miss out on learning something
because of the casualness with which they can relate to the teacher? 

\emph{Tan Jotimanto}: Yes. In Thailand, even if the Ajahn is young or
has about the same number of Rains Retreats, he still is accorded formal
respect as a teacher. It may be natural to feel a level of friendship in
this situation, or not to want to listen to the Ajahn as someone senior
in the hierarchy. But in a Thai wat this would rarely happen, because
the Ajahn is always placed so high. 

\emph{Tan Dhīrapañño}: Take Tan Jayasiri, for example. He will turn
into an Ajahn very soon, and it will be difficult for us to relate to
him. We used to be close to him and tell jokes around him. When he
becomes an Ajahn we will have to change that perception. He will be put
in the special position of a teacher soon. [Laughter]

\emph{Tan Jayasiri}: I think there is a time and place for that. It's
not always the case that the Ajahn plays the role of the formal teacher. 
When a situation in which respect is appropriate arises, we act
accordingly, like a young son with his father. The feeling of a close
relationship makes it easier to talk with each other. 

\emph{Tan Paññāvuddho}: For Western monks it usually requires some
effort to adapt to hierarchical social structures in a skilful way. For
example, in Western culture one tends to relate to people directly, not
in terms of a formal hierarchy. One shows respect through mutual
friendship, openness and trust.

However, in the Buddhist tradition, 
respect and trust of one's teacher and fellow monks can be conveyed by
selflessly and harmoniously acting in accordance with one's formal place
in the relationship. This can be a subtle and refined thing, something
many of us Westerners can learn to do better. It is a challenge to adopt
a new cultural form naturally, without fabricating a new identity that
isn't authentic.

When I live in Thai monasteries, however, it seems
comfortable to fit into the hierarchy and relate to the Ajahn formally. 
Formal separation feels natural. It can also be useful. The Dhamma
teachings flow much better this way. But when I am with many senior
Western monks, it often feels appropriate not to overdo a formality that
is unnatural in our culture. But the form of these relationships changes
depending on the context, whether in Asia or the West, as individuals or
in a group, inside the monastery or out in public. One has to be alert
and sensitive in order to adapt. 

\emph{Tan Dhīrapañño}: I think there are both benefits and drawbacks to
the `equality' form. In Thailand the hierarchy system is so strong that
when I was young I had very little idea about what monks do. The closest
I came to the monks was offering alms-food or listening to a Dhamma
talk. That was it. It never occurred to me that the monk's life was
another lifestyle that I might choose. Today is different from the old
days, when the monastery was also the local school and a community
centre, and good monks were role models for the whole community. 

When I met a Western monk, however, somehow that rigid hierarchy was
weakened. When I spent a week hiking in America, listening and talking
to Ajahn Jayasāro, I felt very close to him. I felt that I could ask any
question I liked and his answers touched me personally. Perhaps my
Western education enabled me to understand the Dhamma better in Western
terms. But I have to say that Ajahn Jayasāro is more Thai than many Thai
people I know. Even when I first met him, during many talks we had, he'd
always correct my use of Thai words. I also think the personality of the
teacher is important. When we like the teacher, we like the subject too. 
I have much gratitude for that chance to have a close relationship to
the Ajahn while in America, something I had not yet experienced in
Thailand. 

But there are also drawbacks to the `equality' form that I can see more
clearly now that I am in robes. In monastic life, identifying with being
`equal' can be detrimental. If everybody is the same, we may not listen
to or show appropriate humility around a more senior monk, especially if
our views tell us that we know more than him. It can seem natural for
the mind to follow its own \emph{kilesas}, but there will be no
spiritual growth then.

Another way to put it is that I think it is best
to recognize equality in terms of Dhamma, while realizing hierarchy is
important when it comes to communal life and the Vinaya. Hierarchy can
be a very skilful means if we use it correctly. Personally, as a monk, I
feel that I want to keep the hierarchical relationship with the Ajahn, 
to keep him high, so I can do what he teaches with respect and not treat
his teaching casually. 

\emph{Tan Paññāvuddho}: Very soon, Tan Jayasiri, you will be going to
Australia. What is it that interests you about practising in the West? 
Why would a Thai monk leave his own country and go to practice Dhamma in
Australia? 

\emph{Tan Jayasiri}: The idea came from Ajahn Ñāṇadhammo. He spoke to me
about this two and a half years ago. He said that the monastery there is
a good place with an excellent teacher. At that time I wanted to
continue to practice at Wat Pah Nanachat. Now the time is appropriate
and I have decided to go. I think it will be valuable for me to be
tested by a new experience. After working as the secretary at Wat Pah
Nanachat, the opportunity for more solitude in Australia will be good. I
can speed up my practice. There I will have more free time. I can set up
my practice schedule the way I like. 

\emph{Tan Paññāvuddho}: In the West many people have been interested in
Dhamma practice for decades, but not many people have become monks. Most
people prefer to keep the lay practice. Conversely, young men in
Thailand may have little exposure to Dhamma practice, but they still
have an interest in going forth. Do you think that in the West in the
future this pattern will change? 

\emph{Tan Jayasiri}: Yes, in the future it will change. But in becoming
a monk, one has to give up so many things. When you become a monk, the
important foundation is the discipline. Someone with many attachments
might feel it is a narrow path and might not feel comfortable. They
might think it is more convenient to practice with just the Five
Precepts. To a Westerner the monastic precepts may seem like a lot at
first. They go against our habit in the lay life of being preoccupied
with making and acquiring things as the way to realize happiness. As
monks we have few requisites. But even though we may be ordained, we can
still have the same perceptions as when we were laymen. When a monk is
ordained on the outside, it doesn't mean he is ordained on the inside. 
We can also feel there are too many precepts, because we still hold the
views and attachments we had in the lay life. So we have to put a lot of
effort into the practice. After some years the results of the practice
will appear. Perceptions and feelings from the lay life will transform
naturally. 

These days, with modern communications and international travel, it is
more possible for people to be exposed to the possibility of practising
the Dhamma through monastic life. In the West you can sometimes see
monks out in society and leading retreats. People can see how a monk
conducts himself and lives his life. With more exposure to this, some
people may become interested in ordination. 

\emph{Tan Paññāvuddho}: Tan Joti, you will be going soon to our branch
monastery in New Zealand. What do you think about the interest of
Westerners in the monastic path? 

\emph{Tan Jotimanto}: In the West now people are interested in
meditation, but without knowing much about the bigger picture of Dhamma
practice. To become a monk is a very great step to take, because one has
to relinquish a lot of attachments. But if Westerners are really
interested, they can do it. Today they can even be ordained in the West, 
but in my opinion there are still advantages to training in the East. 
Because the \emph{sāsana} is so rooted here, it can be very nourishing
to the practice, especially in the early stages of monastic life. But
whether or not to be ordained as a monk is something one needs to know
for oneself. 

\emph{Tan Dhīrapañño}: For myself, I feel it is important to add that
if someone is really seriously about the practice, I would strongly
recommend the monastic life. It makes the foundation in \emph{sīla}
solid and firm. The Buddha laid down the three-fold training of
\emph{sīla, samādhi} and \emph{paññā}, but the \emph{samādhi} that has a
strong foundation in \emph{sīla} is more fruitful and leads more
directly towards wisdom. In the lay life there is often a compromise. 
The monk, on the other hand, can go all the way to the highest goal. The
only limitation is his inner effort, not the situation outside. 

\emph{Tan Paññāvuddho}: For me it is very uplifting to see so many
people across Thailand who have a strong love for Dhamma. That can be a
powerful inspirational and motivational force for a young monk like
myself. Some people in the West have this love of the Dhamma, but most
people don't know anything about the Dhamma at all. How do you feel
about going to a place where many people don't know about Dhamma
practice? They might not even know who Buddhist monks are. [Silence and smiles]

\emph{Tan Paññāvuddho}: Tan Joti, you lived in Manhattan for five
years. Can you imagine going for alms-round on Fifth Avenue? 

\emph{Tan Jotimanto}: No, I can't really imagine that. But if I were in
that situation I would have to accept it if people didn't know what a
Buddhist monk was, or if they looked at me strangely. But then again, 
many strange things happen in New York City. In any case, it is a good
practice to uphold the tradition, and the sight of a simple monk walking
for alms, practising mindfulness, might offer something to the people in
their busy lives. These days, however, people tend to know more about
monks. Still, if I were to go walking in New York, I would be sure to be
prepared to answer any questions that I might receive from the people I
met. 

\emph{Tan Paññāvuddho}: Tan Jayasiri, how would you respond to the
question, `Who are you?' which is sometimes asked of monks in the West, 
perhaps while they are on alms-round? 

\emph{Tan Jayasiri}: If someone asked who we were, I would probably
smile and respond that we are alms-mendicant Buddhist monks. If he asked
me to describe our practice, I would say that we try to live a simple
life. We practice meditation to try to understand the true nature of
things in this world. If we can practice the teachings of the Buddha and
see things clearly as they truly are, we can help to lessen suffering. 
The aim is to have wisdom with the thoughts, emotions and feelings that
arise: `I like this' or `I don't like this', `I want this' or `I don't
want this' -- we can let go of them. If we can keep our minds above the
worldly feelings of happiness and unhappiness, we can begin to find a
true freedom and happiness that do not change. This may be easy to
believe, but it is difficult to do. The practice is not easy. If the
person who asked me had some free time, they could try to practice
mindfulness and meditation to investigate these things on their own. 

\emph{Tan Paññāvuddho}: Tan Joti, how would you respond? 

\emph{Tan Jotimanto}: I would try to explain to him that I am a Buddhist
monk who does not handle or use money, and that it is our tradition each
morning to walk for alms. It might take some time to explain the
Buddha's teachings. What I would say would depend on the person's
background and the situation. If the situation were right, I would say
we practice to let go, not to attain something. We do this through
training to develop virtue, meditation and wisdom. Higher teachings on
non-self and emptiness would be difficult to explain. It would be more
appropriate to start with teachings about \emph{sīla} and \emph{mettā}, 
and when a friendly relationship was established, then we could talk
about something deeper. If he wanted to talk more, I could invite him to
visit the monastery to learn more about Dhamma practice and try it out
for himself. 

\dividerRule

\section{The Interviewees}

This previously unpublished interview is from the original
material for the first edition of `Forest Path'. It was not included in
the original book, probably because of its length. On re-reading, it
seemed a real pity not to include it this time. Tan Paññāvuddho (see page \pageref{pannyavuddho-desc}),
the main editor of the first edition of \emph{Forest Path}, conducted the
interview. The three monks interviewed have all in the meantime been
abroad: Tan Jotimanto went to New Zealand, Tan Jayasiri to Australia, 
and Tan Dhīrapañño stayed with one of his former teachers in America, 
Bhante Guṇaratana. All have now returned to Thailand and are at present
engaged in running their own little monasteries there, Tan Jayasiri in a
remote hermitage called Mettāgiri Forest Monastery in Chayabhumi
Province; Tan Jotimanto and Tan Dhīrapañño co-abboting Wat Pah Boon Lorm
on the Moon River in Ubon Province, a half-hour drive from Wat Pah
Nanachat and closely linked to it.



% Life and Times at Wat Pah Nanachat -- The Teaching of My Life
\setChapterAuthor{Christine Lem}
\setChapterNote{A lay meditator considers the challenges and benefits of practising at Wat Pah Nanachat.}
\chapter[Ten Thousand Joys and Ten Thousand Sorrows]{Ten Thousand Joys and\newline Ten Thousand Sorrows}
\markright{\chapterAuthor}
% Title: Ten Thoughts Joys and Ten Thousand Sorrows
% Author: Christine Lem

When I was asked to write a piece for this Wat Pah Nanachat publication, 
I found it was an opportunity to reflect on and digest the challenges, 
experiences and insights that arose during my extended period of
practising the Dhamma at the monastery. When I arrived at the monastery
I was similar to many women who came with the hope of devoting focused
energy to the practice. Admittedly, many times during my stay I
questioned the benefits of living in a primarily male monastic
environment, but I stayed on because I was open to experiencing the way
of training in the community, and to learning more about the reality of
life as taught by the Buddha. Over time I eventually gained insights
into myself and the nature of life which astonished me. I continually
found it fascinating to learn and understand more about myself through
practising the Dhamma. 

One of the first teachings which amazed me was related to anxiety. I
remember that on four different occasions I asked completely different
questions with different story lines. The same monk began every response
with, `Well, anxiety is like this\ldots{}' I could not understand why he
was focusing on anxiety and not addressing my questions. It took me a
month to figure out that I had anxiety and the story lines were
meaningless. I thought it was not possible that anxiety could exist in
my life. My lifestyle was on the nomadic side, so I had high coping
mechanisms, and this should have meant that I was in tune with
impermanence. But to my surprise, as I began to investigate more deeply, 
I uncovered areas of my life that needed to be reflected on. Spiritual
urgency was ignited. 

The practice situation for women at Wat Pah Nanachat presented the
opportunity to embrace whatever arose in the present moment, whether it
was a pleasant, unpleasant or neutral experience. Living in a monastic
environment gave me the time and space to investigate deeply and examine
what caused my mind to move. Was it due to external conditions or was it
coming from a place inside? What seemed so real and solid in my mind one
minute was not the same the next. During my stay at the monastery I
considered all situations and incidents which moved my mind as valuable
and as opportunities for potential insights. Having to be with things I
didn't always find pleasant required a lot of work. It took energy to go
against the grain. Mindfulness was needed to hold back views and
opinions, and it took a lot of time to be able to see through the
superficial story lines, in order to search internally for a path
leading out of \emph{dukkha}. Idealism, anxiety, views, opinions and
perceptions, as well as joy, insights, understanding, compassion, 
respect, humility and inspiration were all part and parcel of the
teaching of my life. 

While living at Wat Pah Nanachat it was sometimes difficult to assess
the progress in my practice, but after being away from the monastery for
several months I realized that the positive fruits of the practice could
be seen clearly. I have already experienced glimpses of insight that
have had a profound effect on my life. With the encouragement and
support of the Wat Pah Nanachat community, my practice continues to gain
momentum. So where did the experiences and challenges of Wat Pah
Nanachat begin? 

\section{Relating to monks}

The first evident challenge for lay visitors to Wat Pah Nanachat, 
especially women, is the question of how to fit into a monastic training
centre for males. In particular, due to the monastic discipline and
conventions, women might find that they don't fit into Wat Pah Nanachat
as well as they might wish. For me this was an ongoing challenge, though
I felt more at ease as the practice gained strength. The hierarchical
system was a part of monastic training, which helped define how
monastics interacted with each other as well as how they interacted with
laypeople. In addition, the Thai culture has placed monks on a pedestal. 
In some respects it seemed as if they were no longer ordinary beings. An
explanation given by a Thai lay supporter gave me insight into why monks
received special treatment with which Westerners might not be familiar. 
She said, `I'm a householder with many responsibilities and duties, 
that's my \emph{kamma}. It gives me great joy that I can support the
monks, because they can live and dedicate their lives fully to
practising the Dhamma, which I can't do right now.' The generosity of
the Thais affects not only the monastics, but also those laypeople who
visit and live in the monastery. Often Westerners do not fully
understand the inter-dependent relationship between lay Thai Buddhists
and monastics. The laypeople support the monks with simple material
requisites, and the monks support the laypeople by passing on and
exemplifying the Buddha's teachings. 

What strikes Western women at the monastery as surprising are the old
traditional ways of behaviour. In Thai monasteries laypeople sit and eat
after the monks, kneel down or stand lower when speaking with a monk, 
and fulfil the role of cooking and serving the monks. Western women
wonder, `Where does equality fit in at Wat Pah Nanachat? This is the
twentieth century, is it not? These are Western monks who surely must be
aware of modern times.' Western women may suddenly find themselves in a
position where they don't know how to act or relate to Western men in
robes. The question of appropriate protocol, coupled with the need to be
consistent with one's own familiar comfort zone, can produce heightened
levels of anxiety. What is considered appropriate conduct and what is
not? Anxiety increases, especially for those women who are unsure of
themselves or are unfamiliar with the formalities of the monastery. 
Regardless of the fact that they might be new to Thai culture and
monasticism, some women found they needed to know exactly what to do and
how to act in every situation. I also went through this phase. 

It was important to remember that the monks follow a code of discipline
composed of 227 training rules; these were their rules, not mine. This
was a good reminder to relax at Wat Pah Nanachat. I noticed that
Westerners tended to worry about the monks' code of discipline more than
necessary. It was not as though we had adopted the 227 precepts the
moment we entered the monastery. There was a period when I thought Thai
people knew best about the formalities of the monastery. The reality was
that the monks knew best. They were the ones who lived and abided by the
training rules. Sometimes, even when I was familiar with the protocol of
the monastery, anxiety would still arise. I found the mind proliferated
as follows: `There are Thai monks and Western monks. Should I be formal
with Thai monks and informal with Western monks? But then how do I act
with the Westernized Thai monks, or the Malaysian or Japanese monks? 
Maybe I should be formal with all the monks at all times? But if I were
a Western monk, perhaps I would think it was not necessary to be so
formal? Should I continue to talk with the monks I was friends with when
they were laymen?' The thoughts proliferated on and on. After much
anxiety, I learnt to relax a little and search for balance. I found that
each situation and each person was different, and one acted
appropriately according to the circumstances. There were no fixed rules. 
I found that if I was respectful and polite, and was sensitive to the
cultural differences, I couldn't go wrong. 

I went through other phases as well. In Thai society women have to
maintain a certain physical distance from monks, as if women were
dangerous. I would think, `If I were in the West the distance between
males and females would not be a big issue. This monastery is a bit odd. 
What am I doing here? I'm living in a male monastic community!' I
questioned my stay in the monastery. Monastic life and Thai culture are
interesting combinations for Western women. After making a commitment to
the monastery for a period of time, I felt a rebellious rumble in my
mind. I would express exactly what was on my mind. Once I admonished a
bossy novice who rudely interrupted my conversation so that I could
complete his duties. I used strong words and told him to tidy up his own
mess and take responsibility for his actions. I thought I was speaking
the truth, but harmful speech is not considered `right speech' either. 
It was a good teaching for me, as I was beginning to understand that
true and honest thoughts could be dangerous mind states. This was one of
many insights which meant my stay at the monastery was challenging and
fruitful. Although the monks tried to explain the situation at Wat Pah
Nanachat, and how and why the rules and conventions exist as they do, 
Western women said it was difficult to listen to or even hear the words
of these privileged monastics in robes, explaining the way things were
in the monastery. Monks were explaining the reasons for their own
privileges. Women found that it was easier to hear things when a Western
female with Asian roots, living in the monastery for an extended period
of time, explained matters there. 

I once thought that monks at Wat Pah Nanachat were constantly mindful
and that they probably practised twenty-four hours each day. What else
would monks do? I sometimes forgot that I only saw the monks four times
a day, at morning meditation and chanting, the daily meal, teatime, and
the evening meditation and chanting. Well, of course they were mindful
all the time then! Monks were supposed to behave this way during formal
meetings, and especially when sitting on the \emph{āsana} before the
meal with so many people looking in their direction. The brief moments
when I saw the monks gave an impression of heightened mindfulness and
awareness. 

Wat Pah Nanachat can be an intense place for Western women when small
and trivial matters are magnified. There were times when there was not
enough tea, sweets, cushions or chanting books for the people sitting at
the lower end of the hierarchy. Did this happen every time? Was it due
to an inexperienced person setting up, or perhaps due to someone working
with greed that day? Depending on my mind state I was sensitive to these
things and sometimes took them personally, as if someone was directly
trying to create \emph{dukkha} for me. There were many different ways
the mind could proliferate in response to a given situation. 

I also experienced monks who were sensitive to the position of women in
the monastery. I noticed gestures of kindness and concern. There was a
novice who made a separate kettle of cocoa for women so that they
wouldn't have to wait so long, and there were other novices who would
personally pass on the spare sitting cushions to women. At other times
monks would ask if there was enough food or tea for the women, or a monk
would make a sincere apology for his lack of mindfulness towards women. 
All this took effort, energy, humility and awareness of the sensitivity
of women living in the monastery, which was much appreciated. Honest
mistakes and miscalculations do happen. 

It takes at least a few visits and sometimes more to understand Wat Pah
Nanachat. To understand the monastic training set-up more deeply, I
needed to take the time and energy to keep asking questions. At the same
time, I had to be open to receive answers. I found it was important to
dispel doubts by asking questions, in order to benefit from the new
experiences gained from living in a Thai monastery. My doubts or
misunderstandings arose because there were many ways of perceiving a
given situation. I encountered one monk who did not use eye-contact
during a Dhamma discussion with the lay residents. My judgemental mind
said, `He is a young monk and he must lack confidence, be insecure and
have women issues to deal with in his practice.' It was part of my
conditioning that not to make eye contact is odd or impolite, but in a
monastery little or no eye-contact was appropriate for a `good' monk. It
was obvious he was a `good' monk because his Dhamma responses were sharp
and clear, with compassion and wisdom. 

Timing and not taking everything too seriously are also important. I
remember one occasion when someone asked me, `How come all the monks are
so young?' I said, `It's all that meditation they do.' I'm not quite
sure what role humour plays in monasticism, but it certainly helps to
lower unnecessary barriers between people and allow space for new
growth. 

\section{Language and cultural challenges}

The position of women in a male monastic community can blind Western
women to seeing the benefits of the opportunity to practise at Wat Pah
Nanachat. Inability to communicate in the native language adds to the
challenge of practice there. Female Western guests at Wat Pah Nanachat
can find their anxiety level increase due to the lack of
English-speaking Thai women. Thoughts about what to do, where to go and
how to do things might be overwhelming when you cannot understand the
answers, but the Thais will come to your assistance if you look lost. It
is also important to smile. If I ever felt at a loss for words and did
not know what to do, I just gave a great big smile. It is the Thai way. 
Smiling is considered a higher realm of communication. 

Not knowing the Thai language can also be a blessing. I found it was
interesting to observe the movement around the community and the people
in it. Not knowing the right words, and speaking little, I had an
opportunity to experience the monastery on a different level. Many times
the challenges and aversions that came with community living were
limited to thoughts. At those times I saved myself from creating any
unwholesome \emph{kamma} which I would have regretted later -- what a
relief! It was also an advantage not to understand or become acquainted
with kitchen gossip. Reduced mental proliferation was always conducive
to my meditation practice. 

In addition to the language barrier, unfamiliarity with Thai customs
added to my anxiety. In general I found it was a good idea to follow in
the footsteps of the Thais, literally and figuratively, especially when
there were ceremonies at the monastery. The best policy was to go where
they went, sit where they sat and do as they did. For me Thais were good
means of gauging what was deemed appropriate in a monastic community, 
until I developed the confidence to know what was best based on my own
experience. I remember one incident when I placed bananas randomly in a
bowl that was going to be offered to the monks. They underwent a
transformation when I briefly left the kitchen. They were wiped and
cleaned with a wet cloth, both ends were snipped for a tidy appearance
and then they were neatly arranged in a bowl. It was hard for me to
believe, but the Thais staying at the monastery had even stronger
perfectionist tendencies than I did. They were meticulous in everything
they presented to the monks. I made a point of observing this in order
to smooth out any future incidents. Most of the time, if I made a
mistake, it was not actually a major error on my part. Interestingly, I
learnt a lot when I made mistakes, and the memory was recorded in my
mind to ensure the same mistake did not happen again. This put me at
ease on some levels. I felt as if someone was showing me insights into
Thai culture, and protecting me from making any major blunders that
might cause embarrassment to myself and others in the community. 

I found the challenge in the kitchen was that I needed to keep paying
attention to what was useful, and to take the initiative of performing
tasks that were appreciated. The Thais expressed delight and encouraged
me when I cooked Western food. The villagers inquired about what I was
cooking and what kinds of ingredients I used. I took all this interest
as a sign of Thai-style praise, which may have been a bit of projection
on my part. It didn't seem to matter if the Western food looked, tasted
or smelled pleasing to their eye, tongue or nose contact. The important
thing was that it was Western food, and the pot came back fairly empty
so the monks must have liked it. 

One area that the Thais did not encourage me to emulate occurred when
preparing fruits. They were skilled at carving and arranging fruit into
beautiful shapes and designs. Thais cut with the hands in mid-air, 
rarely using a chopping board, so I did the same. Also, the Thai-style
of using a knife was to cut away from the body, rather than towards the
body as in Western style. I did try a few times to cut and design fruit
in a manner pleasing to the eye, using the Thai method with a knife. The
results were disastrous. I could not manage to carve a piece of fruit
with a machete-like knife without the fruit turning into juice in my
hands. I soon noticed that Thai people offered very subtle hints with
regard to where my talents were most useful and beneficial in the
community. I learnt to be alert to how I could be useful, especially in
the kitchen where communication was in the form of body language. When
preparing the meal, the Thais would rarely suggest what to do to Western
women. 

Overall, insights and challenges from my daily experiences and
interactions with Thai culture brought an enormous amount of joy. The
Thais' kindness, generosity, respect and humble nature were qualities I
continuously tried to rediscover in myself and aspire towards. These
qualities, which the Thai community exemplified in their day-to-day
lives, had a tremendous impact on my attitude towards life. The relaxed
and easygoing style of Thai people influenced nearby Westerners. I
examined my resistance to these wholesome qualities which the Thai
community continuously expressed to others. I learnt a lot by simply
noticing, listening, observing reactions, going against the grain of my
cravings, using my favourite word, `whatever', talking little and
smiling a lot. I believe living for any period of time in a Thai
monastic community might enable people to make great progress in their
practice. I was able to let go of fears and anxieties related to my own
cultural conditioning, and open up to a new set of heartfelt wholesome
qualities. What joy and relief it was to let down my barriers of
resistance! 

Another cultural difference that deeply affected me occurred during Thai
Buddhist funerals, where the dead bodies were brought to the monastery
to be cremated in the open air. I appreciated the increased contact with
death, which reduced any fears I had about it. Watching a body being
cremated was a powerful reminder that my body was of the same nature. It
was interesting to witness that these funerals were ordinary, peaceful
and simple, quite a contrast to my experience of Chinese funerals. The
bodies were decaying naturally and not decorated to look beautiful or
full of life. I felt for the first time that funerals were authentic. I
noticed that less fuss and distraction regarding funeral arrangements
made the death more real for those concerned. Anyone who wished was able
to watch the body burn and melt from skin to bones to ashes. The heat
element from the fire was raging and powerful. It had a life force of
its own which felt natural and freeing. It felt as if life and death
were coming together in harmony and balance. Burning a body in the open
air in the centre of a forest monastery seemed quite appropriately in
tune with nature. I was very moved by these funerals. When I die I might
consider Wat Pah Nanachat as an option for my cremation. 

Through interacting with the Thais, I realized that being born in the
West with Chinese roots, I perceived the world in a way that appeared to
be a contrast to the reality of experience. It was an incredible
experience to discover that I knew so little about the true meaning of
life, which I felt was being revealed to me in the monastery. 

\section{Practice in communal life}

I've spent a great deal of time in numerous communities, usually passing
through them. At Wat Pah Nanachat I lived for the first time in a
community for an extended period of time. This made a considerable
difference to my practice. Making a commitment to a community meant
there was no escape from confronting my habits of body, speech and mind. 
I learnt to live with the \emph{kamma} created in the present and to
resolve the \emph{kamma} created from the past. Living in a community
was intense and sensitive buttons were pushed. When practising with
others I found that I was no longer dealing with just my own five
\emph{khandhas}: I had taken on board the entire community's
\emph{khandhas}. If there were twenty people in the community, there
were one hundred \emph{khandhas} to interact and deal with in a
compassionate and skilful way. 

I was also aware that in most communities there always seem to be one or
two odd characters who test all those living in the monastery. They came
in a variety of forms: monks and nuns, Western and Thai, men and women, 
young and old. I noted how these types of characters affected the entire
community, even though I first thought I was the only person in the
monastery who was experiencing major \emph{dukkha} with that particular
person. One Western layman who visited the monastery did not want to
speak to or have any contact with any of the women. At the same time, he
continued to frequent areas where women gathered, like the kitchen. 
These characters were challenging and could lead me to react
unskilfully. I noticed that if I was in a good mood they did not move my
mind; instead, compassion arose. My thoughts about this situation were: 
`I wonder what it would feel like to be living in this person's shoes. 
Sounds like a lot of anger. I wonder where it comes from. His
\emph{dukkha} can't come from me. We don't know each other. We have
never met before. I can't take this personally.' Confusion and
\emph{dukkha} arose as I reacted to his attitude towards women on the
external level. I noticed my thoughts, and what immediately came to my
mind was that he was not supposed to be acting that way or saying those
absurd things: `What is he doing here? This is a monastery!' But in
reality it was not his comments that were causing the \emph{dukkha}. It
was my own internal struggle with idealism that created true
\emph{dukkha}. The scenario was all happening so fast that I could
barely keep up with being mindful. I reacted to external scenarios and
blamed the external world for my \emph{dukkha}, but the real
\emph{dukkha} was my own internal issues. Whenever mindfulness returned, 
the amount of confusion I could create for myself amazed me. Idealism
was an ongoing struggle because it seemed so real and solid. I was
absolutely positive beyond a shadow of a doubt that the \emph{dukkha}
was coming from the people or places who caused me to react with
negative mind states. As I continued to experience scenario after
scenario, I hoped to learn from each and every incident rather than
react to the endless incidents that were all integral parts of life. 

As well as the odd men who came through Wat Pah Nanachat, there were
also women who were challenging. One woman constantly complained about
living at Wat Pah Nanachat, but she continued to stay in the monastery
for quite some time. We had a discussion, and slowly her personal
history, perceptions and feelings began to unfold. I was startled to
hear the amount of suffering she experienced. It was a strong and solid
reality for her. It was difficult to relate or connect with this
\emph{dukkha} because it didn't affect me in the same way. After our
meeting there was a shift in my attitude: I found space and compassion
to allow this person to be who she was. To begin with I reacted
negatively to her continuous stream of complaints. I felt they were
unrealistic or exaggerated. Idealism affected my mind state. I would
think, `If she doesn't like the monastery, why continue living here?' I
lacked the patience to accept what was happening in the present moment. 

I noted how idealism took up a vast amount of my time and energy, so as
to cover up the present reality and create the reality I wanted. I
realized it would be wiser to bring up the effort to work with the
issues occurring in the present moment. My ideal was that I loved
everybody because that was the practice, but in reality I avoided those
I didn't love, which made it easier to love everybody -- what a delusion! 
It would be better to learn to be with people with whom I did not wish
to be. Community living was like that. I was continuously learning and
gaining insight about others and myself in the community. It was an
opportunity to take on the challenge and work with what moved the mind. 
I didn't get overwhelmed with my \emph{dukkha} or try to leave the
monastery by the back door. If I was not going to practise with the
difficulties that came from living in the monastery, what was I waiting
for? I had a tendency to wait for the right moment to really put in the
effort to practise. I'd say things like, `When I find the right teacher
I'll practise; when I find the right community I'll practise; when I'm
in a happy mood I'll practise; when I'm not so tired I'll practise; when
I'm in solitude I'll practise.' The list goes on. The right moment is
here and now, because I might die before the next moment happens. Death
does not wait for the right moment. It's known for bad timing. Imagine
dying waiting for `the right moment to practise'. To talk about the
practice more than practising the practice sounds like hungry
ghost-realm material. It was always good to remind myself that it was
useful to take every opportunity to practise, and not waste one precious
moment. The practice continues. 

Wat Pah Nanachat was an inspiration because people were practising, 
living and working with reality, life as it was happening. I tried to
spend my time being with whatever was in the present moment, and going
against those old habits that continuously pushed me around. I examined
exactly what was pushing my buttons. Did it ever really come from
external conditions? I noticed a variety of incidents that caused my
mind to move in different directions. Sometimes I let it go; other times
I reacted because `I was right and they were wrong'; and there were
times when fire came flying out of my mouth with zero mindfulness. It
was all about watching the mind, fully reaping the consequences and
learning from these incidents. If I was not making the effort to
practise and learn in a conducive and supportive environment like the
monastery, where else was I going to do it? Would I ever find the
perfect place so that I could have perfect meditations? 

\section{Investigating idealism}

After I left Wat Pah Nanachat, a major insight revealed itself to me. 
These reflective periods outside the monastery offered an opportunity to
step back and create some space for the challenging situations I had
experienced when living in a community. The key issues that entered my
mind were idealism, perfectionism and anxiety, which extended into
honesty, trust and refuge. I noticed that I had difficulty in being
honest with myself due to idealism, particularly in relation to
feelings, opinions, thoughts and decision-making. I had practically
taken refuge in idealism; I had put my trust and faith in it because it
wouldn't fail me. That perfectionist idealistic tendency invaded my
mind. I continually denied the reality of where I was at, because I was
not the ideal. I was not as honest as I could be in my practice because
of fear that I was not living up to my idealism. This was a lot of
\emph{dukkha}. 

I remember I was surprised and impressed by talks given by a visiting
Ajahn. I noticed he was exceptionally honest about his past and present
experiences. He openly admitted and discussed his challenges, 
difficulties and complaints. I had thought Ajahns would be above and
beyond these mundane issues. He used personal experiences because he was
not afraid to express his situation freely. He was not embarrassed
because he didn't have anything to hide, while I was caught up in my own
idealism and did not allow enough space for the practice to unfold
naturally. The influence of a highly respected senior monk who spoke in
an alternative style helped me let go of my ideals about the practice. I
suddenly felt I had a lighter load to carry. It was easy to forget that
the Ajahns were just ordinary sentient beings like everyone else. It was
difficult to think they still had \emph{dukkha} like anyone else. It was
their approach, and how they responded to \emph{dukkha} in a
compassionate and wise way, which made the difference. The skilfulness, 
clarity and honesty of their body, speech and mind were beautiful to
experience and inspiring to watch. The presence of senior monks was a
powerful teaching. 

I found Wat Pah Nanachat was conducive to deepening my practice. I felt
the foundation of my practice shake and move, because it immediately
reflected back to me the question of how honest I was with where I was
in it. It was not as though I was going around lying to people about my
practice -- it was that my living, breathing reality consisted of
thoughts related to incidents, such as irritation which arose due to
loud and disrespectful lay visitors to Wat Pah Nanachat. I thought, `I
should not be sinking into these low-level mind states! This is not how
one trains the mind.' I would reflect on how wonderful it was that
people had discovered the Dhamma, and how great it was that they had a
chance to visit the monastery. That was the ideal -- it was a pleasant
thought, but the \emph{dukkha} still wouldn't go away that easily

Sometimes I couldn't clearly see the difference between reality and
idealism. In community life the ideal was that people should be aware of
what tasks needed to be done and then do them, but the reality was that
some people didn't make the effort or didn't care. Then I usually went
through a mind-spin and wondered what kind of practices they were doing. 
How could people use and abuse such a wonderful place? Didn't they hear
the evening talk? It was so powerful. When the reality doesn't match
idealism, \emph{dukkha} escalates. How many times did this happen? I
lost count. A monk once said something to me that had a powerful effect. 
He said, `It's good to have you here, Christine, it's good to have
someone we can trust.' He used the word `trust' often with me, and I
thought to myself, `Well, of course you can trust me. I've got too much
idealism, I would never ever think of doing anything intentionally
harmful to the monastery.' With hindsight, however, I realized that he
was trying to say, `Christine, we trust you, therefore you should trust
yourself.' For me this meant I needed to give myself some credit, and
probably believe and trust that I was okay. I was fine most of the time, 
but I got thrown off course because sometimes my feelings, memories, 
opinions, views, and criticisms all seemed accurate, justifiable and
absolute truth. At other times I was very clear they were \emph{dukkha}, 
\emph{aniccā}, \emph{anattā} and not worth holding on to. I needed to
work on letting go of those absurd fixed ideas. Self-criticism arose, 
and I wondered how I could possibly trust myself when I made so many
deluded mistakes. It was interesting to watch and notice the anxiety, 
idealism and perfectionist tendencies when I could catch them. It was
better than spinning in them -- such a waste of my time and energy. 

In the monastery I was a long-term female lay resident who lived and
worked near the kitchen, spoke English and a little Thai, and had a
shaved head, which led people to believe I was the local information
source. Sometimes I felt I was engaged in idle conversations more than
necessary, which meant less time to do the 101 things on my wish list. 
This increased my craving for solitude. I thought that if I had enough
solitude my whole practice would come together. I was forever
complaining that there was not enough time in the day to do all that I
wished to do. If only I had more time, I would be happy. I would
renounce tea or fast in order to increase the time leading to happiness. 

One of the subjects for frequent recollection is: `Do I delight in
solitude or not? This should be reflected upon again and again by one
who has gone forth.' For me the answer is, `YES. Which mountain, cave or
jungle do I get to meditate in so as to delight in solitude?'
Unfortunately, it was `not recommended' for women to stay in some branch
monasteries unless there were two women together. It was also `not
recommended' for women to go to certain areas of a monastery. The `not
recommended' advice was frustrating for me because I believed that deep
and heartfelt insights came from intensive solitary practice. Women
practitioners who wished to meditate in more isolated areas were bound
to come into contact with the fears, worries and anxieties expressed by
Thai people. Thais asked, `What if ghosts should appear?' What was
considered appropriate, safe and respectable for women were factors that
made it challenging for women who wished to delight in solitude. In
hindsight there was enough solitude, but I let \emph{dukkha} get the
better of me. Living in my \emph{kuṭī} in the forest, sweeping and
meditating were periods of solitude which were overlooked because they
didn't match my ideal type of solitude. It was pure \emph{dukkha} to
have high standards and not be able to maintain or keep up with them. I
experienced the `ideal' type of solitude a few times, but using that
memory to compare the daily life experience of solitude only created
unease in my mind. I didn't appreciate or make good use of ordinary, 
everyday solitude, which probably would have measured up to the dream
type of solitude I frequently envisioned in my mind. 

I noticed that some days I would love Wat Pah Nanachat. It was the best
monastery, with the best monks. It was the most authentic and pure
practice place in all Thailand. Everyone was wholeheartedly into the
practice. We had the most faithful and dedicated villagers who came
every day to support the monastery. Then there were other days when I
said, `I'm leaving tomorrow and I'm never coming back. It's all wrong
here, they should be updating and adjusting their rules of discipline to
keep up with modern times, I'm not going along with these outdated, 
old-fashioned ways of interacting with one another.' Which was real and
which was not? I couldn't believe in these passing thoughts or moods. 
They were too much \emph{dukkha}. I found it both time-consuming and
tiring to resist present conditions at the monastery. I spent too much
energy not accepting the way things were. It was less exhausting to let
go and conform to the way things were, because Wat Pah Nanachat is
actually a fine place to practise. In the monastery I meditated, ate a
little, slept a little and lived simply in the forest. After a while I
noticed, `What else do I really need in life?' I heard myself saying
things like, `I love Wat Pah Nanachat 95\% of the time in order to keep
room to express my frustration and complaints.' Was the dissatisfaction
reality or not? On different days different mind states appeared, 
directed at different situations and different people. The mind's moods
kept changing. What could be so important as to take away my peace of
mind? 

\section{Closing}

I continued to make frequent visits to Wat Pah Nanachat, because I loved
the simplicity, authenticity and purity of the living Dhamma that
existed in the monastery. I felt my life opening powerfully when I came
into contact with the straightforward teachings of the three
characteristics of existence, \emph{aniccā, dukkha} and \emph{anattā}. 
Wat Pah Nanachat provided a good setting in which to continue to
investigate these truths at ever deepening levels. I kept going back
because realistically there was no going back to my old ways of
thinking. In retrospect I can see how Wat Pah Nanachat shook the
foundations under my feet. The experience of living for an extended
period of time in a community of full-time Dhamma practitioners, people
with whom I had not individually chosen to live, became the teaching of
my life. Although Wat Pah Nanachat always managed to provide an array of
challenges, I learnt that the monastery had no inherent \emph{dukkha}, 
just as it had no inherent \emph{sukha} (happiness). I needed to take
responsibility for my own experience of life. 

Women who visit Wat Pah Nanachat will perhaps feel it is similar to a
boys' club. Some women might wish to be part of the boys' club. Who
likes the feeling of being excluded or left out? This is the challenge
for all women who visit Wat Pah Nanachat. But whether one is male or
female, the practice is to keep watching the movement of the mind and
protect it from falling into states of greed, hatred and delusion. Some
women find Wat Pah Nanachat a useful place to practise for a few days, 
others for weeks and months, and some are not yet ready for it. This is
all just fine. It is good to acknowledge where we are at in the present
moment. 

My experience of living in a monastery was intense because it meant I
had to be with things I did not wish to be with, I was separated from
the things I wished to be with and I didn't get what I wished for. There
was no escape except to face the \emph{dukkha}. \emph{Dukkha} was
crystal-clear, and so was the path leading out of \emph{dukkha}. These
were the sorrows and joys of living at Wat Pah Nanachat. It was an
opportunity to deepen my understanding and reflect on my defilements and
habits, in order to learn and know more about myself. I saw myself
investing energy in confronting situations and striving for ideals which
it was beyond my ability to do anything about. Meanwhile, what it was in
my power to change did not seem as appealing to work with. While living
in the monastery I experienced what `real life' was all about. I heard
lay visitors express comments such as, `The monastery isn't real life.'
I felt that in the monastery I began to learn about my multi-dimensional
mind states. My attitude began to change and insights came through
experiencing community life, understanding Thai culture, relating to
monks and observing my idealism. The simplicity of living in a forest
and the ascetic practices of monastic life resonated well. I felt in
tune with this lifestyle because I was drawn to simplicity and to what
was `real', elements which were lacking in my life. For me `real life'
meant reflecting and working with the \emph{dukkha} that appeared in the
present moment. This meant that there was a lot of `real life' in the
monastery! 

I would like to express my sincere appreciation and deep gratitude to
Ajahn Jayasāro, Ajahn Vipassi and the entire Wat Pah Nanachat community, 
and all those wonderful beings who have supported, lived at, and visited
Wat Pah Nanachat. Thank you for your kindness, generosity and the
opportunity to practise at Wat Pah Nanachat. Thank you for the teaching
in my life. 

I hope that what I have written is helpful for all beings who find
themselves at Wat Pah Nanachat, especially women, who might find the
experiences of monasticism, Thai culture and Buddhism challenging. 

\dividerRule

\section{The Author}

Born in Canada of Chinese origin, Christine initially spent six
years in Thailand and India practising the Dhamma. Her introduction to
the Dhamma began with Tibetan Buddhism in India. Eventually her travels
led her to Southern Thailand and a retreat at Wat Suan Mokkh in 1993, 
where she first encountered the teachings of Theravāda Buddhism. Soon
after that retreat she visited Wat Pah Nanachat for the first time in
August 1993. The monastery was part of the Dhamma trail and on her
`list'. During the period when this article was written she was staying
at Abhayagiri Forest Monastery in America for a few months, and
eventually travelled to England to spend a year as an \emph{anāgārika} at
Chithurst Monastery. After leaving the monastery, having decided not to
pursue the monastic life there, she returned to Vancouver and alternated
work with travelling to Asia and spending time practising in forest
monasteries and meditation centres there. Between 2006 and 2008 she
lived in Burma, mainly at Pa-Auk Forest Monastery and Shwe Oo Min. Over
the years she has often called in at Wat Pah Nanachat for a few days or
weeks at a time.



\setChapterAuthor{Sāmaṇera Gunavuddho}
\setChapterNote{Exploring identity and the similarities between his training in music and monastic life.}
\chapter{Hey, man, don't give up your music!}
\markright{\chapterAuthor}

\setChapterAuthor{Sāmaṇera Gunavuddho}
\setChapterNote{Exploring identity and the similarities between his training in music and monastic life.}

\chapter{Hey, man, don't give up your music!}
\markright{\chapterAuthor}

Barely a month in the robes , I am a newly ordained \emph{sāmanera} who
is still trying to understand exactly what has happened to his life.
Sometimes I wake up from sleep in a moment of disorientation and ask,
`Where am I and why am I dressed like this?' I thought my goal in life
was to be a jazz recording artist, but somehow I have made the
transition into the Theravāda  monastic lifestyle. I have a desire to
understand the transition better; and I admit that it is only now,
through writing this piece, that I am able to start investigating the
deeper reasons why I am willing not to play music again. I wish to make
the piece an exploration of my experiences in music practice and
monastic practice, in the hope of understanding better what has
happened.

When I look deep in my heart and ask why I practise Dhamma, I see that
the answer comes forth with great energy. I practise to learn the truth
of how nature works and to do what is good. After the heart has spoken,
I feel a heating-up of the body with an increased flow of blood; my back
straightens up nobly, my mind becomes quiet and my gaze softens. I am
also told by my heart that the same goal was the powerful current that
carried me through all those years of music practice. When this is
revealed I see that I have not given up what I find truly important, and
that there was a natural flow to the recent transition I've made into
the monastic life. Like a raft, music practice was able to take me
part-way across the river, but I have now switched to the raft of
Buddhist monasticism, which I believe has the ability to take me right
across to the shore of liberation. When looking at the similarities
between my music practice and this monastic practice, I am able to feel
a deep sense of gratitude for my past musical experience, while
investigating the differences which mean the monastic practice goes
further towards my goal. I have experienced the role of devotion,
sacrifice, the teacher, solitude, awareness, creativity, effort and
challenge in both my past music practice, and the monastic practice of
the Thai forest tradition.

Devotion

My music practice started with devotion. I was born into a family
devoted to music, where all the conditions for setting me on the path of
music practice were ripe. My family would go to church occasionally, but
music was the religion I practised at home, and I could trust it and use
it to relate to the world. My grandfather had a music store and taught
my father music. My father taught me, and I eventually taught others
after many years of practice. Playing a musical instrument almost seemed
to be a prerequisite for being in the family. A visit to my
grandparents' house always included an offering of a musical performance
by someone. I remember my father would play music on the fishing boat as
we sat meditating on our floating bobbers, in the house while we cleaned
it, and even in the bathroom as he showered. Our house always had many
different instruments to experiment with: drums, saxophones, keyboards,
trumpets and others. Teaching music and selling instruments were what
put food on the table and enabled us to continue our music practice. My
father and I would talk for hours about music; this was one of our main
ways of bonding. I would also carry his instruments to his public
performances in order to watch him play, which he did in a way that was
inspiring and magical for myself and others. He taught the values of
music to me through example from the very beginning of my life, and
after seeing my deep desire to follow in the path of music, he started
giving me formal piano lessons when I was five. I grew up with small
ceramic clowns holding instruments and Christmas ornaments that played
music. Our house was also filled with paraphernalia like mugs, posters,
belts, scarves, T-shirts, neckties and stationery which were all
decorated with musical notes.

A bit obsessive, you may think, but believe it or not, it all made
perfect sense at the time. Upon reflection I think that always seeing
those musical trinkets in my environment strengthened my identity as a
musician. Likewise, someone can strengthen the identity of being a
Buddhist by merely owning a statue or image of the Buddha, but I think
it is more beneficial to use the Buddha statue as a devotional tool to
open the heart and the mind. When I bow to the statue of the Buddha I
have the opportunity to recollect the Buddha's virtuous qualities and
his teaching, and to bring my mind back to the practice. I feel that I
am fortunate in having such an object of devotion as the Buddha, and
there is a feeling that every ounce of effort I give is returned one
hundredfold. To me the goodness that can result from the practice of
Buddhism seems limitless, so it is worthy of limitless devotion.

Sacrifice

When one has deep devotion, one is willing to sacrifice just about
anything. Music always came first in my life, and I would sacrifice many
things without question. Sleep would be sacrificed by waking up early in
the morning or staying up into the late hours of the night, practising
the piano or listening to jazz performances in clubs. After-school
activities, sports and dances didn't seem nearly as important as going
home to practise music. My parents spent a lot of money on music
lessons, instruments and recordings to support the cultivation of my
abilities, and put in many hours behind the wheel of the car to drive me
to lessons and performances. At home my family sacrificed their outer
silence to let me practise, and my sister can testify to the hours and
hours of piano-playing on the other side of her bedroom wall.

Now that I am ordained I have given up my worldly possessions. Bit by
bit, as my instruments, recordings and other tools of the trade made
their exit from my life by being sold or given away, my identity as a
musician started to fade. Having left my position as a teacher and
bandleader in the field where I had experience, I have now traded that
status to become a beginner at the bottom of the line. After being
ordained I found simple actions such as getting dressed hard to do. I
thought I was shown how to dress myself when I was a little boy, but the
familiar pants are gone and the robes are hanging in their place. So
many actions I thought I already knew, such as walking, eating and
sleeping, are now challenged by the monastic training. Even the name
I've used since birth has changed. I once believed there were
possessions and experiences that were necessary for survival, but as I
experiment with living very simply I experience some joy from learning
that my safety doesn't hinge on having those things after all. I would
say that the main thing I thought I knew was that I was a musician, but
where has that gone?

Realizing that I have personally felt a fair bit of thrashing around in
my life in an attempt to `be somebody', I hope to resist the temptation
merely to trade my identity as a musician for that of a monk. Trading
for the identity of an ex-musician is probably another trap. At some
level I understand that attaching to the identity of being a monk is
entirely different from actually being present to the way things are,
through the monastic practice as taught by the Buddha. Sacrificing the
need to attach to an identity is extremely hard to do, but I believe
that the invaluable guidance provided by an experienced teacher on the
spiritual path can make it possible.

Role of the teacher

Sometimes the teacher will select a practice which is not one we would
select ourselves, but out of faith and trust we follow the guidance we
have been given. Great experience gives the teacher the wisdom to see
where the student is weak in the practice, and the ability to protect
the student from training in a harmful way. In both music practice and
monastic practice, the teacher is one to whom we can sacrifice our ego.
I would admit what I did not know to my music teachers, but they could
already hear it in my playing. There was nowhere to hide and it felt
great to be seen.

I think that this attitude of observing and exposing weakness in a
relationship of trust is a central part of the Holy Life. I find that in
the monastic practice, due to the safety of the Dhamma teachings and the
morality practised in the environment, there is much more of an
opportunity to humble oneself and give oneself fully to the teacher. We
can bow with reverence to the teacher and allow this relationship to
develop as a spiritual tool in order to dissolve egotism. There are
opportunities to tend to the teacher by carrying his bag, washing his
feet, washing his clothes, bringing him something to drink, cleaning his
lodging and cleaning his bowl. Even though this man can do these things
for himself and has done for many years, the opportunity to think of
someone besides oneself, even for a moment, can be very powerful.

With my music teachers the relationship was usually only musically and
financially based. I always felt there was a cut-off after I paid for
the lesson and was out the door. If I did not pay for the lessons, the
relationship would end. That's the way it works in the business. Some
teachers would take more time to talk after the lesson, go to the store
and help me buy recordings or invite me to come to their performances,
but that was as deep as the relationship would go in the context of my
life as a whole. The only opportunities to give of myself were to pay
for the lesson, practise what I was assigned, provide supportive energy
in the audience at a performance and maybe give a gift at Christmas. It
would be quite strange to go early to a piano lesson to tidy up the
teacher's studio and bring him a drink on my knees. There just wasn't
room for that.

Solitude

Both musicians and monks spend a great deal of time in solitude in order
to deepen their practice. I remember the small piano rooms at college
dedicated to solitary practice. Each was insulated with white-walled
foam for soundproofing, in order to create an environment with minimal
distraction. Restraining the non-listening senses was an aid to
concentration. It was in those rooms that most of the daily sweat was
released. I spent many timeless hours in these practice rooms, and
sometimes I would emerge surprised to see that it was already dark
outside, or that a snowfall had blanketed the city without my knowing
it. So after years of such experiences, I don't mind practising in the
\emph{kuti}. Having space to oneself in the forest functions as
insulation from many worldly dhammas. I've searched for solitude in both
practices, and I am just starting to understand that true solitude is a
state of mind. Whether from amplified long-haired guitarists, mango
pickers, aeroplanes or loud insects, there is always outside noise. The
silence must come from inside. Just as the silence or `rests' in music
give the listener space to appreciate all the notes, our mental silence
similarly brings awareness and meaning to our actions and the actions of
those with whom we are in contact.

Working with the sangha and helping others

As Buddhists, a central part of our practice is to work with the
community of monks, nuns, laymen and laywomen who make up the Sangha.
Similarly, musicians work with their bandmates to give to and receive
from an audience. In each public performance there is an opportunity for
the band members to give a teaching in the virtues of diligent practice,
by providing a good example through the way they play their instruments
and how they act onstage or offstage. Each composition focuses on
different aspects of life, and if the composition touches something
inside the listener or the performer, the experience can be used as a
point of reference or inspiration for understanding better what is truly
important in their own lives. If the music provides an environment for
releasing blocked energies such as stress and anxiety, there is an
opportunity for healing and insight to occur.

In the jazz world it is completely appropriate for the listener to
respond to the music during the song. Listeners have an opportunity to
be an active part of the improvisation and their response becomes a part
of the music. Some responses may be in the form of shouting out,
`A-MEN!', or `That's right!', or `!'. Clapping and whistling after a
musician has played a solo improvisation are also common responses. If
the band is really `burnin'' or a player is playing so well that they
may be described as being `on fire', the audience's response will
reflect exactly that and be more animated and lively. I see this as a
sharing of energy. The musician directs energy through the instrument,
and the audience transmits energy through verbal and physical responses.
It is this sharing of energy that makes the performance experience so
alive and special.

When people come to hear the Dhamma teaching with sincere interest and
faith, they reciprocate by transmitting the positive energy of the
inspiring truths they hear back to the teacher. The teacher may respond
by touching the heart with humour, which will be met with wholesome
laughter from the listeners. The fearsome demons of negative mental
states can be transformed into little puppies during these special
moments of a powerful Dhamma teaching. A release of fear and anxiety can
bring a sigh of relief and a moment of silence from the listener.
Breathing may become deep and slow. Others may feel open and safe enough
to respond by asking questions that touch tender areas of their lives.
As with jazz improvisation, if the teaching is done naturally in the
moment, the product is much better than anything one might pre-plan.

The development and communication of the most profound insights in music
occur during rehearsals, performances, instruction and jam-sessions.
Through these forms we have the opportunity to help others with the
challenges of life. Personal issues will come up during rehearsals, and
the group contact can help bring awareness of certain issues. If a
musician only plays alone in the practice room, he will never have the
possibility of experiencing the contrasts that expose his habits through
the group. But when a musician plays with a group he can investigate
whether he plays too loud, too soft, too slow, too busily or in balance
with the others.

Similarly, in Buddhist practice the Sangha is the community of disciples
who work together through the traditional form of daily group
activities, ceremonies, instruction and discussion. We constantly
compare ourselves in relation to the way the group practises as a whole.
Do I walk too slowly or too fast? Do I eat more slowly or more quickly
than the others in the group? In relation to the group, how do I speak,
sleep, do chores or put forth effort? Being aware of these habitual
energies is a preliminary step, but in the monastic practice awareness
is encouraged in order to transform unwholesome habits into the
wholesome ones which lead to liberation, as taught by the Buddha. On one
occasion a community member spoke with me about my tendency to chant
louder than others in the group. It was only through chanting with the
Sangha that I had the opportunity to have this tendency brought forth
into my awareness. The old tendency to play louder than some bandmates
also used to crop up in my jazz playing at times. Training with the
Sangha helps me to bring my habits into the light of mindfulness, so
that I can begin to change them for the good of all.

Awareness

In the practice of both music and monasticism repetitive actions are
used to cultivate mindfulness. The scales, finger techniques and
following music theory rules are to the daily life of a musician as
bowing, meditation techniques and following the Vinaya rules are to the
daily life of a monk. Walking, speaking, washing things, moving things,
entering and leaving the \emph{sala} or \emph{kuti} and getting on the
\emph{āsana} are all areas which, if approached with great care and
awareness, become beautiful and magical. How many times do we take off
our bowl lid each day? Even in the smallest of acts there are
opportunities to act with dignity and allow the sacred to enter our
lives. How does one ring the bell? If we listen closely to the way it is
rung, we can learn about the one who is ringing it. Is there a sense of
urgency, frustration or restlessness? Can one hear the focus of the act,
or is there an irregularity? In general, does the execution of our
actions both support the mind in abiding peacefully and inspire others?
For myself I find it very inspirational to witness masterful expression
through the living examples of experienced practitioners.

I was introduced to the Theravāda tradition at Abhayagiri Monastery near
my home in California. I remember how the everyday tea conversation at
Abhayagiri was turned into an art. It was a safe place for people to ask
questions about the practice. Ajahn Pasanno would answer questions from
the laity with ease, and then skilfully allow for a space of silence. It
was a moment of being with things as they were. At this time one could
be conscious of the breath and let what had been said be observed fully.
If the person had another question, they had the opportunity to ask it.
There was a feeling that people could ask all their questions until
there were none left. After that period had passed there was another
time of silence. Since at that time I was an enthusiastic layman with
many questions, I remember feeling very satisfied to be listened to. The
whole approach was peaceful and allowed for the release of anxious
energy.

Usually in life I experience conversations that move quickly, jumping
around from one topic to another. People rarely listen to what has
already been said and might even interrupt each other. Interruption is
not a personal attack but a reflection of the restless mind. When
conversing in the language of music, it is the same restless mind which
interrupts others on the bandstand or interrupts one's own thread of
continuity during solo improvisation. I have noticed that many of these
unsatisfactory conversations tend to be oriented towards trying to be
understood, instead of trying to understand what others are
experiencing. We may have to embrace an uncomfortable feeling if we wish
to truly understand the suffering another is expressing. If we are not
aware that we are uncomfortable with an issue brought up in a
conversation or that we are interrupting others, we will always be
blinded by the restless mind, which selfishly only wants to be
understood. It helps to use creative means to keep the mind malleable
and receptive, in order to listen with understanding to whatever comes
to us in life.

Creativity

Sometimes `life just happens' on the bandstand, and in that moment some
playful creativity may work better than letting it all fall apart. In
jazz improvisation a mistake is worked with on the spot in live time.
For example, if the saxophone player accidentally makes an irritating
high-pitched squeak on his reed, this mistake can be transformed by
deliberately inviting it into the piece. If the player makes the mistake
twice in a row, another band member could notice the rhythmic pattern of
the squeak and incorporate that rhythm in some form of a response.
Sometimes the reed may be difficult to work with, but many times the
squeaking is the result of a player experiencing fear. Think how
someone's voice may crack or squeak if they are excited; it can be
similar to that. The process of befriending the fear happens all in the
moment. There isn't time for thinking, and it takes practice in
creativity to pull this off.

Creativity is a necessity in both practices. Ajahns use creative means
in relating the teaching. One of my favourite devices is humour, and as
a musician I would use humour in compositions or improvisation. I have
noticed that the Ajahns in this lineage use wholesome humour as a
skilful device, to make things light enough for the mind to be malleable
and receptive to the teaching. I bet the Buddha had a good sense of
humour, but unfortunately I don't hear much about this personality
trait. It takes creative means such as this to really reach people. If
they feel bored or just talked at they may get up and leave the Dhamma
teaching behind. Metaphor and analogy are creative devices that the
Buddha used and his disciples continue to use today. The ability to find
creative ways to relate the Dhamma in simple terms that people can
understand is the mark of a masterful teacher. All Ajahn Chah's
teachings that I've heard are of this style. The teaching is clear,
simple and goes straight to the heart.

Effort

In the music world, challenges such as anxiety, fear, absent band mates,
miscommunication, drug and alcohol use, bad attitudes and different
ability levels are just some of the reasons why proficiency can decrease
during a performance. My piano teacher would recommend putting forth
effort through pushing up tempos and practising long hours, in order to
have the strength and endurance needed to persevere in such challenging
situations.

In the monastic training, effort is put into pushing up tempos and
practising long hours in order to cultivate the focus of mind needed to
greet the raw conditions of life exactly as they are in the given
moment. Doing group activities briskly, walking barefoot over long
alms-round routes and sitting in meditation for vigils and retreats are
all ways in which we put forth effort in the monastery. When one puts
effort into making things neat and tidy, there is an opportunity to see
the result of a job well done, which provides a space for future good
actions to arise. I remember seeing a monk working with great vigour as
he shovelled sand on a workday, and that inspirational display of effort
wholesomely affected many areas of my practice long after the workday
was over.

When cultivating music, I found that good results were more likely to
arise if I put effort into maintaining a daily practice. As momentum
built up and less effort was spent on simply getting myself to sit on
the piano bench, I could redirect my energy towards the more refined
aspects of the music. Just as I would wake up early before school as a
child to practise piano, or stay up late at night enduring fatigue in
order to fit in the daily practice which had not happened yet that day,
I now put in that same kind of effort to go to the meditation hall in an
attempt to cultivate good qualities. Even though there are times when I
don't feel like practising, there is also deep joy, and a desire to do
the practice that seems to summon the effort needed for the occasion.

Challenges

When I was introduced to Abhayagiri Monastery, I was given a photocopy
of what visitors needed to know before staying at the monastery. I
remember reading the Eight Precepts and laughing out loud when I got to
what I now affectionately refer to as `number seven'. I didn't
understand why refraining from playing or listening to music would even
make the list. I remember showing this Precept to a few people close to
me and saying, `I can do it for a week, but that's it!'

The one week stay gradually turned into a three-month lay residency.
That period of time was fraught with `number seven' questions. I
remember Ajahn Pasanno gently answering my questions relating to the
monastic practice. I would ask, `Ajahn, can monks give music instruction
if they don't touch any instruments? Can monks at least accompany the
chanting with guitar?' My mind wrestled with absolutes. I wondered
whether music was wrong action. Was I a bad person for playing music? I
noticed from the questions that friends and family members asked that
they also looked at the Precepts more as commandments than training
guidelines. If I looked at `number seven' as `Thou Shall Not\ldots{}', I
could only feel guilt and remorse. Through the instruction at
Abhayagiri, I feel that I was able both to appreciate what my music
practice had given to me and see how the unwholesome environments and
mental states often associated with music practice can be a hindrance to
the Holy Life and true peace of mind. I was also glad to see that by
giving up my music for the practice of the Thai forest tradition, I was
not giving up learning from challenging situations. In this Thai forest
tradition there is a warrior spirit that is willing to go against the
grain by marching fearlessly into the middle of the battle to endure
whatever comes, even if it is uncomfortable. One is encouraged to give
100\% to the training with fierce determination. I find it particularly
challenging at this point in my development to reach a balance between
knowing when it is time to push and time not to push, but I feel that
through the practice of the Thai forest tradition I have an opportunity
to find the middle way by seeing where the boundaries lie.

With my music practice I was constantly looking for challenges in order
to sharpen my skills. Some of them included moving far away from home to
study at one of the world's most intensive music colleges; taking almost
any opportunity to put my practice on the line in a public performance
situation; sharing my most personal original compositions with others;
enduring the discomfort of playing with musicians who were much more
advanced instead of trying to be in the superior position; seeking out
the most qualified and disciplined teachers available; and playing
demanding music that was outside the comfort zone of the traditional
jazz I specialized in.

Similarly, I looked for challenges in the monastic life such as moving
far away from home in order to participate in the intense practice at
Wat Pah Nanachat, led by some of the most disciplined and experienced
practitioners, who keep a strict standard of Vinaya. Through seeking the
opportunity to put my practice on the line by working out in the open
with the Sangha, I expose my most tender moments in the context of a
community where it is impossible to have a secret. Training as a
beginner at the bottom of the line, I am constantly in a position to
learn about all the do's and don'ts of monastery standards, instead of
comfortably sticking to what I learned years ago in the music field. I
feel that living this monastic life is the greatest challenge I can use
to meet my goal of learning the truth of how nature works. The restraint
of the senses feels like a pressure-cooker at times. We don't eat
whenever we wish, there's no flipping on the television for a little
distraction, and there's no sexual activity whatsoever. The rules of
training ask one to give up many means of outer control, and put one in
a position where there is much less room to run away and hide from
long-avoided fears. By going against the grain of these habitual
tendencies, we have the opportunity to learn the ways of the mind.

The monastic life is difficult enough as it is, but I find that being in
Thailand adds another layer of challenge to my practice. The forest
environment in Thailand is constantly changing, and has an aggressive
quality to it that never gives one a chance to relax completely. There
are intense weather conditions of heat and rain. Either mosquitoes and
ants are biting you or you know that they could bite at any time. There
are other insects, scorpions, centipedes, poisonous snakes and wild
animals to be mindful of here in Thailand as well. Recently thousands of
termites took over my \emph{kuti}, swarming all over the walls and door
just inches away from me. I walked through the night with the belongings
that I thought the termites would consider edible and tasty, and found
refuge in the monastery's sewing room. In the mornings on almsround, I
am challenged by walking barefoot over sharp stones, while dodging
occasional shards of glass and ridged bottle caps. Even though there may
be cuts on my feet, walking through the widespread buffalo dung smeared
on the road by the village traffic is sometimes unavoidable. Compared to
the USA, in Thailand I am challenged by the different sanitary
standards, increased threat of disease, different cultural values, the
different language, and being apart from family and the familiar. But
though the challenges I've mentioned have been difficult, they aren't as
bad in the moment as my mind would like to tell me. I also realize that
I am supported through these challenges by generous lay supporters,
Sangha members, family members and, in general, by practising in the
context of a Buddhist country.

So how do I blend music with the monastic life now? I like to appreciate
the sound of dissonance from out-of-tune Sangha members during the
chanting. Other times I enjoy the sound of someone mistakenly hitting
their bowl or spittoon, appreciating the different timbres each
produces. I enjoy the chance to ring the monastery bell and to listen to
others briefly express the causes and conditions of nature through
ringing the bell themselves. And of course, it may not be a surprise to
anyone that I enjoy chanting.

`Hey man, don't give up your music!', counselled my jazz piano-player
friend with a tone of urgency and disappointment, as he saw me move off
the path of music practice that we had once shared. I've been
questioning what I have actually given up and what is happening in my
life, but after the exploration of writing this piece I feel that the
question of `Who am I?' is closer to my heart. There are many layers to
my recent transition from music practice into the monastic life, but
ordaining as a \emph{sāmanera} has been predominantly marked by an
intense questioning of my identity and sense of self. Uncomfortable with
not knowing who I am, I feel that holding the question is more important
than trying to wrap it up neatly with a bow on top, tucking it away as
if I've found a conclusive answer. So in the midst of uncertainty I will
practise on.

\emph{ The Author}

\emph{Sāmanera} \emph{Gunavuddho went on to take full ordination a year
later, with Luang Por Liem, the abbot of Wat Pah Pong, as his preceptor.
He trained in Wat Pah Nanachat and subsequently at Wat Marb Jan in
Rayong Province. In 2007 and 2008 he undertook to help the current
abbot, Ajahn Kevali, to look after Wat Pah Nananachat and Poo Jom Gom
Monastery. Following his tenth Rains Retreat, he joined the community of
Abhayagiri Monastery in California. He has helped spread the Dhamma in
the San Francisco Bay Area, and played an active role in the Buddhist
Global Relief Foundation, a charity devoted to the goal of working to
eliminate global poverty and its associated problems. He has since
decided after 14 years in the robes to live as a lay Buddhist in
Thailand by his name David De Young, and continues spreading the Dhamma
in English in Bangkok.}



\setChapterNote{Two local lay supporters talk about their relationship to the wat.}
\chapter{Refuge in Sangha}
\markright{\chapterAuthor}

\emph{Many people in Thailand have ties to Wat Pah Nanachat. The
villagers of Ban Bung Wai, even those who rarely go to the wat except
for funerals, see the monastery as `theirs'. They feel a sense of pride
about it, and a sense of responsibility. Then there are the people from
surrounding villages, the local town of Warin and the city of Ubon, who
regularly come to make merit, keep the Eight Precepts on Wan Phra days
or practise in the monastery for longer periods. Lay supporters from
Bangkok and other provinces may also come up to stay in the wat during
their holidays. This section consists of the words of Mae Samlee, a
village woman living in a house in the fields outside the monastery, and
Por Khroo, a primary school teacher from Ubon.}

Mae Samlee

`The pain's not been so bad really. My husband ordained as a monk for
fifteen days to make merit for me {[}she smiles at him warmly{]}, and
I've been feeling better ever since. It's just the past three or four
days that have been a bit more difficult.'

Mae Samlee is 55. She has cancer of the spleen which is metastizing. She
has spent many months in hospital over the past year and had two
operations. Now she is back at home in her house among the rice fields
between the monastery and the main road.

`I've been going to monasteries for as long as I can remember. When I
was a young girl my mother would always take me with her when she went
to make merit. After I got married I used to go to Wat Pah Pong on
Observance Days. I loved it so much: making food for the monks,
listening to Dhamma talks, meditating. Then we moved to Kanchanaburi and
stayed there for six years. It was a rough place. My husband became the
village headman and everywhere he went he had to carry a rifle and a
pistol, he said one weapon wasn't enough. Then he read a talk by Luang
Por Chah and we decided to come back to Ubon. My brother-in-law lives in
Bung Wai. He wanted us to come here and said he'd look around for some
land for us. I said,''I don't care how expensive it is, please find us
land close to the monastery, so that I will be able to go every day,
even when I'm old.'' Everything worked out: we got this plot of land
right in front of the wat, we built a house on it and now {[}she smiles
widely{]} it looks like I'm not going to have an old age after all. I
must admit that sometimes I wish I had accumulated more merit in my
life.

`I meditate whenever I can, whenever the pain is not so bad. I chant in
the morning and evenings. Actually, these days I often do the evening
service at three o'clock in the afternoon! The pain usually comes on in
the evening, you see, and I'm afraid it will stop me from chanting. But
today, I've been so excited all day waiting for Ajahn Jayasāro to come
to visit that I've felt fine all day.

`I felt homesick when I was in the hospital in Bangkok having my
operation. It all took such a long time. On the days when I could sit
up, I did the morning and evening chanting normally. When I couldn't sit
up, I chanted as best I could lying down. Then that first time Ajahn
Jayasāro walked into the ward I felt so happy! It was such a wonderful
surprise. And he brought me a little picture of Luang Por Chah, too, to
put on the table by my bed. After he'd gone the other patients were
really curious. ``Who is that Western monk? Where does he come from? How
do you know him?'' I felt much better after he came. Things didn't seem
so bad. I remembered the things he taught me and they were a refuge to
me. I always kept in my mind the virtues of the Buddha, Dhamma and
Sangha.

`Next Wan Phra, if I'm feeling any better, I hope I will be able to go
to the wat. I don't want to miss the morning talk. It's another few days
yet. I hope I'll feel a bit better by then.'

Por Kroo

Por Kroo is a male primary school teacher aged 57 and a member of the
monastery's lay committee.

`I've been going to monasteries since I was a child, when my father used
to take me. I first went to Wat Pah Pong way back in 1959. But it was so
difficult to get there that I didn't keep it up. I was disillusioned by
the monasteries near my home in Ubon. One day I went to invite some
monks from the local monastery to come and eat in my house. I happened
to see them acting in ways completely unfitting for monks. I lost all my
faith. I didn't go near monks again for three or four years. I didn't
even put food in monks' bowls on alms-round. During that time I felt as
if I'd lost my sense of purpose, that I had no refuge and was just
drifting through my life. Then I thought of Wat Pah Nanachat, and it
happened that that year --- it was 1983 or 1984, I think --- the
ministry made it compulsory for teachers to go to listen to Dhamma
teachings in monasteries. Our group came to Wat Pah Nanachat. I walked
into the \emph{sala}, and the first thing I saw was Ajahn Pasanno and
Ajahn Jayasāro sitting on the \emph{āsana} talking together. It was such
a moving sight for me to see two Westerners so restrained and composed
in their bearing; I felt a new wave of inspiration and I started coming
regularly. I was impressed by the dedication of the monks. I came to
know for the first time that monks are not supposed to use money, and I
began to realize the way monks are supposed to live.

`The result of coming to the wat over the years that I see most clearly
is that I'm a lot more calm and patient than I used to be. I 'm
basically quite a forceful, headstrong kind of person. Listening to the
Dhamma and the teachings of the Buddha and trying to put them into
practice, I've seen my mind cool down and become more peaceful. I don't
lose my temper or get angry when I'm provoked in the way that I used to.
My temper has improved a lot. Also, I feel more mindful in my daily
life. I always tell people that I've been to many monasteries, but it's
here that I've received the most beneficial teachings.

`I've had some difficult times at work over the past few years. When I
see something wrong I can't always keep quiet. On occasions when I've
spoken up against corruption, I've been slandered and victimized by my
seniors. At home too it's been hard. My wife was in a motorcycle crash
last year. Worse than that, a few years ago my son died. One day he
returned from work in Bangkok in his boss's car. His boss had driven him
up himself. He said my son had an inoperable brain tumour and didn't
have long to live. That night I spent many hours with my son, teaching
him the Dhamma reflections that I had learned from my teachers. The
following morning I went to the wat, and when I came home in the late
morning I found him lying dead on the couch. It was a terrible shock for
my wife and it took her months to get over it. I've had a lot to endure.
If it wasn't for the Dhamma and the advice and support of the Sangha, I
don't know how I would have coped.



\setChapterAuthor{Ajahn Jayasāro}
\setChapterNote{An interview conducted by Tan Paññāvuddho.}
\chapter{Dtao Dam Forest Monastery}
\markright{\chapterAuthor}
% Title: Dtao Dam Forest Monastery
% Author: Ajahn Jayasāro

Dtao Dam Forest Monastery is a branch monastery of Wat Pah Nanachat,
located in the mountainous, thickly-forested jungle of Mae Nam Noy, Sai
YokNational Park in the Kanchanaburi Province of Thailand. Mostly
through the tireless efforts of the monastery's main lay supporter, a
brave and determined woman called Tivaporn Srivorakul, the pristine,
lush quality of the Dtao Dam forest has been well preserved. Tivaporn
operates a tin mine in the Dtao Dam area, employing Burmese, Karen and
Mon workers who live in this remote border region in order to escape the
social and political strife of present-day Burma. Despite great pressure
on her for many years, she has stood up to all those interested in
destroying the forest.

At the onset of each hot season, the monks and novices of Wat Pah
Nanachat make a three-day \emph{tudong} through the National Park into
the monastery, where they spend two months in retreat. This year (April
 1999), the \emph{tudong} was cancelled due to cattle and
drug-smuggling activity in the outlying area, forcing the Sangha to be
brought in by four-wheel drive vehicles. The following interview
covering these and other issues related to Dtao Dam was conducted by Tan
Paññāvuddho at Ajahn Jayasāro's \emph{dtieng} in the Dtao Dam forest.

\emph{Tan Paññāvuddho}: Tan Ajahn, to begin, could you give a brief
history of Dtao Dam Forest Monastery? How was this place founded, and
how has it developed and evolved to reach its present state today?

\emph{Ajahn Jayasāro}: In 1979 Ajahn Pasanno was on \emph{tudong} in
Kanchanaburi. He got to know Khun Sunan, the owner of another mine in
this part of Kanchanaburi, and she built a \emph{kuti} for him in the
forest near her mine where he spent the Rains Retreat. That mine is now
abandoned -- we pass it as we come into Dtao Dam. Years later the owner
of the Dtao Dam mine, Yom Tivaporn, was struggling to preserve the
forest here in the National Park. She thought this would be a wonderful
place for forest monk\emph{s} to live and train, and hoped that their
presence might also deter hunters and loggers. So she invited Ajahn
Pasanno to bring some monks in and do a retreat here. She offered to
make sure food and any other requisites would be provided, as there's no
village for alms-round. The idea was that if it was appropriate some
basic \emph{kutis} would be built and two or three monks could stay on
for the Rains Retreat. Everything worked out as she hoped.

\emph{Tan Paññāvuddho}: So in what year did Ajahn Pasanno first bring
the monks?

\emph{Ajahn Jayasāro }: About eight years ago. That year two monks spent
the Rains Retreat here in \emph{kutis} built at the foot of this
mountain. Since then monks have spent the rainy season here on a regular
basis. Gradually, over the years, there have been developments. A few
years ago we had a tractor come in and cut a road up to the upper sala.
Three \emph{kutis} have been built up on the ridge there.

As you know, where we are now (to the west of the upper \emph{sala} on a
different mountain, where the inner \emph{sala} is located) is a very
densely-forested plateau through which the stream meanders. We have
twenty \emph{dtiengs} scattered throughout the forest for the monks to
use during the hot season. On this plateau the climate is very pleasant
in the hot season, but extremely damp and unpleasant in the rainy
season; there are many leeches, and if you put a robe out to dry after
washing it, it may still be wet after twenty-four hours. Hence the need
to build the upper \emph{sala} and three small \emph{kutis} on the more
exposed ridge over to the east of here, so monks can stay at Dtao Dam
during the rainy season.

We also built a lower \emph{sala} down at the base of these two
mountains. The monks walk down in the early morning, take their daily
meal and then come straight back up the mountain afterwards. There's no
village for monks to go on alms-round, which makes this quite an unusual
place for monks to live and puts serious restrictions on how many can
stay here on a long-term basis. But we have a lot of lay supporters,
particularly in Kanchanaburi, who give Yom Tivaporn food and provisions
to bring to the monastery when the whole community comes in the hot
season. There are a couple of Burmese families who cook for the monks,
particularly one lady,.

\emph{Tan Paññāvuddho} : Is it common for monks to inhabit forest in
National Parks?

\emph{Ajahn Jayasāro }: About six years ago there was a period in which
the presence of forest monks was looked upon by the government as being
detrimental to the forests. In a number of highly publicized cases monks
were accused of felling trees to build lodgings and developing
monasteries in a way that harmed the forest. A government plan proposed
having all monks living in national parks or reserved forests ejected
from them. In fact, although there have been isolated instances of monks
being insensitive to forests they were living in, this has never been
one of the major problems facing forests in this country. But anyone who
lives out in country areas knows that the presence of monks is the major
factor inhibiting deforestation. It's hard to say to what extent the
plan was devised out of ignorance, and to what extent it was influenced
by the forces who want to get the monks out of the way. Fortunately,
however, there was an about-face -- the plan aroused a lot of
opposition. Tan Chao Khun Payutto wrote an influential pamphlet
explaining the importance of the forest to Buddhism and Buddhist
monasticism, and subsequently a new plan emerged which involved giving
opportunities for monastic communities to help to care for the forest.
The new plan allowed monasteries to apply for permission from the Royal
Forestry Department to look after a certain area of land, from a
thousand \emph{rai} upwards to five thousand \emph{rai}. Looking after
the forest in this case means little more than living in it in very
basic dwellings. The respect that monks command, at least amongst the
local people, is acknowledged to inhibit the destruction of the forest.
The Forestry Department is seriously understaffed and underfunded. The
U-turn regarding the role of monks in forest conservation was more or
less an admission of defeat. They know that they can't prevent the
forest from disappearing and the government prefers to spend its money
on other things.

So we applied to participate in this programme. In fact, we asked for a
lot more land than we were eventually given. The head of the National
Park (suspected by many to be deeply implicated in illegal logging in
the park and now transferred) was not supportive and cut our application
down to a thousand \emph{rai}. The only condition laid down for us to be
here is that we don't do a lot of building and don't cut down any trees.
We have fulfilled our plan to build the three small \emph{salas} and the
three \emph{kutis}, so as for the material development of Dtao Dam, I
can't see that there is really much more that needs to be done. Anyway,
Tivaporn feels that if there was to be any more building, the officials
hostile to her would use it as a pretext to accuse her of something.

\emph{Tan Paññāvuddho}: Who are these officials? Why are they apparently
so hostile to Tivaporn?

\emph{Ajahn Jayasāro }: Well, she's a thorn in the side of the
`baddies'. She stands in the way of certain people making a lot of
money. Every branch of the Thai civil service is plagued by corruption.
It seems, at least from the outside, that the corrupt officials
outnumber the honest ones to a frightening degree. In many places
corruption is institutionalized, difficult to avoid being sucked into
because social pressures to conform in the workplace are so powerful. Of
course there are some who manage it, but if you stay honest it means
you're unlikely to advance very far in your career. Your boss may well
not put you forward for promotions. Some people who are ambitious and
feel they have something to offer society justify their corruption by
arguing that it is the only way they can get into a position where they
can change things for the better.

In the eyes of many people in this country, forests mean money. Dtao Dam
is, to put it bluntly, big bucks. Some of these people who hate Tivaporn
don't see nature as we do when they see a forest, they don't see
wildlife -- they see money in a previous existence. Trees are money.
Land and animals are money. And because there is so much corruption,
there's so much influence-peddling up to the very highest levels. That
means the laws don't have the kind of irrevocable fixed quality that
they have in the West, at least in most people's minds. In Thailand you
always feel that there's a way to get around things. You can often get
things changed, get exceptions made, if you know the right people, or if
you are the right people yourself. There actually was a law prohibiting
gas pipelines through national forests, so a special law was passed to
make it all right. At the same time, a sub-clause allowing mining
operations in national parks somehow slipped in.

There are various kinds of scams for forestry and national park
officials, the most obvious one being turning a blind eye to the illegal
logging interests. As time goes on and good timber becomes increasingly
rare, the whole business becomes more and more lucrative. National parks
can be eligible for grants for reforestation, but the area in question
has first to be classified as degraded forest. So a common ploy is to
light a fire or cut down some trees and then get the designation of the
land changed. Once it has been declared `degraded forest' you can apply
for funds for reforestation. Then you can use one part of the grant for
reforestation but keep the rest for yourself and your henchmen. You can
also let the loggers have more of the original big trees. And so on.

On the national level, forests all over the country are under pressure
from the growing human population. The whole question of land settlement
has become a political hot yam and it brought down the government before
last. The question that has arisen with some urgency over the past few
years is: what do you do when poor and desperate people encroach on
national park or reserved forest in order to clear the land and grow
themselves some food? If you evict them, where are you going to put
them? What will be the political repercussions for political parties
which seek to present themselves as protectors of the poor? The answer
given by the previous government was to pass a law giving these people
squatters' rights, which, in effect encouraged people all over the
country to clear land in national parks for farming.

So there are many pressures on Dtao Dam. Corrupt local politicians,
government officials, businessmen, soldiers, border police: a lot of
ethically challenged people have their eyes on the place. Theoretically
it should not be too hard to protect Dtao Dam. To get wood out there's
only one road, and there's a border police barrier across it and a
barracks overlooking it. But of course, all the people at the police
post are on a very low wage, far from home. People at the National Park
are involved. Everyone takes their share.

\emph{Tan Paññāvuddho} : Tan Ajahn, a couple of weeks ago several of us
climbed to the top of the tallest local mountain peak, where one can get
a view in all four directions. From there it is forest as far as the eye
can see. You have mentioned before that there are plans to build a road
from Burma into Thailand though this mountain pass. Why are various
parties so interested in building a road through such remote mountainous
forest?

\emph{Ajahn Jayasāro }: If you look on a map, you'll find this pass is
the shortest and most direct route between the city of Kanchanaburi and
the Burmese port of Tavoy, which is maybe forty kilometres away. That's
not a long way to cut a road. It would provide Thai industry with direct
access from Bangkok, through Kanchanaburi, to the Andaman Sea -- port
facilities, holiday resorts and so on. The army officers, politicians
etc. who've been buying up land on either side of the prospective road
would make huge profits when the land prices go up. And of course, once
the road is cut you open up the entire forest for exploitation. In every
case that I can think of, a road built through a forested area has
signalled the beginning of the end for that forest. I don't think
there's any question that the road will be built -- it's just a
question of what route it will take. The pass here happens to be the
easiest traverse of this mountain range. Further to the south and to the
north the mountains are much more difficult to cross. This is why in the
past, during conflicts between the Thai or Burmese, this was the pass
that the invading armies would march through. It's a kind of gap in the
mountain range. From where we are sitting right now at an elevation of
about seven hundred metres, we're only about a fifteen-minute walk from
Burma. From an engineering point of view it would be a relatively easy
job. Driving along this route, you'd hardly notice the gradient.

\emph{Tan Paññāvuddho} : So out there, what interested parties besides
the monks are working to preserve this Dtao Dam forest?

\emph{Ajahn Jayasāro }: Well, the Thai environmental movement is in many
ways still in its infancy. Probably only in the last five to ten years
has it had any kind of muscle. It is only very recently that the
authorities have even felt a need to pay lip-service to environment
protection. Of course there have always been people warning against the
destruction of the environment, but during the boom economy that
preceded the economic collapse, many people just assumed that
environmental degradation was the price you had to pay for prosperity.
Preserving forest was considered the concern of romantics or people not
living in the real world. After all, what does a tree contribute to the
gross national product? Now, of course, with the economy on the rocks,
the reasoning is that there are more pressing priorities.

Another telling factor is the perception of forest. Until recently
forests were associated in Thailand, and indeed throughout Asia
generally, with backwardness. Cities mean civilization. For example, you
have the Thai word \emph{Pah-thuean}. `\emph{Pah'} means `forest' and
`\emph{thuean'} means `uncivilized', implying that people who live in
the forest are backward and uncivilized, and that everything associated
with nature is the past. That's been a strong underlying idea in the
view of the urban population, and particularly perhaps among businessmen
and politicians. There has never been any sense of the forest as a
national heritage that should be looked after. And this prejudice is
still very strong. It's only recently that there have been attempts to
introduce a more progressive understanding of nature in the school
curriculum, with the aim of creating a new respect for nature. Perhaps a
period of alienation from nature is necessary. As far as I know, love of
nature did not develop in the West until the Industrial Revolution
either. If you look at the history of Western art, for example, it
wasn't until the eighteenth century that there were pure landscapes,
paintings in which nature was considered an interesting topic in itself,
rather than as merely a backdrop for human beings.

So there are various non-governmental environmental groups at work. But
as far as the government itself is concerned, the Ministry of the
Environment is officially called the Ministry of Industry, Technology
and the Environment, which gives you some idea of the priorities and
conflict of interests there.

\emph{Tan Paññāvuddho} : All three interests lumped together?

\emph{Ajahn Jayasāro }: All lumped together. And it's the same with the
Forestry Department. It has a dual role: firstly, caring for the forest,
secondly, the promotion of agro-forestry. So it's inevitably tied up
with big business and not free to preserve the forest.

There are good people, educated people particularly, who are starting to
appreciate nature and coming together to protect what is left, but they
often feel somewhat helpless and intimidated by all the forces against
them. These people live in the cities, a long way away. As you can see,
I am not particularly optimistic, but there is a bright side. Here at
Dtao Dam we have representatives from the palace helping us.

Right from the first days of the Thai kingdom in Sukhothai, there was a
custom whereby on Wan Phra anyone could go to the palace and ring a
certain bell to request an audience with the King in which to let him
know their grievances. You could discuss a problem where you hadn't been
given a fair deal by the government authorities, for instance, or where
you'd been the victim of corruption. It was a very special appeal court
that bypassed the judicial system. This custom has come down to the
present day. Now it takes the form of an office in the palace to which
anyone can write and appeal. The people who work in this office liaise
with the Queen's private secretaries to investigate these claims, so you
go right to the top. And since the power of the monarchy is strong, this
office exercises a lot of influence outside the conventional power
structures of parliament. People from this office have been instrumental
in getting some of the most corrupt officials transferred, for example.
We also have friends in various conservation bodies, as well as a senior
minister in the present government.

\emph{Tan Paññāvuddho} : What about Nature Care, the environmental
preservation organization founded by Ajahn Pasanno? How is it involved?

\emph{Ajahn Jayasāro }: Nature Care is an NGO, a non-governmental
organization, set up originally in Ubon to help preserve the forest
bordering the Mekong River. Subsequently, with our association with
Tivaporn, Nature Care established a branch in Kanchanaburi. This
facilitates applications for funds from businesses and government bodies
for conservation purposes.

Returning to Dtao Dam again, one of the things that complicates the
issue here is Tivaporn's mine. She has been running this mine since well
before the area was declared a National Park, and she's deeply in debt.
The mine also functions as a kind of welfare programme for over 100
workers and their families, most of whom are undocumented immigrants
from Burma. Her mining concession will last for another four or five
years. But people who wish she wasn't here -- she is the main obstacle
to people destroying the forest -- have gone as far as to threaten her
life on a number of occasions. Others have instigated various rumours to
blacken her reputation. One of the things that people say is that she
invited the monks here as a front. Or they say that she's trying to
create the image of being an environmentalist, when all she is
interested in is keeping her mine going. She has even been accused of
building a private luxury resort here.

\emph{Tan Paññāvuddho}: That's pretty unbelievable!

\emph{Ajahn Jayasāro }: Well, you'd think so, but as the people
spreading it were powerful and influential, this accusation apparently
reached the highest circles of the government and the royal family. As
you know, last year Tivaporn was given a prize as Thai Citizen of the
Year by a leading charity of which the Crown Princess is the patron. The
charity was worried they'd been deceived and decided to investigate. Of
course, they found the charge was baseless. But these kinds of slurs are
leaked to the newspapers. Also, senior army officers are given folders
containing facts about Dtao Dam which distort the picture. For instance,
Tivaporn was shown a folder two weeks ago in which her signature for an
application for funds for reforestation of Dtao Dam had been forged.
It's not clear whether someone is attempting to embezzle these funds or
is trying to accuse her of receiving the funds and not using the money
for reforestation. But this kind of thing is going on all the time. All
the various parties involved, even the environmental groups, have their
own contacts, and they hear these things. Naturally, they're not always
sure what to believe. So the policy for us has been to invite these
people in to see for themselves.

Another problem down the road is when the mining concession ends. One of
the plans is to have various kinds of scientific projects going on,
especially ones related to biodiversity and botanical research. This is
an area rich in biodiversity. A botanist was here a few weeks ago and he
was extremely excited by what he saw, all kinds of things that he'd
never come across before. So there is a hope that there will be some
kind of botanical or biological station at the base of the mountain,
where graduate students can come and do research.

\emph{Tan Paññāvuddho} : What about the animal life? We've seen all
kinds of exotic animals, from elephants to black panthers to white
tigers, and bizarre-looking creatures unlike anything I've ever seen
before.

\emph{Ajahn Jayasāro }: That's why we have those animal-sighting forms,
to document that these animals are really here. It's a matter of getting
this information into the hands of the people with the right intention
towards forests.

\emph{Tan Paññāvuddho} : Tan Ajahn, you've alluded to it somewhat, but
could you articulate your role as Abbot and as a forest monk in
addressing these problems?

\emph{Ajahn Jayasāro }: Well, as you know conventionally speaking, I am
the Abbot of Dtao Dam, but most of the year I live at Wat Pah Nanachat.
I'm able to come out for a month or two in the hot season, but I keep in
contact by telephone with Tivaporn when I'm back in Ubon, to stay
abreast of what's going on and give her support and encouragement. She
gets bullied and slandered a lot. It's a lot to put up with, apart from
the ordinary pressures of running a business in adverse circumstances
and struggling to make enough money to keep going. You know what an
incredible drive it is in and out from here, and you know how often she
does that. She goes out for a day, then comes back in again, then drives
a truck all the way to Phuket to sell the ore from the mine, then drives
all the way back up again to meet with all these academics and
scientists here for a visit tomorrow. I've asked her to phone me right
away if anything really difficult or heavy comes up. For example, she
gave us the advance warning about the cattle and drug-smuggling going on
across the border this year.

Also, a role that one plays as a forest monk who is also an Abbot is to
be a liaison or central figure. Being the Abbot of a large monastery,
and having been in Thailand for many years, I've come to know a lot of
people and I can help interested parties get in contact with each other.
So to summarize my role, I'd say it involves first, giving moral,
spiritual support, Dhamma teaching, encouragement and reflections.
Second, in the social role as the Abbot of a well-known monastery who
knows lots of people, I can help the right people get in touch with each
other.

\emph{Tan Paññāvuddho} : Seems potentially like a pretty adversarial
situation. How do you manage to maintain a non-partisan position in a
scenario that is very partisan?

\emph{Ajahn Jayasāro }: The abbot of a monastery usually plays the role
of being the referee or the impartial resort for both sides in a
dispute, whereas here it's a little different in that we're part of it.
I'm not absolutely equanimous about this. I'm not totally impartial. I'm
definitely on the side of the people who want to save this forest. But I
find it important to avoid the `us' and `them' way of thinking. Also, I
don't personally have to confront these people trying to destroy the
forest the way Tivaporn does. I've met very few of the leading figures.
And Thai society being what it is, one always keeps up social
proprieties. If someone were to wish bad things for the forest monastery
and curse us to our backs, if he met me he'd probably bow and speak very
politely.

For a monk the principle is always, `What's correct according to
Dhamma-Vinaya?; what's wholesome, what's right?', and standing up for
that in certain circumstances, while being sensitive to time and place
and the way to go about things. And not to come across as being
adversarial. For example, if something illegal is going on in the
forest, I wouldn't confront the person directly, but I might try to see
their superior, or superior's superior. Rather than being a problem
between me and someone, it's a problem in the wider community that calls
for the right people to be alerted.

\emph{Tan Paññāvuddho} : In personal terms, training here in this
remote, wild forest has been a very enjoyable and profound experience.
Could you put into words why it is so important for monks to train in
forests? What are the advantages? How do we reflect on and learn from
nature in the context of Dhamma practice?

\emph{Ajahn Jayasāro }: Well, the practice of Dhamma is one in which
it's very important to develop the ability to calm the mind, to make the
mind peaceful and concentrated. That being so, it's essential to have a
conducive environment in which there is nothing too jarring or too
exciting. So we lead a very simple life, one bared down to the
essentials, not surrounded by anything man-made or anything that's going
to pull you out of yourself. Living in a forest, there's nothing really,
nowhere for your eyes to go -- just greenness and trees all around. It
automatically encourages you to incline inwards. The natural rhythms of
the forest, of the trees and the streams, give a sense of uplift and
well-being to our minds. They ground us. This provides a very important
foundation for the meditation practice. It comes to feel natural to be
by yourself and you come to delight in solitude. Sitting meditation and
walking meditation become obvious ways to spend time, not something you
have to push yourself into. I think that for most people support from
the environment is still vitally important for success in practice.
Ajahn Chah would often talk of the relationship between physical
seclusion and seclusion from the Hindrances.

The forest is not quiet, but it is tranquil and it is teaching you the
laws of nature all the time. The things you see around you are just
natural phenomena. You're surrounded by birth, ageing, sickness and
death, arising and passing away in the most raw and obvious forms. As
you reflect on those principles internally, your contemplations find a
resonance outside of you. The phases of the moon, dawn and dusk, the
play of heat and cold, the whole natural environment attain an increased
profundity, because they express the nature of things you're
investigating internally. You feel a sense of harmony and a seamless
unity between the inner and the outer.

Living at the foot of a tree, keeping the \emph{dhutanga} practices, we
also have this wonderful feeling of being the inheritors of a tradition
that stretches back for over 2,500 years. We are not living so
differently at all from the way the great monks of the Buddha's time
lived. That sense of being a part of something larger, something noble
that stretches in an unbroken line right back to the Buddha: I think
that's a very wholesome feeling, one that a monk may cherish.

This particular forest, being home to so many wild animals, gives us the
opportunity to look at fear, anxiety and attachment to the body in a
very direct way, seeing the effect they have on our mind, the sense of
urgency they give us. While living in the forest, a lot of these
phrases: `a sense of urgency', `making every moment count' -- teachings
we've read about and studied -- really come alive. In a way it's
difficult to articulate, I feel a sense of rightness, a feeling of `This
is exactly how I should be living', and `This is how a monk lives'.

\emph{Tan Paññāvuddho} : In the Suttas, so many passages from the
Buddha's enlightenment to the Parinibbāna take place under trees. The
Buddha, always lived and practised in the forest when possible. With the
forest disappearing, and the subsequent likelihood that this will be the
last generation of forest monks, how do you see a Buddhist monk
responding to a predicament like this?

\emph{Ajahn Jayasāro }: Well, you don't have any choice really. There's
not so much that can be done. As you become a more senior monk and you
have more responsibilities and opportunity to teach the Dhamma to lay
Buddhists, you can at least point out the value of the forest, how
little remains and the need to look after it. It's also important to
point out the relationship between the forest and the Buddhist religion.
Then as a younger monk, just make the best of the forested areas
available while you can. But I think it's really important to have young
monastics come out and experience this way of practice right from the
beginning of their monastic career, because it can create such a strong
impression. You know you'll remember this for the rest of your life.
Hopefully you will continue to have the possibility to keep coming here
or to places like this to train. But even if that is not the case, for
monks to have the experience of living simply in a forest like this,
even once, has a ripple effect. As monks become more senior and have
their own monasteries and their own disciples, they will pass on their
love of nature and appreciation of the role of solitude in monastic
life.

But you know, I must admit I sometimes feel that this destruction is not
going to stop until every last tree outside of private hands is gone. To
effect a real change, it has to start in the schools. Last week, when I
went out to see what was going on with the various arson fires in the
area, Tivaporn was running a retreat for school kids in the village at
the edge of the forest. And the kids loved it -- they really responded
to the teaching about nature very well. At one point the children were
asked what their parents do for a living, and three of them replied that
it was illegal logging. It's the same everywhere. In Huay Ka Kaeng, just
north-east of here, there is a lot of wild forest. It's labelled a World
Heritage Site. It's also the place where a forest park official called
Seup Nakasatheean killed himself as a gesture to call attention what is
happening to the forests across Thailand. A foundation was set up in his
name and the army was sent in to look after and patrol the land. But
still there is cutting going on to this day. I myself don't see any
fundamental changes being made until there comes a point where laws are
laws, and whoever breaks the law is wrong and is dealt with
appropriately.

But I feel reasonably confident about the prospect of saving Dtao Dam.
It is really hard to say when you don't know everything going on, really
hearing everything second or third-hand. Still, overall, I think there
are enough people with influence who know about Dtao Dam to keep this
tract of forest intact.

\emph{Tan Paññāvuddho} : You mentioned that you went to a conference
about Dtao Dam last year. What was that like?

\emph{Ajahn Jayasāro }: Well, yes, it was strange for me. One has the
idea that Dtao Dam is a forest far away from anywhere else, where we
come on retreat every year. But at this conference I walked into a large
room with academics who all seemed to be experts on Dtao Dam. I was
wondering where these people came from. They were talking about the
biology and topography and all kinds of esoteric subjects.

\emph{Tan Paññāvuddho} : Ajahn, when I'm practising \emph{sīla, samādhi,
paññā} in long periods of solitude in the forest, I sometimes get the
sense that this in itself feels like the most direct, authentic response
to the threat of this forest being destroyed, not to mention all the
other environmental, economic and socio-political calamities on earth.
But I find it challenging to articulate this. If I were to try to
explain it to somebody who hasn't had much experience with meditation, I
wouldn't know how to put it into words. But the sense of authenticity
seems true and real.

\emph{Ajahn Jayasāro }: Yes, I agree, but I also have the same kind of
difficulty in explaining it, why it is best to practise in this way. One
thing I said the other day: if we look at the root of all mankind's
self-imposed difficulties, there is a common underlying cause. We find
that because mankind doesn't know himself, he constantly acts in
conflict with his own best interests, living merely as the puppet of
desires, fears and delusions. But practising \emph{sīla, samadhi, paññā}
deals with these things at their very roots. That way one really works
with the whole structure of what is going on, rather than just
responding to a particular expression of it. We study the mind to
understand what greed is, what hatred is, what delusion is, even in
their most subtle forms, while developing the skilful means to abandon
them, to let them go. That seems to me to be as an intelligent response
as any other.

But with trying to save the forest or whatever, I myself am always wary
of falling into the trap of `I've got to do it', or `We've got to do
this'. Once you fall into this `We've got to!' mentality, you've lost it
already. Yes, I will do what I can. But who knows what will happen?
These things occur due to causes and conditions, many of them way out of
my control.



\emph{In the following little sequel, Ajahn Siripañño, who has been the
abbot of Dtao Dam Monastery for the last five years, gives an overview
of some of the developments which have taken place since the original
article was written.}

Since the publication of this article, a whole host of factors regarding
the situation at Dtao Dam have changed, mainly for the better. What
remain almost entirely unchanged are the pristine nature of the jungle
there and the ongoing presence of a small number of forest monks
dedicated to living in the open, under the beautiful tree canopy, with
the call of gibbons and crickets echoing around.

Regarding Khun Tivaporn and the mine, in 2003 the mine lease expired,
and with it permission for the hermitage. Certain forces within the
National Park Department tried their best to force the monks to leave,
presumably hoping to exploit the area commercially in some way. However,
pressure from monastery supporters, the national press and, finally,
from within the National Parks Department itself, resulted in the
monastery being given permission to stay on, with the agreement to be
renewed every five years. More stability was achieved when in 2009 the
Thai government created a nationwide `Buddhist National Parks' project,
with the specific aim of enabling legitimate monasteries to stay in
forest areas and make use of national park land for Dhamma practice,
while at the same time relying on the Sangha for help in preserving the
forests by keeping an eye open for illegal hunting, logging, forest
fires and other dangers. In fact, with the closure of the mine and the
departure from the area of Tivaporn's family and all the workers, the
hermitage has become a completely isolated spot right on the Burmese
border, with very few people coming in and out other than a seasoned
bunch of off-road vehicle drivers.

As for Tivaporn herself, this hardy woman, now approaching 70, runs a
community centre on the property surrounding her house on the outskirts
of Kanchanaburi Town, teaching seminars on environmental matters and
sustainable livelihood. She also maintains a centre in the village
nearest to Dtao Dam, Tung Ma Sa Yo, which provides work opportunities
for her former mine-workers, many of whom remain undocumented due to
Thai bureaucratic complexities.

The road linking Bangkok and Davoy (now usually marked on maps as Tawei)
is currently under construction. Luckily, the possibility that it might
actually pass right through the Dtao Dam area was averted when
sufficient pressure from environmental groups made it clear that this
would not be acceptable. The route now passes some twenty kilometres
south of the monastery.

Food is mainly brought in by monastery supporters, cooked by one or two
resident workers and supplemented by a vegetable garden and forest
fruits and vegetables, roots, shoots and herbs which can be collected.

Every year a group of monks and novices from Wat Pah Nanachat go there
to spend the hot season in the same way as they have done for some
twenty years. The three original \emph{kutis} provide enough shelter for
a small group of monastics to spend each Rains Retreat there. During the
last two Decembers a group of students, parents and teachers from
Panyaprateep school have visited for a few days with Ajahn Jayasāro.

Animals, large and small, are still in evidence. Two years ago a herd of
elephants strolled through the hermitage, completely demolishing our
inner sala. Tigers and other forest cats are still being sighted, and
recently a protective mother bear rushed at one of our monks high up on
a mountain after her cub had wandered towards him. (Admittedly the monk
was using a mobile phone at the time, which might have alarmed her even
more. Both the monk and the monastery Nokia survived unscathed). A
rustling in the bushes may turn out to be the reclusive Dtao Dam itself
(a six-legged black turtle), a porcupine with eight-inch quills or the
bizarre Malayan tapir, a cow-sized beast resembling a combination of
elephant and rhino. These incidents, though, are very rare. What are
experienced daily are the beautiful singing of gibbons and the chatter
of monkeys. A strange rhythmic whooshing sound high in the sky will be
the flight of a giant hornbill, or even a pair mated for life, as they
fly through the valleys looking for their regular spots to feast on figs
and insects in the tree canopy. All this and more, set to music: night
and day the jungle noises beat out a samba to rival any carnival. The
streams are still flowing, and the waterfall cascades down several
levels before levelling off at the foot of the mountain on the top of
which the monks dwell. The sun rises in the east over Thailand and sets
in the west over Burma; the timeless rhythm of nature undisturbed by
man's whims and fancies. Long may it be so.



\setChapterAuthor{Paññāvuddho Bhikkhu}
\setChapterNote{Practising in the ancient way at the foot of a tree.}
\chapter{Learning Forest Dhamma}
\markright{\chapterAuthor}
% Title: Learning Forest Dhamma
% Author: Paññāvuddho Bhikkhu

\section{Alone with other creatures}

\begin{quote}
\small
`As long as the monks see their own benefit in wilderness
dwellings, their growth can be expected, not their decline.'
\quoteref{D.II.77; A.IV.20%
\footnote{Bhikkhu Aparihaniya-Dhamma, translated by Ṭhānissaro Bhikkhu.}
}
\end{quote}

Emerging from meditation while nestled in an isolated spot deep within
the folds of the forest certainly makes a wonderful way to greet the
day. As I sit here on my \emph{dtieng} in Dtao Dam, the crescendo of the
birds and insects celebrating the crack of dawn has subsided, and shafts
of light begin to seep through the trees. The beams of sunshine from the
early morning sun rising over the mountainous horizon produce a dance of
light throughout the trees and imbue the forest air with a glowing
yellow-orange hue. 

Here in the lush, tropical forests of Dtao Dam I often catch myself
marvelling at the degree to which the forest brims with life. The
natural surroundings pulsate with a vibrant energy. From the gurgling
flow of water in the creek, to the bass undertone of humming bees, to
the punctuated, high-pitched cries of barking deer, the forest provides
a constant symphonic medley of sound. At dawn the energy peaks; while
nocturnal creatures return to their abodes, the rest of the fauna awaken
with the sun to embark on a search for a new day's meal. A little bird
perched just on the front ledge of my \emph{dtieng} sings a small song. 
Then with a sudden quiver of its wings it slices through the
timelessness of a moment. I note how a detached observation of nature's
rhythmic movements brings peace, ease and a sense of release to the
ceaselessly thinking mind. In my Dhamma practise the pristine wilderness
inspires me to turn within and attempt to attune to that which is true. 
The whole environment encourages me to develop a deeper awareness of the
present moment and practise letting go. 

Suddenly, while sitting on my \emph{dtieng}, I hear a commotion in the
brush several metres away. The noises are insistent in their beckoning. 
Noting a subtle ripple of sensation from my abdomen up to the crown of
my head, I anticipate the onset of some mental proliferation. I make an
effort to bring attention to the breath to re-establish mindfulness in
the present moment, but the mind doesn't cooperate. Instead the sounds
trigger a reminiscence of a similar occasion a year ago. Immediately I
feel myself reliving the scene. 

I have just completed my early morning meditations, concluding with a
chant of the \emph{Mettā Sutta}, the Buddha's discourse on
loving-kindness for all beings. I check my clock and discover that I am
a little late for the descent down the mountain to take the daily meal. 
But this is my favourite part of the day and I want to soak up a bit
more meditation in the radiant morning sun. The fresh dew-drops on the
abundant flora indicate the cool, crisp moistness of the air, and I feel
quite content to remain sitting here, comfortably tucked away in my
robes. 

I re-close my eyes to sit in meditation for a couple more minutes, but a
sudden crashing sound in the brush competes for my attention. To hear
sounds, even loud sounds, in the forest is so common that I tell myself
not to take note and to return to the meditation. But as the volume and
frequency of the fracas in the brush increase, I can't refrain from
speculation. I realize that whatever is going on is coming closer. 
Unable to control my curiosity, I open my eyes and slowly rotate my neck
to look over my left shoulder. It takes my vision a moment to focus. The
greenness of the forest all seems to melt together like an Impressionist
painting. And then I see it. The form of an animal is darting to and fro
between clusters of bamboo and through the dense underbrush.

I make out
the shape of a large sandy-beige feline-like form, about human size, 
jumping up and down and zipping left and right in an almost playful
fashion. Two butterflies flutter their wings over its head. Is it
chasing butterflies or following some ground animal through the brush? I
can't see. Everything is taking place so quickly. But whatever is
happening, this feline creature apparently doesn't see or smell me and
is on a collision course with my \emph{dtieng}. Not really thinking of
anything, I find myself making a noise by gently but audibly clearing my
throat. The large cat instantly stops jumping around, ducks low behind
some tall grass and shrubs, then shoots clear out of sight. It had been
only three or four metres from my \emph{dtieng}. 

\section{Finding resolve in the forest}

The proverbial forest monk story revolves around the encounter with a
tiger. As the fiercest predator to be found in the wild, this large
flesh-eating character unquestionably rules the jungles of Southeast
Asia. Tigers notoriously reek with the smell of death on their breath, 
usually strong from a recent kill. Yet interestingly, although there are
countless documented instances of forest monks meeting up with tigers in
recent recorded Thai history, there is not one single known case of any
forest monk being killed by one of these beings. I had just met up with
a relative of one of these regal beasts, (later I discovered the animal
I saw was probably an Asian Golden Cat, not a properly striped
full-grown Bengal Tiger), and fortunately I did not become the first
victim on the list. 

Now a year later, as I sit at the foot of a giant tree at a new spot in
the same Dtao Dam forest recollecting this encounter, I contemplate why
fear did not arise. Why do I feel so at home in this seemingly wild and
uncontrollable environment? Practising in the forest in accordance with
a forest monastic tradition dating back to the time of the Buddha, I get
a gut sense of the authenticity of this form of training, although it is
a far cry from my upbringing and education in America. For a forest monk
there are a bare-bones honesty and naked simplicity to daily life. 
Everything is a teacher. Every moment is geared toward awakening. Ajahn
Chah points the way: 

`Whether a tree, a mountain or an animal, it's all Dhamma, everything is
Dhamma. Where is this Dhamma? Speaking simply, that which is not Dhamma
doesn't exist. Dhamma is Nature. This is called the \emph{saccadhamma}, 
the True Dhamma. If one sees Nature, one sees Dhamma; if one sees
Dhamma, one sees Nature. Seeing Nature, one knows the Dhamma.'%
\footnote{Ajahn Chah: ``Dhamma Nature'', The Collected Teachings of Ajahn Chah Vol.III, page 11.}

\section{And elsewhere}

`Where is the Buddha? We may think the Buddha has been and gone, but the
Buddha is the Dhamma, the Truth of the way things are. The Buddha is
still here. Regardless of whoever is born or not, whether someone knows
it or not, the Truth is still there. So we should get close to the
Buddha, we should come within and find the Dhamma. When we reach the
Dhamma we will see the Buddha and all doubts will dissolve.'%
\footnote{Ajahn Chah: ``No Abiding'', The Collected Teachings of Ajahn Chah Vol.I, page 34 (adapted)}

In the solitude of the deep forest, however, inspiration in practise is
always quickly tempered by the work at hand -- overcoming the
\emph{kilesas}. For me, battling the \emph{kilesas} involves a continual
struggle with some deeply ingrained proclivities: always catching myself
ruminating about some aspect of the past, present or future; trying to
let go of and unlearn desires that have been drummed into me by society
to become `somebody' or to achieve `something'; being mindful of the
arising and passing away of moods, emotions and unskilful habitual
tendencies; and forever uprooting and investigating deeply entrenched
perceptions. 

With nowhere to go and nothing new to see, I experience a spaciousness
and lightness of mind that allow some deep stuff to percolate up, 
flooding the mind with a deluge of memories. I'm reminded of various
accounts about the experience of the mind just before death. Here at
Dtao Dam I've managed to review what seems like my whole life, 
remembering some of the tiniest details, recollecting where I did no
more than spit on the ground. Yet when I can get beyond this recursive
thinking, there's absolutely nothing to do all day to distract me from a
full-on practise of sitting and walking meditation. Any patterns of
greed, hatred and delusion are given room to manifest in their most
subtle forms. The process of birth, sickness, ageing and death, 
internally and externally, becomes so obvious. As Ajahn Chah liked to
say, `We practise to understand just this much'. 

In similar intensive meditation situations of the past I often found
support and encouragement though practising with others and surrendering
to a retreat schedule. But at Dtao Dam there is no retreat schedule to
which to surrender. Although I have gratitude for those years of formal
retreats in America -- they now provide me with an invaluable array of
tools to help put my time to wise use -- I recognize that the level of
surrender here on a long-term retreat in solitude is of an entirely
different order. I realize that in the past I often motivated myself to
practise through working with a teacher or a group of
fellow-practitioners. Here, although the Sangha is scattered in isolated
spots throughout the surrounding Dtao Dam forest, when it comes down to
it I must muster up the gumption and resolve to maintain an impeccable
standard of discipline on my own. In solitude a more honest and natural
kind of effort replaces any determination fuelled by hubris. 

The anchor for my practise of awareness is \emph{ānāpānasati}, 
mindfulness of breathing. I learn to come back to the breath in the here
and now, again and again and again. With one-pointed awareness of the
breath in the present moment, I practise quieting the mind, cutting off
the incessant internal chatter and, as Carlos Castaneda says, `stopping
the world'. As the practise moves towards a balanced sense of serenity
and tranquillity, I note how investigative energy begins to transform
the very base of conditioned consciousness. Instead of relating to
nature by dividing experience into dichotomous fictions of `self' and
other, as my mind becomes more silent I see the possibility of
experiencing things simply and truly as they are. 

Recollecting family and spiritual companions, teachers and students, and
wishing I could offer them a realization of peace and truth, elevates my
mind and gives it motivation. But over time I witness my mind
oscillating between inspiration and a more humble recognition of how
much there is to do.

Slowly I learn to see these passing emotions as
just more mind states. Could it be any other way? I keep the goals of
the Buddha's path clear in my mind, but the art is learning how to
relate to these goals in a skilful way. The sincere desire for true
freedom from the compulsions of craving is usually the most direct way
to give rise to right effort. Over the weeks and months at Dtao Dam, I
gradually learn how to exert an effort that is wholehearted and
rigorous, while at the same time balanced, measured and at ease with
letting go. Ajahn Chah remarks: 

`The worldly way is to do things for a reason, to get some return, but
in Buddhism we do things without any gaining idea \ldots{} If we don't
want anything at all, what will we get? We don't get anything! Whatever
you get is just a cause for suffering, so we practise not getting
anything \ldots{} This kind of understanding which comes from
[practising Dhamma] leads to surrender, to giving up. Until there is
complete surrender, we persevere, we persist in our contemplation. If
desires or anger and dislike arise in our mind, we aren't indifferent to
them. We don't just leave them out but rather take them and investigate
to see how and from where they arise. We see them clearly and understand
the difficulties which we cause ourselves by believing and following
these moods. This kind of understanding is not found anywhere other than
in our own pure mind.'

\section{The challenge to live in harmony with nature}

In the forest monk life there are various themes which undergird our
practise, to give us a form in which to surrender. One main theme is
that of \emph{nekkhamma}, simplicity and renunciation. The Buddha and
our teachers urge us to eat little, sleep little, talk little and
practise a lot. We're far away from any distraction. The nearest Thai
village is over forty kilometres away, three hours by a difficult
four-wheel drive journey. When it rains the road is easily washed out. 
So here at Dtao Dam the sense of \emph{viveka} -- solitude, quietude
and detachment from the world -- is real. Furthermore, we undertake a
number of \emph{dhutaṅga} practices to cultivate a spirit of simplicity
and renunciation in relation to our four requisites of food, shelter, 
clothing and medicine. We live at the foot of trees on small open-air
bamboo platforms, take just one meal a day in one bowl, wear and sleep
with our robes and get by with a modest supply of communal medicines. 

At the same time, the natural habitat and wildlife make me feel deeply
enmeshed in nature. Biologists and botanists who visit speak with great
enthusiasm about the ecological diversity of the surroundings. The place
is a tropical paradise. Exotic funky-looking palms and ferns abound. 
Ancient hardwood trees tower over dense thickets of bamboo. It takes
seventeen people with arms stretched to form a ring around the base of
the tree where I sit this year. The tree-top provides a home for a
cornucopia of life-forms. There are dozens of bee colonies, whose
beehives at this time of the year drop every so often like grenades from
the branches over a hundred feet above. Ancient ferns sprout out from
the hardwood branches. A family of hornbills makes its nest in the tree
as well. Indeed, looking up at the top I find a whole ecosystem. In
several recent visits to Dtao Dam forest a group of birdwatchers spotted
and catalogued over two hundred bird species, some of which were thought
to be extinct in Thailand. And I can only speculate about how many
\emph{devas} make their homes here. 

Throughout the forest, water flows everywhere. From the cusp of the
mountain ridge to the depth of the valley several hundred metres below, 
the creek cascades into a meandering staircase of waterfalls which
furnish the community with invigorating showers. Small pools at their
bases offer fresh-water baths -- that is, if we can withstand the
curious nibbling on the skin from schools of colourful fish which dart
about. We have agreed not to use soap products in or near the water when
bathing, washing by gently scrubbing with sand or taking a bucket of
water to a spot away from the creek if we use soap. Ajahn Chah, having
spent many years living in the forest, would teach his monks how to live
off the forest in harmony, while keeping the strict precepts of the
Vinaya. He would describe the different trees and plants, indicating
which ones could be used for medicines or allowable food. So although we
have hardly any possessions beyond a few simple requisites, we rarely
feel a sense of lack in such a natural environment. The whole experience
creates an attitude of mind that easily learns to let go and live in
harmony with the surroundings. 

While living in such an environment may sound quite idyllic from a
romantic standpoint, on a practical front a number of difficulties
exist. The hour-long daily climb up the mountain after the meal is
always a hot and sweaty affair. Any water drawn from the creek used for
drinking must be filtered and then boiled to prevent sickness. And
perhaps the most incessant challenge comes from the impressive array of
insects and creepy-crawlies that have to be reckoned with: ticks, biting
horse flies, bees, termites, ants, mosquitoes, spiders, snakes, 
scorpions, rats and centipedes. Bloodsucking leeches, albeit harmless
creatures, can also cause quite a mess of blood. If not bitten by a
leech, my body always manages to get cut, scraped, bloodied and bruised
in some manner. 

It requires constant effort to keep my few possessions dry from the rain
and free from the creatures that ascend the \emph{dtieng}. The nights
are cold and damp. When I awaken in the night on the \emph{dtieng}, I
often have the sense that I am open prey for any large flesh-eating
creature. It is not uncommon in the middle of the night to hear the
footsteps of animals, or even the breathing next to my \emph{dtieng} of
some confused animal such as a bear or wild boar, seemingly trying to
determine what it has bumped into. Invariably these animals smell who I
am and leave me alone. On more rare occasions, monks have come across
deer being attacked by wild dogs or a panther. Other monks have seen
tigers while doing walking meditation. Some locals have even reported
running into wild rhinoceroses. One recent night a monk walked down the
path from his \emph{dtieng} to investigate a curious sound of bamboo
being munched upon, only to find, to his astonishment, a herd of wild
elephants. The earth shook, rumbling as if there had been a small
earthquake, as the elephants fled in surprise. The one creature in the
forest that does offer a serious health hazard is the malarial mosquito. 
This year five members of our community of about twenty have contracted
the malarial parasite from a mosquito bite and have had to be taken out
to a hospital for treatment. Mosquito nets offer some protection, but
the insects can apparently bite at any time of the day, not just dawn
and dusk. 

All in all I experience a renouncing of many comforts and securities I
didn't even realize I had. The tenuous, uncertain nature of the body
really draws me within, in an urgent search for a peace unconditioned by
these external phenomena. If I complicate my daily life by holding on to
any attachments or acting in an unwholesome way, the suffering and
negative \emph{kamma-vipāka} seem almost instant. 

Lastly, the sense of urgency in practise is heightened by the fact that
the forested Dtao Dam area, which has undoubtedly taken centuries to
grow, could be gone or nearly destroyed within the next few years. The
hardwood trees fetch a good price for loggers in the timber industry and
the wild animals are prized by hunters. National Park officials have
been known to burn the forest to get reforestation funds. I can vividly
remember sitting in meditation at the upper \emph{sāla} late one night, 
with a clear vista of the forest for miles, and observing lines of fire
apparently set by arsonists blazing from mountain to mountain. Now
commercial interests want to cut a road through the heart of this
pristine, virgin forest to expedite the transfer of goods from a port in
Burma to Bangkok. First-hand accounts from fellow monks who have done
\emph{tudong} throughout Thailand indicate that forests like this, which
one generation ago covered this country, are now almost non-existent. 
When I think about this it makes me want to practise even harder. 

\section{Cultivating the Sublime Mind}

A second theme penetrating many aspects of our practise is the diligent
development of \emph{sīla, mettā} and wholesome, radiant states of mind. 
For Buddhist monks, the primary precept guiding our relation to the
world is harmlessness. Expressed in a positive way, this means the
cherishing of all life. Interestingly, it also functions as our greatest
protection when living in a wild environment. Giving great importance to
our \emph{sīla} and actively practising the \emph{brahma-vihāras} of
\emph{mettā}, \emph{karuṇā}, \emph{muditā} and \emph{upekkhā} gives us
a skilful method to work with fear. Dealing with the wild animals is not
a matter of bravado or machismo. Asserting a self against nature only
gives rise to conflict. Instead we learn through the cultivation of
\emph{mettā} to emanate a kindness that gives no footing for fear to
arise. Indeed, in the forest we can study how fear and love are like
darkness and light -- the presence of one drives out the other. In the
forest we are instructed not to go out and look for wild animals, nor to
shun them. We just attempt to look at whatever comes our way with
equanimity. When we face wild animals with \emph{mettā} and succeed in
letting go of fear, we can unearth a deep Dhamma treasure buried beneath
the fear. This can open us up to a new taste of freedom and ease. It is
a common theme in the poems written by monks and nuns at the time of the
Buddha: 

\begin{quote}
`I am friend to all, companion to all,\\
sympathetic to all beings,\\
and I develop a heart full of mettā,\\
always delighting in non-harming.'
\quoteref{Thag. 648.%
\footnote{Adapted from K.R. Norman's translation (Pāli Text Society Translation Series No 38, Oxford 1990)}}%
\end{quote}

Holding to the principles of non-violence and harmlessness, a monk trains
not to lift a finger, even in self-defence, to harm another sentient
being. Many of the 227 major training precepts in the Vinaya cultivate a
respect for animal and plant life in the most refined ways. Monks are
prohibited from digging the earth, trimming foliage or uprooting plants. 
When drawing drinking or bathing water from the creek, we must carefully
check the water for any beings visible to the eye. If there is even a
tiny mosquito larva the water cannot be used, or it must be filtered and
any living creatures returned to the water. 

These details of the monk's discipline might appear excessive, but they
create a new attitude of mind in relation to nature. We endeavour to
look upon everything in nature as worthy of care and respect. Rather
than being a source of material for use and consumption, nature is
understood as a process that incorporates our very life. I am aware that
the \emph{dtieng} upon which I meditate and sleep is constructed from
dead bamboo, and when I move away the bamboo rots in the torrential
rains and becomes a natural part of the forest carpet, as it would
anyway. As alms mendicants, Sangha members depend upon lay support for
the building of these simple structures in the first place, so we learn
to relinquish any sense of ownership.

The practise of living in harmony
with nature also extends to the method we use to wash our robes. 
Laypeople offer a piece of heartwood from a jackfruit tree, which the
monks in turn chop into small chips and boil in water, making a
\emph{gaen-kanun} concoction for washing. The \emph{gaen-kanun} has a
marvellous disinfectant and deodorizing effect that lasts for days. If a
robe washed in it becomes sweaty, hanging it in the sunlight gives it a
natural freshness in minutes. In understanding that we are an aspect of
nature, the emphasis is upon living in harmony, attuned to nature's
processes. The fortnightly recitation of the Pāṭimokkha is scheduled
according to the lunar cycle, occurring every new and full moon. By
forgetting the date and month of the worldly calendar and just living
according to the patterns of the sun and the moon, we create a sense of
timelessness. There's no time to practise awakening except the present
moment. 

This shift in attitude generates some positive results. To begin with, 
wild animals respond differently. They can intuitively sense
harmlessness and any accompanying fearlessness. When we encounter wild
animals in the forest, they seem simply to mirror what they sense. Many
people also respect the strict ethical standards of forest monks. Lay
Buddhists consider it auspicious to have forest monks around, and as a
subsequent effect the monastery protects not only the forest in its own
immediate area, but the whole forest around it. Lay Buddhists in Asia
find inspiration if their monks are putting forth a lot of effort in
practise. Although few people make it into Dtao Dam, the ones who do
come from throughout Thailand, and range from businessmen to ecologists
to military historians. 

Another interesting development in Thailand is that a tradition of
ordaining trees has been introduced to protect the forest from logging
-- tying a semblance of the \emph{gaen-kanun}-coloured forest monk's
robe around the trunks of trees. Even the most callous Thai logger will
think twice before killing a tree with a monk's robe tied round it. To
be sure, there are glaring exceptions to this tendency to respect the
trees and the wildlife in and around the forest monastery, but generally
the presence of monks has a strong deterrent effect on deforestation, 
inhibits the hunting of wild animals and engenders an increased
appreciation and love of nature. As we are forest monks, our
teachers encourage us to reflect consciously on the value of the forest, 
and bring to mind that our presence and sincerity of practise are
intended to be a force for preservation. At Dtao Dam Ajahn Jayasāro
spurs us on with an analogy: 

`Living in a forest threatened with extinction is like encountering a
human being on the side of the road, injured or with an illness, 
possibly dying. One doesn't worry about the person's previous behaviour, 
inquire about their nationality or ethnicity, wonder whether the human
being is rich or poor, young or old, famous or not. In any case, human
life is sacrosanct. You attend to the threat on the person's life by
taking them to a hospital or doing everything possible to try to save
them. Similarly, just as human life has intrinsic value, so does the
life of the forest, with the multitudes of life within it. Who knows, 
the forest may contain a rare plant species that leads to a cure for
cancer. Or maybe not -- one way or the other, a forest supports the
lives of countless beings, and if it is subject to destruction, that
merits an immediate and appropriate response towards protection and
preservation.'

\vspace*{2\baselineskip}

\section{The intrinsic power of mind}

A third major theme of the forest Dhamma practise, in many ways the
point of retreating to the forests, is to develop \emph{sati, 
sampajañña} and \emph{samādhi}. Ajahn Ṭhānissaro (abbot of Wat Mettā in
California) once challenged my enthusiasm for practising at a place like
Dtao Dam, perhaps sensing my enchantment with the exoticism of the
retreat setting, by remarking, `It's a good thing as long as it helps
you with your meditation'. This echoes a relevant exhortation by the
Buddha in the Saṃyutta Nikāya of the Pāli Canon:

\vspace*{\baselineskip}

\begin{quote}
`Monks, the establishing of mindfulness is to be practised with the thought, ``I'll watch after myself.'' The establishing of mindfulness is to be practised with the thought, ``I'll watch after others.'' When watching after yourself, you watch after others. When watching after others, you watch after yourself.

`And how do you watch after others when watching after yourself? Through cultivating [the practise], through developing it, through pursuing it. This is how you watch after others when watching after yourself.

`And how do you watch after yourself when watching after others? Through endurance, through harmlessness, through a mind of goodwill, \& through sympathy. This is how you watch after yourself when watching after others.'

\quoteref{S.V.168\footnote{\href{http://www.accesstoinsight.org/tipitaka/sn/sn47/sn47.019.than.html}{http://www.accesstoinsight.org/tipitaka/sn/sn47/sn47.019.than.html}}}
\end{quote}

With the simplicity of the \emph{viveka} environment, all the energies
of the day can be focused resolutely on cultivating the Four Foundations
of Mindfulness. \emph{Sati} functions as a fulcrum for every aspect of
our practise. It is the \emph{sine qua non} of the spiritual life. In
whatever posture one finds oneself, there is the determination to give
rise to the \emph{sati} and \emph{sampajañña} of the situation, knowing
the body as body, feelings as feelings, the mind as mind and dhammas as
just dhammas. We practise to know things just as they are, impermanent
and empty of an inherent self. \emph{Sati} and \emph{sampajañña} both
bolster the strength of the \emph{samādhi} cultivated in the formal
sitting practise, and operate as extensions of it. Ideally, \emph{sati}
and \emph{sampajañña} form a seamless continuum of awareness and
investigation throughout the entire day and night. They are the presence
of mind that is life itself; without them there is heedlessness. When we
are heedless, as the Buddha said, it is as if we are dead. 

Over an extended period of diligent practise, I begin to experience how
the power and knowledge of \emph{sati} and \emph{sampajañña} grow
organically and build upon themselves as they develop. The power and
knowledge intrinsic to these qualities of mind are not derived from
force or coercion. Once in motion, with continual effort, they naturally
deepen. By their inherent nature, the faculties of \emph{sati} and
\emph{sampajañña} are ready for development in all human beings. And
although they can be aimed toward a variety of ends, our task in Dhamma
practise is to use \emph{sati} and \emph{sampajañña} as tools to be
awakened by all things. We're not trying to concoct beautiful theories
and ideologies, but to develop a penetrative clarity in the study of
moment-to-moment experience. We learn from Dhamma, manifest in the
nature around us, through \emph{opanayiko}, turning inwards. We
endeavour to continue to take the practise yet further, looking deeply
at the nature of the mind. Indeed, by getting in touch with the
pristine, natural state of the mind, we let the rigid separation between
inside and outside become deliquescent, giving rise to a more unified
awareness. 

The breath, in constant flow between the inner and outer natures, is an
ideal \emph{samatha} meditation object for myself and many others. As I
learn to let the breath breathe itself naturally, accompanied by a
suffused and unwavering awareness, I try to let the doer of the
meditation process disappear. Only when the doer steps out of the way
and the knower of the breath lets go of any attachment and
identification with the process, does \emph{samādhi} mature. But as soon
as my mind proliferates -- `How can I give impetus to the \emph{samādhi
nimitta}?' or, `Will this lead to \emph{jhāna}?' -- a taint of craving, 
a glimmer of grasping and a subtle sense of lack all obfuscate the
knowing, and my mind becomes distracted. With distraction there is no
possibility to cultivate awareness of phenomena just as they are. With
even a subtle sense of contrivance, luminosity fades. But fortunately, 
there's always the next breath, another mind moment, to begin anew. 

\section{The wisdom to let go}

According to the Buddha's teaching, \emph{sīla} and \emph{samādhi} are
the ground for \emph{paññā} to arise. With a mind brought to
malleability, sharpness, clarity and deep peace through \emph{samādhi}, 
the defilements are at least temporarily held at bay and the
investigative faculty of \emph{paññā} is tractable for making some
decent headway. Again from Ajahn Chah: 

`With right \emph{samādhi}, no matter what level of calm is reached, 
there is awareness. There is \emph{sati} and \emph{sampajañña}. This is
the \emph{samādhi} which can give rise to \emph{paññā}, one cannot get
lost in it \ldots{} Don't think that when you have gained some peace of
mind living here in the quiet forest, that's enough. Don't settle for
just that! Remember that we have come to cultivate and grow the seeds of
\emph{paññā}.'

Reading the Suttas, I am struck by the plethora of passages in which a
monk or a nun, given the teachings by the Buddha, retreats to meditative
solitude at the foot of a tree and in no long time `does what is to be
done'. That is, the monk or nun sees the five \emph{khandhas} as
impermanent and empty, puts an end to \emph{dukkha}, liberates the mind
from the \emph{saṃsāric} treadmill of birth and death and realizes
complete Awakening. It sounds utterly straightforward. Yet in my
practise, without constant heedfulness it can be natural for mindfulness
to lose touch with the present moment and allow the \emph{kilesas} to
slip in.

Memories from the past, coupled with creative imagination about
the future, perpetually enchant and fascinate. Or conversely, my mind
can feel immured by the extremes of languor and impetuousness. When
practising in America I never considered myself to be interested in
sense desires. Now though, living the renunciant life of a monk in the
meditative solitude of the forest, I am more acutely aware of sense
impingement and the lure of sensuality. The ostensible comfort of the
familiar and the secure and the ensuing entanglements of the worldly
life can seem more alluring than ever before. Such is the pathos of
\emph{dukkha}! 

If I do get a bit of sustained success in the practise, I notice that a
subtle sense of pride in living a pure and good life can enter the mind. 
As I attempt to live a simple, selfless life, the sense of self
unwittingly tends to emerge in new and unforeseen forms. It wants to
claim ownership of any goodness and wisdom that might arise. Here Ajahn
Chah continues the encouragement: 

`For the ultimate in the practise of Buddhist Meditation, the Buddha
taught the practise of letting go. Don't carry anything around! Detach! 
If you see goodness, let it go. If you see rightness, let it go. These
words, ``Let go'', do not mean we don't have to practise. They mean that
we have to practise the method of letting go itself. The Buddha taught
us to contemplate all of the dhammas, to develop the Path through
contemplating our own body and heart. The Dhamma isn't anywhere else. 
It's right here, not somewhere far away! It's right here in this very
body and heart of ours.'

On still other occasions I find the sense of self assert itself through
doubt. My mind wonders, `Would it be better to go out and help others? 
What use am I to the world sitting at the foot of this tree? What if I
don't awaken to the unconditioned truth of Nibbāna? Isn't it a bit
presumptuous for me to think that I can realize the ultimate truth?' The
Theravāda forest masters' arousing exhortations in this respect are
echoed in Patrul Rinpoche's classic introduction to Tibetan Buddhism, 
`The Words of My Perfect Teacher': 

`Until you have overcome wanting anything for yourself, it would be
better not to rush into altruistic activities \ldots{} The ancient
[practitioners] had these four goals: Base your mind on the Dhamma, 
base your Dhamma on a humble life, base your humble life on the thought
of death, base your death on a lonely cave. Nowadays we think we can
practise Dhamma alongside our worldly activities, without the need for
bold determination, courage, and difficult practices, all the while
enjoying comfort, well-being and popularity \ldots{} But how could there
be a way to marry Dhamma and worldly life? Those who claim to be doing
so are likely to be leading a good worldly life, but you may be sure
that they are not practising pure Dhamma. To say that you can practise
Dhamma and worldly life at the same time is like saying that you can sew
with a double pointed needle, put fire and water in the same container
or ride two horses (simultaneously) in opposite directions. All these
things are simply impossible. Could any ordinary person ever surpass
Sākyamuni Buddha? Yet even he found no way of practising Dhamma and
worldly life side by side.'

Another reflection that sits powerfully in my mind is the story of a Zen
Buddhist monk who, as I remember the story, was asked, `What would you
do if you were told that you had twenty-four hours to live?' He
responded, `Sit straight zazen \emph{samādhi}' -- that is concentrate
the mind in meditation in the sitting posture. The questioner persisted, 
`What about your vow to liberate all sentient beings?' The monk
retorted, `That is the most direct, complete way to liberate all
sentient beings'. 

Our teachers remind us that the mind absorbed with the bliss of
\emph{samādhi} is far from intoxication. With a pliant and supple mind
primed for the work of investigation from one-pointed concentration, 
it's natural for insight into the tenuous, transient nature of
conditioned existence to deepen. In the forest I notice that when my
mind is in a peaceful state, ordinary discursive consciousness
dissatisfies; thoughts, even astute ones, are of their nature ephemeral
and capricious. To indulge in the thinking mind gives me a feeling akin
to not having taken a shower or brushed my teeth for days. 

When I realize dispassion towards mental proliferation the practise
takes on a greater immediacy for me. \emph{Dukkha} is ubiquitous. It is
imperative to understand the source of the incessant torrent of
suffering, and to tread the path to realize its cessation. Resorting to
belief systems, philosophical explanations of ultimate truth or
supplication to an external being only takes me further away from peace. 
Any approach that is bound up with the five \emph{khandhas} is still
within the realm of \emph{dukkha}. But how do I penetrate or transcend
the conditionality of the five \emph{khandhas} while still operating
from within their realm? I find this to be the heart of the
investigation. 

In his book \emph{Heartwood of the Bodhi Tree}, Ajahn Buddhadāsa
mentions that in the \emph{Majjhima Nikāya} the Buddha is asked to
summarize his teachings in one statement. To this the Buddha responds, 
`\emph{Sabbe dhammā nālaṃ abhinivesaya}' -- nothing whatsoever should
be clung to. Implicit in this utterance is the teaching of \emph{anattā}
-- not-self, voidness of a separate self, emptiness of a soul-entity. So
in the Dhamma practise I attempt to see though the seemingly \emph{a
priori} concepts of `I', `me' and `mine'. Ajahn Chah explains: 

`So we practise not getting anything. Just this is called ``making the
mind empty''. It's empty, but there is still doing. This emptiness is
something people don't usually understand, but those who reach it see
the use of knowing it. It's not the emptiness of not having anything, 
it's emptiness within the things that are here.'

\section{Investigation of birth and death}

In my Dhamma practise, the steady contemplation of birth and death
brings the practise right to the heart. The forest affords numerous
opportunities to confront matters of life and death. Walking along the
forest paths, one frequently comes across a snake eating a frog or
lizard, swallowing the animal whole and head first. The process takes
maybe twenty minutes and the snake, having once overcome his victim, 
rarely stops the process because of a curious onlooker. 

In the investigation of birth and death I make a repeated effort to come
to peace with the fact that my body -- this bag of flesh and bones --
although not likely prey, will certainly die and is not really mine. Its
constituent parts, comprised of the four elements, are constantly
returning to their nature. When I look closely, I can begin to see this
process of birth and death at every moment. We have all known many
people who have died. I find the practise most honest and powerful, 
however, when I am able to weigh the fact of death not just for others, 
but for my own body. 

So contemplation of my death helps brings the essence of the practise
into the present moment. Ajahn Chah would ask newcomers to his
monastery, `Did you come here to die?' (Interestingly, the word `die' is
the same in Thai and English). This kind of vital question catapults us
from the dogmatic slumbers of our everyday existence into awakeness. 
Making an investigation into death with continuity and sincerity
breathes a heedful clarity into our daily life. The Buddha exhorts his
monks to recollect frequently: `Has my practise borne fruit with freedom
or insight, so that at the end of my life I need not feel ashamed when
questioned by my spiritual companions?' By frequently reflecting
on death, I find my understanding of life and death gradually takes on
an earthy honesty. Death can become just like an old friend. Connecting
this with the fact that death spurs urgency, I then ask, `Within just
this, what is it that does not die?'

\section{Everything or nothing?}

So as a community of forest monks, our endeavour at Dtao Dam is to make
progress on the noble path to Nibbāna. Naturally, in this process our
gift to the forest is our very practise of \emph{sīla}, \emph{samādhi}
and \emph{paññā}. Our aim is to tilt the balance of virtuous and evil
forces in the world decisively towards goodness. And if a positive
evolutionary change is to take place in the world in any significant and
fundamental way, from a Buddhist perspective it must grow from an
enlightening shift or awakened transformation in consciousness. Without
such a change, any attempts to heal the world are just band-aid remedies. 
Although perhaps well intentioned and important, these efforts are not
enough. The Buddha has indicated that the human being has the potential
to go beyond the \emph{saṃsāric} realm of \emph{dukkha} altogether, and
settling for anything less would be to sell ourselves short. 

But, to see beyond the clutches of conditioned existence necessarily
entails experiential knowledge of the unconditioned. In other words, to
paraphrase Einstein, the most significant problems we face cannot be
solved by the same level of thinking which created them. Similarly, in
the practise of Dhamma we see that \emph{dukkha} is a problem which
cannot be extinguished by the same mindset that fabricates it in the
first place. Hence the imperative to develop the path to go beyond our
conditioned perspective. And the Buddha and the \emph{arahants} show us
that the goal of realizing the unconditioned is achievable in this life, 
and worth any sacrifice. 

The teachings of the Buddha also indicate that the effects of the
profound transformation of Awakening reverberate far and wide throughout
the web of life, although perhaps in ways imperceptible to the
unenlightened eye. We're more interconnected than we think. Systems
theorists make a parallel point when they contend that a seemingly small
input at the beginning of a process can have huge ramifications in the
big picture. And quantum physicists concur that the effect of something
as small as one electron making contact with another electron might not
bear any fruition until over a thousand years later. From a Buddhist
perspective, it's the accumulation of the moment-to-moment efforts to
practise \emph{sīla, samādhi} and \emph{paññā} and tread the path to
Nibbāna over the long haul that count. 

So given that every small action can carry significant \emph{kammic}
weight, before we (as monks) presume to know what is the best way to
help others, and before we become too engaged in resolving worldly
matters, we need to be solid in our realization of Dhamma. Again, we can
draw a parallel with contemporary science. David Bohm, the eminent
quantum physicist, also holds that the process of changing the world
occurs first through transforming the mind:

\begin{quote}
`A change of meaning [within the human mind] is necessary
to change this world politically, economically and socially. But that
change must begin with the individual; it must change for him \ldots{}
if meaning is a key part of reality, then, once society, the
individual and relationships are seen to mean something different a
fundamental change has taken place.'\footnote{\emph{Unfolding Meaning: A
Weekend of Dialogue with David Bohm}}
\end{quote}

In any case, for the meditator who has success in the practise, all
these explanations are superfluous. From reading the Suttas and hearing
the words of contemporary masters, we see that the enlightened, 
liberated mind validates itself and inherently knows what is the best
thing to do in order truly to help others. Even the teachings and the
practise become just the raft to the other shore. They are the finger
pointing at the moon, not Awakening itself. As Ajahn Chah indicates: 

`The Buddha laid down \emph{sīla}, \emph{samādhi} and \emph{paññā} as
the Path to peace, the way to enlightenment. But in truth these things
are not the essence of Buddhism. They are merely the Path \ldots{} The
essence of Buddhism is peace and that peace arises from truly knowing
the nature of things \ldots{} Regardless of time and place, the whole
practise of Dhamma comes to completion at the place where there is
nothing. It's a place of surrender, of emptiness, of laying down the
burden \ldots{}'

So here at Dtao Dam, through various aspects of the practise of Forest
Dhamma, I have come to find that many of the Buddha's discourses in the
Suttas which once seemed recondite or beyond me are now clearer and easy
to apply to my life. Pāli words such as \emph{nekkhamma}, \emph{viveka, 
sīla, mettā, sati, sampajañña, samādhi} and \emph{paññā} become the
vernacular of everyday situations. Far from being ossified or esoteric
teachings from two and a half millennia ago, the Suttas become living
teachings. They are urgent reminders and vivid pointers to the way it is
with the body and mind right here and now. I ask myself, can my practise
with heartfelt dedication make progress towards the goal of awakening, 
dropping the burden of \emph{dukkha} and at last transcending the
vicious cycle of birth and death? 

My mind returns once more to the time of the cat encounter. It is
forty-eight hours later, and I have just spent another pleasant morning
of meditation at my \emph{dtieng}. I pick up my robe and carry-bag and
set out for the morning descent down the mountain for the meal. Having
taken only a few steps on the path, I suddenly see a large black animal
up ahead, but I am not fully sure what it is. It has apparently just
stepped off the path about twenty metres ahead of me and has hunched
itself right behind a large cluster of brush. By sight I can't make out
the type of animal for certain from its general physical form, but I am
instantly reminded of the large black panther (three and a half metres
in length!) sighted several times wandering around the Dtao Dam forest
in recent weeks. (Several nights earlier one of the novices had been
circled by the black cat while walking the thirty-minute walk from the
upper to the inner \emph{sāla} late at night with only his candle
lantern. The workers at the mine also saw it one morning recently). In
any case, the animal is far too large to walk up to on the path and it
seems to have its gaze fixed dead centre on me. 

Spontaneously, and perhaps with a bit of over-confidence and
fearlessness from my previous cat encounter, I clear my throat as if to
signal an indication of my presence and wish to walk by. The animal does
not budge. Then I feel a wave of energy coming from this impressive cat
sweep over me. It has a very strong and fierce feeling-tone. If I had to
put it into words, it would be something like: `Who in the world do you
think you are to be telling me to get off this path? I am in charge
here!'

This feeling that I get from the stealthy animal on the path is unlike
anything I have ever experienced in the forest in daylight hours. To be
sure, at night on countless previous occasions I have heard the
footsteps of a large animal padding though the brush, gently crushing
the bamboo leaves as it moves. On those occasions fear sometimes would
arise, but I have never actually seen any animal that I recognized as a
tiger. This morning, however, I am looking at a giant black cat. There
is uncertainty and I feel a profound presence of death. I collect my
\emph{sati} and slowly take a couple of steps backwards without turning
round. My hands instinctively go into \emph{añjalī}, palms joined in
reverential salutation, and I start to gently chant the \emph{Mettā}
\emph{Sutta}, the Buddha's words of loving-kindness which I had chanted
moments before at my \emph{dtieng}. I close my eyes and tap into the
energy I had felt so strongly that morning. And with the mind imbued
with \emph{mettā}, I practise letting go. 

After finishing the chant about three minutes later, I open my eyes and
can no longer make out any large black animal in the brush. Keeping my
hands in \emph{añjalī}, I chant homage to the Buddha (\emph{Namo tassa
bhagavato arahato sammā-sambuddhassa}) and proceed down the path. 
Whatever was previously there has silently walked away. While walking
down the mountain that morning, my mind naturally turns to the
contemplation of death and the empty and impersonal nature of the five
\emph{khandhas}. The energy from the encounter gives me a definite
penetrative push. During these moments I feel a deep and meaningful
trust in the Buddha and the path of Dhamma. To the extent that I have
been able to devote my life to the cultivation of the path of
\emph{sīla, samādhi} and \emph{paññā}, my heart feels good and true. In
the face of what appeared to me as death, or at least a purveyor of it, 
the question of what really matters hit home. I experience an
appreciation for my life that resonates deeply in my bones. 

Yet while contemplating the empty nature of the five \emph{khandhas} and
feeling immeasurable gratitude for the Buddha's teaching, there is a
wonderful and ineffable mystery to it all, a kind of `Don't know' mind. 
The experience is light and peaceful. Although my heart feels full of
\emph{mettā} and connected to the beauty of everything around it, 
\emph{paññā} insists that everything is completely empty, with nothing
really there at all. And for once I don't experience the pressing need
to reconcile this seeming contradiction. How to realize these two
insights as not separate but one is the practise of Forest Dhamma that I
continue to learn. 

\clearpage

\section{The Author}
\label{pannyavuddho-desc}

Tan Paññāvuddho carried on his training in Wat Pah Nanachat and
branch monasteries in Thailand before joining the community at Buddha
Bodhivana Monastery, north-east of Melbourne, Australia. Having returned
to Thailand after his sixth Rains Retreat as a monk, he spent many
months in solitude in a cave in the Pak Chong district of Korat
province. In 2005, the year of his seventh Rains Retreat, Tan
Paññāvuddho travelled to Wat Boonyawad in Chonburi, where he entered the
Rains Retreat practising under Tan Ajahn Dtun Thiracitto's guidance. A
few days into the Rains Retreat he failed to come out for the alms-round
or the meal, which was unusual for him. A check in his \emph{kuṭī} led to the
very sad news, which spread through Thailand and very quickly all round
the world, that this much-loved Sangha member had died in a simple
accident in his bathroom. Subject for many years to fatigue caused by
low blood pressure and dehydration, he had fainted early in the morning. 
His forehead had struck a corner post as he fell, forcing his head
sharply back with tremendous force and breaking his neck. His death, the
doctors concluded, would have been instantaneous and painless. His
cremation, carried out at Wat Pah Nanachat, was attended by monks and
laypeople from all over Thailand, Ajahn Kalyāṇo, the abbot of Buddha
Bodhivana Monastery, Melbourne, and Tan Paññāvuddho's brother from the
US.

As a Sangha member Tan Paññāvuddho was known for his keen
intellect, sensitivity and kindness in community situations, and most of
all for his absolute dedication to training in the way of the Buddhist
monk. This publication, `Forest Path', is one example of his creative
skills: as well as writing this chapter, he was one of the main
collaborators on the whole project. His love of the Thai forest
tradition was very deep, despite his earlier background in the Zen
school and a strong appreciation of Mahāyāna and Vajrayāna teachings. 
When he died it was a certain source of comfort that he was truly
practising in the way he loved, and living with a teacher, Tan Ajahn
Dtun, for whom he had much devotion and respect.



\setChapterAuthor{Ajahn Piak}
\setChapterNote{An interview with a senior disciple of Ajahn Chah conducted by\linebreak Ajahn Chandako.}
\chapter{What it Takes to Reach the Goal}
\markright{\chapterAuthor}
% Title: What it takes to reach the goal
% Author: Ajahn Piak

During the first few years of his monastic career, a young monk's
training is divided between Wat Pah Nanachat and other branch
monasteries of Wat Pah Pong. One of the disciples of Ajahn Chah who has
helped to train Wat Pah Nanachat monks is Tan Ajahn Piak, abbot of Wat
Pah Cittabhāvanā, a branch monastery situated to the north of Bangkok. 
The following conversation with Tan Chandako took place in 1998. 

\emph{Tan Ajahn Piak}: The \emph{Kruba Ajahns} rarely say anything
directly about Nibbāna because it is beyond a normal person's realm of
possible experience. Even if the people listening believe the
explanation, it still doesn't actually help them much, and if they don't
believe it they may make a lot of bad \emph{kamma} for themselves. So
the \emph{Kruba Ajahns} usually refer to it using metaphors or refuse
to speak of it at all, only teaching the path to get there. 

The important thing is to keep going straight without stopping. For
example, say you want to go to Fa Kram Village over there; if you follow
the path and keep walking you'll get there in a short time. If you stop
to take a look at something and then chat with people, then go off with
them to see something else, it will take a long time before you reach Fa
Kram, if ever. But the reality is that almost everybody gets sidetracked
or at least stuck in \emph{samādhi}, thinking that they've arrived
already. Even Luang Por Chah was stuck for a while; Tan Ajahn Mahā Boowa
for six years; Ajahn Tate for ten years; Ajahn Sot (Wat Pak Nam) for
twenty years. 

\emph{Tan Chandako}: Because to all intents and purposes it appears to
be full enlightenment? 

\emph{Tan Ajahn Piak}: Yes. There seem to be no \emph{kilesas}
whatsoever. Everything is clear. Many people don't make it past this
stage. Other people practise for five Rains Retreats, ten Rains
Retreats, and still feel they haven't made much progress and get
discouraged. But one has to keep in mind that it is always only a very
few people who have the \emph{pāramī} to reach the goal. Compare it with
the US President or the Thai King. Out of an entire nation of millions
of people, only one person at a time has the \emph{pāramī} to be in the
top position. You have to think in terms of what you are going to do to
set yourself above the crowd, creating the causes and conditions for
future liberation. Effort in the practice is what makes the difference. 
There are thousands of monks in Thailand who ordain with the sincere
intention of realizing Nibbāna. What sets people apart, why some succeed
while others don't, is mainly due to their level of effort, as well as
the effort they've put forth in the past. A person has to train himself
to the point where it becomes an ingrained character trait to be
continuously putting forth effort, whether he's around other people or
alone. Some people are very diligent as long as there is a teacher or
other monks watching, but as soon as they're alone their effort
slackens. 

When I was a young monk and my body was strong, I'd stay up later than
everyone else walking \emph{jongrom} and see the candles in the other
\emph{kuṭīs} go out one by one. Then I'd get up before the others and
watch the candles gradually being lit. It wasn't that I had it easy. The
\emph{kilesas} in my heart were always trying to convince me to take a
rest: `Everyone else has crashed out. Why shouldn't you do the same?'
The two voices in my head would argue: `You're tired. You need a rest. 
You're too sleepy to practise.' `What are you going to do to overcome
sleepiness? Keep going.' Sometimes the \emph{kilesas} would win, but
then I'd start again and eventually they weakened. 

\emph{Tan Chandako}: It's often when \emph{samādhi} or \emph{vipassanā}
has been going well that \emph{kilesas} seem to arise the most. At such
times it seems I've got more \emph{kilesas} than ever. Is that normal? 

\emph{Tan Ajahn Piak}: Very normal. The average person has a huge amount
of \emph{kilesas}. Just to recognize that one has a lot of
\emph{kilesas} is already a big step. Even the \emph{sotāpanna} has many
\emph{kilesas} to become free from, much work to be done. Even at that
stage it's not as if everything is \emph{sabai}. It's as if there is a
vast reservoir of \emph{kilesas} below us which gradually come to the
surface, and it's not easy to know how much is remaining. Just when you
think you've fully gone beyond a particular \emph{kilesa}, it will arise
again. This happens over and over. The only thing to do is to keep using
\emph{paññā} to keep pace with the \emph{kilesas}, meet and let go of
them as they arise in the present. 

\emph{Tan Chandako}: Have you ever met or heard of anyone who has
attained \emph{magga-phala} by only contemplating and not practising
\emph{samādhi}? 

\emph{Tan Ajahn Piak}: No, if you want a straight answer. \emph{Samādhi}
is essential for the mind to have enough power to cut thoroughly through
the \emph{kilesas}. However, if one is practising \emph{vipassanā} with
the understanding and intention that it will lead to the development of
\emph{samādhi} at a later stage, this is a valid way to go about it. 

The character of almost all meditation monks, both Thais and those born
in Western countries, is such that they need to use a lot of
\emph{paññā} right from the very beginning in order to gradually make
their minds peaceful enough to be able to develop \emph{samādhi}. Only a
very small percentage of Thais, and possibly no Westerners, are the type
to develop \emph{samādhi} fully before beginning \emph{vipassanā}. 

\emph{Tan Chandako}: Can it be said how deep and strong \emph{samādhi}
must be in order to attain \emph{magga-phala}? 

\emph{Tan Ajahn Piak}: It must be strong enough to be still and unified
as one, without any thinking whatsoever. There will still be awareness
-- knowing what one is experiencing. 

\emph{Tan Chandako}: According to whether one is in a remote location or
in a busy monastery, should one's Dhamma practice change or remain the
same? 

\emph{Tan Ajahn Piak}: Dhamma practice takes on a different character if
you are in the city or are busy with duties in a monastery. In the
forest there are few external distractions and it is easy to make the
mind peaceful. If you have many sense contacts and dealings with other
people, it is essential to figure out how not to pick up other people's
emotional vibes (\emph{arom}). Otherwise what happens is that the people
around us feel lighter, while we feel heavier and heavier. It's
necessary to be able to completely drop mental engagement as soon as
interactions with other people have finished. Otherwise all the
conversations and emotions of the day are floating around in the
\emph{citta} when one goes to sit in meditation. 

It's easy to say, `Just be mindful' and `Don't pick up other people's
baggage', but it is very difficult to do. Luang Por Chah could take on
the problems and sufferings of others without picking up any of them
himself, because his \emph{citta} was very strong. The people around him
didn't know what was happening. They just knew that they felt cool and
happy around Luang Por. But this is not a practice for beginners. Most
people just get burned out.

Practising in the forest is easier, and I
recommend that you should try as much as possible not to get involved
with too many responsibilities, especially being an abbot. If someone
tries to tell you that you are selfish and should be helping others, 
reflect that this is due in large part to the conditioning from Western
society. If the Buddha had thought that way, we never would have had a
Buddha. In order to put your mind at rest, reflect on the goodness
you've done and rejoice in the \emph{pāramī} that you're creating. Those
who try to help others too much before they've helped themselves will
never be able to teach or help beyond the superficial. If their
teachings mislead others due to their own ignorance, they can make a lot
of negative \emph{kamma}. Many of the Wat Pah Pong monks try to emulate
Luang Por in his later years, when he would talk with people all day, 
rather than his early years of difficult practice. But it was precisely
those years in the forest that made Luang Por into the great teacher
that he was. 

\emph{Tan Chandako}: Have you ever heard of anyone attaining
\emph{magga-phala} by any means other than analyzing the body into its
component parts and elements? 

\emph{Tan Ajahn Piak}: No. At the very least, when the \emph{citta} is
clearly known as \emph{anattā}, the knowing mind will return to knowing
the body thoroughly as \emph{anattā} as well. 

\emph{Tan Chandako}: In one of Luang Por Chah's Dhamma talks he says
that even for \emph{arahants} there are still \emph{kilesas}, but like a
bead of water rolling off a lotus petal: nothing sticks. How do you
understand this? 

\emph{Tan Ajahn Piak}: Luang Por liked to use language in unconventional
ways in order to get people's attention and make them think. What he was
referring to was the body -- the result of previous \emph{kamma} --
but the \emph{citta} was completely devoid of \emph{kilesas}. Normally
people use other terms to refer to the body and the physical
\emph{dukkha} of an \emph{arahant}, but Luang Por was quite creative in
his use of the convention of language. 

\emph{Tan Chandako}: I've heard that while still a student, before you'd
met Luang Por Chah, you had a vision of him. 

\emph{Tan Ajahn Piak}: That's right. I'd intended to return, [to New
York, to finish a master's degree in business management] but soon
after I'd begun to meditate I had a clear vision of a monk whom I didn't
recognize, chewing betel nut. I went to see many of the famous
\emph{Kruba Ajahns} at that time -- Luang Por Fun, Luang Por Waen --
but when I met Luang Por Chah I recognized him from the vision and
figured that he would be my teacher. 

When I began to consider ordaining instead of completing my studies, my
family tried hard to dissuade me, but I found meditation so peaceful
that everything else felt like \emph{dukkha}. 

\dividerRule

\section{The Authors}

Tan Ajahn Piak still lives in his monastery to the north of
Bangkok. Any fields surrounding it are long gone and now the Bangkok
suburban sprawl has engulfed Wat Pah Cittabhāvanā. The 2011 flooding saw
the monastery submerged under a couple of metres of water. However, 
Ajahn Piak still provides a refuge for those seeking the Buddha's path. 
His reputation as a meditation teacher has grown, and his emphasis on
combining the cultivation of \emph{samādhi} with staying up all night brings
many people to practise under him. Despite poor health he has begun
travelling and teaching abroad in recent years, most notably in
Malaysia, Singapore, Australia and New Zealand.

Tan Chandako carried on training in Thailand under various
teachers, and also spent periods of time in Perth, living in Bodhiñāṇa
Monastery. He spent a year in Wat Pah Nanachat as Vice-Abbot in 2002, 
before seeking a place to settle down. A Rains Retreat in the Czech
Republic led to his return to Australia and finally to Auckland, New
Zealand, where in 2004 he was invited by the ABTA (Auckland Theravāda 
Buddhist Association) to establish a monastic residence on their
recently-acquired property not too far from the city. Thus Vimutti
Monastery was born, and an extensive programme of tree-planting and
construction has been under way since then. Additional land has been
purchased to provide something of a buffer zone.

As well as his responsibility for running the monastery, Ajahn Chandako
provides regular teaching and retreats both at the monastery and in
various other parts of New Zealand. Every year he comes to Thailand and
visits his home in the US, where he also conducts retreats. 



\setChapterAuthor{Tan Acalo}
\setChapterNote{A monk visits his family after three years at Wat Pah Nanachat.}
\chapter{Acceptance, Forgiveness and Deep Blue Empathy\ldots{} Going Home}
\markright{\chapterAuthor}
% Title: Acceptance and Forgiveness
% Author: Ajahn Achalo

It had been three years since I'd spent any time in the country of my
birth, when I unexpectedly had the opportunity to return to Australia
with one of my teachers. Tan Ajahn Anan had been invited to Melbourne,
to visit a newly established meditation hermitage and give teachings at
the local Buddhist society. Several other monks were going and we would
be passing through the cities of Sydney and Canberra, staying in Thai
wats and then later going on to Melbourne. In Melbourne I would take
leave of my teacher and travel to Queensland to spend time with my
mother and father. As a monk, one tries to practise in all situations.
In Australia I would be close to the members of my family. I would have
to honour our own relationship and also their relationships with others.
I would have to be considerate of their lifestyles and views, yet at the
same time maintain my own loyalties. This article explores some of the
challenges, along with what to me were some of the more significant and
moving times during my visit.

An earlier incident is a good way to introduce Tan Ajahn Anan. It was
the middle of my second Rains Retreat, and I was staying for the first
time at a Wat Pah Pong branch monastery where everyone except myself and
another English-speaking monk was Thai. During one evening meditation
session I was concerned about myself. I was stuck in a negative mood
that just wouldn't move. All the other monks appeared so sweet and kind,
and I was sitting there being angry, thinking about what was wrong with
everyone and everything: `Maybe everyone else's moods arise and pass
away, but maybe mine won't! Maybe I'm just too defiled to be a monk.' I
faintly knew all these thoughts were silly, yet somehow I couldn't stop
them and it was very uncomfortable. The Rains Retreat can be a tense or
difficult period, and most monks will experience some kind of negativity
at some time or other during those three months. All this I knew, but
this particular time I couldn't help making a big deal out my negative
mood.

After the evening sitting the bell was rung routinely and we chanted in
Pali and Thai, sharing merit and taking refuge, bowing our respects to
the Buddha and our teacher. As the monks left the room I lingered a
little, wanting to be alone. When I left a few minutes later, passing
through the rear door, I noticed that Tan Ajahn Anan was seated alone on
a wooden dais in the corner of the balcony surrounding the hall.
Intuitively I lowered to my knees and respectfully approached. With a
characteristic firmness that was also caring, he looked at me, raising
his chin in acknowledgement, and asked, as if he had been waiting,
`\emph{Achalo, tam-mai mai sa-ngop ley?}' (Why are you not peaceful at
all?) Defeated, I couldn't actually articulate anything, but I knew he
was not angry at me; he was not demanding that I answer. After what
seemed like a very long but exquisitely empathic moment, he said to me
in carefully practised English, `You think a lot about Australia', a
statement and a question to which I replied, `Yes it's true.' Then he
continued, `When you think about the future you throw your mindfulness
in the dirt.' Another pause. `When mindfulness is not strong you cannot
let go of \emph{arom}.' Then we talked a little more about how I'd been
feeling and what to do about it. The Ajahn explained that the
mindfulness of a good meditator is usually a nice clean white colour,
while the mindfulness of an enlightened being is radiantly clear and
impenetrable like a diamond. But when a person is lost in some kind of
delusion, mindfulness degenerates to the colour of dirt or mud. After
allowing me some time to describe my difficulties of late, this kind and
wise teacher assured me that it was ordinary for young monks to
experience such things. He then encouraged me very gently to keep to
myself for a few days, try to eat lightly and refrain as much as
possible from thinking about the future, letting mindfulness
re-establish itself.

It was true I had been obsessing about the future, a transgression in
the Buddhist sense if ever there was one, particularly for someone who
has made a firm resolve to cultivate awareness in the present. But I'd
been caught off guard when just a month before I'd found that there was
an opportunity to go to Australia and that my family was eager to help
with the ticket. I had previously assumed that it would be a few more
years before I returned to Australia. As a young monk still learning the
ropes, I would ideally choose to stay within the most clear and
supportive of contexts. However, the opportunity to return to my country
with one of my teachers seemed a blessing too providential to refuse.
Now that there was an upcoming travel date to fuel proliferation, I had
got lost in remembering, fantasizing and planning. Certainly there were
things to think about. I would have to prepare my mind for the change of
locations and consider what rules of discipline to brush up on, so that
I could maintain my practices while travelling. I would also have to
consider my relationship with the members of my family. How should I
relate to them? Over the years I'd been fairly diligent in maintaining
correspondence with my family. Basically they had been supportive of my
choice to be a monk. My mother gave me her blessing and encouragement
before I'd even asked for it. I recollected with affection the words she
spoke to me before I last left Australia: `You seem very happy and the
monastery sounds like a safe place. I always knew you'd find what you
needed to do with your life. I never knew what you should do, that's why
I used to worry about you so much! But I knew you'd find your niche. If
you want to become a monk you have my blessing.' Indeed I had the
blessing of my parents, for they were happy that I seemed well, but at
the same time I sensed an absence of true empathic appreciation. They
appeared to be unconcerned about spiritual matters, and even though they
were supportive, there was always, perhaps reasonably enough, the
lament: `We're glad that you're happy, but we miss you and it's sad that
you have to live so far away.'

At Wat Pah Nanachat a few days before our departure I was meditating
alone in the main \emph{sala}. I had been busy getting things together
for my trip, and had decided to take a few quiet moments to collect my
mind. When I opened my eyes, a layman approached and said that he wanted
to offer me the Buddha statue placed by his side. Saying that the Abbot
would be along in ten minutes, I suggested that he wait and offer it to
him. He went away for a minute, but then came back and said that he
wished to offer it to me, so this time I received it happily.
Ironically, this small figure was in the same standing posture as the
large Buddha statue in the ceremonial hall where I'd been ordained. I
remembered the morning of my ordination in pre-dawn darkness, looking up
at an ominous black figure standing with both hands held in the posture
named `holding back the waters'. His arms and hands were straight and
taut, with palms exposed in a gesture that seemed to be making a firm
command. At that time I had seen the gesture as an emanation of
assertive compassion, compelling me to see the importance and urgency of
my opportunities. The figure I held in my hands was a more androgynous
and serene one. With both palms held out, he seemed to be saying
something more along the lines of, `Truly I come in absolute peace'.
When I showed it to Tan Ajahn Jayasāro, he suggested that I let my
parents be its new owners. As I looked at the statue's expression, I had
a feeling that my mother would adore it but I wasn't quite sure. It was
definitely a religious object, and, as my parents do not profess any
faith they might in fact find it objectionable. But I decided to trust
my feeling and take it anyway.

On the day of our departure in Bangkok, at the airport moving towards
the check-in counter, a robust short-haired woman was vigorously
employing her feet to kick and push her backpack along the ground, while
heaving another large bag in her arms. I was standing in the queue
alongside her with my teacher and three other senior monks. They were
quiet and composed in manner. The young woman then began arguing with
the delicately mannered Thai assistant behind the counter, in an English
that screamed `Australia!' `This is the bag I want to take onto the
plane.' `I'm sorry, but it's too big', came the quiet and polite reply.
The passenger continued, obstinate and confrontational, `But they let me
take it last time! They did! This same bag! They let me take it last
time!' After several more imploring but quiet pleas, the check-in
assistant gave in and allowed her to keep the large bag. As a rule Thai
people abhor public conflict. At once I felt deeply embarrassed and a
terrible sense of trepidation about our journey. The incident reminded
me of some of the defects in the character of my countrymen. As I looked
to my right I could see a row of monks who had come to bid farewell to
their teacher. Their robes were immaculate, their countenance serene;
they were sitting together in a neat row, talking quietly among
themselves. The contrast created an uneasy tension. I shouldn't have
taken it so personally. I should just have accepted things as they were,
and taken solace in the equanimity and wisdom of the other monks and my
teacher. But as the only Australian in our group, I felt somehow
responsible. We would be flying economy class on Qantas, the Australian
national carrier, and I couldn't help worrying that the plane would be
full of poorly mannered drunken people who would be rude to my fellow
monks. The ground staff were organising a first-class classification for
our luggage and a security escort onto the plane, and I was acutely
aware then of the grace, humility and kindness of so many of the Thais.

We were the last passengers to enter the plane, on stopover from London.
As we were led through the aisle to our seats, I was struck by the sheer
number of big-bodied white people. At the same time I was relieved to
see that for the most part they appeared personable and harmless enough.
As my anxiety level lowered, I forgivingly remembered that for the most
part that is the way Australians are. After sitting down I noticed this
was a fascinating realm, caught in limbo, but which in many respects
could already be considered part of Australia. The air was thick with a
familiar slouching accent and hearty laughter. The airline staff walked
swiftly, confidently and heavily up and down the aisles, stopping
occasionally to receive loud orders or deliver food and drink with an
equally enthusiastic command to enjoy. I must admit I really did enjoy
the spectacle, and was surprised by the amount of affection I felt for
everyone inside the plane.

In the aircraft my mind at last felt clear and it was easier to feel
prepared for the impending change of situation. After four or so hours
in the air, the Thai monk next to me said that he would really like to
meet the captain and have a look at the cockpit. He'd seen pictures
before but never the real thing. Seeing the eager look on his face, I
knew I would have to ask one of the cabin crew. `Why not?' I thought,
`they might just say yes', and so, feeling stupid, I asked. Twenty
minutes later there were three Buddhist monks in the cockpit and the one
by my side was positively delighted. Rolling 180 degrees before us were
scores of magical billowy cumulus clouds, illuminated by the clear blue
sky of mid-afternoon. Eager to know how the pilots were trained, the
senior monk asked many questions and we were both impressed by their
impeccable discipline. The monks drew parallels between the discipline
of the pilots and our own training. Much to my surprise, the co-pilot
then confessed that he thought he was a Buddhist. Our visit to the
cockpit was timely, as the captain announced that in a few minutes we
would come to the north-western corner of the Australian continent.
Feeling sad for the other two monks, still on the lower level, I asked
if it would be possible to send for them. My friends departed,
satisfied, and a few minutes later my teacher and another monk were with
us in the cockpit. As we approached land I was surprised by an
unfamiliar burst of patriotism, as if this really was my country.
Through the many windows the view was fantastically clear. A delicate
white strip of sand gracefully traced the coastline of the semi-arid
ochre-coloured earth, defining it as separate from the huge expanse of
ocean directly below. Before us the continent seemed endless, and once
over land I was emotional for a different reason. Years ago I had left
this country seeking a lifestyle with purpose and an authentic spiritual
training. Now I had returned physically, and as I reflected that one of
my much-revered teachers was sitting just a little to my left, I felt
exultant at such good fortune.

On arriving in Sydney, I was met by my younger by eighteen months
brother, and it was weird to look into a face so similar to mine. He was
easy in manner and happy to see me, and I felt the same. Over the
following few days I had the opportunity to spend some time with both my
younger brother and elder sister, who came to visit me at the Thai
temple and also took me on a picnic. The talk was mostly about jobs and
careers, politics and current affairs. We all felt affection and joy at
being able to be with one another after such a long time. Upon returning
to the wat however, I felt a little scattered, and was grateful to
participate in the evening chanting and meditation with the resident
monks and a few local Thais. Throughout the entire trip the visiting
monks were surrounded by the warmth and courtesy of faithful Thai
Australians, and were taken on many sight-seeing tours by their eager
hosts. In Sydney my Dhamma brothers were as impressed by the clean and
free-flowing roadways and the large suburban houses as they were by the
Opera House,and the deep harbour and its famous old bridge. Although
there were no particular teaching engagements in Sydney, the Thai
community truly relished the presence of Tan Ajahn Anan. Some asked
questions about meditation practice or inquired about his monastery and
our lineage, and several people subsequently made plans to visit and
spend some time practising back in Thailand.

From Sydney we drove to Canberra and all, including myself, were
awestruck at the vast expanses of pasture lands so sparsely inhabited.
Heavy spring showers had turned the land a gorgeous rich green, with
large patches of lilac wild flowers. Staying in a Thai monastery once
again, we were pleased to see that the Wat in Canberra was very well
supported. Each evening a good number of laypeople came to chant,
meditate and listen to either a taped Dhamma talk or a teaching by one
of the Ajahns. Ajahn Anan gave a teaching on the second evening.

Having been away from Australia for some years, and then returning
surrounded by Thais, I felt I could contemplate aspects of the country
of my birth with a greater degree of detachment, asking myself such
questions as what nationality actually meant. What did it mean to be
Australian and what were the core values of this culture? I wasn't
looking for answers so much as stimulating reflection. In Canberra, the
nation's capital, the War Museum and Parliament House were particularly
thought-provoking. In Parliament House, walking through vast rooms and
corridors, I wondering whether I was really close to the heart of
Australia, in this place that had cost over a billion (Australian)
dollars to build. I was certainly grateful to such an abundant and
well-organized society, and all the opportunities it had made available
to me. But somehow I felt unmoved. One notable comment by the other
English-speaking monk seemed to capture something of my sentiments: `The
design seems to be full of complicated patterns, shapes and angles, but
it doesn't seem to be going anywhere or pointing to anything. Like
cleverness just for the sake of cleverness.' Admittedly, sitting in the
Senate Room and House of Representatives with my teacher was certainly
fun, and there is one particular image that I will always remember.
There was a curving corridor; large panes of plate glass looked over a
perfectly tended garden lawn. Above the mirror-like granite floor hung
an impressively lofty roof, emanating soft, even golden light. A huge
hand-woven woollen rug with a bold geometric design of black with reds
and yellows lay beneath, between two large dark leather sofas. On one of
these large sofas I and a Thai layman sat, and sitting alone on the
other was my teacher. Amid such contrived beauty, cleanliness and order,
the look on my teacher's face was probably that of equanimity, yet that
word somehow doesn't capture it -- cool, relaxed, detached and unmoved.,
But the air around him was not a vacuum. He seemed in fact to be making
a pronouncement, saying something about where to place one's efforts and
attention in order to know what is most worthy of reverence.

In Melbourne we were received by members of the Buddhist Society of
Victoria. It was an inspiring sight to see a room full of practising
Buddhists of many different ethnic origins. Here in the evenings Tan
Ajahn gave several Dhamma teachings followed by questions and answers,
all of which were translated. Many practitioners were keen to ask
specific questions about their meditation practice, and Tan Ajahn was
pleased with their energy and willingness to learn. People stayed for
hours listening to the answers to questions and all the evenings ended
quite late. It was a pleasure to be practising the dhammas of listening
and meditating among sincere Australians. As most of my Dhamma
experience is associated with Thailand, these occasions helped me to
feel more the global relevance of the Dhamma, and less isolated from
where I had begun my life. Tan Ajahn instructed me that I must practise
very hard to develop my skills well, so that I could come back and truly
help these people. Although I was flattered by his faith in me, I was
more than a little daunted by the prospect.

After four days in Melbourne it was time for me to separate from my
teacher. He and the others would stay on there for a while and then fly
to our branch monastery in Perth. I would go alone to Sydney, travelling
to the Blue Mountains to spend a few days with my eldest sister and her
children. From there I would go back to Sydney to meet one of the monks
and we would fly to Brisbane together. Having a friend who was a monk as
my companion would be a valuable support, and I felt lucky that he had
been willing to meet up with me again. After performing the ceremony for
asking forgiveness, I asked Tan Ajahn Anan if he had any advice. He
simply answered that I should take care of my heart and mind. We were
drinking afternoon tea when a kind Sri Lankan doctor arrived to pick me
up. As I drank the last few mouthfuls of tea, I became aware that I was
suddenly feeling very vulnerable. Moments later I noticed a distinct
lightness in my heart. The feeling grew, my face smiled broadly as if of
its own will and I was left feeling supported and safe. I realized that
my teacher was radiating loving-kindness toward me, a highly developed
ability for which Tan Ajahn Anan is renowned.

Away from my teacher and the Thais who surrounded him like a comforting
universe, it was interesting to see that I did feel a little less safe
and confident. However, having a few days alone in Sydney was relevant
to my experience of going home. It was like going on a pilgrimage into
my past. As a young adult these streets were the backdrop of my life,
and I was keen to walk them alone, to observe my thoughts and notice if
there had been any changes. With nothing but my robe and the underground
ticket offered by a lay supporter, I walked down the main street of the
central business and shopping district. Observing people's faces, I
could remember the way I used to live my life and the thoughts I used to
think. Most people looked physically healthy but distracted and tense,
in a hurry. While walking along the street I was determined to think
thoughts of kindness and keep in mind my faith in the Buddha's teaching.
Although I was a little fearful, it was interesting to see that among
all the varied inner-city subcultures, there were very few comments,
sneers or noticeable reactions to my presence. I had wondered if people
would move away or be uncomfortable standing close to me at traffic
lights or while waiting on the train platform. Repeatedly people stopped
close by and seemed quite at ease, so much so that my faith in the human
capacity to sense a commitment to harmlessness grew. Towards the end of
my urban pilgrimage, an elderly Aboriginal woman caught my attention.
She was sitting on a park bench outside the Town Hall, by the entrance
to the underground. As I walked towards the stairs, she looked me
briefly in the eyes, then raised her joined hands to her forehead in the
traditional Buddhist gesture of respect. I felt as though I'd been
blessed by a true native elder. This incident touched me profoundly, as
it was so completely unexpected. I looked her in the eyes with
tremendous appreciation, and I'm sure she began to blush.

Visiting my eldest sister in the Blue Mountains entailed a significant
shift in modes of relating. I was glad that I'd had a few weeks to
adjust to being busier and talking more. It was also interesting that
for this leg of the journey I was not accompanied by any other monks. My
sister's ten-year-old son fulfilled the Vinaya requirement for the
presence of another male, creating a lot of space for spending very
natural time talking in her home. Before living overseas I had been
especially close to this sister. Dianne trained as a nurse and a
midwife, and had also travelled through Asia as a backpacker in her
mid-twenties. With the arrival of three delightful children she had
ceased caring for other people's babies to pay attention to her own. She
has a love of good natural food, yoga, alternative medicine and Asian
arts. We always got on well together. After living for a year or so in
Thailand, my sister fell out with her husband and a long and painful
separation ensued, complete with custody battles and ugly court cases.
That time in Thailand was difficult for me too. I wanted to be
physically and emotionally present for my sister and her children.
Feelings of love and guilt often tugged at my heart. I'm quite sure that
had the difficulties begun before my departure, I wouldn't have left
such a situation. As it happened, though, I'd relinquished all my
possessions and was already wearing the \emph{sāmanera's} robe when the
news reached me.

Seeing Dianne waiting on the train platform with her beautiful
fairy-like three-year-old daughter I was relieved to notice that she was
radiant with life. I still remember her first few words: `God look at
you! You look great\ldots{} I suppose I can't give you a hug? \ldots{}
How strange! \ldots{} Oh well, doesn't matter! \ldots{} Wow it's great
to see you \ldots{} The colour of your robe is beautiful, so earthy and
natural.' When I explained to her that the dye was handmade from
boiled-down heartwood, she was overjoyed and started telling me how she
had begun taking African drumming classes, and that the drums were
carved from the heartwood of big old mango trees. Later she admired the
handiwork of my crochet bowl cover and hand-made monk's umbrella. Things
settled into a new kind of order. My sister and her children were quite
well, indeed, all having a lot of fun. The new house they were renting
was perched on a small hill which backed directly onto an expansive area
of undulating ranges, carpeted in a reserve of native Australian bush.
The forest was a beautiful backdrop to our many long conversations. It
was a relief to be able finally to listen, to hear the many details of
the past years of struggle, to encourage her in maintaining a generous
heart and in learning to forgive, a process in which she was already
well established. She and her children performed a show of tribal
rhythms and dance on their drums, and Dianne later played some songs on
the piano, the very songs she used to rehearse in our family home and to
which I would wake up as a five-year-old all those years ago. Admittedly
these were strange activities to be participating in as a monk, yet in
such a context they seemed harmless enough.

I also visited my niece's and nephew's school classes and gave a talk
about the lifestyle of forest monks. Happily, the school children were
attracted by the forest monk uncle, asking many questions about my daily
routine, meditation and Thai culture. Some curious questions were put to
me: `Can you make yourself float up in the air?' `I know you can't eat
at night, but if you're really hungry can't you just have some crackers
or something?' They were amazed when I explained that I didn't have a
refrigerator or cupboard, or even electricity. On both occasions I left
classrooms hearing such exclamations as: `That's what I'm gonna do. When
I grow up, I'm gonna be a monk!' I had intended these talks to be a gift
to my young niece and nephew, to show them that although I lived far
away, I still cared for them. There had been some reservations in my
mind, however, as I was not sure that such exotic spirituality would be
appreciated by all. The fears were unfounded. When I asked my nephew
whether any of the kids thought I was weird, he answered, `Nuh \ldots{}
They all thought you were cool!'

A monkcompanion met me for the flight between Sydney and Brisbane, which
was probably the tensest time during my travels in Australia. When, a
person is propelled several hundred miles in a few minutes and thousands
of feet above the ground from where a life was recently being conducted,
I expect it is normal to become very circumspect. I thought a lot during
that flight. Now that after so many phone-calls I would really be
meeting my parents and again staying in their home, the nagging little
thoughts which had been lingering in the recesses of the mind came
clearly into view and expressed themselves: `I do live a long way away
\ldots{} A son being a Buddhist monk would be challenging to most
conservative parents \ldots{} Maybe they've been pretending to be
supportive out of fear of estrangement \ldots{} Maybe the truth is that
they have many reservations, and once I am again within their sphere of
influence they will be angry or possessive.' As I mused over the
possibilities, I laughed to reflect that they probably had uneasy
thoughts themselves: `He might be demanding or difficult to take care of
\ldots{} Maybe he will try to convert us! \ldots{} Maybe he'll just want
to meditate and be by himself all the time.' Remembering the small
Buddha statue, I recollected his message and reminded myself that I was
also coming in peace. Foremost in my mind was that we should all have a
pleasant and relaxed time, to be remembered happily. I had already
decided not to try to teach my parents anything about Buddhist
spirituality unless they asked out of their own interest. I hoped this
attitude would make them feel open and unthreatened. But even with such
a sure game-plan, the nervousness persisted. As we started to descend, I
decided that there was nothing to do but let go of wanting to control
and be with the anxious, gurgly feelings in the stomach. It would be how
it would be, I'd done everything I could to try to set things up right.
On landing I asked my friend to wait a minute so that we could leave
last and take things slowly. Walking down the connecting corridor
between aircraft and terminal I was embraced by an amazing sense of
familiarity. The weather in the other cities we'd visited recently had
been surprisingly cool to our seasoned tropical bodies. Brisbane, an
hour south of my parents' house, is notably subtropical. It is the
region where I spent my childhood, and though I could not say exactly
what they were, the very air seemed pungent with familiar smells.

Reaching the end of the corridor and seeing my mother and father, I
noticed that they looked older than I remembered, but I was glad to see
such big smiles and joyful light streaming from their eyes. My father
moved forward as though he were about to embrace me. Slowing a little, I
looked at him firmly -- they had promised to save their hugs for once
we got home. As I took the last few deliberate steps, he responded to my
hesitation with a hearty, `You're my son, I haven't seen you for a long
time and I'm gonna give you a hug.' In his strong embrace I felt
humbled, childlike, with a disarming kind of happiness. Concerned
thoughts whispered in the back of my mind: `He hasn't honoured my wishes
to be discreet and restrained. He's not going to respect the boundaries
that a monk needs to maintain.' Gently I pushed the thoughts aside. For
a sixty-year-old Caucasian man to publicly hug another man,
shaven-headed and clad in religious robes, seemed a gesture too
beautiful to be censured. There was also a feeling of acceptance. The
boldness of his affection signalled that he had indeed respected my
decisions. By his side, my mother smiled a little awkwardly but didn't
express any demands for physical affection. I introduced them both to
Tan Neng, the monk who had accompanied me, and then presented my mother
with the gift which I had carefully wrapped myself. She gazed at the
black and gold paper, testing the weight in her hand and I caught a
sparkle in her eye -- my mother always loved gifts! My father insisted
on helping with the bags, saying: `When you didn't come with the others,
I thought, ``Uh-oh! He's missed the plane''\ldots{} When we saw you
walking down you did look very peaceful. You look healthy but you seem
to have lost some weight.'

In the car my mother opened her gift. `She's beautiful', she exclaimed,
then kissed the statue and held it to her breast. I explained that it
was an image of the Lord Buddha, and that it would probably be best if
she didn't kiss it. My mother promised that she would find a nice place
for it in the living room. In the company of a Thai \emph{bikkhhu} I was
embarrassed by my parents' lack of familiarity with Buddhist customs,
but happy that they appreciated the gift. Things went well at my
parents' house. Although Tan Neng and I continued to eat our main meal
from our alms bowls alone on the veranda, Ajahn Jayasāro had given
special permission to have breakfast with my parents, to enjoy a casual
time for chatting. There were many simple joys to rediscover: looking at
childhood photographs, eating favourite foods specially home-cooked,
walking through the local forest reserves. And after discussing at
length what would be a most suitable and practical gift in the future, I
drew a pattern and taught my mother how to sew the monk's lower robe.

One day, after about a week or so with my parents, Tan Neng and I spent
the day at a meditation centre in the Tibetan tradition and my father
came to pick us up. Much to my astonishment, a little more than halfway
home, I caught myself speaking to my father in a sharp and angry voice.
He'd said something about some of my habits in adolescence, and an
unexpected ball of fire began flaring in my abdomen. Do parents have a
special gift for poking the most sensitive spots in their children's
hearts? I implored myself to abide with the feeling and contain it
skilfully, hoping that my father wouldn't say anything else to
exacerbate the situation, but inevitably he did and out flew my
grievances. In his eyes I had been in some ways an obnoxious teenage
son. To me he had appeared in some respects to be a distant man who
could have paid his children a little more attention. Was there any
point to discussing these things after so long? Sitting in the strained
silence afterwards, I was sure the rest of my stay would be tainted by
this unfortunate outburst and couldn't help feeling sad.

Returning to Thailand for a moment, in the forest monasteries one finds
oneself with many hours to consider things deeply, to investigate
affections and disaffections and the layers and aspects of attachments
as they present themselves. I must have asked myself a hundred times
what I missed about Australia, and was frequently surprised to find that
except for a few close relationships, I didn't really miss much. It
seems that once we've firmly established new habits and a new lifestyle,
old memories and old habits slowly fade, and unless deliberately
recollected become increasingly distant. But one memory that would
always assert itself hauntingly was that of the beach and the ocean. I
spent my entire childhood and early adult life within a short distance
of the deep blue Pacific. Along the entire east coast, from spring
through autumn the water is cool but not really cold, and when people go
to the beach it may be for a half-day or indeed an entire day. A day at
the beach is an Australian institution, and if the sun is shining there
are always as many people in the water as there are on the sand. The
surf of the east coast had received me in its waters on thousands of
occasions over many years, propelling me between waves, pummelling my
body or allowing me to float out past the breakers, stilled by the sheer
immensity below, around and beyond. The ocean seemed to possess an
awesome power, being capable of empathizing with anything that I could
feel, offering a watery cushion between myself and the demands of living
back in the shore-bound world. Once I entered the sea, hours could go by
before I returned from this watery other-world. I was never a surfer,
but for as long as I can remember I was a swimmer. Sometimes in the dry
or hot seasons in Thailand my body would lament and grieve, aching for
the temporary respite that can be had in the Pacific.

During my trip to Australia I had several opportunities to visit the
ocean. One time in particular was especially magical. It was the weekend
and I was walking along the beach with my father. The terrible feelings
from the incident in the car seemed to have evaporated. Surprisingly,
what had appeared to be a disaster had triggered a warm opening in my
father's heart. Perhaps it was wisdom coming from age, or perhaps it was
my long absence and impending departure that stimulated a kind of
urgency in him. His past habit would have been to talk little if at all
about such tension, but this time he wanted to discuss things, with an
interest in the details. What ensued were several beautiful
conversations and an opportunity to learn many unknown things about one
another. On this fantastically bright day my father was wearing his
swimming trunks, I was wearing my robe and my twin brother had lent me
his sunglasses. We walked for many minutes along an endless beach, and
then turned round to enjoy the long walk back. The smell of the salt
air, the tumbling roar of the waves and my father by my side engendered
an exquisite sense of reunion. Our discussions led us to the place where
I could lovingly acknowledge my father's honourable qualities. Walking
over the silk-like sand we discussed each other's good qualities, with
the sky above a clear and brilliant blue and the ocean beneath it
darker, rich and deep. Hearing about my father's own experiences with
his father and teachers, I felt sure he had done a better job than the
examples given him. Learning about his past opened up a space in my mind
which placed him in a larger context. Aware of my pain and his pain, I
experienced a vast sense of empathy and wondered if there were ever a
father and son who did not feel such things. As we approached our spot
on the sand, my sun-bronzed twin brother who had driven up for the
weekend waved from his beach towel, pleased to see us together.

The remaining few days went by easily. On the last day we performed a
small ceremony where my parents formally offered me a new lower robe. I
had drawn the pattern, my mother had sewn the fabric and my father had
helped with the dyeing. On the way to the airport we visited the
memorial where my grandmother's cremated remains were stored and I
recited some funeral chants. I had been unable to come to her funeral,
so this was a kind of symbolic goodbye and it felt good to be able to
offer the chanting.

Once again in an airport, another neat expensive new building of
well-considered design, sitting in large comfortable chairs arranged
around a coffee table to give the sense of lounge-room intimacy, my
family and I drank some `goodbye espresso'. Brisbane Airport is a large
cube-like open space, with veritable walls of glass and huge skylights
which nourish the fully grown palm trees that towered around us inside.
The polished marble and chrome and the sophisticated displays of the
duty-free shops gave the place the air more of a modern art gallery than
a transit centre. The monk who had accompanied me earlier had left for
Bangkok a week before. After five weeks in Australia I felt markedly out
of place sitting there. My father and eldest brother were talking about
car insurance, my twin brother was exploring the options for
rust-proofing his new four-wheel drive, while my sister-in-law was
complaining that her work was boring. Nervous and self-conscious, I
reacted to my family's discussions by feeling impatient. I was guilty of
indignation and could fairly have been accused of having judgemental and
critical thoughts. At this and other times during the preceding weeks I
had felt exasperated by the nature of some of the conversation from the
members of my family. It would have been nice had that not been so, as
that would have been most comfortable. But there were times when I was
confounded by what appeared to me to be my family's disinterest in
engaging their experience of life with integrity. I had to remind myself
several times to see things in a suitable context.

As a member of a community of dedicated contemplatives, I had begun to
take the qualities of meditation and frank investigation for granted. If
you were to define a religious life by the qualities of honesty,
morality and generosity, then the members of my family are in fact quite
religious. I had to remember simply to be grateful for the Sangha back
in Thailand, to recognize that they were people dedicated to spiritual
cultivation and to respect the members of my family for the decent
people that they are. I should add, however, that at times my critical
thoughts were not based in negativity. As an observer I could see that
some of my family's habits were bringing them more pain and danger than
happiness. When these came up in conversation I wanted to interject.
More often than not, however, I bit my tongue, as I know that people
must feel a lot of trust and be in an open, willing frame of mind before
criticisms can be skilfully received.

Now at the airport, my frustrations were running high. Within minutes I
would have to leave them once again, and somehow it seemed important to
say something of integrity. Deciding there was little to lose, I
interrupted the flow and announced that I needed to say something:
`Mother, father, thank you very much for looking after me and making my
visit comfortable and enjoyable. I really am grateful and impressed with
the way you made so much effort and took so much time for me. If there's
anything that I said or did that was inappropriate, please forgive me.
And if there was anything that I didn't say or do which I should have,
please forgive that also.' `No, it was great having you here, mate!'
said my father, while my mother looked at me in a moment of
uncharacteristically intense concentration, and said all of a sudden,
`And will you please forgive me?' The look in her eyes was surprisingly
emotional. As I expect is the case for everyone, my mother and I have
had our share of confusion and pain. As an adult and as a meditator, one
who has had ample time to notice his own shortcomings and human
fallibility more and more, I have grown to appreciate both my parents
more and more and to feel very much at peace with them. The past ten or
so days had been very pleasant and it seemed there was no particular
`thing' to forgive. But I felt I knew somehow what she was saying. When
separated from loved ones by large distances and time, it is normal to
ask oneself such questions as, `Did I do something wrong for them to
want to stay away?'; or to have worrying thoughts like, `Maybe I
shouldn't have said those things in the past', or `Maybe I should have
done more or given more.' These thoughts had occasionally been in my
mind in Thailand. Being close to each other and allowing the worries a
moment of acknowledgement seemed important. It was with earnest
sincerity and much relief that I could look into her caring eyes and
truthfully say, `I forgive you.' I could have said much more, but that
phrase seemed to say just enough.

Both my parents are over sixty years old now and they are in some
respects creatures of habit. We have discussed the possibility of their
coming to visit me in Thailand, and although they try to sound willing,
I sense that perhaps they're just not that adventurous. Sadly, it may
once again be years before I see them. Indeed, as the Buddha encourages
us to consider, one or all of us may pass away, so this might have been
our last opportunity to see each other. It may sound precious or
excessive, but it was important to give them these messages and to let
them know, `I really am okay with you people. I care for you, and when I
think of you it will be kindly.' The sudden shift to frankness and
resolution left me feeling a little bewildered. And then it was time to
go. My father followed me to the escalator and insisted upon another
hug. Alone with him at the top of the escalator, I felt something very
consciously for the first time: that I was a man just as he was a man.
Indeed, I actually stood a little taller than he. There was an uncanny
sense of mutual respect and separateness, yet at the same time I felt
closer to him than I ever had before.

Going down the escalator and through customs wearing my new lower robe,
I was quite astonished by the sense that things had gone wonderfully,
that in fact I couldn't have hoped for a more positive visit. There were
many events that I could later share with my teachers and the other
monks. Tan Ajahn Anan had been very supportive of this visit with my
family. I would tell him how my mother had decided to offer the Buddha
statue flowers fresh from her own garden whenever she was thinking about
me, so that she could feel that she was giving something instead of just
worrying.

Once a person has taken the mendicant's vows, the question of exactly
where home is becomes primary. For me, though, I suppose home is the
place that I miss most. After five weeks in Australia, my years in
Thailand had begun to feel like a short vacation I'd had years ago.
Although I'd had many pleasant experiences and met many good people, I
was missing the quiet of the forest monasteries. I missed my Dhamma
brothers and the simplicity and clarity that come from monastic routine.
After having such a pleasant time with the members of my family, I
anticipated that perhaps upon returning to Thailand I would feel even
more at ease. After coming home, I felt that I could once again go home.

\emph{About the Author}

\emph{As well as living in Thailand, Tan Acalo went on to spend time
with Tan Ajahn Pasanno at the then recently-established Abhayagiri
Monastery in California, and with Luang Por Sumedho at Amaravati
Monastery in England. Following his fifth Rains he went to live in Wat
Marb Jan, to continue his training with Ajahn Anan for a few years.
After more time in the UK and then in Melbourne, where he spent the
Vassa with Ajahn Kalyano, he returned to Thailand, where following the
2010 Rains Retreat he was invited to take up residence on a piece of
land in the beautiful hills of Petchaboon Province, starting Ānandagiri
Forest Monastery which nestles at a meeting point between central,
northern and north-east Thailand.}



\setChapterAuthor{Tan Yātiko}
\setChapterNote{An account of a period practising at the monastery of Tan Ajahn Mahā Boowa.}
\chapter{Two months at Wat Pah Baan Taad}
\markright{\chapterAuthor}
% Title: Two months at Wat Pah Baan Taad
% Author: Tan Yātiko

Mosquito repellent? Check. Ground sheet? Check. Road map of Isan? Check.
Spare flashlight batteries? Check. Water filter? Check. Candle lantern?
Check.

So I was packed and ready for my upcoming \emph{tudong}. I had decided
that I would start from Wat Pah Baan Taad, the famous monastery of Ajahn
Mahā Boowa, and from there walk through much of the province of Udorn
Thani in the north-east of Thailand. My plan was pretty loose, if you
wanted to call it a plan. I would keep my ears open for tips on good
teachers and monasteries in the area that monks would recommend. It
would be my first real \emph{tudong} on my own, and I was looking
forward to the adventure and the uncertainty that \emph{tudong} would be
sure to offer. 

The \emph{tudong}, however, never even began. 

I remember going to Ajahn Paññāvaddho's \emph{kuṭī} the day after I
arrived at Wat Pah Baan Taad to introduce myself and ask permission to
stay. Ajahn Paññāvaddho is a senior English monk of over forty rains and
is in charge of the practical running of the monastery, while Luang Por
Mahā Boowa occupies his time in receiving guests
and teaching. Ajahn Paññāvaddho's \emph{kuṭī} is a simple wooden
structure in the forest, built on concrete posts about four feet high. 
Inside it is perhaps some ten by ten feet, and it has a rectangular
porch out front, large enough to seat four or five people. Just in front
of this old well-worn \emph{kuṭī} is a sandy meditation path that
stretches some eighteen paces, shaded by the shadow cast by a tall
wooden pole. Ajahn Paññāvaddho's solar panel, which he uses to recharge
batteries and run his water heater, is secured atop the pole. I coughed
as I entered the sandy yard, and when he invited me up I ascended the
\emph{kuṭī} steps and saw him sitting on the step between the
\emph{kuṭī} door and the porch, with knees up near his chest and his
elbows on his knees. 

I felt immediate affection for Ajahn Paññāvaddho. His gentle old eyes
glowed at me as I bowed to pay my respects, and his informal greeting, 
`So, how's it going?' quickly made me feel at ease. I explained to him
that I wanted to spend a few nights at the monastery, pay respects to
Ajahn Mahā Boowa and then head off on my \emph{tudong} trip through the
province of Udorn Thani. Ajahn Mahā Boowa, he told me, had gone away to
Bangkok until New Year, still two weeks away, but I would be welcome to
stay until he came back. Well, it wasn't what I had planned, but I
thought that this would be an ideal opportunity to spend some time at
the monastery and receive some teachings from Ajahn Paññāvaddho. 
Besides, after Ajahn Mahā Boowa came back I could just carry on with my
\emph{tudong} as I had planned. So I decided to stay on till Ajahn Mahā
Boowa returned. 

That afternoon Ajahn Paññāvaddho took me personally around the
monastery. It's a fairly small place, maybe some 200 acres in area, and
with a forbidding chicken population one's first impression is that it's
a rather unlikely place to come and calm your nerves. Within the
monastery confines there are about forty little shelters, mostly
open-air wooden platforms with iron roofing, and a handful of larger
\emph{kuṭīs}. He took me to one of the \emph{kuṭīs} which had recently
been evacuated by one of the senior monks, and suggested I sweep around
the \emph{kuṭī} and spend the day meditating. With the monastery being
about as peaceful as a chicken-farm, and with Ajahn Mahā Boowa's current
fund-raising programme to help the country out of its financial crisis
 (he had then raised over a billion Thai Baht as an offering to the
National Reserve), I wasn't expecting much out of my meditation. I
prefer the silence of caves to the clucking of chickens and I have
always been sensitive to noise when I sit. But quite unexpectedly, I
found that as I sat in my hut my mind felt brighter and stronger than
usual. `Must be a fluke,' I thought to myself. But no, there was no
denying it, after a few days I found that my meditations had improved; I
was feeling much more focused. I mentioned this to Ajahn Paññāvaddho, 
and he said it was the power of living in Ajahn Mahā Boowa's monastery. 
`He's a special man', he said with an air of understatement, `and that's
why your meditation is going so well here. Living under a teacher like
Luang Por is a critical part of the practise.'

Well, I'm not so sure that
just being there had any magical influence, but it's certainly true that
I took to reading Ajahn Mahā Boowa's talks like never before, and I felt
a confidence in the tradition and in the man that surprised even me. 
Being in the presence of a realized being helps one to know that this is
not a path `I' am walking on, and one for which `I' must have the
qualifications to succeed, but rather that the path is a simple cause
and effect process. Just as any one of these realized masters was once
as deluded and self-centred as I am now, so too can this set of five
\emph{khandhas} one day be as free of suffering and delusion as theirs. 

Afternoon tea would take place at the dyeing shed shortly after 1 p.m.,
and then I would have the chance to ask questions about Dhamma. During
my stay there I had many inspiring conversations with Ajahn Paññāvaddho
over a hot drink. A small handful of Westerners would gather round
outside the shed and squat down on tiny five inch-high wooden stools. 
One of the resident monks would make Ajahn Paññāvaddho's usual, a cup of
black unsweetened Earl Grey, and would make me a cup of black coffee. 
I'd set my stool on the dusty soil to the left of Ajahn Paññāvaddho, 
while a dachshund, a poodle and a few cheerful dalmatians would
playfully compete for our attention around the shed. As the
conversations evolved, almost invariably the talk would turn to or at
least encompass Dhamma. 

Often when talking Dhamma with someone it's not so much the content of
what is said that educates, but rather the way a point is made, 
fashioned exactly to illuminate an aspect of practise you may have
overlooked or forgotten. Ajahn Paññāvaddho had that valuable knack of
being able to offer a pointful reflection on any given topic of Dhamma
that one might raise. I described to him once how my meditation was
losing some of its sense of direction and my mind was beginning to
wander again. Ajahn Paññāvaddho's response was almost uncannily fitting. 
He told me that whatever we do in meditation, the main point is to
undermine the defilements. That's it. Simple words, nothing I couldn't
have thought of myself, but their simplicity and directness gave me a
new outlook on meditation and I found that my sittings improved. I
noticed that I had been getting bogged down with techniques and
strategies, without focusing on the actual purpose. A good teacher is
one who can notice these subtleties in a student and point them out. 

That being the case however, the most educative thing about being with
Ajahn Paññāvaddho was just his relaxed and mild presence. He was a man
with nothing to prove to anyone, and he was not out to convince me of
anything. One point he would often make was that a meditator has to have
\emph{samādhi} before he can get very far with \emph{vipassanā}. 
`Without \emph{samādhi}', he'd say with a gentle authority, `it's
unlikely that one will have the emotional maturity to deal skilfully
with profound insights. When deep insights occur, one can be left with
the sense that there is no place for the mind to stand, and this can be
very unsettling. When one has \emph{samādhi}, one has a comfortable
abiding and one has the freedom to ease into one's insights with skill.'
I then asked, `What about the need for \emph{jhāna} on the path? Do you
feel it's necessary to have \emph{jhāna}?' A reserved look appeared on
his wrinkled face, and in his soft voice he explained, `The thing is
that people will often talk about \emph{jhāna} without knowing what it
really is. Ajahn Mahā Boowa prefers to use the word \emph{samādhi}. We
need \emph{samādhi}, and the more we have the better, but we must not
neglect our investigation.'

One of the practise techniques that is often encouraged by the forest
Ajahns is the investigation of the body. This practise will involve
visualizing a part of the body, focusing on it and studying it, so as to
achieve a soothing and dispassionate calm and counterbalance an
underlying infatuation with the body that lies within the unenlightened
mind. It is a practise which for those unacquainted with Buddhism is
often unpalatable and hard to understand. Ever since I ordained I had
had a chronic avoidance of this technique and I asked the Ajahn about
it. He said that the fact that I avoided it probably showed there was
much for me to learn in that area: `That's your front line of
investigation.' He went on to say, `Why do you think you are so hesitant
to pick up this practise? I'm not saying it is necessarily an
appropriate meditation theme for you to work with simply because you are
resistant to it, only that you should at the very least understand the
nature of that resistance before you just go ahead and believe in it.'
So I looked at the resistance deeply and found it hard to see exactly
what it was all about. After a few weeks of being in the monastery and
taking up a considerable amount of contemplation of the body as my
meditation theme, I began to feel a certain weariness with it, and felt
a kind of depression coming up. I told this to Ajahn Paññāvaddho. He
smiled knowingly and shook his head as if to say, `Yup, that makes
sense', and explained, `Yes, this can happen to some people if they are
doing a lot of body contemplation. It can feel disappointing if you
suddenly start to see the reality of the body's unattractiveness and
impermanence. After all, most of us have been running around glorifying
it for lifetimes.'

I sat and listened to what the Ajahn was saying, somehow sensing that
the truths that he was talking about weren't sinking in. I tried to
force myself to believe what he was saying, but found that I couldn't. 
And yet I trusted the man and I trusted in the truths of which he was
speaking. In the end I found myself stuck between a reality I couldn't
accept and a trust I couldn't deny. I asked him how I should proceed if
I found this kind of meditation left me feeling a bit down. His answer
was reassuring. He said I shouldn't force it and that it was important
to balance this meditation technique with other techniques that I found
uplifting. For years one of my major meditation themes had been
contemplation of the Buddha, something that always brought me great joy. 
I told this to Ajahn Paññāvaddho and he said that it would be an
excellent counterpart to body contemplation. 

While at Wat Pah Baan Taad I actually had very little contact with Ajahn
Mahā Boowa himself. He kept quite aloof from the monks and gave more
attention to the laity. I remember that a few days after he had returned
from Bangkok, word went round that he would be giving a talk at the
dyeing shed that night at eight o'clock. `That must be nice', I thought
to myself. I imagined the group of monks gathering around Luang Por
in the cool of a chilly Isan night, warmed by a fire lit in the wood
stove, listening to stirring personal Dhamma from the master himself. 
Not quite. I approached the dyeing shed around five minutes to eight and
saw Ajahn Mahā Boowa studying some newspaper clippings that had been
posted on the wall of the shed. There wasn't a soul around and I thought
to myself, `What an opportunity to approach Ajahn Mahā Boowa and say a
few words!' So I somewhat meekly approached Luang Por, holding my hands
up in the traditional gesture of respect, when suddenly from out of the
shadows of the night appeared a nervous young monk, tugging at my robes
and, whispering in Thai, `Come here! Come here!' gesturing me to steal
away behind a pile of buckets and squat down. As we squatted there in
the silence with eyes glued to the Ajahn, I noticed that to my left
behind the water tank another shadowy figure was squatting, and then to
my right, closer to the wood stove, was another one. This one had a pair
of headphones on, and some electronic gadgetry with little red LED
indicators that danced up and down to the crackle of the fire. He
reminded me of some FBI agent or private eye. In fact the whole scene
was a kind of \emph{déjà vu} of some half-forgotten memory from high
school. I was both amused and confused, and it took more than a few
seconds to realize what was going on. Ajahn Mahā Boowa wasn't giving a
talk to the monks, he was giving a talk to a certain layperson, and the
resident monks were gathering around in the dark to hear and even tape
the conversation. 

It was strange. I didn't understand. Why was Ajahn Mahā Boowa so aloof
from the monks? He was known for his uncompromising strictness and his
readiness to scold, and I found it hard to understand to what purpose. 
Was this really coming from a heart of \emph{mettā}? As with any
situation one wants to understand, one has to understand it from within
its own context. It wasn't until I had stayed there for several weeks
that I began to get a feeling for how Ajahn Mahā Boowa motivates his
monks. It isn't through playing the charming, soft, lovable role of a
guru. He does not become informal with any of his disciples, and this
serves to establish a very specific role in their relationship which is
felt and understood by those who practise under him, but may not be felt
and understood by newcomers. It was obvious that Ajahn Mahā Boowa cares
for the monks, he cares for the world, he cares for the religion, and he
sees the sincere practise of monks and nuns as an immeasurable
contribution to the virtue in the world. He wants to protect the high
standards of Dhamma and his life is dedicated to that valuable end: 

`Don't waste your time letting any job become an obstacle, because for
the most part exterior work is work that destroys your work at mental
development \ldots{} These sorts of things clutter up the religion and
the lives of the monks, so I ask that you not think of getting involved
in them.'

For Ajahn Mahā Boowa the role of monastics is clearly defined. He says
it is those who have gone forth who are the `\ldots{} important factors
that can make the religion prosper and serve as witnesses [of the
Dhamma] to the people who become involved with it, for the sake of all
things meritorious and auspicious \ldots{}' It is the monastics who
preserve the actual teachings through their practise, and it is the
teaching which will help promote goodness and virtues in all those who
come in contact with it and take an interest in it. 

It seems that one of the dangers he sees for monks is over-association
with laypeople. Particularly in a country like Thailand, where so many
of the population have great faith and respect, I think he sees that
respect as a possible source of corruption for monks. Maybe that is
partly why he doesn't seem to over-value impressing laypeople with
politeness. He shows no interest in receiving respect for mere social
niceties, and when he teaches the monks he warns them of the danger: `No
matter how many people come to respect us, that's their business. In
practising the Dhamma we should be aware of that sort of thing, because
it is a concern and a distraction, an inconvenience in the practise. We
shouldn't get involved in anything except the contact between the heart
and the Dhamma at all times. That's what is appropriate for us\ldots{}'

In one of his talks he tells a story about a childhood experience. It
seems his father was a short-tempered man, and one day during the
evening meal his father began to turn on him and his brothers: `You're
all a bunch of cow manure. All of you. I don't see a single one of you
who is going to ordain as a monk and make something of yourself\ldots{}
none of you except maybe\ldots{} Boowa over there. He might have what
it takes\ldots{} but other than him, you're all a bunch of cow manure, 
you'll never come to anything.' The scrawny little Boowa was so shaken
by this first and only `compliment' from his father that he got up from
the meal and ran out of the house with tears welling up in his eyes. He
leant up against one of the water tanks with his head on his arm, trying
to collect his emotions and thoughts. There was no denying the feelings
rushing through him. His fate had been sealed. `It's decided,' he felt
in the pit of his stomach, `I'll be a monk, and if I'm going to be a
monk, I'll do it right. I'll spend however long it takes to pass the
third grade of Pāli studies and then I'll go off to the mountains and
spend my life meditating.'

And that's exactly what he did. It's perhaps this determined nature that
comes across most clearly in his talks, and it is what he seeks to
cultivate in his disciples. He makes frequent references to the need for
whole-heartedness in one's practise and candidly recounts his own
attitude as a young monk: `From the very start of my practise, I was
really in earnest, because that's the sort of person I was. I wouldn't
just play around. Wherever I would take my stance, that's how it would
have to be. When I set out to practise, I had only one book, the
Pāṭimokkha, in my shoulder-bag. Now I was going for the full path and
full results. I was going to give it my all, give it my life. I was
going to hope for nothing but release from suffering.'

He is uncompromising in the teaching and devotes a large part of its
contents to the dangers of being consumed by the changes of modern
society and all its trappings. I have heard the occasional person
comment that Ajahn Mahā Boowa is old-fashioned. I consider the comment
misses the point of Dhamma practise. Yes, he keeps to most of the old
traditions and may not be in touch with the younger generation and where
they are, but that means we must use our own wisdom to unearth where he
is and learn from that. There is a well-spring of wisdom in the forest
tradition which people are liable to overlook because it doesn't answer
their questions in the way they hope. Ajahn Mahā Boowa is not interested
in dressing up the Dhamma to make it tastier medicine to swallow, and
this may make him distasteful to the modern spiritual seeker. But what
he does teach is powerful and transformative for those who are willing
to commit themselves to it. 

One of his constant refrains to monks is to seek out the forests, the
hills, the lonely places, and meditate. Distractions not only interfere
with our cultivation of meditation; more seriously, they can delude us
into taking the worthwhile for worthless and the worthless for
worthwhile. Computers, books, worldly conversation -- it's not so much
that he sees them as bad in themselves, but he does see them as great
dangers to meditating \emph{samaṇas} and puts them in their proper
perspective. Regarding material society, Ajahn Mahā Boowa says: 

`We've gone way out of bounds. We say we've progressed, that we're
advanced and civilized, but if we get so reckless and carried away with
the world that we don't give a thought to what's reasonable, noble, or
right, then the material progress of the world will simply become a fire
with which we burn one another and we won't have a world left to live
in.'

And elsewhere he suggests:

`The teachings of the religion are an important means for putting
ourselves in order as good people living in happiness and peace. If you
lack moral virtue, then even if you search for happiness until the day
you die, you'll never find it. Instead you'll find nothing but suffering
and discontent.'

While there is almost certainly no forest monastery in Thailand that
could bring in anywhere near as much money as Wat Pah Baan Taad, Ajahn
Mahā Boowa still lives in a simple wooden hut. As he approaches his late
eighties he still eats at only one sitting. His monastery still has no
electricity running into it (though generators are used for certain
things), and thus the monks clean the Dhamma-hall by the light of candle
lanterns. Apparently the King of Thailand has repeatedly offered to
build a new modern Dhamma-hall for the monastery, but Ajahn Mahā Boowa
has insisted on keeping the original old wooden structure in use. So in
a sense he is a bit old-fashioned; he's certainly not intoxicated with
progress and development, because he sees virtue and insight as the most
significant factors in peace and happiness. 

When I first arrived at the monastery I was not without some critical
feelings. I wondered if there might not be a better system of running a
monastery than having hundreds of laypeople donating food to a monk who
was in the midst of receiving his fifteenth bowl-full that morning. But
when I looked closer, when I looked from within, I could see that Ajahn
Mahā Boowa is operating from a fundamental premise that nothing has more
value than the Dhamma, and indeed that its value is immeasurable. That's
what the fault-finding mind tends to forget. I would become frustrated
with the speed with which the monks there ate or their style of going on
alms-round (they walk very fast), because I failed to remember that
behind all this, in the backdrop, is an enlightened mind which is
creating the conditions for profound transformation within individual
minds. In one of his Dhamma talks Ajahn Mahā Boowa says the happiness of
the world is like the happiness of a prisoner, and the happiness that
comes from Dhamma is like the happiness of freedom: 

`[As we practise] the happiness that comes from the outside world
-- in other words, from the current of the Dhamma seeping into our
heart -- we begin to see, step by step, enough to make comparisons. We
see the outside world, the inside world, their benefits and drawbacks. 
When we take them and compare them, we gain an even greater
understanding -- plus greater persistence, greater stamina \ldots{} The
more peace we obtain, the greater the exertion we make. Mindfulness and
wisdom gradually appear. We see the harm of the tyranny and the
oppressions imposed by the defilements in the heart. We see the value of
the Dhamma, which is a means of liberation. The more it frees us, the
more ease we feel in the heart. Respite. Relief.'

That's what it all comes down to. He wants his monks and lay supporters
to practise. And when I read one of his books or listen to one of his
talks I can sense the intelligence that goes into them, and the will
that is doing part of the work for me by inspiring me and instructing
me. 

When the time came for me to leave Ajahn Mahā Boowa's monastery, I felt
richer and more connected to the forest tradition. For three or four
years I had had a strong appreciation of the Suttas, but not having
known Ajahn Chah in his teaching days, I had not had the good fortune of
training under a fully enlightened master. My visit to Ajahn Mahā
Boowa's monastery and subsequent visit to the monastery of one of his
great disciples, Ajahn Wanchai, opened my eyes to the Sangha, the
enlightened Sangha, as a living force in this world, shaping and
moulding the understanding and right intentions of those who come into
contact with it. 

\clearpage

\section{The Author}

Tan Yātiko did finally go on \emph{tudong}, spending some of his years as
a junior monk in the Dhammayut forest
monasteries of Udorn Thani, most notably with Tan Ajahn Wanchai. He
returned to Wat Pah Nanachat to offer his assistance to the community in
2003. For the next four years he was the senior monk at Dtao Dam Forest
Hermitage in Kanchanaburi Province. Following a Rains Retreat in his
home country of Canada in 2008, he joined the community at Abhayagiri
Monastery in California where he is still a resident monk, assisting Tan
Ajahn Pasanno in the running of the monastery and the training of the
monks. Over the years he has maintained the tudong spirit of an
unsupported alms mendicant, walking in Thailand, India and, most
recently, California.

\section{Note}

Venerable Ajahn Mahā Boowa passed away on 30 January 2011 in his
\emph{kuṭī} in Wat Pah Baan Taad.

Many of the teachings by Luang Por Mahā Boowa are available in English
in `\emph{Samaṇa}', published for free distribution by Forest Dhamma
Books.



\setChapterAuthor{Ajahn Jayasāro}
\setChapterNote{An edited version of a Dhamma talk given to the Sangha during the Rains Retreat of 1998.}
\chapter{Faith in the Quest}
\markright{\chapterAuthor}
% Title: Faith in the Quest
% Author: Ajahn Jayasāro

As a child I was fascinated by quests: I loved reading about Greek
heroes facing tests of their endurance and ingenuity as they sought some
great treasure, and of the knights of the Round Table searching for the
Holy Grail. In my teens I discovered the Buddhist vision of life as a
spiritual quest. At a time when I was finding the values of the world in
which I lived hollow and inane, the Buddha's declaration of the search
for freedom from ignorance and attachments as the truly Noble Quest
seemed irrefutable to me, and it still does today. I became convinced
that any human endeavour alienated from the Noble Quest, no matter how
conventionally worthy it may be, was ultimately trivial. 

As monastics we take on the form of the Buddhist monk with its
discipline and regulations, not as a new identity, but out of a
recognition that in our chosen quest we need a certain degree of
structure and support. Any quest worthy of its name involves facing up
to demons, avoiding quicksands and disregarding sirens. With the backing
of Sangha and faith in the value of the quest we can achieve this. 

Faith has been an unpopular word in some Western Buddhist circles, 
especially with those people who have felt bitter about their theistic
upbringing and seen in Buddhism something more `scientific'. For myself
I like the word, and find `confidence', the other popular choice as a
translation for `\emph{saddhā}', too mundane. Anyway, however this term
is rendered into English, I think we must first acknowledge that we
can't do without it. Nobody can prove that there is such a thing as
enlightenment, but if we don't have faith that there is our practice is
unlikely to go very far. Faith clarifies the goal, focuses our efforts
and fills us with energy. Ultimately it is wisdom rather than faith that
moves mountains, but it is faith that impels us to move them in the
first place, and faith that sustains us through the inevitable
frustrations that dog our efforts. 

There are so many things that human beings can search for in life: 
security, wealth, power, fame, respect, love, immortality and even
annihilation. Many people spend their life searching without ever
clarifying exactly what it is they want; all they know is that whatever
it is, they haven't found it yet. Some give up their quest, some turn to
drink or drugs, others become bitter and cynical. Many people search for
the ultimate experience, doing outlandish things merely because nobody
has ever done them before. They want `the challenge', they crave the
adrenaline. They want to stand out from the crowd. Everyone is afraid
that their life doesn't mean anything. But all experience, from cleaning
out a dingy drain filled with human hair to sexual bliss in Shangri-la, 
lies within this very narrow circumscribed realm which the Buddha called
the \emph{āyatanas}. We see that none of the exotic, mind-boggling
things that people get up to in the world ever transcend the sense
spheres. No matter how much money people have and how well-endowed they
are with worldly blessings, wherever they go they are still stuck fast
within the realm of the sense bases. No matter how exalted the aesthetic
experience, nobody is ever going to be able to see with their eyes a
form which is anything other than a form. Form is just form. It arises
and then it passes away. That's all it knows how to do -- quite pathetic
when you think about it. \emph{Samsāra} just doesn't live up to the
hype. When we understand a form as a form, the content of that form
starts to lose its power to enthral or enslave us, depress or enrage us. 
Sounds are only ever just sounds. No matter what sound it might be, the
nature of a sound is always the same -- it's just this much, unstable
and inconstant. Sound is just sound. Odours are just odours. Flavours
are just flavours. Physical sensations are just physical sensations. 
Thoughts, moods, emotions are just that, and they can't ever be more or
less than just that -- impermanent, empty and ownerless. Not one of the
\emph{āyatanas} can be sustained and enjoyed for an indefinite length of
time. 

This, I feel, is where the whole concept and idea of renunciation starts
to become so compelling, uncommon sense. Racing around, struggling and
striving, just to be able to experience more forms, more sounds, more
odours, more flavours, more physical sensations, more emotions, thoughts
and ideas, starts to appear tiresome. Is it really a satisfactory way to
spend a human existence? Prince Siddhartha and countless men and women
after him have started their spiritual quests with the conviction that
there must be something more to life than this. 

But in the world, pursuing sense desires is the norm. Enjoying certain
kinds of feeling and avoiding others is the aim. The more experiences
you have, the fuller your life is considered to be. Intensity and
passion are seen as ends in themselves. From an early age we absorb the
idea that romantic love is the supreme fulfilment -- poetry, novels, 
television, movies all tell us the same thing. Freedom is usually
considered to lie in consumption; the more money we have, the greater
the choice of hamburger fillings available, the freer, it seems, we are
to consider ourselves. Even financial markets are called free. In the
monastic life, however, we are willing to look into everything carefully
in order to see what is truly what. Our role in the world is to step
back a little from the pace and pressures of the world, to investigate
and penetrate the nature of existence. We do this with the understanding
that the quest is not exclusively intellectual. Success will depend a
great deal on our moral integrity and emotional maturity. In the
uncovering of truth, our effort is not to try to make all our ideas
conform to what's written in the Buddhist books, but to use the
teachings as working hypotheses. Do they explain human experience
adequately, completely? With the power of \emph{sati, samādhi} and
\emph{paññā}, we learn to see the truth, just as it is. 

What we're offered by the Buddha is a teaching that opens our eyes to
look at life and at experience. We investigate: `What is this?' `What am
I?' `What is this life \ldots{} this body \ldots{} this mind?' And
humble acknowledgement that we don't know is the motor for the search. 
We cherish a faith in the ultimate value of the search for a direct
experience of truth. Ajahn Chah once spoke of a bird waking up and
realizing it's in a cage. No matter what the cage is like, even if it's
a most beautiful ornate cage with gold bars, once the bird clearly
understands what a cage is and that liberation lies outside it, it can
never be content with its old life again. We undertake the spiritual
life to seek freedom from the confines of the cage. 

The path we follow, the way we live our life as \emph{samanas}, only
becomes really sensible and meaningful in light of the aspiration for
transcendence. We must believe in the vision of freedom outside the
bonds and attachments founded on a sense of self. That freedom comes
from fully understanding, moment by moment, the nature and mechanics of
bondage. We practise to understand the nature of the five
\emph{khandhas} and the six \emph{āyatanas}. The reason why attachment
to forms, sounds, odours and all the \emph{āyatanas} is so treacherous
is, quite simply, because they don't last. Today I went to visit our lay
supporter Mae Jorm in the hospital. Her cancer is now at the stage where
she doesn't even have enough energy to swallow water. Her throat was dry
and she desperately wanted some water, but when it was poured into her
mouth she couldn't swallow it. The water started to drip from the side
of her mouth. It was a heartbreaking thing to see. Her body and sense
organs are shutting down. I have known this woman since I was a novice
and soon she will be dead. 

Our eyes start to go, our ears start to go, and even if the forms are
still there, they are not for us any more. Sounds were there before we
were born and will be there after we have died. No one has ever become
free through looking at beautiful things. No one has ever become free, 
liberated from the vicious round of birth and death, through listening
to beautiful music or hearing beautiful sounds. We might have become
quite peaceful for a short while, but there was no wisdom -- we were
merely fine-tuning the quality of our distraction. 

With basic faith and confidence in the teachings of the Buddha, we get a
foretaste that there is such a thing as freedom, and that it is
realizable. It is possible. There is hope. There is a path. But as long
as there is still attachment to the five \emph{khandhas}, as long as
there is still clinging to delight in the physical body, to delight in
feeling, perception, thoughts, ideas, emotions and sense-consciousness, 
there is still delight in \emph{dukkha}. The Buddha made this very
clear. The attachment to the five \emph{khandhas} is attachment to
\emph{dukkha}. One who is attached to \emph{dukkha} cannot be free from
\emph{dukkha}. 

So there is really nowhere to go -- nowhere new, anyway. Wherever we
go, we're always going to be in exactly the same place. We'll be in a
place where there are forms, sounds, odours, tastes, physical sensations
and mental events. This is it. We've already got the whole package. No
matter whether we're trekking through a pristine Himalayan valley or
struggling through a crowd in a frowning city, we have the same work to
do. We can see the truth of things wherever we are. Of course, certain
environments are more conducive to the work than others -- that is why
the Buddha established monasteries and a monastic order -- but even so, 
wherever we are, in whatever posture we find ourselves, we can do the
work of developing awareness, turning the light within. We seek to learn
from whatever it is, learning to see things as dhammas rather than as
`this person', `that person' and `this and that'. We apply ourselves
steadily to the process of de-conditioning and re-education. Stopping
the rot. Making a fresh start, again and again. Infinitely patient. 
Until the work is done. 

With the faith that the Buddha was fully enlightened, with the trust
that the teachings which he shared with human beings and \emph{devas}
for forty-five years are true, and with the conviction that the
\emph{ariya sāvakas} truly penetrated those teachings, it follows that
each one of us, wherever we're from, wherever we were born, whatever
language we speak, man, woman, old or young; we all bear within us this
capacity to realize the truth. Human beings can attain Awakening, can
realize Nibbana, because we're fish in the water. Why shouldn't fish be
able to understand what water is? It's all around us, it's all within
us. All we have to do is learn how to open our eyes. 

It's common amongst Buddhist practitioners, however, to realize that
their strong sense of \emph{saddhā} or faith in the Buddha, Dhamma and
Sangha, a profound trust and confidence in the truth of the Buddha's
teachings, is not matched by faith in their capacity to realize that
truth. But without this faith in ourselves the five \emph{indriyas} have
no opportunity to mature. This lack of faith in our potential for
enlightenment is crippling and unwise. The doubt is based on a mistaken
way of looking at ourselves. Swallowing the whole myth of the
independent `I' gives us spiritual indigestion. We can't force ourselves
to have faith, and we don't need to. We merely have to remove the wrong
thinking that prevents faith from arising, and start paying more
attention to our experience. 

Our tradition makes an important distinction between two levels of
truth: the conventional and the absolute. The term `conventional truth'
refers to the conditioned, phenomenal or relative sphere. In this sphere
it is valid to talk about `self', `human beings', `monasteries' and
`monastic orders'. `Absolute truth' refers to the way things are, 
unmediated by concepts and bias; in this sphere language and thought are
transcended. The wise person uses conventional truths in order to
communicate, but he is not fooled by them. When this way of exposition
is understood, certain Buddhist teachings which would otherwise remain
quite puzzling become clear, in particular those regarding `self'. 

A number of references in the Suttas to `self' -- for example, the
famous saying that the self should be the refuge of the self and the
instructions on various kinds of self-development -- are expressions on
the conventional level. They do not clash with the `absolute' truth of
\emph{anattā}. The teaching of \emph{anattā} does not mean that the
Buddha is refuting the self on the conventional level; he is simply
reminding us not to confuse a useful social fiction with ultimate
reality. There is no independent, timeless `I', no unmoving centre of
experience, no soul-entity, no separate ego-identity, as we assume. 
However closely we look we cannot find `one who acts', `one who thinks', 
`one who does' and `one who wanders from life to life'. But there is a
conventional `I'. One teacher puts it well: `There is a self; it's just
not permanent'. 

Our discouragement in the practice frequently comes from trying to
imagine how this limited `I' could possibly realize the unlimited. How
could this bounded `I' realize the unbounded? Having posed a question
based on false premises (that the `I' is real) we naturally conclude
with a false answer, `No, I can't see how realizing Nibbāna could ever
happen; it just doesn't seem possible'. In other words, how could little
old me ever realize something so marvellous? The gap seems too wide. How
can this person realize the truth? 

Well, that's exactly the point, isn't it? It can't. This `person'
doesn't realize the truth. Rather, it's through understanding what this
`person' is, that truth is revealed. The realization of the Third Noble
Truth leads to the unveiling and manifestation of Nibbāna. In the words
of the Suttas, it involves `upturning something that has been
overturned'. It is a `shining of light in the darkness'. Nothing new is
created; what occurs is a radical re-appreciation of experience and a
recognition of something which has always already existed. The deathless
element is also a birthless element. It is not something that is brought
into existence. Instead, those things which conceal or envelop it are
removed. If we can grasp this point, we can feel a new surge of energy. 
We see that any sense of inadequacy we might feel is founded on
attachment to the conventional self as being ultimately real. At this
point our effort and energy, our persistence in practice are greatly
strengthened, and the nagging doubt about our capacity to follow the
path to its end may even disappear in a flash. We start to give what it
takes. 

If doubts arise in practice, investigate the discouragement, the
uncertainty and hesitancy as mental states. Watch when questions like, 
`Could I possibly reach the same level as Ajahn Chah and Ajahn Mun and
all those great teachers?' arise. If we dwell on `me' and `my' personal
history, `my' foibles and idiosyncrasies, it seems ridiculous even to
imagine that we could ever possibly attain the same level as monks like
that. But the essence of the practice is not -- what a relief! -- the
gradual perfection of character and personality; it is the understanding
of character and personality as conditioned phenomena. Certainly, 
unwholesome features like selfishness, jealousy, anxiety, etc. 
inevitably abate through practice, but the idea is not to mould our self
into a new, more `spiritual' being. Character and personality are not
and never have been who we are. They are not self. They are not anything
ultimately real. So we have to learn to stand back from the idea of
becoming an enlightened being. Otherwise, when we contemplate, `Is
enlightenment really possible for me?' and we are unsure -- is this
hubris, worldly ambition, spiritual materialism? -- we may decide, `No, 
not for me' and then falsely dignify that wrong view by calling it
humility. 

\emph{Asmimāna}, the subtle conceit `I am', is the crux of the problem, 
the spoiler, the fly in the cosmic soup. It's the most difficult thing
to see through, because the self-assumption is the foundation on which
unenlightened human beings build their whole world-view. The existence
of an independent self-existent `I' seems obvious; everyone takes it for
granted, it's common sense. That's why it's so hard to face up to the
realities of life -- birth, old age, sickness and death -- we see them
as things that happen to `me'. It's `my' dilemma, it's `my' problem. `I'
was born, `I'm' getting old and `I'm' going to die. In Thai the word
\emph{samkan} means `important'. But sometimes the word is used as a
verb, to \emph{samkan tua}, which means to give importance to self. And
we give importance to self in so many ways, not just in our arrogance
and pride, but also in our humility. Or in being anything at all. We see
this clearly when we compare ourselves with others, considering
ourselves as better, as equal, as worse and so on. This is where we
`\emph{samkan tua}' This is where we most prominently uphold the myth of
self. 

The Buddha said that the practice which most directly opposes or
undermines \emph{asmimāna} is \emph{aniccā} \emph{saññā}, the
contemplation or constant recollection of transience and change. 
Investigation of the impermanence and inconstancy of phenomena enables
us to see that those things we've always assumed to be solid are in fact
not solid; that which we think is permanent is not permanent at all. 
This solid `I' who does things, has experienced things, has highs and
lows, ups and downs, is not a coherent entity at all. If you take a
light, a candle, a torch, and you wave it around in a circle fast
enough, you get the illusion of a fixed circle of light. But in fact
there is no such thing. The same principle applies to our investigation
of the mind and the five \emph{khandhas}. Through the practice of being
fully awake and alert in the present moment, the truth of change becomes
manifest. Mindfulness slows things down, at least subjectively. Suddenly
we have time. There are gaps. There is a sense of things not moving so
fast anymore. And when there is that penetrative awareness and presence
of mind, there is the opportunity for circumspection and for the
recognition of things arising and passing away. The arising and passing
away of the five \emph{khandhas} may be seen as a simple impersonal
truth. We know consciousness as just that, without having to add
anything to it. With mindfulness and wisdom we don't make a big story
out of things any more. We experience the episodes of our life more as
haikus than as scenes from a fat and portentous autobiographical novel. 

Faith is what keeps us going through the difficult times. Faith and
endurance. Human beings in extreme conditions show an incredible
capacity for endurance -- the prisoners of war on the Death Railway in
Kanchanaburi during the Second World War are a good example -- but
whether people survive ill-treatment and deprivation seems often to lie
more in whether they want to endure, rather than whether they can. Those
who don't see the purpose or the value of endurance are alienated from
their innermost resources and die. They lose their will to live, or we
might say their faith. In spiritual life, our capacity to endure through
the ups and downs, the dark nights and deserts and sloughs of despond, 
is dependent on our wanting to do so. And if we want to, it's because we
believe it to be worthwhile. This is faith. 

Blind dogmatic faith was sharply criticized by the Buddha. He taught a
faith that welcomes the critical faculty and does not claim to be more
than it is. He pointed out that it is possible to have a strong faith in
something and be completely mistaken. The strength of the feeling is not
a proof. He taught us to take the teachings as working hypotheses and
then put them to the test of experience. Observe yourself and the world
about you. Many years ago I experienced a small epiphany, one that
greatly increased my faith in the capacity for radical change in my
life. What I experienced wasn't an intellectual proof of that capacity, 
but it had a deep emotional significance for me which has not faded. 

I was travelling in a bus through a huge desert. The journey was to take
fifteen to twenty hours and there was almost nothing to be seen on
either side of the bus, just sand and rock. At the time I was in my late
teenage years and at a pretty low ebb in my life. I'd been in India
practising meditation and just starting to feel that I was making some
progress, but then I had to leave because my money ran out. During the
many adventures I had as I travelled westwards, there was this
underlying feeling that I had squandered a marvellous opportunity. 
Something had been lost. So I was travelling through a seemingly endless
expanse of desert. Looking out of the window, all I could see was just
sand and rock everywhere. I remember thinking, `That's me, just sand and
rock, as far as the eye can see \ldots{}' Every time I looked out of the
window this thought just kept coming up: `That's me, sand and rock.'
Then I must have dozed off. During the night quite an unusual thing
happened; there was a rain storm in the desert. As I came round I could
feel straightaway that it was cooler and fresher. I looked out through
the window and couldn't believe my eyes. Throughout the desert and on
the rocky outcrops were these beautiful flowers, a profusion of the most
vibrant yellow, mauve and turquoise blooms! It struck me as a miracle. 
How could these flowers exist in such a place? Where did they come from? 
Just a few hours ago there were endless stretches of sand and rock. Now
there were beautiful wild flowers everywhere! The flowers were not big
flowers, just tiny little flowers, but they sprang up in such a short
time. And as I was already in a metaphorical frame of mind, the beauty
and surprise of the experience made me think, `I've got all those little
flowers in my heart, they're dormant in my mind, and all they need is
just a little bit of rain on them'. And so with that thought a big smile
came over my face and I felt, `Yes, I can do it'. 

Even the desert can sprout flowers. Even when our mind is feeling dry, 
lifeless and dull, if we just keep at the practice, continue the
development of the five \emph{indriyas}, sprinkling the water of Dhamma, 
of mindfulness, clear comprehension and sincere effort, skilfully
applying all the Buddha's wonderful teachings that we've learnt, then we
can create freshness and beauty in the mind. There's always a way
forward. There's always a way to peace. This is the hope that the Buddha
held out to us. All mental conditions are just that: they're conditions. 
They change. And we can influence the nature of that change through
seeing life just as it is, by doing something wise about it through our
study and practice of Dhamma. 

So with \emph{saddhā} in the path, in the quest for truth, seeing its
value, there arises a \emph{viriya} independent of all the passing
feelings of inspiration, depression, like, dislike and pleasure --
those are all part of it, they're not something outside of it. Practice
is developing this right, wise attitude to practice and not taking all
those feelings so seriously -- not taking the person who seems to
experience them so seriously. 

Practically speaking, it is \emph{sati} and \emph{samādhi} that enable
us to see what is what. The practice of \emph{samatha} meditation, 
concentration on an object, is in fact a kind of mindfulness practice. 
It's training our awareness to maintain an uninterrupted conjunction
with an object, for example, the breath. Mindfulness of breathing has
been called the king or the crown jewel of meditation objects, because
it may be used both as a means of calming the mind and also for the
direct penetration of \emph{aniccā, dukkha} and \emph{anattā}. Arousing
the feeling that the breath is more important, more interesting, more
fascinating than anything else in the world makes the practice progress. 
We generate the deep faith that mindfulness of breathing can take us to
liberation. This faith energizes the mind. 

When the mind becomes calm, notice how your attitudes and values change. 
There is a recognition that stuffing the mind full of thoughts and
fantasies is pointless, that dwelling in even subtle forms of anger, 
ill-will or greed is painful and a waste of time; that searching the
universe for pleasant sensory experiences is demeaning and irrelevant. 
You wonder that you never thought about getting out of these traps
before. \emph{Samādhi}, the deep peace and happiness of mind, brings
forth a very different kind of logic from that of the busy mind. 
Suddenly there is a sense of sadness for the time that you've allowed
the mind to hang out with the hindrances, all the time that has been
squandered! You think, `How could I have been so foolish?' To the
peaceful mind, only peace makes sense. 

The mind stabilized by \emph{samādhi} loses its habitual reaction to
objects, which is to rush towards the pleasant and away from the
unpleasant. Without \emph{samādhi} the mind has no home. It has no
dwelling-place, it has nowhere it really wants to be. And so when faced
with objects, the mind rushes around, moving towards the pleasant and
away from the unpleasant, and hovering around the neutral, not quite
sure whether to move towards it or away from it. But peace of mind has a
stabilizing effect on this process. Suddenly it's almost as if the mind
is too content; it just can't be bothered to make a fuss about things
any more. The mind in \emph{samādhi} is quite happy to be where it is, 
at home. 

But the mind is not peaceful all the time. Awareness of the value of
\emph{samādhi} can be lost again. When we're in the hindrance-mode, then
\emph{samādhi} seems so far away. All the teachings about peace of mind
seem like pious platitudes and the practice doesn't really gel. We may
even find ourselves trying to avoid meditation, though we still aspire
to its fruits. But if we are willing to go against the grain, once the
mind starts to become calm and \emph{sati} and \emph{sampajaññā}
increase, that kind of negative thinking appears foolish once more. The
pacification and clarification of the mind's intrinsic power seem so
obviously the most intelligent thing that we could be doing. We see how
state-specific are our thoughts about life. 

If the mind takes joy in its object, chooses it wholeheartedly, then
what starts to become clear is the inherently peaceful nature of the
mind. The meditator experiences clarity, transparency, brightness and
purity; he connects with the strength, resolution and firmness of the
concentrated mind. At the same time, with \emph{samādhi}, we are aware
of a flexibility, suppleness and malleability in the mind. Put into
words, that sounds self contradictory, doesn't it? How is that possible? 
How can the mind be both firm, resolute and rock-solid, and yet at the
same time flexible and pliable? Well, why not? It's not a logical
theorem. It is `\emph{paccatam}', to be realized by each person for
themselves. 

With the practice of \emph{samādhi} the meditator samples the initial
wonders of the inner world. He reaches the gates to the marvellous, 
something few human beings ever experience. Here is where the mind
begins to intuit its full power and potential, and is exhilarated by
that. The meditator sees how unsatisfactory and superficial ordinary
sense-consciousness is -- it's as if human beings are just skating
around on dirty ice looking for water, never aware of the beautiful, 
cool flow beneath their feet. 

As the mind becomes imbued with \emph{sati} and \emph{samādhi}, the
powers of this penetrative awareness can be applied. In accordance with
its nature, the mind will move and flow towards the objects of
investigation and contemplation. The mind emerging from \emph{samādhi}
is naturally ripe for the emergence of \emph{paññā}. With \emph{paññā}, 
what becomes most clear to us is that every aspect of our experience, 
everything that we can perceive and conceive, has the same value. We
enter a calm egalitarian land. Everything does exactly the same thing: 
it arises and then passes away. For the first time the nature of
experience far outweighs the significance of its content. We make a
radical switch or revolution, from obsession with the contents of
experience to the cool, clear-eyed appreciation of the process or
contour of experience, this rising and passing away. With insight and
understanding of the process of rising and passing away, it's here that
\emph{asmimāna}, the upholding of the idea of self, raising the flag of
`me' and `mine', starts to be undermined. 

So as we progress down the path, we come to understand that part of our
development as human beings is the gradual maturing of our understanding
of happiness. The increasingly subtle and profound forms of human
happiness developed through the five \emph{indriyas} of \emph{saddhā, 
viriya, sati, samādhi} and \emph{paññā} are sometimes invisible to
others, especially those who do not practise. They're not objects of
possession and they're not founded upon the \emph{āyatanas}, and yet
these qualities truly sustain the human heart. But if we find ourselves
trapped in that no-man's land where we have given up some of the coarser
pleasures based on gratification of sense desire, but do not yet have
any real access to the higher, more subtle and refined pleasures, 
enjoyments and happiness of the path, then we need to be very patient
and dwell in faith in the Buddha. 

Even though I don't think my critical faculty is lacking in any way, 
after devoting most of my life to studying and practising the Buddha's
teachings to the best of my abilities, I have yet to find a single
teaching that I have been able to disprove. This gives me a great deal
of faith in those aspects of the Dhamma that I have not yet verified. 
It's like a map. If you have found it to be trustworthy in one area of
the landscape, you find it unlikely to be at fault in another. The
Buddha teaches that the practice of Dhamma brings happiness to the human
heart. We trust the Buddha's teachings not by dismissing doubts, but by
putting our life on the line. Faith does not entail the mere acceptance
of a philosophy. Buddhist faith is the faith to do. It is a trust in our
capacity, a belief in our own potential; something we can put to the
test. The daily practice may sometimes feel little more than a stumble
or crawl, but through faith our underlying effort and sincerity is
unwavering. Eventually, attainment of the goal is assured. 



%\chapter{Coming to Visit Wat Pah Nanachat}
%
Wat Pah Nanachat (The International Forest Monastery) is situated in a
small forest in the north-east of Thailand, about fifteen kilometers
from the city of Ubon Ratchathani. In 1975 Ajahn Chah established it in
order to give foreigners who do not know the Thai language and culture
the possibility of a traditional monastic training. English serves as
the primary language of communication and instruction. Our community
consists of monks, novices and postulants from a wide range of
nationalities. There is no permanent nuns' community at Wat Pah
Nanachat. Women interested in a monastic commitment are invited to
contact our affiliated nuns' community at Amaravati Buddhist Monastery,
Great Gaddesden, Hemel Hempstead, Hertfordshire HP1 3BZ, Great Britain.

Although Wat Pah Nanachat is not a meditation centre, there are
facilities for a limited number of male and female guests to stay at the
monastery and practise with the resident monastic community. We like our
guests to follow the daily routines of the monks as much as possible,
and join in all communal meetings and work activities. As the teachers
of the forest tradition stress, in monastic life qualities like
cooperation, respect and self-sacrifice facilitate both communal harmony
and individual growth in the practise. Generally the training at Wat Pah
Nanachat aims to follow the Dhamma-Vinaya, the teachings and code of
monastic discipline laid down by the Buddha, respecting both their
letter and their spirit. The monastic life encourages the development of
simplicity, renunciation and quietude. It is a deliberate commitment to
this way of life that creates a community environment where people of
varied backgrounds, personalities and temperaments can cooperate in the
effort to practise and realize the Buddha's path to liberation.

\section{Practice Schedule}

In general guests have many hours a day for study and meditation
practise, so to make the best use of the situation it is advisable to
have had previous meditation experience in a retreat setting and
exposure to Buddhist teachings.  Below is an outline of the standard
daily routine, which varies from time to time.

\begin{tabular}{l p{80mm}}
3:00 AM & Morning wake-up bell\\
3:30 AM & Morning meeting: chanting and meditation\\
Dawn & Monks go out to surrounding villages on alms-round. Lay guests sweep the monastery or help in the kitchen.\\
8:00 AM & The meal (followed by short instructions and teachings by the Abbot)\\
9:30 AM & Chores period\\
4:00 PM & Afternoon drink\\
6:15 PM & Evening meeting: chanting and meditation\\
\end{tabular}

The schedule may be supplemented by periods of group practise, communal
work or Dhamma instructions according to the needs of the community.
After the meal the Abbot or a senior monk is available to receive
visitors and resident guests and answer questions. Four times in a lunar
month, on Wan Phra (the Buddhist holy day), the community observes a
late-night vigil, during which there is the opportunity to discuss
aspects of Dhamma practise with one of the senior monks.

Much of the day is reserved for private practise, using the time for
sitting and walking meditation in either one's private hut in the forest
or one of the meditation halls. Regarding meditation instructions at Wat
Pah Nanachat, rather than utilizing only one particular technique, we
aim to have our practise include all aspects of daily life, however
simple and ordinary, as opportunities to develop mindfulness and other
spiritual qualities such as diligent effort, joy, contentment, patience
and faith. In time, the virtuous qualities that grow out of such a
training gather strength and contribute towards deeper peace and
concentration leading to insight and the growth of liberating
wisdom.

\section{The Buddhist lay training guidelines (precepts)}

Lay guests who stay at Wat Pah Nanachat are expected to abide by the
traditional eight Buddhist precepts. The first five form the basic
guidelines for conduct leading to harmony and self-respect. The other
three precepts encourage a spirit of renunciation and simplicity and are
among the fundamental principles of monastic practise.

The five training precepts:

\begin{itemize}

  \item \emph{Harmlessness:} to refrain from intentionally taking the life of
    any living creature.

  \item \emph{Trustworthiness:} to refrain from taking anything that is not
    given.

  \item \emph{Chastity:} to refrain from all sexual activity.

  \item \emph{Right Speech:} to refrain from false, abusive, malicious or
    disharmonious speech and worldly gossip.

  \item \emph{Sobriety:} to refrain from taking intoxicating drinks or drugs
    (smoking is prohibited at all times at the monastery).
\end{itemize}

The three renunciation precepts:

\begin{itemize}

  \item \emph{To refrain from eating after midday:} The monastery practise is
    to eat one meal a day in one bowl at one sitting. This frees time
    for meditation and enhances simplicity of life.

  \item \emph{To refrain from using entertainment such as music, dance,
    playing games and beautifying or adorning the body with jewellery or
  makeup:} This assists in focusing the mind's attention inwards towards
  Dhamma.

  \item \emph{To refrain from using high or luxurious beds or seats and from
    indulging in sleep:} This develops the qualities of wakefulness,
    mindfulness and clear awareness in all postures and all activities
    throughout the day.

\end{itemize}

These training precepts are guidelines for good conduct in body and
speech and provide a necessary foundation for the development of
mindfulness, clear comprehension and meditation in our endeavour to
cultivate the Noble Eightfold Path. The precepts serve to promote
harmony within the community through restraining unwholesome speech and
action. These fundamental principles of training cultivate the
self-discipline necessary for spiritual development and are taken up as
an act of deliberate personal choice and initiative.

\section{Staying as a guest}

If you wish to come and stay at Wat Pah Nanachat, you need to write in
advance to the guest monk and allow several weeks in which to receive a
written response. We only have limited space for guests and are often
booked up, so it is good to write well in advance. Please understand
that it is the wish of our community to not be publicly available by
telephone or email. Guests are accepted initially for three days. If
they wish to stay longer, they can consult the guest monk or the Abbot.
The best time to arrive is before 8:00 am, in order to take part in the
meal and meet with the guest monk.

Resident lay guests in Wat Pah Nanachat wear traditional Thai lay
monastic attire: loose white and long trousers with a white shirt for
men, and a white blouse and long black skirt for women. Men staying
longer than one week are asked to shave their heads, beards and
eyebrows. Guests are advised to be in good physical and mental health
and to have health coverage or travel insurance. If you have previously
had any serious mental illnesses, please inform us openly about them, so
we can be sure that your stay in the monastery won't give rise to major
problems for you and the community. There is no malaria at Wat Pah
Nanachat.

While the monastery provides bedding and a mosquito net, guests are
expected to supply other requisites (e.g.~a good flashlight/torch, an
alarm clock, flip-flop sandals, candles, mosquito repellent and
toiletries). A padlock for locking away personal valuables is very
useful. The monks are happy to share food and drinks that are offered to
them with the lay guests each morning, but as it is part of the
renunciant tradition to accept whatever is offered, they are unable to
arrange any special diets for the guests or residents. Please either do
not bring electronic gadgets like mobile phones, portable computers,
cameras, etc. with you, or lock them away in the monastery safe. These
things create a worldly atmosphere which impinges on the simple
meditative lifestyle in the monastery. Also, this is a strictly
non-smoking monastery. Please note that the financial expenses of the
monastery are completely covered by donations out of faith and free will
from our lay community, whether local or international.

\clearpage

If you would like to visit and stay at Wat Pah Nanachat, please write a
letter (suggesting possible dates) to:

The Guest Monk\\
Wat Pah Nanachat\\
Bahn Bung Wai\\
Warin Chamrab\\
Ubon Rachathani 34310\\
THAILAND

Some more information about life at Wat Pah Nanachat can be found on the
monastery's website:

\href{http://www.watpahnanachat.org/}{www.watpahnanachat.org}


%
%\chapter{Transliteration of Thai Words}
%
%\chapter{Glossary}

\end{document}
