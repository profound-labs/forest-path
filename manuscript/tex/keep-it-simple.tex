% Title: Keep it Simple
% Author: Ajahn Pasanno

When considering the Dhamma, if you look at it in one way it is quite
complicated, quite complex; there is a lot to know, a lot to figure out, 
a lot of information to digest. Looked at another way, it is quite
straightforward -- it is just a matter of following it, of doing. There
is a certain element, particularly, it seems, in the Western
temperament, which makes us believe that the more information we have, 
the better we will get to know about something, and so the more
information we have, the better we should be at practising the Dhamma. 
This is actually not true. 

A lot of importance should be given to patience: to being able to be
patient with one's experience, observing oneself, observing the world
around one and learning to trust the observer, the watcher, the ability
of the human mind to pay attention to itself. When we talk about
liberation or enlightenment, we are actually just talking about paying
attention, what the attention is directed towards. So it means learning
to observe oneself, one's experience, to recognize the quality of the
mind. 

The Buddha particularly emphasized the quality of suffering, of
unsatisfactoriness. The Four Noble Truths are based on observing this
quality of unsatisfactoriness. It is something to be known. 
Understanding unsatisfactoriness is a duty to oneself. The problem is
how we relate to the world around us. The way we relate to each other
means that we tend either to create or experience unsatisfactoriness. 
Then we hold on to it, cling to it, judge it, try to avoid it; we create
incredible scenarios around it, we look for someone to blame because of
it, or we feel sorry for ourselves. So we create a whole range of
reaction around \emph{dukkha}. But the Buddha says that all we have to
do is just know it. This quality of knowing is to be turned to, to be
focused on our experience, and then we learn to recognize that this
knowing is a point of balance: not affirmation or rejection, not wanting
or not wanting. It is the balancing of the faculties of the mind. The
body and the mind are the tools we have for experiencing the world. We
revolve around the sense faculties of the body and the faculties of the
mind: the ability to create and experience emotional tones of happiness, 
suffering or neutrality, the ability to remember, to conceptualize, to
put labels on things through perception, the ability to act in a
volitional way, to initiate thought processes and be conscious of the
world around us. These are the tools we have. The practice of the Dhamma
is learning the quality of knowing: knowing the world around us, both
the material world of the physical body and the sense spheres of the
mind, the faculties of the mind; just knowing, not reacting to the
proliferation around them, but just being with the knowing. 

But although this practice is just knowing, it seems complicated because
the simplest things are difficult to sustain. So we need to develop
certain tools, certain qualities. The Dhamma provides a theoretical
framework that may look complicated but facilitates this knowing. It
requires us to come back to the human heart, which is capable of
knowing, capable of peace, capable of creating hell round us and capable
of creating celestial worlds. We have to see this point of clarity and
stillness within us in order to stop creating worlds around us. Once
Ajahn Chah and a group of his disciples went to visit a well-known
disciple of Ajahn Mun, Luang Por Khao. Ajahn Sumedho was one of the
group. They listened to a Dhamma teaching, and when they were leaving, 
Ajahn Sumedho as the most junior monk in the group was the last to leave
the room. Just as he was leaving, Luang Por Khao rose quickly and came up
to him, and since Ajahn Sumedho did not know much Thai at the time, 
Luang Por Khao pointed at his heart and said, `It's all here, it's all
here.' All the talking, the explanations, come back to the heart, we
have to see this clearly and pay attention to the mind, to the heart. 

This is the reason the Buddha gave the teaching of the Four Noble
Truths, as it is the heart, the mind that motivates us. All sentient
beings prefer happiness to suffering, so we are motivated to try to free
ourselves from suffering. Often, however, our attempts to do so are
either superficial or misguided, and only lead to a temporary
appeasement of suffering. We put off really dealing with it to the
immediate or distant future. The Buddha said that it should be dealt
with by understanding its causes, because we can only understand
something when we understand its cause. He pointed out that often our
misapprehension of the truth, of reality, is due to \emph{avijjā}, 
non-knowledge. \emph{Avijjā} is often translated as ignorance, but it is
really the lack of true knowledge. Through this lack of true knowledge, 
different kinds of desire are created: desire to seek out sensual
gratification; desire for the affirmation of self, for becoming; desire
for self-negation, for annihilation, the pushing away of experience --
not wanting to experience things is also a desire. So this pushing and
pulling, this grasping after experience is the real cause of our
suffering. 

And so it is the relinquishing of desire that brings about the cessation
of suffering; we do not relinquish suffering itself. But if we try to
push away its immediate cause and find something more satisfactory, this
is not dealing with the real causes, which have to be seen for what they
are and relinquished. Letting go is something that one needs to feel
consciously; letting go of holding in our hearts the emotional reactions
to experiences and relationships and judgements, how things should or
should not be. It is relinquishing the whole of this, letting go of all
of it, that is summed up in a short teaching the Buddha gave when he
said, `All dhammas are not to be clung to.' This is like the core of his
teaching -- everything has to be relinquished, given up. 

The nature of desire is to hoard, to cling, to attach to things, to hang
on. We have to establish attention to this tendency in our practice and
try to go against it, to let all of it go. When you actually see
suffering, you want to let go of it. The more clearly you see suffering, 
the more willing you are to let go. It is somewhat similar to the method
of trapping monkeys. A small hole is cut in a coconut; it should be just
big enough for a monkey to put his paw in. A piece of some hard fruit is
put inside the coconut. When a monkey comes, being very curious, he puts
his paw in, and finding the fruit, he grabs it. Then he is stuck, as the
hole is too small to get his fist out with the fruit in it. When the
hunter comes, the monkey keeps pulling at his fist but will not let go
of the fruit. His desire for gratification is stronger than the
recognition of the suffering that will follow when the hunter grabs him
by the scruff of his neck. If the monkey could really see the suffering, 
it would be easy for him to let go of the fruit and get away. 

We do the same thing. Suffering is there all the time, but we do not
relinquish it because we do not see it clearly enough. As soon as we see
it, we should let it go. But we do not recognize suffering -- aversion, 
ill-will, anger -- and carry it around with us for long periods of
time: minutes, hours, days, weeks, months, years, because we can justify
it in some way. We are also able to suffer tremendously over things
which we perceive could give us pleasure -- and which may even be
pleasurable on a certain level, but suffering is inherent in them. The
clear recognition of suffering is therefore related to the ability to
let it go. And the ability to let go is clearly related to the degree of
awareness and mindfulness, the stability of knowing. So we come back
again to this quality of knowing: to the establishment of awareness, the
establishment of mindfulness. 

The purpose of this path, the whole point of our practice, is to
facilitate this quality of clarity. Mindfulness or awareness is not
passive; there is a sense of moral responsibility within it, a sense of
patience and endurance, the ability to bring up effort. Our path lies in
developing virtue or \emph{sīla} to see our actions and speech clearly
and take responsibility for them in a moral sense. And we need to
develop the quality of renunciation and the quality of wisdom in our
practice, to question, to investigate, to reflect. 

At the level of mental training we train to bring forth effort, training
to recognize ways of cultivating the wholesome and letting go of the
unwholesome. This is developing stability of mind, concentration, 
steadiness of mind. The steadiness of mind that has to be developed is
an emotional steadiness in the sense of the heart and mind, not in the
sense of the analytical mind. It is the ability not to be drawn by our
habitual preferences, our wanting and not wanting, but to establish
stability. Concentration sometimes has a sense of focusing, of
exclusion. Exclusion, the blinkering of the mind, does not lead to a
really stable and still mind. There has to be an openness, not reacting
to likes and dislikes, an ability to observe, staying with the knowing. 
So the steadiness pertains to the ability of recognition, the ability to
observe without a sense of focusing in an exclusive way. 

So we need to develop the qualities of investigation. And the Buddha has
given the parameters, the boundaries of investigation, what to
investigate and the tools for investigating experience. The structures
of the Four Noble Truths, the five \emph{khandhas} or aggregates of
being and the six sense spheres are tools for the delineation of our
experience. They enable us to recognize the patterns of our mind, the
patterns of our experience. 

So when we sit in meditation, it is very important to have a structure, 
a framework, to guide us in investigating our experience. If we just sit
and watch the breath going in and out, pretty soon either the mind
starts wandering and gets hooked to something or other, or else it
becomes bored and collapses on itself, and you sit in a state of
dullness. When the mind is in \emph{samādhi} it is ready for work, the
work of a meditator, which is to investigate one's experience, to
investigate what it is that motivates one, what causes the mind to
proliferate, what it is that creates suffering. What brings a point of
balance to the mind? These are questions that need to be investigated
when we are engaged in meditation. Sometimes we sit and wait for an
illumination to descend upon us and free us from all confusion, but
that's not how the mind works. In order to understand the mind and
ourselves more clearly, we have to apply the mind to look at and
investigate the actual problems we keep running up against in our
experience. This is where we develop the exercise of coming back to the
breath to encourage mindfulness, using the in-breath and the out-breath
to clarify the movement of the mind. We use the breath as it keeps going
in and out as a framework to see where the mind is moving, to clarify
its movement. This close observation within the framework of the
meditation object clarifies and makes us understand the nature of the
mind. 

Calmness of mind is not obtained by shutting things out or forcing the
mind to a point of stillness. The more you force the mind, the more
tense it becomes. What is needed is application of the mind, using the
tools set out for us by the Buddha. With the application of effort the
practice takes on all kinds of meanings, and one finds that gaps in
one's understanding are filled in and doubts and misunderstandings
overcome. So this aspect of wisdom is not just passive knowledge or a
piece of information that you get from a book or a teacher. It is
arrived at through applying one's mind, investigating one's mind
honestly. Often our minds create distractions for themselves, creating
stories around ourselves. Unless you see the mind for what it is, you
will keep buying into these stories, into all that proliferation. So to
let go we have to develop clear understanding, relinquish the mind's
creations and proliferation, let it all cease. 

As we become more and more familiar with knowing, we are able to find a
quality of relinquishment, a point of stillness within that knowing. 
Ajahn Chah used a lovely image to describe the proliferations of the
mind. He compares them to the wheels of an ox cart, which create deep
tracks that seem to go on endlessly. The wheels are not all that big, 
but the tracks are very long. The purpose of our training and our
practice is to stop that ox cart, to let it come to rest. And this is
where our practice should be going, in the direction of a point of rest. 

\dividerRule

\section{The Author}

Tan Ajahn Pasanno was the Abbot of Wat Pah Nanachat for seventeen
years, between 1980 and 1997. During that period the monastery's
reputation as a training centre for monastics in the Ajahn Chah
tradition grew both in Thailand and abroad. In 1996 plans for the
beginnings of a new branch monastery were under way in California, under
the guidance of Ajahn Amaro. When Ajahn Pasanno joined the project with
a view to sharing the leadership of the new monastery as co-Abbot, 
wonderful as this news was for California, it meant that the Wat Pah
Nanachat community and the Ubon laity would be saying goodbye to their
much-loved teacher.

Ajahn Pasanno nevertheless left Thailand to begin Abhayagiri Buddhist
Monastery in Mendecino County, California, which is now a thriving
forest monastery with a recently-opened branch near Portland, Oregon. In
2010 Ajahn Amaro was invited to return to Amaravati Monastery in England
to take up the post of Abbot there, leaving Ajahn Pasanno as the sole
Abbot at Abhayagiri. 

Luang Por Pasanno continues to inspire men and women to practice
Dhamma and lead the holy life. He combines the many facets of the modern
Western Buddhist monastic culture with maintaining close ties with the
Thai Sangha. On his annual visits to Thailand he always finds time to
stay for a while at Wat Pah Nanachat, where the whole community can
still continue to benefit from his teachings.

