% Title: Progress without Movement
% Author: Samanera Khemavāro

My first encounter with the Dhamma was about eight months ago, at the
end of 1998. A friend of mine, Al, was undergoing short-term ordination
as a monk in a temple near Bangkok, and invited me to come along to the
ceremony and spend a week at the monastery. Al's teacher was an English
monk, Phra Peter. When I first heard some of the Buddha's teachings from
Phra Peter, even though the concepts were all new to me, something
resonated in my head. It was as if he was expressing something deep
inside but inchoate in my consciousness. The teachings on \emph{kamma}
and \emph{sīla} -- if you do good, then good things will happen to you, 
and your goodness will protect you from harm -- these were teachings
that I felt and had always tried to live by. 

\section{New experience, yet familiar}

Though I believed in morality and ethics, I remained sceptical due to
their association with Christian puritanism and self-righteousness. The
Buddhist approach seemed to have a different tone. I found the teachings
on keeping precepts quite attractive. I was taught that this is
something you offer to the world and not something that is demanded from
above. Another appealing aspect of Buddhism for me is that we are
responsible for our own enlightenment. The Buddha discovered the path to
Nibbāna, but it is up to each one of us to make the effort and walk down
that path. The first time Phra Peter told me about the goal of Buddhism, 
which is to do good, refrain from doing evil and purify oneself, it felt
so natural and familiar that I thought to myself, `If I were to
verbalize the criteria or goal of my life, this would be it.'

That whole week at the monastery in Nakorn Sawan, I felt a bit odd. 
While everything was new to me, there was something vaguely familiar
about the monastic setting. I felt really at ease. For example, the
daily devotional chant, even though it was in Pāli and I couldn't
understand a word, gave me great inspiration, and so I went to all
morning and evening sessions. Phra Peter also introduced me to
meditation. Growing up in Los Angeles, a place of never-ending New
Age/spiritual fads and fashion, I was prejudiced against meditation. 
Rather hastily, I lumped it with all that trendy newfangled stuff. My
initial impression of meditation and yoga was that they were something
for bored corporate wives with little to do, the type who would only
drink hyper-hygienic sparkling water with just a twist of lime, 
organically grown by a politically correct commune, and do some yoga or
meditation before visiting their spiritual guru to have their auras
examined. 

Phra Peter felt my keen interest and offered to be my teacher if I
wanted to pursue this religious path further. While I was grateful for
the offer, I felt that a major component of being a monk was discipline, 
and, unfortunately this temple was a city temple which was somewhat lax
in its interpretation of the Vinaya rules. For example, one monk owned a
car and drove it around on the temple grounds. So Phra Peter told me
about Wat Pah Nanachat -- he had spent some time at Amaravati, an
affiliate branch monastery in England -- and what he told me interested
me, so I decided to visit it. 

\section{Past life, fast -- work hard, play hard}

I still remember the powerful surges of conflicting emotions during my
first few days at Wat Pah Nanachat. The monastery is only an hour's
flight away from Bangkok, but my lifestyle there as a
stockbroker seemed worlds apart from the lifestyle of the monastery. As
a stockbroker, my life revolved around information, a constant flow
 (sometimes a whirlwind) of information. A large part of the job is to be
able to sift through the stream of information and determine which piece
of news will have an impact on the stock market. Hence there is a
relentless search for the most updated news and `new' news. By nine
o'clock in the morning I would have read four newspapers (two local, one
regional, and one international); then I would continue to scan for
further news updates from international news services (Reuters and
Bloomberg) and check with the research department regarding recent
developments regarding companies, as well as broad economic and
political trends. 

Working in such a fast-paced environment, one tends to maintain the
momentum throughout the day. After work I would rush to the gym for a
quick workout, then meet up with friends and colleagues for drinks and
then dinner. I would be out until about 10:30 - 11:00 p.m. two or three
times during the week. The weekends would be filled with brunches and
lunches, dinners and clubs. Sometimes I would be literally running from
one appointment to the next. Rarely would I be home before midnight. And
then there would be weekend trips to Phuket, Chiang Mai, Hong Kong or
Singapore. It would not be uncommon for me to go to the airport from
work on Friday night and come back to the office on Monday morning
straight from the airport, having spent the weekend in Hong Kong or
Singapore. I was brought up with the motto `work hard, play hard'. 
Unfortunately, nobody told me about contentment. So in spite of all the
sensory diversions and options available to me, during the past couple
of years I had felt bored and disenchanted with life. Things started to
slow down at work due to the economic recession. Regarding my personal
life, I began to notice that no matter where I was or what I was doing, 
there was an undercurrent of boredom and existential anxiety. I would be
at some `fabulous party' or the `in' club; and then such feelings would
come over me. I would look around and realize that everybody looked as
lost as I was, and seemed to be trying to fill up their lives with the
same type of material possessions, clothes and cars; and sensory
diversions like going to restaurants and clubs and travelling to strange
and exotic places, or self-annihilation through drugs and alcohol. 

\section{Life in the monastery -- paradoxes and parables}

Coming from such a fast-pace and sensory-driven world to Wat Pah
Nanachat, where it seemed that the only sounds were from the swaying
bamboo bushes and leaves falling, was a bit of a shock. It was like
running on the treadmill with headphones on, listening to music with the
volume on high, and suddenly having someone come up and pull out the
plug. Coming from such a sensual world, I remember feeling quite lost
and ill at ease at times with the calm and stillness of monastery life. 
I can recall experiencing many mood swings during my first week. Yet
overall I was quite attracted to the simple and peaceful life in the
monastery, and the structured environment of having scheduled activities
throughout the day really appealed to me. 

In contrast to the myriad choices in the outside world, this structured
and simple monastic lifestyle may seem stifling and monotonous. But
nothing could be further from my experience so far. As a layperson I was
rushing from one sensory experience to another, constantly planning
where I should be next, but never really being present wherever I was. 
If I was eating dinner, my mind would be planning where to go
afterwards: `Should I go to so and so's party or hit the bars, or both?'
And then there was the dilemma of tomorrow: with whom should I go to
lunch? And that would lead to where we should go for lunch, and the same
for dinner, and then bars and clubs, and on and on. The irony in such a
`go-go' and glamorous life, was that everything ended in boredom or, 
worse, oblivion. Most of the time I could not remember what I did
yesterday. I remember thinking: `I got cheated somehow! I have done
everything I am supposed to do. They all told me that if I worked hard, 
followed all the rules, and paid my dues, success would come, and with
it everlasting happiness. By all accounts I am a poster-child of
success. I am barely in my thirties, bringing home six-figure pay
cheques, dining at the best restaurants, taking holidays anywhere in the
world and buying whatever I want. Yet I feel so bored and discontented. 
This is utterly, utterly unfair!' In my fast-paced life there were
endless variations, yet my habitual way of reacting limited my world. 

A good image to illustrate this is a small circle flying through space: 
although the space surrounding the circle is infinite, my habitual way
of responding and seeing things limited my vision to just that small
circle. In a similar way, the Wat Pah Nanachat logo of a lotus in a
square is for me a meaningful image for monastic life. Although the
lotus is contained in the square, it has endless variety in terms of its
positions. Here at the monastery the slight variations and nuances in
monastic life fascinate me. Every day I am excited to wake up to the
endless permutations of life in the monastery. How mindful will I be
during the meal? Walking on alms-round? In my interactions with
different members of the Sangha? The scenery of the sunrise over the
rice fields outside the gate is a constant source of surprise and
delight. Mundane things like the texture of the gravel road I walk on
during the alms-round attract my interest. Is it soft and muddy from the
rain last night, or is it hard from being baked by the hot sun
yesterday? What is the sensation on the soles of my feet? Why do they
hurt more today than yesterday? Am I mindful of my steps or am I off
somewhere plotting a revolution? And then there are my mind states in
the morning. Am I happy and relaxed? Or a bit anxious and irritated? And
what is the cause of these different feelings? And am I mindful of them
as feelings, or do I get caught in them? 

I am still quite mystified by the paradox of how rich and diverse life
at the monastery can be. On one level it can seem quite repetitious and
regimented. With few exceptions, the same things take place every day. 
We go on alms-round at 5:30 in the morning and have our one meal at
about 8:00; after that an hour of chores, followed by more work or
meditation, and then it is tea-time at 4:30 p.m. Yet within that
regulated environment there are countless variations and permutations in
the surroundings, and in myself as well.

It is through this repetitive
and structured environment that I learn about myself, how I perceive and
react to my surroundings. There are many levels to the practice. For
example, at mealtime, how mindful am I when walking down the line to
collect the alms-food? Did I exercise self-restraint and take only a few
pieces of mango, and leave some for the people behind me in the line? Or
did my defilements overwhelm me, so I filled half my bowl full of
mangos? Am I exercising sense restraint in terms of keeping to myself, 
or am I anxiously looking at the front of the line to see which monk is
taking more than his share of the mangos and feeling ill-will towards
him? 

\section{The practice -- walking the walk}

Two areas of the monastic practice that I find interesting in its
contrast to my lay life are the practice of meditating throughout the
night on Observance Day (Wan Phra) and eating once a day. Before being a
stockbroker I was an investment analyst, which entailed writing research
reports about companies listed on the stock market. Working for ING
Barings, one of the top international brokerages in Thailand, I had a
heavy workload and strict publishing deadlines. The company's mantra was
`Publish or Perish'; hence it was not uncommon to work through the night
to meet a particular deadline. As a matter of fact, I had to `pull an
all-nighter' about once a month to get a certain report published by the
deadline. During these all-night sessions we had lots of help to keep
the adrenaline going. There would be a group of people at the office to
help finish the report, and then there were TV and radio, and pizza and
beer. There was much talking and running around to complete the final
details. 

\section{Altered states, altered egos}

At Wat Pah Nanachat we are encouraged to stay up all night and meditate
on Wan Phra, which falls about once every week. But instead of all the
sensory stimuli to help keep the adrenaline going and
the body awake, the only help in that area is a cup of coffee at
midnight. Other than that, one is supposed to meditate quietly by
sitting or walking. Needless to say, staying awake all night is more
challenging without the aid of external stimuli such as TV or radio, but
working with my mind states from 2:00 to 4:00 a.m. has been quite
revealing. One moment I can be feeling dull and sleepy, the next
restless and resentful. The following is the type of internal chatter
that took place during the last Wan Phra at 3:15 a.m.

`I should be in bed. This is a silly practice, staying up all night. 
It's a dead ritual without any rhyme or reason. What am I trying to
prove, anyway? How much \emph{samādhi} could I get in this current state
of stupor? Where is everybody, especially the monks? Why aren't they up
meditating? And why is that senior monk nodding off? He has been doing
this for a while, you would think he'd have got over this problem. He
doesn't seem very developed anyway, and looks as if he hasn't much to
show for all those years. Maybe it's not him, maybe it's the practice. 
Maybe it does not work after all. And why doesn't that stupid clock move
any faster? My knees are sore, my back hurts, and I hate this place.'

This is hardly the picture I have of myself as a calm, collected and
compassionate person. But the beauty of the teaching is that we are
taught to accept things for what they are, being open to all the aspects
of our personality, the good, the bad, and the ugly. The practice has
been helpful for me in recognizing and dealing with my weaknesses and
shortcomings. It is liberating to realize that I have these unwholesome
mind states, but also that they are just mind states, and to be aware of
them as such and not get caught in them or identify with them. 

One of the biggest challenges I have faced so far has been not eating
after noon. Part of the forest tradition discipline is that monastics
only eat one meal a day. With few `medicinal' exceptions such as dark
chocolate, sugar and butter, no solid food should be consumed after
midday. Before coming to Wat Pah Nanachat, food was not an issue for me. 
I have never had a weight problem and can eat pretty much whatever I
want, but always in moderation. However, in the monastery, with so few
outlets for my desires to express themselves, food has taken on a
disproportionate role. I constantly think of things I can eat or
reminisce about all the nice dining experiences I have had. As an
\emph{anāgārika}, one of my responsibilities is to prepare the afternoon
drinks for the Sangha. This entails being in the kitchen and around
food, which has been a challenge for keeping the food precept. Part of
my problem in dealing with the food issue has been that I do not see the
logic in being able to eat dark chocolate in the afternoon, but not a
banana. However, after several discussions with senior monks, I am
beginning to realize that the purpose of the ascetic practices of one
meal a day and not indulging in any worldly behaviours is to calm the
mind, which is conducive to achieving \emph{samādhi} in meditation. 

As a beginner meditator, here again I had that feeling of \emph{déjà
vu}. Not that I was entering the \emph{jhānas} in my first week of
meditation, but at the end of most sessions I have a sense of calmness
and centredness that I find quite refreshing. There is not so much
restlessness, and the preoccupation with food is not so gripping. My
perception of my surroundings seems to be enveloped in a mist of
goodwill and gentility. The irony about just sitting is that in contrast
to my blind pursuit of happiness and excitement in my lay life, which
ended up in boredom and desperation, by just sitting in my \emph{kuṭī}
and counting my breaths, I am finding enthusiasm and contentment. 

\section{Conclusion}

While I have had my share of frustrations and disappointments at Wat Pah
Nanachat when dealing with my own defilements, overall I am finding the
experience fascinating and delightful. And though my monastic life has
been somewhat short, only six months, I am finding joy. There is
excitement, but it is a different kind of excitement from what I found
when working on the stock market. 

I was ordained as a \emph{sāmaṇera} before the Rains Retreat and plan to
remain one for one year. I feel quite fortunate in having had the
opportunity to be living as a monastic so soon after my introduction to
the Dhamma, but I also realize that there is much work to be done. While
I have a strong sense of responsibility to be diligent in putting forth
the efforts required of a monastic, there is also a sense of thrill and
anticipation on this journey of self-discovery. 

%\dividerRule

\clearpage

\section{The Author}

Soon after this piece was published Sāmaṇera Khemavāro took monk
ordination and spent a further couple of years in Thailand. He
subsequently moved to Australia, taking up residence initially at
Bodhiñāṇa Monastery, where he lived for seven years. For the last three
years he has been the Abbot of Wat Buddha Dhamma, a very secluded forest
monastery a couple of hours' drive from Sydney.

